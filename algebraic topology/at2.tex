
\section{Lecture 6 9/16}
Today we continue our discussion of homological algebra. Let $M$ be an $R$-module. 
\begin{defn}[resolution]
    A resolution of $M$ is a positively graded chain complex $\{P_\bullet, \partial_\bullet\}$ such that 
    \begin{enumerate}
        \item $H_n(P_\bullet)=0$ for all $n>0$
        \item $H_0(P_\bullet)=\frac{P_0}{\im\partial_1}\cong M$, where $\partial_1:P_1\to P_0$.
    \end{enumerate}
    We say $\{P_\bullet,\partial_\bullet\}$ is a free resolution if $P_i$ is a free $R$-module for each $i$.
\end{defn}
For resolutions, we prove the following two things: first, free resolutions always exist; second, every $R$-module map can be extended to a map between their resolutions (with extra assumptions) and these extensions are unique up to homotopies.
\begin{prop}
    For any $M$, a free resolution for $M$ exists.
\end{prop}
\begin{proof}
    We proceed this inductively. Defien $P_0$ to be $R\la M\ra$, where it is the free $R$-module defined on the set $M$. Note that 
    \begin{equation*}
        R\la M\ra\twoheadrightarrow M \text{ is surjective}: \la m\ra\mapsto m
    \end{equation*}
    Let $K$ be the kernel of this map, we have an isomorphism:
    \begin{equation*}
        \epsilon: P_0/K\cong M
    \end{equation*}
    Define $P_1$ as $R\la K\ra$, note that $P_1\twoheadrightarrow K$, then we define 
    \begin{equation*}
        \partial_1: P_1\to P_0
    \end{equation*}
    to be the composite 
    \begin{equation*}
        P_1\twoheadrightarrow K\subset P_0
    \end{equation*}
    Now we consider $P_2$: let $K_1\subset P_1$ be the kernel of $\delta_1$, define $P_2=R\la K_1\ra$, then define $\partial_2; P_2\to P_1$ to be the composite"
    \begin{equation*}
        P_2\twoheadrightarrow K_1\subset P_1
    \end{equation*}
    note that $\ker\delta_1/\im\delta_2=K_1/K_1=0$. Then we define $K_2=\ker\delta_2$, define $P_3=R\la K_2\ra, \dots$
\end{proof}
Just like the above proposition, the next theorem uses induction. 
\begin{thm}[extension theorem]
    Let $\{P_\bullet^M, \delta_\bullet^M, \epsilon_M\}$ be a free resolution on $M$, and let $\{P_\bullet^N, \delta_\bullet^N, \epsilon^N\}$ be an arbitrary resolution of $N$. Then given a map of $R$-modules $f: M\to N$, we may extend it to a map of chain complexes:
    \begin{equation*}
        f.: \{P_\bullet^M, \delta_\bullet^M\}\to \{P_\bullet^N, \delta_\bullet^N\}
    \end{equation*}
    such that the following diagram commutes:
    \[\begin{tikzcd}
        {H_0(P_\bullet^M)} & {H_0(P_\bullet^N)} \\
        M & N
        \arrow["{H_0(f_\bullet)}", from=1-1, to=1-2]
        \arrow["{\epsilon_M}", from=1-1, to=2-1]
        \arrow["{\epsilon_N}", from=1-2, to=2-2]
        \arrow["f"', from=2-1, to=2-2]
    \end{tikzcd}\]
    Moreover, given any two extension $f_\bullet^1, f_\bullet^2$ of $f$, we have a chain homotopy $h_\bullet$ between $f_\bullet^1, f_\bullet^2$.
\end{thm}
Remark: if $f_\bullet$ makes the diagram commute, and $g_\bullet$ is homotopic to $f_\bullet$, then $g_\bullet$ also makes the diagram commutes: homotopy classes work the same on homologies (they are the same on the nose).
\begin{proof}
    We will constructu $f_\bullet$ as follows. We construct $f_i$ inductively on $i$. Consider the diagram:
    \[\begin{tikzcd}
        \dots & \dots \\
        {P_1^M} & {P_1^N} \\
        {P_0^M} & {P_0^N} \\
        M & N
        \arrow[from=1-1, to=2-1]
        \arrow[from=1-2, to=2-2]
        \arrow[from=2-1, to=3-1]
        \arrow[from=2-2, to=3-2]
        \arrow["{\textcolor{blue}{f_0}}", dashed, from=3-1, to=3-2]
        \arrow[from=3-1, to=4-1]
        \arrow[from=3-2, to=4-2]
        \arrow["f", from=4-1, to=4-2]
    \end{tikzcd}\]
    Since $P_0^M$ is free, and $\epsilon_N$ is surjective, we may lift $f$ on generators of $P_0^M$ by lifting the geneartors of $P_0^M$ to elements in $P_0^N$. (Note: this lift may not be unique), but this lift extends uniquely to define $f_0$. We notice that the bottom square 
    \[\begin{tikzcd}
        {P_0^M} & {P_0^N} \\
        M & N
        \arrow["{\textcolor{blue}{f_0}}", dashed, from=1-1, to=1-2]
        \arrow[from=1-1, to=2-1]
        \arrow[from=1-2, to=2-2]
        \arrow["f", from=2-1, to=2-2]
    \end{tikzcd}\]
    commutes on homologies $(H_0(P_0^M), H_0(P_0^N))$. Now we construct $f_1$:
    \[\begin{tikzcd}
        \dots & \dots \\
        {P_1^M} & {P_1^N} \\
        {P_0^M} & {P_0^N} \\
        M & N
        \arrow[from=1-1, to=2-1]
        \arrow[from=1-2, to=2-2]
        \arrow["{\textcolor{blue}{f_1}}", dashed, from=2-1, to=2-2]
        \arrow[color={rgb,255:red,153;green,92;blue,214}, from=2-1, to=3-1]
        \arrow[from=2-2, to=3-2]
        \arrow["{f_0}", color={rgb,255:red,153;green,92;blue,214}, dashed, from=3-1, to=3-2]
        \arrow[from=3-1, to=4-1]
        \arrow[color={rgb,255:red,153;green,92;blue,214}, from=3-2, to=4-2]
        \arrow["f", from=4-1, to=4-2]
    \end{tikzcd}\]
    We will follow the purple path above. Recall that $\epsilon_M: H_0(P_0)=P_0/\im(\partial_1^M)\to M$ is an isomorphism. We consider the composite: $f_0\circ\partial_1^M=g$, we have 
    \begin{align*}
        \epsilon_N\circ g&=\epsilon_N\circ f\circ\partial_1^M\\
        &=f\circ\epsilon_M\circ\partial_1^M\\
        =0
    \end{align*}
    This implies that $\im(g)\subset\ker(\partial_N)=\im(\partial_1^N)$. We can lift the generators of $P_1^M$ to elements of $P_1^N$. (Once chosen a lift, one can extend this uniquely to define $f_1$). Then we construct $f_2,f_3,\dots$ the same way by considering $f_n\circ\partial_{n+1}$ and show that it is in the kernel of $\partial_n^N$ and lift it to define $\partial_{n+1}$.

    Now we homotopy time. Assume $f_\bullet^1, f_\bullet^2$ are two lifts of $f$, we construct $h: P_\bullet^M\to P_{\bullet+1}^N$. We define $h_\bullet$ inductively, starting with $h_0$ below:
    \[\begin{tikzcd}
        {P_1^M} & {P_1^N} \\
        {P_0^M} & {P_0^N} \\
        M & N
        \arrow["{{f_1}}", dashed, from=1-1, to=1-2]
        \arrow["{\partial_1^M}"', from=1-1, to=2-1]
        \arrow["{\partial_1^N}", from=1-2, to=2-2]
        \arrow["{\textcolor{red}{h_0}}", dashed, from=2-1, to=1-2]
        \arrow["{f_0^1,f_0^2}"', dashed, from=2-1, to=2-2]
        \arrow["{\varepsilon_M}"', from=2-1, to=3-1]
        \arrow["{\varepsilon_N}", from=2-2, to=3-2]
        \arrow["f", from=3-1, to=3-2]
    \end{tikzcd}\]
    We have $\epsilon_N(f_0^1-f_0^2)=0$, then 
    \begin{equation*}
        f_0^1-f_0^2\in\ker\epsilon_N=\im\delta_1^N
    \end{equation*}
    we may lift $f_0^1-f_0^2$ on generators of $P_0^M$, where $h_0: P_0^M\to P_1^N$. Hence 
    \begin{equation*}
        (h_{-1}\circ\delta_{-1}^N)+\delta_1^N\circ h_0=f_0^1-f_0^2
    \end{equation*}
    Inductively, we assume $h_m$ exists for $m\leq n$, then 
    \[\begin{tikzcd}
        {P_{n+2}^M} & {P_{n+2}^N} \\
        {P_{n+1}^M} & {P_{n+1}^N} \\
        {P_n^M} & {P_n^N}
        \arrow["{{f_1}}", dashed, from=1-1, to=1-2]
        \arrow["{\partial_{n+2}^M}"', from=1-1, to=2-1]
        \arrow["{\partial_{n+2}^N}", from=1-2, to=2-2]
        \arrow["{\textcolor{red}{h_{n-1}}}", dashed, from=2-1, to=1-2]
        \arrow["{f_{n+1}^1,f_{n+1}^2}", dashed, from=2-1, to=2-2]
        \arrow["{\partial_{n+1}^M}"', from=2-1, to=3-1]
        \arrow["{\partial_{n+1}^N}", from=2-2, to=3-2]
        \arrow["{\textcolor{red}{h_n}}"{description}, dashed, from=3-1, to=2-2]
        \arrow["f", from=3-1, to=3-2]
    \end{tikzcd}\]
    consier the expressions 
    \begin{equation*}
        g_{n+1}:=f_{n+1}^1-f_{n+1}^2-h_n\circ\partial_{n+1}^M
    \end{equation*}
    we can check (by diagram chasing), $\partial_{n+1}^N\circ g=0$. This implies that 
    \begin{equation*}
        \im(g_{n+1})\subset\im(\partial_{n+2}^N)
    \end{equation*}
    we can construct $h_{n+1}$ to get the map 
    \begin{equation*}
        \delta_{n+2}^N\circ h_{n+1}=g_{n+1}=f_{n+1}^1-f_{n+1}^2-h_n\circ\partial_{n+1}^M
    \end{equation*}
    i.e.
    \begin{equation*}
        \partial_{n+2}^N\circ h_{n+1}+h_n\circ\partial_{n+1}^M=f_{n+1}^1-f_{n+1}^2
    \end{equation*}
    hence we are done!
\end{proof}
\begin{cor}
    Any two free resolutions of an $R$-module $M$ are homotopy equivalent: given two free resolutions $P_\bullet^M, Q_\bullet^M$, there exists extension of $\text{id}: M\to M$ and such that 
    \begin{equation*}
        f_\bullet: P_\bullet^M\to Q_\bullet^M, g_\bullet: Q_\bullet^M\to P_\bullet^M
    \end{equation*}
    such that 
    \begin{equation*}
        g_\bullet\circ f_\bullet= \text{id}, f_\bullet\circ g_\bullet= \text{id}
    \end{equation*}
\end{cor}
\begin{prob}[HW(2.1)]
    Prove this corollary.
\end{prob}

