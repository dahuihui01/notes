
\section{Lecture 6 9/16}
Today we continue our discussion of homological algebra. Let $M$ be an $R$-module. 
\begin{defn}[resolution]
    A resolution of $M$ is a positively graded chain complex $\{P_\bullet, \partial_\bullet\}$ such that 
    \begin{enumerate}
        \item $H_n(P_\bullet)=0$ for all $n>0$
        \item $H_0(P_\bullet)=\frac{P_0}{\im\partial_1}\cong M$, where $\partial_1:P_1\to P_0$.
    \end{enumerate}
    We say $\{P_\bullet,\partial_\bullet\}$ is a free resolution if $P_i$ is a free $R$-module for each $i$.
\end{defn}
For resolutions, we prove the following two things: first, free resolutions always exist; second, every $R$-module map can be extended to a map between their resolutions (with extra assumptions) and these extensions are unique up to homotopies.
\begin{prop}
    For any $M$, a free resolution for $M$ exists.
\end{prop}
\begin{proof}
    We proceed this inductively. Defien $P_0$ to be $R\la M\ra$, where it is the free $R$-module defined on the set $M$. Note that 
    \begin{equation*}
        R\la M\ra\twoheadrightarrow M \text{ is surjective}: \la m\ra\mapsto m
    \end{equation*}
    Let $K$ be the kernel of this map, we have an isomorphism:
    \begin{equation*}
        \epsilon: P_0/K\cong M
    \end{equation*}
    Define $P_1$ as $R\la K\ra$, note that $P_1\twoheadrightarrow K$, then we define 
    \begin{equation*}
        \partial_1: P_1\to P_0
    \end{equation*}
    to be the composite 
    \begin{equation*}
        P_1\twoheadrightarrow K\subset P_0
    \end{equation*}
    Now we consider $P_2$: let $K_1\subset P_1$ be the kernel of $\delta_1$, define $P_2=R\la K_1\ra$, then define $\partial_2; P_2\to P_1$ to be the composite"
    \begin{equation*}
        P_2\twoheadrightarrow K_1\subset P_1
    \end{equation*}
    note that $\ker\delta_1/\im\delta_2=K_1/K_1=0$. Then we define $K_2=\ker\delta_2$, define $P_3=R\la K_2\ra, \dots$
\end{proof}
Just like the above proposition, the next theorem uses induction. 
\begin{thm}[extension theorem]
    Let $\{P_\bullet^M, \delta_\bullet^M, \epsilon_M\}$ be a free resolution on $M$, and let $\{P_\bullet^N, \delta_\bullet^N, \epsilon^N\}$ be an arbitrary resolution of $N$. Then given a map of $R$-modules $f: M\to N$, we may extend it to a map of chain complexes:
    \begin{equation*}
        f.: \{P_\bullet^M, \delta_\bullet^M\}\to \{P_\bullet^N, \delta_\bullet^N\}
    \end{equation*}
    such that the following diagram commutes:
    \[\begin{tikzcd}
        {H_0(P_\bullet^M)} & {H_0(P_\bullet^N)} \\
        M & N
        \arrow["{H_0(f_\bullet)}", from=1-1, to=1-2]
        \arrow["{\epsilon_M}", from=1-1, to=2-1]
        \arrow["{\epsilon_N}", from=1-2, to=2-2]
        \arrow["f"', from=2-1, to=2-2]
    \end{tikzcd}\]
    Moreover, given any two extension $f_\bullet^1, f_\bullet^2$ of $f$, we have a chain homotopy $h_\bullet$ between $f_\bullet^1, f_\bullet^2$.
\end{thm}
Remark: if $f_\bullet$ makes the diagram commute, and $g_\bullet$ is homotopic to $f_\bullet$, then $g_\bullet$ also makes the diagram commutes: homotopy classes work the same on homologies (they are the same on the nose).
\begin{proof}
    We will constructu $f_\bullet$ as follows. We construct $f_i$ inductively on $i$. Consider the diagram:
    \[\begin{tikzcd}
        \dots & \dots \\
        {P_1^M} & {P_1^N} \\
        {P_0^M} & {P_0^N} \\
        M & N
        \arrow[from=1-1, to=2-1]
        \arrow[from=1-2, to=2-2]
        \arrow[from=2-1, to=3-1]
        \arrow[from=2-2, to=3-2]
        \arrow["{\textcolor{blue}{f_0}}", dashed, from=3-1, to=3-2]
        \arrow[from=3-1, to=4-1]
        \arrow[from=3-2, to=4-2]
        \arrow["f", from=4-1, to=4-2]
    \end{tikzcd}\]
    Since $P_0^M$ is free, and $\epsilon_N$ is surjective, we may lift $f$ on generators of $P_0^M$ by lifting the geneartors of $P_0^M$ to elements in $P_0^N$. (Note: this lift may not be unique), but this lift extends uniquely to define $f_0$. We notice that the bottom square 
    \[\begin{tikzcd}
        {P_0^M} & {P_0^N} \\
        M & N
        \arrow["{\textcolor{blue}{f_0}}", dashed, from=1-1, to=1-2]
        \arrow[from=1-1, to=2-1]
        \arrow[from=1-2, to=2-2]
        \arrow["f", from=2-1, to=2-2]
    \end{tikzcd}\]
    commutes on homologies $(H_0(P_0^M), H_0(P_0^N))$. Now we construct $f_1$:
    \[\begin{tikzcd}
        \dots & \dots \\
        {P_1^M} & {P_1^N} \\
        {P_0^M} & {P_0^N} \\
        M & N
        \arrow[from=1-1, to=2-1]
        \arrow[from=1-2, to=2-2]
        \arrow["{\textcolor{blue}{f_1}}", dashed, from=2-1, to=2-2]
        \arrow[color={rgb,255:red,153;green,92;blue,214}, from=2-1, to=3-1]
        \arrow[from=2-2, to=3-2]
        \arrow["{f_0}", color={rgb,255:red,153;green,92;blue,214}, dashed, from=3-1, to=3-2]
        \arrow[from=3-1, to=4-1]
        \arrow[color={rgb,255:red,153;green,92;blue,214}, from=3-2, to=4-2]
        \arrow["f", from=4-1, to=4-2]
    \end{tikzcd}\]
    We will follow the purple path above. Recall that $\epsilon_M: H_0(P_0)=P_0/\im(\partial_1^M)\to M$ is an isomorphism. We consider the composite: $f_0\circ\partial_1^M=g$, we have 
    \begin{align*}
        \epsilon_N\circ g&=\epsilon_N\circ f\circ\partial_1^M\\
        &=f\circ\epsilon_M\circ\partial_1^M\\
        =0
    \end{align*}
    This implies that $\im(g)\subset\ker(\partial_N)=\im(\partial_1^N)$. We can lift the generators of $P_1^M$ to elements of $P_1^N$. (Once chosen a lift, one can extend this uniquely to define $f_1$). Then we construct $f_2,f_3,\dots$ the same way by considering $f_n\circ\partial_{n+1}$ and show that it is in the kernel of $\partial_n^N$ and lift it to define $\partial_{n+1}$.

    Now we homotopy time. Assume $f_\bullet^1, f_\bullet^2$ are two lifts of $f$, we construct $h: P_\bullet^M\to P_{\bullet+1}^N$. We define $h_\bullet$ inductively, starting with $h_0$ below:
    \[\begin{tikzcd}
        {P_1^M} & {P_1^N} \\
        {P_0^M} & {P_0^N} \\
        M & N
        \arrow["{{f_1}}", dashed, from=1-1, to=1-2]
        \arrow["{\partial_1^M}"', from=1-1, to=2-1]
        \arrow["{\partial_1^N}", from=1-2, to=2-2]
        \arrow["{\textcolor{red}{h_0}}", dashed, from=2-1, to=1-2]
        \arrow["{f_0^1,f_0^2}"', dashed, from=2-1, to=2-2]
        \arrow["{\varepsilon_M}"', from=2-1, to=3-1]
        \arrow["{\varepsilon_N}", from=2-2, to=3-2]
        \arrow["f", from=3-1, to=3-2]
    \end{tikzcd}\]
    We have $\epsilon_N(f_0^1-f_0^2)=0$, then 
    \begin{equation*}
        f_0^1-f_0^2\in\ker\epsilon_N=\im\delta_1^N
    \end{equation*}
    we may lift $f_0^1-f_0^2$ on generators of $P_0^M$, where $h_0: P_0^M\to P_1^N$. Hence 
    \begin{equation*}
        (h_{-1}\circ\delta_{-1}^N)+\delta_1^N\circ h_0=f_0^1-f_0^2
    \end{equation*}
    Inductively, we assume $h_m$ exists for $m\leq n$, then 
    \[\begin{tikzcd}
        {P_{n+2}^M} & {P_{n+2}^N} \\
        {P_{n+1}^M} & {P_{n+1}^N} \\
        {P_n^M} & {P_n^N}
        \arrow["{{f_1}}", dashed, from=1-1, to=1-2]
        \arrow["{\partial_{n+2}^M}"', from=1-1, to=2-1]
        \arrow["{\partial_{n+2}^N}", from=1-2, to=2-2]
        \arrow["{\textcolor{red}{h_{n-1}}}", dashed, from=2-1, to=1-2]
        \arrow["{f_{n+1}^1,f_{n+1}^2}", dashed, from=2-1, to=2-2]
        \arrow["{\partial_{n+1}^M}"', from=2-1, to=3-1]
        \arrow["{\partial_{n+1}^N}", from=2-2, to=3-2]
        \arrow["{\textcolor{red}{h_n}}"{description}, dashed, from=3-1, to=2-2]
        \arrow["f", from=3-1, to=3-2]
    \end{tikzcd}\]
    consier the expressions 
    \begin{equation*}
        g_{n+1}:=f_{n+1}^1-f_{n+1}^2-h_n\circ\partial_{n+1}^M
    \end{equation*}
    we can check (by diagram chasing), $\partial_{n+1}^N\circ g=0$. This implies that 
    \begin{equation*}
        \im(g_{n+1})\subset\im(\partial_{n+2}^N)
    \end{equation*}
    we can construct $h_{n+1}$ to get the map 
    \begin{equation*}
        \delta_{n+2}^N\circ h_{n+1}=g_{n+1}=f_{n+1}^1-f_{n+1}^2-h_n\circ\partial_{n+1}^M
    \end{equation*}
    i.e.
    \begin{equation*}
        \partial_{n+2}^N\circ h_{n+1}+h_n\circ\partial_{n+1}^M=f_{n+1}^1-f_{n+1}^2
    \end{equation*}
    hence we are done!
\end{proof}
\begin{cor}
    Any two free resolutions of an $R$-module $M$ are homotopy equivalent: given two free resolutions $P_\bullet^M, Q_\bullet^M$, there exists extension of $\text{id}: M\to M$ and such that 
    \begin{equation*}
        f_\bullet: P_\bullet^M\to Q_\bullet^M, g_\bullet: Q_\bullet^M\to P_\bullet^M
    \end{equation*}
    such that 
    \begin{equation*}
        g_\bullet\circ f_\bullet= \text{id}, f_\bullet\circ g_\bullet= \text{id}
    \end{equation*}
\end{cor}
\begin{prob}[HW(2.1)]
    Prove this corollary.
\end{prob}
Next we deinfe Tor functors (pretty hard things). 
\begin{defn}[tor functors]
    Let $N$ be an $R$-module, recall the functor 
    \begin{equation*}
        -\otimes_RN: Mod_R\to Mod_R
    \end{equation*}
    we define a collection of functors 
    \begin{equation*}
        \text{Tor}_R^i(-,N): Mod_R\to Mod_R, i\in\mathbb{N}
    \end{equation*}
    given an object $M$ in $Mod_R$, let $\{P_\bullet^M, \partial_\bullet^M, \epsilon_M\}$ be a free resolution of $M$, define $\text{Tor}^i(M,N)$ to be 
    \begin{equation*}
        \text{Tor}^i(M,N)=H_i(P_\bullet^M\otimes_RN, \partial_\bullet^M\otimes\text{id}_N)
    \end{equation*}
    where 
    \begin{equation*}
        \dots\to P_i^M\otimes N\xrightarrow{\partial_i\otimes\text{id}}P_{i-1}^M\otimes_RN\to\dots
    \end{equation*}
\end{defn}
We make the remark that there is a choice involved in picking the free resolution, but this is unique since homotopies are the same on homologies.
\begin{prob}[HW(2.2)]
    For all $i$, show that  $\tor_R^i(M,N)$ is a well-defined functor, and any other choice of free resolution of all objects yields an isomorphic functor. Hint: use the above corollary.
\end{prob}
\begin{prob}[HW(2.3)]
    Show that 
    \begin{enumerate}
        \item $\tor_R^i(R,N)=0$ for all $i>0$
        \item $\tor_R^i(M\oplus M', N)\cong\tor_R^i(M)\oplus\tor_R^i(N)$, given by the natural isomorphism.
    \end{enumerate}
\end{prob}
We claim that $\epsilon_M:P_0^M\to M$ induces the following isomorphism
\begin{equation*}
    \tor_R^0(M,N)\cong M\otimes_RN
\end{equation*}
and $\tor_R^i(M,N)$'s are called the highest derived functors of $-\otimes_RN$.

\section{Lecture 7 9/18}
We continue with our discussion of tor functors. We claim that 
\begin{prop}
    The natural isomorphism gives the following 
    \begin{equation*}
        \tor_R^0(-,N)\cong-\otimes_RN
    \end{equation*}
    i.e., for any $M$,
    \begin{equation*}
        \tor_R^0(M,N)\cong M\otimes_RN
    \end{equation*}
\end{prop}
\begin{proof}
    By definition, $\tor_R^0(M,N)$ is the $0$-th hoology of 
    \begin{equation*}
        \dots\to P_1^M\otimes_RN\xrightarrow{\partial_1\otimes\text{id}_N}P_0^M\otimes_RN\to0\to0\to\dots
    \end{equation*}
    this implies that 
    \begin{equation*}
        \tor_R^0(M,N)=\frac{P_0^M\otimes_RN}{\im(\partial_1\otimes\text{id}_R)}
    \end{equation*}
    We complete the proof using the following lemma.
    \begin{lem}
    We claim that the functor $-\otimes_RN$ is right exact, meaning that give a sequence of $R$-modules,
    \begin{equation*}
        A_1\xrightarrow{f}A_0\xrightarrow{g}M\to 0
    \end{equation*}
    that is exact at $A_0$ and $M$, the following sequence:
    \begin{equation*}
        A_1\otimes_RN\xrightarrow{f\otimes\text{id}}A_0\otimes_RN\xrightarrow{g\otimes id}M\otimes_RN\to 0
    \end{equation*}
    is also exact at $A_0\otimes_RN$ and $M\otimes_RN$.
    \end{lem}

    If we assume the lemma for now, then applying it to 
    \begin{equation*}
        P_1\xrightarrow{\partial_1^M}P_0\xrightarrow{\epsilon}M\to 0
    \end{equation*}
    then we are done!

    We prove the lemma now: exactness of $M\otimes_RN$ implies that $g\otimes\text{id}$ is sujrective. givn that $g;A_0\to M$ is sujective, every generator of $M\otimes n$ in $M\otimes_RN$ s of the form $g\otimes\id(a\otimes n)$ for some $a\in A_0$. This implies that $g\otimes\id$ is surjecctive. 

    Next, we need to show that $\ker(g\otimes\id)=\im(f\otimes\id)$. It is clear that $\supset$ holds, hence it suffices to show $\subset$. Let $K=\ker g\otimes\id$, we need to show that 
    \begin{equation*}
        \frac{A_0\otimes_RN}{K}\to\frac{A_0\otimes_RN}{\im(f\circ\id)}
    \end{equation*}
    is surjective. It is enough to construct a map:
    \begin{equation*}
        M\otimes_RN\to\frac{A_0\otimes_RN}{\im(f\circ\id)}
    \end{equation*}
    by the first isomorphism thoerem in algebra and the fact that $g\otimes\id$ is surjective. To get such a map, we need to construct a bilinear map 
    \begin{equation*}
        M\times N\to\frac{A_0\otimes_RN}{\im(f\circ\id)}
    \end{equation*}
    defined as
    \begin{equation*}
        (m,n)\mapsto (a, n)
    \end{equation*}
    and let $a=g^{-1}(m)$ be a choice of the preimage. We remark that there could be many choices of $a$, but the difference $a_1-a_2$ comes from $f$, since $A_0$ is exact. This implies that this map is well-defined. This implies that the above map is surjective. There for 
    \begin{equation*}
        M\times N\to\frac{A_0\otimes_RN}{\im(f\circ\id)}\xrightarrow{g\otimes\id}M\otimes_RN
    \end{equation*}
    this composition is surjective. (Two surjective maps and maps to identity=isomorphism).
\end{proof}


\begin{warn}
    We saw that tensor product preserves surjectivity, but it does not necessarily preserve injectivity. Namely, if we replace the statement of the claim with SES 
    \begin{equation*}
        0\to A_1\xrightarrow{f}A_0\xrightarrow{g}M\to 0
    \end{equation*}
    and consider 
    \begin{equation*}
        0\to A_1\otimes N\to\dots
    \end{equation*}
    $f$ need not to be injective.
\end{warn}
Next we see the sufficient condition for $\tor_R^i$ to vanish for all $i\geq 2$.
\begin{cor}
    If $R$ is a PID, then $\tor_R^i(M,N)=0$ for all $i\geq 2$.
\end{cor}
\begin{proof}
    Consider the free resolution of $M$:
    \begin{equation*}
        0\to K\to R\la M\ra\to 0\to\dots
    \end{equation*}
    such that $R\la M\ra/K\cong M$.
    Recall that all submodules of a free module are free, so we can just take $K=P_1$, then we have 
    \begin{equation*}
        0\to K\otimes_RN\to R\la M\ra\otimes_RN\to 0\to 0\to\dots
    \end{equation*}
    so the only homologies are $\tor_R^0, \tor_R^1$.
\end{proof}
\begin{prob}[HW(2.4)]
    Calculate $\tor_\Z^1$ and $\tor_\Z^1(\Z/n\Z,\Z/m\Z)$ for $m,n>0$. (Note: $m,n$ could be equal or not).
\end{prob}
\begin{defn}[ext functor]
    Fix an $R$-module $N$, consider the functor 
    \begin{equation*}
        \text{Hom}_R(-;N):Mod_R^{op}\to Mod_R
    \end{equation*}
    Define the functors $\ext_R^i(-,N): Mod_R^{op}\to Mod_R$ as follows:
    \begin{equation*}
        \ext_R^i(M,N)=H^i(\text{Hom}_R(P_\bullet^M,N))
    \end{equation*}
    where $P_\bullet^M$ is a free resoltuion of $M$. 
\end{defn}
We note that if $R$ is a PID, then $\ext_R^i(M,N)=0$ for all $i\geq 2$. 
\begin{prop}
    We have 
    \begin{equation*}
        \ext_R^0(M,N)\cong\text{Hom}(M,N)
    \end{equation*}
\end{prop}
\begin{proof}
    This requires the following lemma:
    \begin{lem}
        If 
        \begin{equation*}
            A_1\xrightarrow{f}A_0\xrightarrow{g}M\to 0
        \end{equation*}
        is right exact, then 
        \begin{equation*}
            0\to\text{Hom}_R(M,N)\to \text{Hom}_R(A_0,N)\to 0
        \end{equation*}
        is exact at $\text{Hom}_R(M,N)$ and $\text{Hom}_R(A_0,N)$.
    \end{lem}
\end{proof}
\begin{prob}[HW(2.5)]
    Prove the above lemma.
\end{prob}
\begin{prob}[HW(2.6)]
    Prove the following statements about the $\ext$ functor.
    \begin{enumerate}
        \item \begin{equation*}
            \ext_R^i\left(\bigoplus_{\alpha}M_\alpha,N\right)\cong\prod_{\alpha}\ext_R^i(M_\alpha,N)
        \end{equation*}
        \item \begin{equation*}
            \ext_R^i(M,\prod_{\alpha}N_\alpha)\cong\prod_\alpha\ext_R^i(M,N_\alpha)
        \end{equation*}
        \item Calculate \begin{equation*}
            \ext_\Z^1(\Z/n\Z,\Z/m\Z)
        \end{equation*}
    \end{enumerate}
\end{prob}
Next we state and prove Algebraic Kunneth theorem.
\begin{thm}[AKT]
    Let $R$ be a PID, and let $\{M_\bullet,\partial_\bullet^M\}, \{N_\bullet, \partial_\bullet^N\}$ be PGCC of $R$-modules such that $M_i$ is free for all $i$. Then there exists a SES:
    \begin{equation*}
        0\to\bigoplus_{i+j=n}H_i(M)\otimes_R H_j(N)\xrightarrow{X} H_n((M\otimes_RN)_\bullet)\to\bigoplus_{i+j=n-1}\tor_R^1(H_i(M_\bullet), H_j(N_\bullet))\to 0
    \end{equation*}
    where $X$ denotes the algebraic crossproduct.
\end{thm}
\begin{proof}
    \textcolor{red}{too long, will type up later} 
\end{proof}

\section{Lecture 8 9/23}
\begin{cor}
    Let $R$ be a field, then the algebraic crossproduct induces an isomorphism:
    \begin{equation*}
        0\to\bigoplus_{i+j=n}H_i(M)\otimes_\F H_j(N)\cong H_n(M\otimes_\F N)
    \end{equation*}
    where the isomorphism is given by the algebraic crossproduct $X$.
\end{cor}
\begin{cor}[Universal Coefficient Theorem]
    Let $\{M_\bullet,\partial_\bullet^M\}$ be a chain complex of free $\Z$-modules, and let $R$ be any commutative ring, then there is a SES:
    \begin{equation*}
        0\to H_n(M_\bullet)\otimes_\Z R\xrightarrow{f} H_n(M\otimes_\Z R)\to \tor_\Z^1(H_{n-1}(M),R)\to 0
    \end{equation*}
    where $f$ is injective but not necessarily surjective (the failure to be surjective is measured by $\tor_R^1$).
\end{cor}
\begin{proof}
    Use AKT with 
    \begin{equation*}
        N_i=\begin{cases}
            R, i=0\\
            0, i>0
        \end{cases}, \quad 
        H_i(N)=\begin{cases}
            R, i=0\\
            0, i\neq 0
        \end{cases}
    \end{equation*}
    Hence $(M\otimes_\Z N)_\bullet =M_\bullet\otimes_\Z R$.
\end{proof}
\begin{prob}[HW(2.7)]
    Prove the UCT in cohomology: let $\{M_\bullet, \partial_\bullet^M\}$ be a chain complex of free $\Z$-modules, let $R$ be any commutative ring, then there exists SES 
    \begin{equation*}
        0\to \ext_\Z^1(H_{n-1}(M),R)\to H^n(\text{Hom}_\Z(M_\bullet,R))\xrightarrow{\beta}\text{Hom}_\Z(H_n(M),R)\to 0
    \end{equation*}
    Hint: use the same proof for AKT, instead of $\otimes$ with $N$, you take the Hom into $R$.
\end{prob}


\newpage
\chapter{Singular Cohomology}
We begin with some basic definitions.
\begin{defn}[$n$-simplex]
    The standard $n$-simplex $\Delta_n\subset\R^{n+1}$ is defined as 
    \begin{equation*}
        \Delta_n=\left\{x\in\R^{n+1}: x=\sum_{i=0}^nt_ie_i, t_i\geq 0, \sum_{t_i}=1\right\}
    \end{equation*}
    where $e_i, 0\leq i\leq n$ are the standard basis vectors of $\R^{n+1}$.
\end{defn}
\begin{defn}[face]
    Let $0\leq i\leq n$, then the $i$th face $F_i$ of $\Delta_n$ is the $(n-1)$-simplex 
    \begin{equation*}
        F_i=\{x\in\Delta_n:t_i=0\}
    \end{equation*}
\end{defn}
\begin{defn}[singular chain complex]
    Given a topological space $X$, the singular chain complex of $X$, with $\Z$ coefficients, denoted as $S_\bullet(X,Z)$ is defined as 
    \begin{equation*}
        S_i(X,Z)=\begin{cases}
            0, i<0\\
            \Z\la\Delta_i(X)\ra, i\geq 0
        \end{cases}
    \end{equation*}
    where $\Delta_i(X)$ is the set of continuous maps from $\Delta_i\to X$. We define $\partial_n:S_n(X,\Z)\to S_{n-1}(X,\Z)$ as follows:
    \begin{equation*}
        \partial_n\la f\ra=\sum_{i=0}^n(-1)^i\la f\circ F_i\ra
    \end{equation*}
    where $\la f\ra$ is a generator of $\Delta_i(X)$, and $f:\Delta_X\to X$, where 
    \begin{equation*}
        f\circ F_i=\Delta_{n-1}\to\Delta_n\xrightarrow{f}X
    \end{equation*}
    Note to complete this definition, one needs to check that $\partial^2=0$, which we did in class. \textcolor{red}{might include this later}
\end{defn}

\section{Lecture 9 9/25}
Recall that last time, we defined the singular chain complexes $S_\bullet(X,\Z)$ with $\Z$-coefficients:
\begin{equation*}
    S_i(X,\Z)=\begin{cases}
        0, i<0\\
        \Z\la\Delta_i(X)\ra, i\geq 0
    \end{cases}
\end{equation*}
where $\Delta_i(X)$ is the set of continuous maps from $\Delta_i$ to $X$. Now we discuss some variations of this concept.
\begin{defn}[relative singular chain complex with $\Z$-coefficients]
    Let $A\subset X$ be a subspace, define $S_\bullet(X,A\Z)$ by 
    \begin{equation*}
        S_i(X,A,\Z)=\begin{cases}
            0, i<0\\
            \frac{\Z\la\Delta_i(X)\ra}{\Z\la\Delta_i(A)\ra}, i\geq 0
        \end{cases}
    \end{equation*}
    note that the quotient is still free.
\end{defn}
We note that $S_\bullet(X,A,\Z)$ is a chain complex with the following $\partial$ maps such that the following diagram commutes:
\[\begin{tikzcd}
	{S_i(A,\Z)} & {S_{i-1}(A,\Z)} \\
	{S_i(X,\Z)} & {S_{i-1}(X,\Z)} \\
	{S_i(X,A,\Z)} & {S_{i-1}(X,A,\Z)}
	\arrow["{\partial_i}", from=1-1, to=1-2]
	\arrow[from=1-1, to=2-1]
	\arrow[from=1-2, to=2-2]
	\arrow["{\partial_i}", from=2-1, to=2-2]
	\arrow[two heads, from=2-1, to=3-1]
	\arrow[two heads, from=2-2, to=3-2]
	\arrow["{\partial_i}", from=3-1, to=3-2]
\end{tikzcd}\]
\begin{defn}[$S_\bullet(X,A,R)$]
    We define $S_\bullet(X,A,R)$, where $R$ is any commutative ring, and 
    \begin{equation*}
        S_\bullet(X,A,R)=S_\bullet(X,A,\Z)\otimes_\Z R
    \end{equation*}
    it is a chain complex of $R$-modules with $\partial_i$ induced from $S_\bullet(X,A,\Z)$.
\end{defn}
The last variation is as follows:
\begin{defn}[singular cochain complex]
    Define the singular cochain complex of $R$-modules $S^\bullet(X,A,R)$ as follows:
    \begin{equation*}
        S^i(X,A,R):=\text{Hom}_\Z(S_i(X,A,\Z),R)=\text{Hom}_R(S_i(X,A,R),R)
    \end{equation*}
    where $\partial^i$ is induced from $\partial_i$ in $S_\bullet(X,A,R)$.
\end{defn}
We make the following remark: if $A=\emptyset$, then $S_\bullet(X,A,\Z)=S_\bullet(X,\Z)$. Previously, we did UCT for chain complexes of free $\Z$-modules, here we state the universal coefficient theorem for singular chain complexes:
\begin{thm}[Universal Coefficient Theorems]
    \begin{enumerate}
        We have some SES's:
        \item There exists a short exact sequence 
        \begin{align*}
            0\to\bigoplus_{i+j=n}H_i(X,A,\Z)\otimes H_j(Y,B,\Z)&\to H_n(S_\bullet(X,A,\Z)\otimes S_\bullet(Y,B,\Z))\to\\
            &\bigoplus_{i+j=n-1}\tor_\Z^1(H_i(X,A,\Z),H_j(Y,B,\Z))\to 0
        \end{align*}
        where $H_i(X,A,Z)=H_i(S_\bullet(X,A,\Z))$
        \item There exists a SES:
        \begin{align*}
            0\to H_i(X,A,\Z)\otimes_\Z R\xrightarrow{f} H_i(X,A,R)\to\tor_\Z^1(H_{i-1}(X,A,\Z),R)\to 0
        \end{align*}
        again $f$ is injective, and the failure to be surjective is measured by $\tor_\Z^1$.
        \item There exists a SES:
        \begin{equation*}
            0\to\ext_\Z^1(H_{n-1}(X,A,\Z),R)\to H^n(X,A,R)\xrightarrow{\beta}\text{Hom}_\Z(H_n(X,A,\Z), R)\to 0
        \end{equation*}
    \end{enumerate}
    note all the above assumes $R$ is a PID.
    
\end{thm}
We next introduce the category $\ptop$.
\begin{defn}[$\ptop$]
    The category $\ptop$ has objects pairs $(X,A)$ where $A\subset X$ is a subspace of a toplogical space $X$. where 
    \begin{equation*}
        mor_\ptop((X,A),(Y,B))=\text{ set of continuous maps from $X\to Y$ that sends $A$ to $B$}
    \end{equation*}
    i.e., the image of $f$ in $A$ is contained in $B$.
\end{defn}
\begin{thm}
    $S_\bullet(X,A,R)$ is a functor from $\ptop\to ch_R^+$, and $S^\bullet(X,A,R)$ is the contravariant functor from $\ptop$ to $coch_R^+$.
\end{thm}
\begin{proof}
    To show that it is a functor, we know it's defined on objects, we now define it on morphisms. Given $f:(X,A)\to (Y,B)$, we define 
    \begin{equation*}
        f_*:S_i(X,A,R)\to S_i(Y,B,R)
    \end{equation*}
    as follows:
    \begin{equation*}
        (f_*)_i\left[\la g:\Delta_i\to X\ra\right]:=\left[\la f\circ g: \Delta_i\to Y\ra \right]
    \end{equation*}
\end{proof}
We need to check that it commutes in a ladder as follows:
\[\begin{tikzcd}
	{F_j} \\
	{\Delta_i} & X & Y
	\arrow[from=1-1, to=2-1]
	\arrow["g", from=2-1, to=2-2]
	\arrow["f", from=2-2, to=2-3]
\end{tikzcd}\]
we have
\begin{align*}
    \partial_i\circ (f_*)_i[\la g:\Delta_i\to X\ra]&=\partial_i[\la f\circ g:\Delta_i\to Y]\\
    &=\sum_{j=0}^i(-1)^j(f_*)_{i-1}[\la g:F_j\to X\ra]\\
    &=(f_*)_{i-1}\sum_{j=0}^{i}(-1)^j[\la g:F_i\to X\ra]\\
    &=(f_*)_{i-1}\circ\partial_i
\end{align*}
Moreover, it is easy to see that 
\begin{equation*}
    (f\circ g)_*=f_*\circ g_*
\end{equation*}
\begin{defn}[singular homolgoy]
    The $n$th singular homology with coefficients in $R$ is the composite functor:
    \begin{equation*}
        \ptop\xrightarrow{S_\bullet(X,A,R)}ch_R^+\xrightarrow{H_n}Mod_R
    \end{equation*}
    and similarly for cohomologies.
\end{defn}
\begin{example}
    We consider the following simple example $X=\pt$, and $A=\emptyset$, $S_\bullet(\pt,R)$ since the set $\Delta_i(\pt)$ is a singleton $i\geq 0$. So $S_\bullet(\pt,R)$ looks like
    \begin{equation*}
        \dots\to R\to R\xrightarrow{\partial_2}R\xrightarrow{\partial_1}R\to 0\to\dots
    \end{equation*}
    where
    \begin{equation*}
        H_i(\pt,R)=\begin{cases}
            0, i\neq 0\\
            R, i=0
        \end{cases}
    \end{equation*}
\end{example}
\begin{defn}[path-connected]
    A space $X$ is path-connected if given any $a,b\in X$, there exists a continuous path $\gamma:[0,1]\to X$ such that 
    \begin{equation*}
        \gamma(0)=\alpha, \quad \gamma(1)=b
    \end{equation*}
\end{defn}
\begin{prop}
    If $X$ is path-connected, then 
    \begin{equation*}
        H_0(X,R)\cong R
    \end{equation*}
    (implying that $H_0$ a homology group, could be tiny!)
\end{prop}
\begin{proof}
    Recall that by definition, we have 
    \begin{equation*}
        H_0(X,R)=\frac{R\la\Delta_0(X)\ra}{\im(\partial_1)}
    \end{equation*}
    where $\partial_1:R\la\Delta_1(X)\ra\to R\la\Delta_0(X)\ra$. We consider the homomorphism: 
    \begin{equation*}
        \varepsilon: R\la\Delta_0(X)\ra\to R
    \end{equation*}
    such that $\varepsilon\la x\ra=1$, for generator $x\in X=\Delta_0(X)$. Notice that 
    \begin{equation*}
        \partial_1\la\gamma\ra=\la\gamma(1)\ra-\la\gamma(0)\ra
    \end{equation*}
    Hence 
    \begin{equation*}
        \varepsilon\partial_1\la\gamma\ra=\varepsilon\la \gamma(1)\ra-\varepsilon\la \gamma(0)\ra=1-1=0
    \end{equation*}
    Hence $\varepsilon$ extends to a surjective map. 
    \begin{prob}[HW(2.8)]
        Show that $\varepsilon$ is also injective.
    \end{prob}
\end{proof}

Next we stated Eilenberg-Steenrod Axioms.
\textcolor{red}{will fill in later}

\section{Lecture 10 9/30}
\textcolor{red}{we restated ES axioms} at the beginning of class.


