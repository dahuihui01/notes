\section{Lecture 10/09}

Last time we showed that 
\begin{equation*}
    S_\bullet(X,R)\otimes_R S_\bullet(Y,R) \text{ and } S_\bullet(X\times Y,R)
\end{equation*}
are chain homotopic. We now make a few remarks:
\begin{enumerate}
    \item The above theorem extends to a relative natrual equivalence: using naturality of the non-relative theorem: $EZ: S_\bullet(X,A,R)\otimes S_\bullet(Y,B,R)\to S_\bullet(X\times Y, X\times B\cup A\times Y, R)$
    \item On the level of cochain cmplexes, we have corresponding cochain homotpy equivalences: 
    \begin{equation*}
        EZ^*: S^\bullet(X\times Y, R)to S^\bullet(X,R)\otimes S^\bullet(Y,R)
    \end{equation*}
    and more generally for relative equivalences.
    \item If $X=Y$, then there is a nice classification of the following composite:
    \[\begin{tikzcd}
        {S_\bullet(X,R)} & {S_\bullet(X\times X,R)} & {S_\bullet(X,R)\otimes S_\bullet(X,R)}
        \arrow["{\Delta_*}", from=1-1, to=1-2]
        \arrow["{EZ^{-1}}", from=1-2, to=1-3]
    \end{tikzcd}\]
    given by the Alexander-Whitney diagonal formula, where $\Delta_*$is the map induced by the diagonal map:
    \begin{equation*}
        \Delta:X\to X\times X
    \end{equation*}
    where $x\mapsto (x,x)$.
\end{enumerate}
Next we talk about the Alexander-Whiteney construction.
\begin{defn}
    The Alexander-Whiteney diagonal is a natural map:
    \begin{equation*}
        AW: S_\bullet(X,R)\to S_\bullet(X,R)\otimes_R S_\bullet(X,R)
    \end{equation*}
    such that 
    \begin{equation*}
        AW\la f\ra=\bigoplus_{p+q=n}\la f\circ Fr^p\ra\otimes \la f\circ Bk^q\ra
    \end{equation*}
    where $Fr,Bk$ means front and back, and $f:\Delta_n\to X$ is a continuous map and 
    \begin{equation*}
        f\circ Fr^p
    \end{equation*}
    denotes the restriction of $f$ to the front $p$-face given by 
    \begin{equation*}
        Fr^p=\{x=(t_0,\dots, t_n):t_i=0, \forall i>p\}
    \end{equation*}
    and the back $q$-face is 
    \begin{equation*}
        Bk^q=\{x=(t_0,\dots, t_n): t_i=0, \forall i<n-q\}
    \end{equation*}
\end{defn}
Note that 
\begin{equation*}
    \la f\circ Fr^p\ra\in S_p(X,R)
\end{equation*}
and 
\begin{equation*}
    \la f\circ Bk^q\ra\in S_q(X,R)
\end{equation*}
\begin{prob}[HW(3.3)]
    Show that $AW\la f\ra$ is a chain map, i.e., it commutes with the differential on both sides.
\end{prob}
\begin{thm}
    The Alexander-Whitney diagonal is naturally homotopic to the following composite:
    \[\begin{tikzcd}
        {S_\bullet(X,R)} & {S_\bullet(X\times X,R)} & {S_\bullet(X,R)\otimes S_\bullet(X,R)}
        \arrow["{\Delta_*}", from=1-1, to=1-2]
        \arrow["{EZ^{-1}}", from=1-2, to=1-3]
    \end{tikzcd}\]
\end{thm}
\begin{proof}
    Consider the functors:
    \begin{equation*}
        g,g':Top\to\ach_R
    \end{equation*}
    given by 
    \begin{equation*}
        g(X)=S_\bullet(X,R), g'(X)=S_\bullet(X,R)\otimes S_\bullet(X,R)
    \end{equation*}
    consider the collection 
    \begin{equation*}
        M=\bigsqcup_k M_k, M_k=\{\Delta_k\}
    \end{equation*}
    and let $i_k\in S_k(\Delta_k,R)$ be $\la id_{\Delta_k}\ra$. Then notice that $g$ is free with respect to $M$ and $g'$ is acyclic with respect to $M$ by the algebraic Kunneth theorem. Hence the second part of the AMT syas that the two natrual transformations $AW$ and $EZ^{-1}\circ\Delta_*$ are naturally chain homotopic.
\end{proof}
Next we prove the ES axiom 2.
\begin{prop}
    Singular homology satisfies the homotopy axiom.
\end{prop}
\begin{proof}
    \textcolor{red}{fill in later}
\end{proof} 
Next we will prove the ES A3, the Excision axiom. We introduce a few definitions before doing so. We will talk about barycentric subdivision.

\begin{defn}[barycenter]
    Let $\Delta_n$ denote the $n$-simplex, let
    \begin{equation*}
        \{e_{i_0},\dots, e_{i_k}\}
    \end{equation*}
    be some collection of vertices of $\Delta_n$. where 
    \begin{equation*}
        I=\{i_0,\dots,i_k\}\subset\{0,\dots, n\}
    \end{equation*}
    The barycenter of $I$ is 
    \begin{equation*}
        b(I)=\frac{1}{k+1}\sum_{i\in I}e_i\in \Delta_n
    \end{equation*}
    Namely, $b(I)$ is the center of the face spanned by vertices $e_i$ where $i\in I$.
\end{defn}
Let $\sigma\in S_{n+1}$ be a permutation of $\{0,\dots, n\}$. Define an $n$-simplex $B(\sigma)$ to be the subspace of $\Delta_n$ with vertices:
\begin{equation*}
    \{b(\sigma\{0,\dots,n\}), b(\sigma\{1,\dots,n\}), \dots, b(\sigma\{n\})\}
\end{equation*}
note that $b$ does not change under permuattions, i.e., 
\begin{equation*}
    B(\sigma)=\left\{\sum_{j=0}^nt_jb(\sigma\{j,\dots,n\}):t_j\geq 0, \sum_jt_j=1\right\}
\end{equation*}
\begin{defn}
    The barycentric subdivision is a natural transformation 
    \begin{equation*}
        L: S_\bullet(X)\to S_\bullet(X)
    \end{equation*}
    such that given $\la f\ra\in S_n(X)$, we define 
    \begin{equation*}
        L(\la f\ra)=\sum_{\sigma\in S_{n+1}}(-1)^{sgn(\sigma)}\la f\circ B(\sigma)\ra
    \end{equation*}
\end{defn}
\begin{prob}[HW(3.4)]
    Show that $L$ is a chain map, i.e., 
    \begin{equation*}
        \partial L=L\partial
    \end{equation*}
    Hint: in $\partial L$ the internal faces cancel off.
\end{prob}
\begin{cor}
    By AMT, we have $L^1, L^2,\dots, L^k$ areall chain homotopic to the identity transformation.
\end{cor}

We make the remark that by the naturality of $L$, we see that $L$ extends to a natural transformation
\begin{equation*}
    L: S_\bullet(X,A,R)\to S_\bullet(X,A,R)
\end{equation*}
such that $L^k$ is chain homotopic to the identity for all $k\in\mathbb{N}$.

Next time, we will talk about the Excision Axiom A3.

\section{Lecture 10/14}
We start the proof of Excision Axiom.
\begin{defn}[good cover]
    A good cover $\{U_i\}$ of a topological space $X$ is a cover such that $\{int(U_i)\}$ is an open cover of $X$. 
\end{defn}
We define chain complex 
\begin{equation*}
    \{S_\bullet^U(X,R),\partial_\bullet\}
\end{equation*}
by 
\begin{equation*}
    S_n^U(X,R)=\bigoplus_{f\in\Delta_n^U(X)}R\la f\ra
\end{equation*}
where $\Delta_n^U(X)$ is the set of all continuous maps from $\Delta_n$ to $X$ that land in some $U_i$, and the differential is given by the usual formula.

The relative version is as follows: if $A\subset X$ then $\{A\cap U_i\}$ is a good cover of $A$, which we also denote by $U$. Define 
\begin{equation*}
    S_\bullet^U(X,A,R)=\left\{S_\bullet^U(X,R)/S_\bullet^U(A,R);\partial_\bullet\right\}
\end{equation*}
is the relative version of the defintion.
\begin{prop}
    The canonical inclusion 
    \begin{equation*}
        i: S_\bullet^U(X,A,R)\to S_\bullet(X,A,R)
    \end{equation*}
    induces an isomorphism in homology:
    \begin{equation*}
        i_*: H_*(S_\bullet^U(X,A,R))\cong H_*(X,A,R)
    \end{equation*}
\end{prop}
\begin{proof}
    First we reduce to the case where $A=\emptyset$ using the 5-lemma. Consider the laddder of LES:
    \[\begin{tikzcd}
        \dots & {H_n(S_\bullet^U(A,R))} & {H_n(S_\bullet^U(X,R))} & {H_n(S^U(X,A,R))} & \dots \\
        \dots & {H_n(A,R)} & {H_n(X,R)} & {H_n(X,A,R)} & \dots
        \arrow[from=1-1, to=1-2]
        \arrow[from=1-2, to=1-3]
        \arrow["{i_A}", from=1-2, to=2-2]
        \arrow[from=1-3, to=1-4]
        \arrow["{i_X}", from=1-3, to=2-3]
        \arrow[from=1-4, to=1-5]
        \arrow["{i_{(X,A)}}", from=1-4, to=2-4]
        \arrow[from=2-1, to=2-2]
        \arrow[from=2-2, to=2-3]
        \arrow[from=2-3, to=2-4]
        \arrow[from=2-4, to=2-5]
    \end{tikzcd}\]
    So if we show that $i_A, i_X$ are isomorphisms, then $i_{(X,A)}$ is also an isomorphism by the 5-lemma. So we consider 
    \begin{equation*}
        i_*: H_*(S_\bullet^U(X,R))\to H_*(X,R)
    \end{equation*}
    First we show that $i_*$ is injective, consider 
    \[\begin{tikzcd}
        {S_{n+1}^U(X)} & {\textcolor{blue}{\beta}\in S_{n+1}(X)} \\
        {S_n^U(X)} & {\textcolor{blue}{\partial\beta}\in S_n(X)}
        \arrow["i", from=1-1, to=1-2]
        \arrow["\partial", from=1-1, to=2-1]
        \arrow["\partial", from=1-2, to=2-2]
        \arrow["i", from=2-1, to=2-2]
    \end{tikzcd}\]
    Let $[\alpha]\in H_n(S_\bullet^U(X))$ such that $i_*[\alpha]=0$. Then 
    \begin{equation*}
        \alpha=\partial\beta
    \end{equation*}
    for some $\beta\in S_{n+1}(X)$. Now apply $L^k$ for some $k>>0$. Then 
    \begin{equation*}
        L^k(\beta)\in S_{n+1}^U(X)
    \end{equation*}
    We know that 
    \begin{equation*}
        L^k(\beta)-\beta=(D\partial+\partial D)(\beta)
    \end{equation*}
    for some natural homotopy $D$, i.e., 
    \begin{align*}
        L^k(\beta)-\beta&=D\partial(\beta)+\partial D(\beta)\\
        &=D(\alpha)+\partial D(\beta)
    \end{align*}
    Hence $\beta=L^k(\beta)-\partial D(\beta)-D(\alpha)$, hence 
    \begin{equation*}
        \partial\beta=\partial(L^k(\beta)-D(\alpha))=\alpha
    \end{equation*}
    but $L^k(\beta)-D(\alpha)\in S_{n+1}^U(X,R)$, by subdivision and naturality, we see that 
    \begin{equation*}
        [\alpha]=0
    \end{equation*}
    in $H_n(S_\bullet^U(X,R))$. 
    
    Next we show that $i_*$ is surjective. Given $[\alpha]\in H_n(X,R)$, replace $[\alpha]$ with $[L^k(\alpha)]$ for $k>>0$ using the fact that $L^k$ is chain homotopic to the identity, but 
    \begin{equation*}
        L^k(\alpha)\in S_n^U(X,R)
    \end{equation*}
    so $i_*$ is surjective.
\end{proof}
\begin{cor}
    Singular homology satisfies Excision Axiom A3.
\end{cor}
\begin{proof}
    \textcolor{red}{fill in later}
\end{proof}

\begin{thm}[Mayer-Vietoris sequence]
    Let $V_1,V_2$ be such that $int(V_1)\cup int(V_2)=X$, then there is a LES:
    \[\begin{tikzcd}
        \dots & {H_i(V_1\cap V_2)} & {H_i(V_1)\oplus H_i(V_2)} & {H_i(X)} \\
        & {H_{i-1}(V_1\cap V_2)} & {H_{i-1}(V_1)\oplus H_{i-1}(V_2)} & {H_{i-1}(X)} \\
        & \dots
        \arrow[from=1-1, to=1-2]
        \arrow["s", from=1-2, to=1-3]
        \arrow["t", from=1-3, to=1-4]
        \arrow[curve={height=-6pt}, from=1-4, to=2-2]
        \arrow[curve={height=-6pt}, from=2-4, to=3-2]
    \end{tikzcd}\]
    where $s$ is induced by the map $(i_1)_*\oplus -(i_2)_*$, where 
    \begin{equation*}
        i_1:V_1\cap V_2\to V_1
    \end{equation*}
    and 
    \begin{equation*}
        i_2: V_1\cap V_2\to V_2
    \end{equation*}
    and $t$ is induced by the map 
    \begin{equation*}
        (j_1)_*+(j_2)_*
    \end{equation*}
    where 
    \begin{equation*}
        j_1:V_1\to X, j_2:V_2\to X
    \end{equation*}
\end{thm}
\begin{prob}[HW(3.5)]
    Calculate the homology of the Riemannian surface $\Sigma_g$, $g$ stands for $g$-holes. For example, $\Sigma_0=\mathbb{S}^2$, $\Sigma_1=S^1\times S^1$.
\end{prob}

Next we talk about products in cohomology. We now introduce and study 4 types of natural pairings.
\begin{enumerate}
    \item Topological crossproduct in homology:
    \begin{equation*}
        \bigoplus_{i+j=n}H_i(X,A)\otimes_R H_j(Y,B)\xrightarrow{X_\bullet}H_n(X\times Y, A\times Y\cup X\times B)
    \end{equation*}
    \item Topological cross product in cohomology:
    \begin{equation*}
        \bigoplus_{i+j=n}H^i(X,A)\otimes_R H^j(Y,B)\xrightarrow{X_\bullet}H^n(X\times Y, A\times Y\cup X\times B)
    \end{equation*}
    \item Cup product in cohomology 
    \begin{equation*}
        \bigoplus_{i+j=n}H^i(X,A)\otimes_R H^j(X,B)\xrightarrow{\cup} H^n(X,A\cup B)
    \end{equation*}
    \item Cap product between (co)homologies:
    \begin{equation*}
        \bigoplus_{i+j=n}H^i(X,A)\otimes_R H_n(X,A\cup B)\xrightarrow{\cap} H_{n-i}(X,B)
    \end{equation*}
\end{enumerate}

We start with the first type: topological product in homology. We define the natural pairing
\begin{equation*}
    \bigoplus_{i+j=n}H_i(X,A)\otimes_R H_j(Y,B)\xrightarrow{X_\bullet}H_n(X\times Y, A\times U\cup X\times B)
\end{equation*}
as the composite:
\[\begin{tikzcd}
	{H_i(S_\bullet(X,A))\otimes H_j(S_\bullet(Y,B))} & {H_n(S_\bullet(X\times Y, X\times B\cup A\times Y))} \\
	& {H_n(S_\bullet(X,A)\otimes_R S_\bullet(Y,B))}
	\arrow["{X_\bullet}", from=1-1, to=1-2]
	\arrow["{X_{alg}}"', from=1-1, to=2-2]
	\arrow["{EZ(\cong)}"', from=2-2, to=1-2]
\end{tikzcd}\]
where $X_{alg}$ denotes the algebraic cross product:
\begin{equation*}
    X_{alg}([\alpha]\otimes[\beta]):=[\alpha\otimes\beta]
\end{equation*}
We note that $X$ can be seen as a natural transforamtion between the functors $H_i\otimes H_j$ and $H_{i+j}$ from 
\begin{equation*}
    PTop\times PTop\to R-Mod
\end{equation*}

(II) Next we talk about topological crossproduct in cohomology. To define topological cross product, we first observe that for any $R$ modules $V,W$, there is a natural map 
\begin{equation*}
    Hom(V,R)\otimes_R Hom(W,R)\xrightarrow{\lambda} Hom(V\otimes W,R)
\end{equation*}
and 
\begin{equation*}
    \lambda(\varphi\otimes\psi)(a\times b)=\varphi(a)\psi(b)
\end{equation*}
We then define 
\begin{equation*}
    \bigoplus_{i+j=n}H^i(X,A)\otimes_R H^j(Y,B)\xrightarrow{X_\bullet}H^n(X\times Y, A\times U\cup X\times B)
\end{equation*}
as the composite
\[\begin{tikzcd}
	{H^i(S^\bullet(X,A))\otimes H^j(S^\bullet(Y,B))} & {H^n(S^\bullet(X\times Y, X\times B\cup A\times Y))} \\
	{H^n(S^\bullet(X,A)\otimes S^\bullet(Y,B))} & {H^n(Hom(S_\bullet(X,A)\otimes_R S_\bullet(Y,B),R))}
	\arrow["{X^\bullet}", from=1-1, to=1-2]
	\arrow["{X_{alg}}", from=1-1, to=2-1]
	\arrow["\lambda", from=2-1, to=2-2]
	\arrow["{EZ^{-1} (\cong)}"', from=2-2, to=1-2]
\end{tikzcd}\]
Note: as before, $X$ can be seen as a natural transformation between the two obvious functors.

Recall the map 
\begin{equation*}
    \beta: H^i(X,A)\to Hom(H_i(X,A),R)
\end{equation*}
\begin{prob}[HW(3.6)]
    Show the following diagram commutes:
    \[\begin{tikzcd}
        {H^i(X,A)\otimes H^j(Y,B)} & {H^{i+j}(X\times Y, X\times B\cup A\times Y)} \\
        {Hom(H_i(X,A),R)\otimes Hom(H_j(Y,B),R)} & {Hom(H_{i+j}(X\times Y, X\times B\cup A\times Y),R)} \\
        & {Hom(H_i(X,A)\otimes_R H_j(Y,B),R)}
        \arrow["{X^\bullet}", from=1-1, to=1-2]
        \arrow["{\beta\otimes\beta}"', from=1-1, to=2-1]
        \arrow["\beta", from=1-2, to=2-2]
        \arrow["\lambda"', from=2-1, to=3-2]
        \arrow["{X_\bullet^*}", from=2-2, to=3-2]
    \end{tikzcd}\]
    where $X_\bullet^*$ denotes the dual map induced by the cro
    sproduct in the homology.
\end{prob}


\section{Lecture 10/16}



\section{Lecture 10/21}



\section{Lecture 10/23}