\chapter{Category Theory}
\textbf{Instructor:} Nitu Kitchro, 
\textbf{Office Hours}: Monday after class, \textbf{TA: } Anna Matsui
\section{Lecture 1 8/26}
\begin{defn}[Category]
    A category $\mathcal{C}$ consists of the following data:
    \begin{enumerate}
        \item A collection of objects denoted as Ob$(\mathcal{C})$
        \item Given two objects $X,Y\in$ Ob$(\mathcal{C})$, a collection of morphisms between $X,Y$, $f:X\to Y$, denoted as mor$_\CC(X,Y)$.
        \item (Composition) We have mor$_\CC(X,Y)\times mor_\CC(Y,Z)\to mor_\CC(X,Z)$ that satisfies associativity
        \begin{equation*}
            f\circ(g\circ h)=(f\circ g)\circ h
        \end{equation*}
        \item (Identity) There is a distinguished morphism for each $X$, $Id_\CC(X,X)$ such that given any $f\in mor(X,Y)$, we have $f\circ id_X=id_Y\circ f=f$.
    \end{enumerate}
\end{defn}
In this course, we will make the assumption that in all the categories that we work with, Ob$(\CC)$ need not be a set, but given any $X,Y\in Ob(\CC)$, mor$(X,Y)$ will always be a set. Now we talka bout some examples of categories.
\begin{example}[Sets]
    Let Ob$(Sets)$ be all the sets in the universe. Given $X,Y$ sets, mor$(X,Y)$ be all the set maps from $X$ to $Y$, and $id_X$ is the identity map.
\end{example}
\begin{example}[Top]
    Let Ob$(Top)$ be all the topological spaces, and mor$(X,Y)$ be all the continuous maps from $X$ to $Y$.
\end{example}
\begin{example}[Vect$_\F$]
    Let $\F$ be a field, and let Ob be all the $\F$-vector spaces. Then mor$(V,W)$ is all the $\F$-linear homomorphisms from $V$ to $W$, where $id_V$ is the identity homomorphism.
\end{example}
\begin{example}[Posets]
    Fix a poset $P$, let Ob$(P)$ be the collection of elements in $P$, and given $p,q$ we define 
    \begin{equation*}
        mor(p,q)=\begin{cases}
            *, \text{ if } q\leq p\\
            \emptyset, \text{ otherwise }
        \end{cases}
    \end{equation*}
\end{example}
\begin{prob}
    \textbf{HW(Q1): check this is a category}
\end{prob}
\begin{example}[Opposite category]
    Given a category $\CC$, there is another category called the opposite category, denoted as $\CC^{op}$, where 
    \begin{enumerate}
        \item The objects are the same as $\CC$
        \item Given $X,Y\in$ Ob$(C^{op})$, we have mor$_{op}(X,Y):=$ mor$_\CC(Y,X)$. 
        \item Moreover, given $f\in mor_{op}(X,Y), g\in mor_{op}(Y,Z)$, then $g\circ f$ in $C^{op}$ is $f\circ g: Z\to X$.
    \end{enumerate}
\end{example}
Naturally, we define isomorphisms now.
\begin{defn}[isomorphism]
    Given a category $\CC$, and a morphism $f\in mor_C(X,Y)$, we say $f$ is an isomorphism if there exists $g\in mor_C(Y,X)$ such that 
    \begin{equation*}
        f\circ g=Id_Y, g\circ f=Id_X
    \end{equation*}
\end{defn}
Now we introduce maps between categories.
\begin{defn}[functor]
    Given categories $\CC,\mathcal{D}$, a functor $F:C\to D$ is the following;
    \begin{enumerate}
        \item Given an object $X$ in $\mathcal{C}$, $F(X)$ is an object in $D$. 
        \item Given a morphism $f: X\to Y$, $F(f)$ is a functor $F(f): F(X)\to F(Y)$. Moreover, it satisfies the following:
        \begin{enumerate}
            \item $F(id_X)=id_{F(X)}$
            \item $F(f\circ g)=F(f)\circ F(g)$. Alternatively, we can rewrite this condition as the following: 
            \[\begin{tikzcd}
                {mor(X,Y)\times mor(Y,Z)} & {mor(X,Z)} \\
                {mor(F(X), F(Y))\times mor(F(Y),F(Z))} & {mor(F(X),F(Z))}
                \arrow[from=1-1, to=1-2]
                \arrow["{mor(F)\times mor(F)}", from=1-1, to=2-1]
                \arrow["{mor(F)}", from=1-2, to=2-2]
                \arrow[from=2-1, to=2-2]
            \end{tikzcd}\]
            such that this diagram commutes.
        \end{enumerate}
    \end{enumerate}
\end{defn}
\begin{prob}
    \textbf{HW(Q2): functors take isomorphisms to isomorphisms.}
\end{prob}
Now we talk about some examples of functors.
\begin{example}
    $F: Top\to Set$, where $X\mapsto X$, where the latter is a set, and $f\mapsto f$ as set maps.
\end{example}
\begin{example}
    Let $\F$ be a field, and $F: Sets\to \text{Vect}_\F$, where $X\mapsto \F\la X\ra$, where $\F\la X\ra$ is the free vector space over $\F$ on the set $X$.
\end{example}
\begin{prob}
    \textbf{HW(Q3): extend this to a functor by defining $mor(f)$ and show this is a functor.}
\end{prob}
\begin{example}
    Let $\F$ be a field, then the following is a functor, $F: Sets^{op}\to\text{Vect}_\F$, where
    \begin{equation*}
        h  F: X\mapsto Maps(X,\F)
    \end{equation*}
\end{example}
\begin{prob}
    \textbf{HW(Q4)}: show this extends to a functor by defining $F(f)$, and show it is a functor.
\end{prob}

\section{Lecture 2 8/28}
\begin{defn}[contravariant functor]
    Let $F:\CC\to\mathcal{D}$ is a contravariant functor from $\CC^{op}\to\mathcal{D}$, (equivalently, $\CC\to\mathcal{D}^{op}$).
\end{defn}
\begin{prob}
    \textbf{HW(Q5):} Show that the following functor $F$ from Vect$_\F$ to Vect$_\F$ extends to a contravariant functor, where 
    \begin{equation*}
        Ob_F: V\mapsto V^*=Hom(V,\F)
    \end{equation*}
    i.e., define the morphism function and show it is a contravariant functor.
\end{prob}
We remark that we can define a category of categories: let $Cat$ be the category of categories, with morphisms as functors, and note that objects or morphisms in this case are both not sets!
\begin{defn}[natural transformation]
    Given functors $F,G: \CC\to\mathcal{D}$, a natural transformation $T$ from $F$ to $G$ is the following: $T: F\Rightarrow G$:
    \begin{enumerate}
        \item given object $X\in Ob(\CC)$, $T(X)\in mor(F(X),G(X))$
        \item Given $f\in mor(X,Y)$, the following diagram commutes:
        \[\begin{tikzcd}
            {F(X)} & {F(Y)} \\
            {G(X)} & {G(Y)}
            \arrow["{F(f)}", from=1-1, to=1-2]
            \arrow["{T(X)}"', from=1-1, to=2-1]
            \arrow["{T(Y)}", from=1-2, to=2-2]
            \arrow["{G(f)}"', from=2-1, to=2-2]
        \end{tikzcd}\]
        where $mor_F, mor_G$ is the identification function on morphisms by functors $F,G$
    \end{enumerate}
    If for all $X$, $T(X)$ is an isomorphism, then this natural transformation is called a natural isomorphism.
\end{defn}
In other words, this natural transformation is how one takes a functor $F$ and turn it to another functor $G$. We will (in a homework) show there exists natural transofrmation between the following two functors.
\begin{example}
    Consider $F,G: Vect_\F\to Vect_\F$, define 
    \begin{equation*}
        F(V)=V\otimes_\F V/_{\la a\otimes b-b\otimes a\ra}=V\otimes_\F V/\Sigma_2, G(V)=(V\otimes_F V)^{\Sigma_2}=\{\alpha\in V\otimes_\F V: \sigma(\alpha)=\alpha\}
    \end{equation*}
    Both are vector spaces are fixed under ``swaps.'' Then a natural transformation can be defined as follows $T(V):$
    \begin{equation*}
        T(V): a\otimes b\mapsto a\otimes b+b\otimes a
    \end{equation*}
\end{example}
\begin{prob}
    \textbf{HW(Q6):} For the above $F,G$
    \begin{enumerate}
        \item Show that $T$ defines a natural transformation from $F$ to $G$. 
        \item Find conditions on $\F$ for $T$ being a natural isomorphism.
    \end{enumerate}
\end{prob}
Next we define limits and colimits.
    Let $\mathcal{C},\mathcal{D}$ be categories, $d$ be an object in $\mathcal{D}$, then we can define a functor $F_d: \mathcal{C}\to\mathcal{D}$ such that for any object $c$ in $\mathcal{C}$,
    \begin{equation*}
        F_d(c)=d, F_d(f)=Id_d
    \end{equation*}
    In other words, this is the ``constant functor'' on $\mathcal{D}$, i.e., every object is sent to $d$, and every morphism is sent to $id_d$.
\begin{defn}[colimit]
    Given any functor $F:\mathcal{C}\to\mathcal{D}$, the colimit of $F$, denoted as $\colim(F)$ is an object in $\mathcal{D}$ endowed with a natural transformation:
    \begin{equation*}
        \varphi_F:F\Rightarrow F_{\colim(F)}
    \end{equation*}
    such that given any other object $d$ in $D$ and a natural transformation 
    \begin{equation*}
        \varphi: F\Rightarrow F_d
    \end{equation*}
    there exists a unique morphism in $\mathcal{D}$, $f:\colim(F)\to d$ making the following diagram commute: for any $X,Y,g$:
    \[\begin{tikzcd}
        {F(X)} && {F(Y)} \\
        & {\text{colim}(F)} \\
        & d
        \arrow["{F(g)}", from=1-1, to=1-3]
        \arrow["{\varphi_F}", from=1-1, to=2-2]
        \arrow["\varphi"', curve={height=12pt}, from=1-1, to=3-2]
        \arrow["{\varphi_F}"', from=1-3, to=2-2]
        \arrow["\varphi", curve={height=-12pt}, from=1-3, to=3-2]
        \arrow["{\textcolor{red}{f}}", from=2-2, to=3-2]
    \end{tikzcd}\]
\end{defn}
Next we prove some facts about colimits and give an example, where $\colim(F)$ exists.
\begin{prop}
    If $\colim F$ exists, then $\colim F$ is unique up to isomorphisms.
\end{prop}
\begin{proof}
    Let $\colim(F), \colim(F)'$ be two colimits that satisfy the criteria. They are both objects in $\mathcal{D}$, then we get a morphism $f:\colim(F)\to\colim(F)'$, and likewise $g:\colim(F)\to\colim(G)'$, then
    \begin{equation*}
        f\circ g:\colim(F)'\to\colim(F)'
    \end{equation*} 
    is the only morphism, and is the identity morphism. Similarly for $g\circ f$.
\end{proof}
Next we demonstrate a fact via an example.
\begin{thm}
    Let $\mathcal{C}$ be a category where $Ob(\mathcal{C}), mor(X,Y)$ are all sets. Let $F: \mathcal{C}\to\text{Top}$ be any functor, then $\colim(F)$ exists.
\end{thm}
\begin{proof}
    Define $\colim(F):=\bigsqcup_{c}F(c)/\sim$, where $\sim$ is induced by the equivalence relation given by 
    \begin{equation*}
        y\sim F(f)y
    \end{equation*}
    where $y\in F(C_1), f:C_1\to C_2, F(f)x\in F(C_2)$. The natural transformation we endow on $F$ as $\varphi_F:F\Rightarrow F_{\colim(F)}$:
    \begin{equation*}
        \varphi_F: F(C)\mapsto \bigsqcup_{C\in Ob(C)}F(C)/\sim
    \end{equation*}
\end{proof}
\begin{prob}
    \textbf{HW(Q7):} Show that $\colim(F), \varphi_F$ is indeed a colimit. 
\end{prob}
We note that colimits also exist (the same argument goes through) if we replace $\text{Top}$ with groups, sets, but with slightly different constructions, replacing disjoint unions with products, etc.

\begin{defn}[limit]
    Given a functor $F: \mathcal{C}\to\mathcal{D}$, the limit of $F$, denoted as $\lim(F)$ is an object of $\mathcal{D}$, endowed with a natural transformation:
    \begin{equation*}
        \varphi_F: F_{\lim(F)}\Rightarrow F
    \end{equation*}
    such that given any other object $d\in Ob(\mathcal{D})$ and a natural transformation 
    \begin{equation*}
        \varphi: F_d\to F
    \end{equation*}
    there exists a unique $f: \lim F\to d$ such that the following diagram commutes:
    \[\begin{tikzcd}
        & {\lim F} \\
        & d \\
        {F(X)} && {F(Y)}
        \arrow["{\textcolor{red}{f}}", from=1-2, to=2-2]
        \arrow["{\varphi_F}"', curve={height=12pt}, from=1-2, to=3-1]
        \arrow["{\varphi_F}"', curve={height=-12pt}, from=1-2, to=3-3]
        \arrow["\varphi"', from=2-2, to=3-1]
        \arrow["\varphi", from=2-2, to=3-3]
        \arrow["{F(g)}"', from=3-1, to=3-3]
    \end{tikzcd}\]
\end{defn}
Just like colimits, limits are unique up to isomorphisms. 
\begin{prob}
    \textbf{HW(Q8):} Given $F:\mathcal{C}\to\mathcal{D}$, consider $F^{op}:\mathcal{C}^{op}\to\mathcal{D}^{op}$, then 
    \begin{equation*}
        \lim F=\colim F^{op}
    \end{equation*}
\end{prob}
The above problem is interpretation of diagrams and essentially we just reverse all the maps.


\chapter{Homologies, Cohomologies}
\section{Lecture 3 9/4}
Today we define (co)chain complexes: let $R$ be a commutative ring, let $Mod_R$ denote the category of $R$-modules and $R$-module maps.
\begin{defn}[chain complex]
    A chain complex of $R$-modules is a collection of $R$-modules and $R$-modules maps 
    \begin{equation*}
        \dots\to M_{i+1}\xrightarrow{\partial_{i+1}}M_i\xrightarrow{\partial_i}M_{i-1}\xrightarrow{\partial_{i-1}}\dots
    \end{equation*}
    such that $\partial_i\circ\partial_{i+1}=0$ for all $i$. In other words, the image of previous map is contained in the kernal of the subsequent map. In short, we have 
    \begin{equation*}
        \partial^2=0
    \end{equation*}
    We will denote a chain complex by $\{M.; \partial.^M\}$.
\end{defn}
Next we introduce morphisms between chain complexes.
\begin{defn}[morphism between complexes]
    Let $\{M.;\partial.^M\}, \{N.;\partial.^N\}$, a morphism $\{f.\}$ between chain complexes is a ``ladder'' such that the following commutes:
    \[\begin{tikzcd}
        \dots & {M_{i+1}} & {M_i} & {M_{i-1}} & \dots \\
        \dots & {N_{i+1}} & {N_i} & {N_{i-1}} & \dots
        \arrow[from=1-1, to=1-2]
        \arrow["{\partial_{i+1}^M}", from=1-2, to=1-3]
        \arrow["{f_{i+1}}"', from=1-2, to=2-2]
        \arrow["{\partial_i^M}", from=1-3, to=1-4]
        \arrow["{f_i}"', from=1-3, to=2-3]
        \arrow["{\partial_{i-1}^M}", from=1-4, to=1-5]
        \arrow["{f_{i-1}}"', from=1-4, to=2-4]
        \arrow[from=2-1, to=2-2]
        \arrow["{\partial_{i+1}^N}", from=2-2, to=2-3]
        \arrow["{\partial_i^N}", from=2-3, to=2-4]
        \arrow["{\partial_{i-1}^N}", from=2-4, to=2-5]
    \end{tikzcd}\]
    Moreover, we define composition of morphisms:
    \begin{equation*}
        \{f.\}\circ \{g.\}:=\{(f\circ g).\}
    \end{equation*}
    where $\{g.\}:\{M.;\partial.^M\}\to\{N.;\partial.^N\}$, and $\{f.\}:\{N.;\partial.^N\}\to \{L.;\partial.^L\}$, which is simply vertical stacking.
\end{defn}
\begin{prob}
    \textbf{HW(Q9):} Prove that chain complexes of $R$-modules form a category $\text{ch}_R$.
\end{prob}
There are interesting functors $F:\text{ch}_R\to Mod_R$, and we begin with the following one:
\begin{defn}[$H_n$, $n$th-homology]
    Given $n\in\mathbb{Z}$, there is a functor 
    \begin{equation*}
        H_n: \text{ch}_R\to Mod_R
    \end{equation*}
    defined as follows:
    \begin{equation*}
        H_n(\{M.;\partial.^M\}):=\ker\partial_n^M\big/ Im \partial_{n+1}^M
    \end{equation*}
    and for $f:\{M.;\partial.^M\}\to\{N.;\partial.^N\}$, we define: $H_n(f): H_n(\{M.;\partial.^M\})\to H_n(\{N.;\partial.^N\})$,
    \begin{equation*}
        H_n(f)[x]:=[f_n(x)]
    \end{equation*}
    where $[x]\in H_n(\{M.;\partial.^M\})$.
\end{defn}
\begin{proof}
    We need to show $H_n$ is well-defined on objects and morphisms. We need to check that $Im\partial_{n+1}\subset\ker\partial_n$. This is a consequence of $\partial^2=0$.

    On morphisms: for $x\in\ker\partial_n^M$, we have $f_n(x)\in\ker\partial_n^N$. This is we have 
    \begin{equation*}
        \partial_n^N(f_n(x)=f_{n+1})(\partial_n^M(x))=0
    \end{equation*}
    Moreover, we need to check that this desn't depend on the choice of representatives, i.e., we can check that 
    \begin{equation*}
        Im\partial_{n+1}^M\mapsto 0
    \end{equation*}
    Let $x=\partial_{n+1}^M(y)$, we have 
    \begin{equation*}
        f_n(x)=f_n(\partial_{n+1}^M(y))=\partial_{n+1}^N(f_{n+1}(y))=0
    \end{equation*}
    \[\begin{tikzcd}
        {M_{n+1}} & {M_n} \\
        {N_{n+1}} & {N_n}
        \arrow["{\partial_{n+1}^M}", from=1-1, to=1-2]
        \arrow["{f_{n+1}}"', from=1-1, to=2-1]
        \arrow["{f_n}", from=1-2, to=2-2]
        \arrow["{\partial_{n+1}^N}"', from=2-1, to=2-2]
    \end{tikzcd}\]
\end{proof}
Next we talk about homotopy between morphisms between chain complexes.
\begin{defn}[homotopy]
    Given two morphisms, $f.,g.:M.\to N.$, a chain homotopy $h.$ between them is a collection of $R$-modules maps, for all $n\in\mathbb{Z}$,
    \begin{equation*}
        h_n:M_n\to N_{n+1}
    \end{equation*}
    such that 
    \begin{equation*}
        \partial_{n+1}\circ h_n+h_{n-1}\circ\partial_n=f_n-g_n
    \end{equation*}
    denoted as $\partial h+h\partial=f-g$.
    \[\begin{tikzcd}
        {M_{n+1}} & {M_n} & {M_{n-1}} \\
        {N_{n+1}} & {N_n} & {N_{n-1}}
        \arrow[from=1-1, to=1-2]
        \arrow["{f_{n+1}}"', from=1-1, to=2-1]
        \arrow["{\textcolor{blue}{\partial_n^M}}", from=1-2, to=1-3]
        \arrow["{\textcolor{red}{h_n}}", from=1-2, to=2-1]
        \arrow["{f_n/g_n}", from=1-2, to=2-2]
        \arrow["{\textcolor{red}{h_{n-1}}}", from=1-3, to=2-2]
        \arrow["{f_{n-1}}", from=1-3, to=2-3]
        \arrow["{\textcolor{blue}{\partial_{n+1}^N}}"', from=2-1, to=2-2]
        \arrow[from=2-2, to=2-3]
    \end{tikzcd}\]
\end{defn} 

\begin{prob}
    \textbf{HW(Q10):} Show that homotopy is an equivalence relation between morphisms. Hint: replace $h_n$ with $-h_n: M_n\to N_{n+1}$. 
\end{prob}
\begin{proof}
    Reflexive is shown by defining $h_n$ to be the zero map. For symmetry, we choose $-h_n$. Transitive is a ladder.
\end{proof}
\begin{prop}
    Let $h.$ be a chain homotopy between $f.$ and $g.$, then we have an equality
    \begin{equation*}
        H_n(f.)=H_n(g.)
    \end{equation*}
    where $H_n(f.), H_n(g.): H_n(M.)\to H_n(N.)$.
\end{prop}
\begin{proof}
    Given $[x]\in H_n(M.)$, we have 
    \begin{align*}
        H_n(f)[x]&=[f_n(x)]\\
        &=[g_n(x)+\partial h.(x)+h.\partial(x)]\\
        &=[g_n(x)+\partial h.(x)]\\
        &=[g_n(x)]\\
        &=H_n(g)[x]
    \end{align*}
\end{proof}
Next we define a new category.
\begin{defn}[$Hch_R$]
    Define the category $\text{Hch}_R$ as follows:
    \begin{enumerate}
        \item $Ob(\text{Hch}_R)=Ob(\text{ch}_R)$
        \item $mor_{\text{Hch}_R}(M.,N.)=mor_{\text{ch}_R}(M.,N.)/\sim$, where $\sim$ is the homotopy equivalence.
    \end{enumerate}
\end{defn}
\begin{prob}
    \textbf{HW(Q11):} Show that $Hch_R$ is a category, admitting a functor 
    \begin{equation*}
        F: ch_R\to Hch_R
    \end{equation*}
    such that the following diagram commutes:
    \[\begin{tikzcd}
        {\text{ch}_R} && {\text{Hch}_R} \\
        & {\text{mod}_R}
        \arrow["F", from=1-1, to=1-3]
        \arrow["{H_n}"', from=1-1, to=2-2]
        \arrow["{H_n}", from=1-3, to=2-2]
    \end{tikzcd}\]
\end{prob}
Next we introduce long and short exact sequences.
\begin{defn}[exactness]
    Firstly, given a pair of $R$-module maps, 
    \begin{equation*}
        X_1\xrightarrow{f}X_2\xrightarrow{g}X_3
    \end{equation*}
    we say that the above is exact at $X_2$ if $\ker(g)=\im(f)$. Hence given a sequence of $R$-module maps, 
    \begin{equation*}
        \dots\to X_{i+1}\to X_i\to X_{i-1}\to\dots
    \end{equation*}
    this is called a long exact sequence if it is exact at all $X_i$. Finally, given a pair of $R$-module maps,
    \begin{equation*}
        0\to X_i\xrightarrow{f}X_2\xrightarrow{g}X_3\to 0
    \end{equation*}
    This is a short exact sequence, and $f$ is injective, $g$ is surjective. 
\end{defn}
\begin{prob}
    \textbf{HW(Q12):} Prove the following:
    \begin{enumerate}
        \item Given  LES, 
        \begin{equation*}
            \dots\to X_{i+1}\xrightarrow{f_{i+1}}X_i\xrightarrow{f_i}X_{i-1}
        \end{equation*}
        show the following is a short exact sequence:
        \begin{equation*}
            0\to\ker(f_i)\xrightarrow{i}X_i\xrightarrow{f_i}\ker(f_{i-1})\to 0
        \end{equation*}
        \item Prove the 5-lemma. Given the below sequence, exact at positions $X_i, Y_i$, where $i=2,3,4$, and assume the diagram commutes and if $t_1,t_2,t_4,t_5$ are isomorphisms, show that $t_3$ is also an isomorphism.
        \[\begin{tikzcd}
            {X_1} & {X_2} & {X_3} & {X_4} & {X_5} \\
            {Y_1} & {Y_2} & {Y_3} & {Y_4} & {Y_5}
            \arrow["{f_1}", from=1-1, to=1-2]
            \arrow["{t_1}"', from=1-1, to=2-1]
            \arrow["{f_2}", from=1-2, to=1-3]
            \arrow["{t_2}"', from=1-2, to=2-2]
            \arrow["{f_3}", from=1-3, to=1-4]
            \arrow["{\textcolor{red}{t_3}}"', from=1-3, to=2-3]
            \arrow["{f_4}", from=1-4, to=1-5]
            \arrow["{t_4}"', from=1-4, to=2-4]
            \arrow["{t_5}"', from=1-5, to=2-5]
            \arrow["{g_1}", from=2-1, to=2-2]
            \arrow["{g_2}", from=2-2, to=2-3]
            \arrow["{g_3}", from=2-3, to=2-4]
            \arrow["{g_4}", from=2-4, to=2-5]
        \end{tikzcd}\]
    \end{enumerate}
\end{prob}
Next we state the most important theorem in chain complexes:
\begin{thm}[The snake lemma]
    Let $A.\xrightarrow{f.}B.\xrightarrow{g.}C.$ be a SES of chain complexes, i.e., 
    \begin{equation*}
        A_n\xrightarrow{f_n}B_n\xrightarrow{g_n}C_n
    \end{equation*}
    is a short exact sequence of all $n$. Then there exists a LES of homology groups. 
    \[\begin{tikzcd}
        && {H_{n+1}(C)} \\
        {H_n(A)} & {H_n(B)} & {H_n(C)} \\
        {H_{n-1}(A)} & {H_{n-1}(B)} & {H_{n-1}(C)} \\
        {H_{n-2}(A)}
        \arrow["{\delta_{n-1}}", curve={height=12pt}, from=1-3, to=2-1]
        \arrow["{H_n(f)}", from=2-1, to=2-2]
        \arrow["{H_n(g)}", from=2-2, to=2-3]
        \arrow["{\delta_n}", curve={height=6pt}, from=2-3, to=3-1]
        \arrow["{H_{n-1}(f)}", from=3-1, to=3-2]
        \arrow["{H_{n-1}(g)}", from=3-2, to=3-3]
        \arrow["{\delta_{n-1}}", curve={height=6pt}, from=3-3, to=4-1]
    \end{tikzcd}\]
\end{thm}

\section{Lecture 4 9/9}
Today we prove the snake lemma. We will refer to this following diagram throughout the proof.
\[\begin{tikzcd}
	{A_{n+1}} & {B_{n+1}} & {C_{n+1}} \\
	{A_n} & {B_n} & {C_n} \\
	{A_{n-1}} & {B_{n-1}} & {C_{n-1}} \\
	{A_{n-2}} & {B_{n-2}} & {C_{n-2}}
	\arrow["{f_{n+1}}", from=1-1, to=1-2]
	\arrow["{\delta^A}"', from=1-1, to=2-1]
	\arrow["{g_{n+1}}", from=1-2, to=1-3]
	\arrow["{\delta^B}"', from=1-2, to=2-2]
	\arrow["{\delta^C}"', from=1-3, to=2-3]
	\arrow["{f_n}", from=2-1, to=2-2]
	\arrow["{\delta^A}"', from=2-1, to=3-1]
	\arrow["{g_n}", from=2-2, to=2-3]
	\arrow["{\delta^B}"', from=2-2, to=3-2]
	\arrow["{\delta^C}"', from=2-3, to=3-3]
	\arrow["{f_{n-1}}", from=3-1, to=3-2]
	\arrow["{\delta^A}"', from=3-1, to=4-1]
	\arrow["{g_{n-1}}", from=3-2, to=3-3]
	\arrow["{\delta^B}"', from=3-2, to=4-2]
	\arrow["{\delta^C}"', from=3-3, to=4-3]
	\arrow["{f_{n-2}}", from=4-1, to=4-2]
	\arrow["{g_{n-2}}", from=4-2, to=4-3]
\end{tikzcd}\]
\begin{proof}
    First we define the map $\delta_n: H_n(C)\to H_{n-1}(A)$. Let $[x]\in H_n(C)$, then $x\in\delta^C$, where $\delta^C:C_n\to C_{n-1}$. We define 
    \begin{equation*}
        \delta[x]=[y], y\in A_{n-1}
    \end{equation*}
    as follows: for $x\in C_n$, $g_n: B_n\to C_n$ is surjective, hence there exists $b\in B_n$ such that $g_n(b)=x$. Then consider $d=\delta^B(b)$, since the diagram commutes, we have 
    \begin{equation*}
        d\in\ker g_{n-1}\Rightarrow d\in\im f_{n-1}
    \end{equation*}
    Let $y\in A_{n-1}$ be this unique $y$ such that $f_{n-1}(y)=d$, where uniqueness is by $f_{n-1}$ is injective. This is indicated in the below diagram:
    \[\begin{tikzcd}
        {A_{n+1}} & {B_{n+1}} & {C_{n+1}} \\
        {A_n} & {\textcolor{red}{b}\in B_n} & {\textcolor{red}{x}\in C_n} \\
        {\textcolor{red}{y}\in A_{n-1}} & {\textcolor{red}{d}\in B_{n-1}} & {C_{n-1}} \\
        {A_{n-2}} & {B_{n-2}} & {C_{n-2}}
        \arrow["{f_{n+1}}", from=1-1, to=1-2]
        \arrow["{\delta^A}"', from=1-1, to=2-1]
        \arrow["{g_{n+1}}", from=1-2, to=1-3]
        \arrow["{\delta^B}"', from=1-2, to=2-2]
        \arrow["{\delta^C}"', from=1-3, to=2-3]
        \arrow["{f_n}", from=2-1, to=2-2]
        \arrow["{\delta^A}"', from=2-1, to=3-1]
        \arrow["{g_n}", from=2-2, to=2-3]
        \arrow["{\delta^B}"', from=2-2, to=3-2]
        \arrow["{\delta^C}"', from=2-3, to=3-3]
        \arrow["{f_{n-1}}", from=3-1, to=3-2]
        \arrow["{\delta^A}"', from=3-1, to=4-1]
        \arrow["{g_{n-1}}", from=3-2, to=3-3]
        \arrow["{\delta^B}"', from=3-2, to=4-2]
        \arrow["{\delta^C}"', from=3-3, to=4-3]
        \arrow["{f_{n-2}}", from=4-1, to=4-2]
        \arrow["{g_{n-2}}", from=4-2, to=4-3]
    \end{tikzcd}\]

    We first need to check that $[y]$ does not depend on the choice of $b$. Let $g_n(b_1)=g_n(b_2)=x$, then 
    \begin{equation*}
        g(b_1-b_2)=0\Rightarrow b_1-b_2=f_n(a), a\in A_n
    \end{equation*}
    let $y_1, y_2$ be those determined by $b_1, b_2$, then 
    \begin{equation*}
        f_{n-1}(y_1-y_2)=\delta^B(b_1-b_2)=\delta^B(f_n(a)), a\in A_n
    \end{equation*}
    Because the following diagram commutes,
    \[\begin{tikzcd}
        {\textcolor{red}{a}\in A_n} & {B_n} \\
        {A_{n-1}} & {B_{n-1}}
        \arrow["{f_n}", from=1-1, to=1-2]
        \arrow["{\delta^A}"', from=1-1, to=2-1]
        \arrow["{\delta^B}", from=1-2, to=2-2]
        \arrow["{f_{n-1}}"', from=2-1, to=2-2]
    \end{tikzcd}\]
    we then have 
    \begin{equation*}
        y_1-y_2=\delta^A(a)
    \end{equation*}
    i.e., $[y_1]=[y_2]$, as they only differ by $\im\delta$.

    \begin{prob}
        \textbf{HW(Q13):} Check that if $x\in\im\delta^C$, then $\delta_n[x]=0$.
    \end{prob}
    \textcolor{red}{the proof is not finished, too lazy to tex it up}
\end{proof}

Next we review the tensor products of $R$-modules. We first review $R$-bilinear maps 
\begin{defn}[bilinear maps]
    Let $M,N, P$ be $R$-modules, an $R$-bilinear map $f:M\times N\to P$ is a map such that 
    \begin{enumerate}
        \item $f$ is linear in both coordinates, we have $f(m_1+m_2,n)=f(m_1,n)+f(m_2,n)$, and similarly, $f(m,n_1+n_2)=f(m,n_1)+f(m,n_2)$.
        \item For all $r\in R$, we have $f(rm,n)=f(m,rn)=rf(m,n)$.
    \end{enumerate}
\end{defn}
Next we define tensor products.
\begin{defn}[tensor product]
    A tensor product of $M\times N$ is an $R$-module denoted by $M\otimes_R N$ such that 
    \begin{enumerate}
        \item $M\otimes_R N$ comes endowed with an $R$-bilinear map 
        \begin{equation*}
            M\times N\xrightarrow{\varphi}M\otimes_RN
        \end{equation*}
        \item given any other $R$-bilinear map $f: M\times N\to P$, there exists a unique $R$-module map $\psi$ such that the following diagram commutes:
        \[\begin{tikzcd}
            {M\times N} & {M\otimes_RN} \\
            P
            \arrow["\varphi", from=1-1, to=1-2]
            \arrow["f"', from=1-1, to=2-1]
            \arrow["\psi", from=1-2, to=2-1]
        \end{tikzcd}\]
    \end{enumerate}
\end{defn}
It is not clear that $M\otimes_RN$ exists or not. In fact, they exist!
\begin{thm}[$M\otimes_RN$ exists]
    Define $M\otimes_RN=R\langle M\times N\rangle/K$, where $R\langle M\times N\rangle$ is the free $R$-module on the set $M\times N$. We define $K$ as the submodule generated by the following four relations:
    \begin{enumerate}
        \item $\la(m_1+m_2,n)\ra-\la(m_1,n)\ra-\la(m_2,n)\ra$
        \item $\la(m,n_1+n_2)\ra-\la(m,n_1)\ra-\la(m,n_2)\ra$
        \item $r\la(m,n)\ra-\la(rm,n)\ra$
        \item $r\la(m,n)\ra-\la(m,rn)\ra$
    \end{enumerate}
    Moreover, the map $\varphi:M\times N\to M\otimes_RN$ given by 
    \begin{equation*}
        (m,n)\mapsto \la(m,n)\ra:=m\otimes_Rn
    \end{equation*}
\end{thm}
\begin{prob}
    \textbf{HW(Q14):} show that $M\otimes_RN$ is a tensor product.
\end{prob}

\section{Lecture 5 9/11}
We continue with the tensors of $R$-modules. Let $f:A\to B$ an an $R$-module map, let $N$ be some fixed $R$-module, then $f$ induces maps:  $f\otimes id: A\otimes_R N\to B\otimes_R N$, 
\begin{equation*}
    f\otimes id: a\otimes n\mapsto f(a)\otimes n
\end{equation*}
and $id\otimes f: N\otimes f: N\otimes_R A\to N\otimes_R B:$
\begin{equation*}
    id\otimes f: n\otimes a\mapsto n\otimes f(a)
\end{equation*}
\begin{prob}
    \textbf{HW(Q15(a)): } Show that the following maps induce functors:
    \begin{enumerate}
        \item $-\otimes_R N: Mod_R\to Mod_R$, where 
        \begin{equation*}
            A\mapsto A\otimes_R N, f\mapsto f\otimes id
        \end{equation*}
        \item $N\otimes_R -: Mod_R\to Mod_R$, where \begin{equation*}
            A\mapsto N\otimes_R A, f\mapsto id\otimes f
        \end{equation*}
    \end{enumerate}
\end{prob}


\begin{prob}
    \textbf{HW(Q15(b)):} Show that one has the following natural isomorphisms:
    \begin{enumerate}
        \item $0\otimes_R M\cong 0$, and $0\otimes_R -\cong F_0$ (recall the definition of $F_0$ as a functor).
        \item $R\otimes_R M\cong M$, and $R\otimes_R-\cong id$.
        \item $M\otimes_R N\cong N\otimes_R M$, and $M\otimes_R-\cong -\otimes_R M$.
        \item $M\otimes_R(N\otimes_R K)\cong (M\otimes_R N)\otimes_R K$.
        \item $(M\oplus N)\otimes_RK\cong (M\otimes_R K)\oplus(N\otimes_R K)$.
    \end{enumerate}
\end{prob}
For convenience, we introduce the following definition:
\begin{defn}[positively graded chain complex]
    A positively graded chain complex $\{M.;\partial.^M\}$ is a chain complex so that $M_i=0$ for all $i<0$. The category of positively graded chain complexed is denoted as $ch_R^+$.
\end{defn}

We have our first important theorem for $ch_R^+$.
\begin{thm}
    There exists a functor $\otimes_R$ and a natural transformation $X$ such that the following diagram of functors commutes up to some $X$:
    \[\begin{tikzcd}
        {ch_R^+\times ch+R^+} & {ch_R^+} \\
        {Mod_R\times Mod_R} & {Mod_R}
        \arrow["{\otimes_R}", from=1-1, to=1-2]
        \arrow["{H_i\times H_j}"', from=1-1, to=2-1]
        \arrow["{H_{i+j}}", from=1-2, to=2-2]
        \arrow["{\textcolor{blue}{X}}", Rightarrow, from=2-1, to=1-2]
        \arrow["{\otimes_R}"', from=2-1, to=2-2]
    \end{tikzcd}\]
    where $X: \otimes_R\circ(H_i\times H_j)\Rightarrow H_{i+j}\circ\otimes_R$ is a natural transformation.
\end{thm}
We note that the existence of $X$ means this diagram doesn't commute ``on the nose,'' but these two composition functors are the same up to some natural transformation. Before we given the proof, we recall that $Ob(C\times D)=Ob(C)\times Ob(D), mor((X,Y), (X',Y'))=mor(X,Y)\times mor(X',Y')$.
\begin{proof}
    We define $\otimes_R$ of positively graded chain complexes as follows: let $\{M.;\partial.^M\}, \{N.;\partial.^N\}$ be two PGCC. Define $\{M\otimes_RN.;\partial.^M\}$:
    \begin{equation*}
        (M\otimes_RN)=\bigoplus_{i+j=n}(M_i\otimes_RN_j)
    \end{equation*} 
    note that the RHS is always a finite sum. Moreover, $\partial^\otimes$ is defined as follows:
    \begin{equation*}
        \partial^\otimes: (M\otimes_RN)_n\to (M\otimes_RN)_{n-1} \text{ is defined on the component $M_i\otimes_R N_j$ (from the RHS) }
    \end{equation*}
    and 
    \begin{equation*}
        \partial^\otimes(m_i\otimes n_j):=\partial^M(m_i)\otimes n_j+(-1)^im_i\otimes\partial^N(n_j)
    \end{equation*}
    It is easy to check that $\partial^\otimes\circ\partial^\otimes=0$.

    Now we've show $ch_R^+\otimes_Rch_R^+$ is well-defined, it remains to define $X$, the natural transformation. We define 
    \begin{equation*}
        X: H_i(M.)\otimes_R H_j(N.)\to H_{i+j}(M.\otimes_R N.)
    \end{equation*}
    again, it suffices to define $X$ on elementary tensors.
    \begin{equation*}
        X: [\alpha]\otimes[\beta]\mapsto [\alpha\otimes\beta]
    \end{equation*}
    we need to check that 
    \begin{enumerate}
        \item $\partial^\otimes(\alpha\otimes\beta)=0$ if $\partial^M(\alpha)=0$ and $\partial^N(\beta)=0$.
        \item If $\alpha=\partial(r)i$, then notice that $\partial^\otimes(r\otimes\beta)=\alpha\otimes\beta$, similarly for $\beta$. This would show that $X$ is well-defined. 
    \end{enumerate}
    It is straightforward to check that $X$ commutes with morphisms in $ch_R^+\times ch_R^+$.
\end{proof}

Next we define cochain complexes and cohomologies. 
\begin{defn}[cochain]
    A cochain of $R$-modules $(M^\bullet, \partial_M^\bullet)$ is a sequence of $R$-module maps:
    \[\begin{tikzcd}
        \dots & {M^i} & {M^{i+1}} & {M^{i+2}} & \dots
        \arrow[from=1-1, to=1-2]
        \arrow["{\partial^i}", from=1-2, to=1-3]
        \arrow["{\partial^{i+1}}", from=1-3, to=1-4]
        \arrow[from=1-4, to=1-5]
    \end{tikzcd}\]
    such that $\partial\circ\partial=0$.
\end{defn}
Cochain complexes form a category, with morphisms $\{f^\bullet\}$ form a ladder: 
\[\begin{tikzcd}
	\dots & {M^i} & {M^{i+1}} & {M^{i+2}} & \dots \\
	\dots & {N^i} & {N^{i+1}} & {N^{i+2}} & \dots
	\arrow[from=1-1, to=1-2]
	\arrow["{\partial^i}", from=1-2, to=1-3]
	\arrow["{f^i}", from=1-2, to=2-2]
	\arrow["{\partial^{i+1}}", from=1-3, to=1-4]
	\arrow["{f^{i+1}}", from=1-3, to=2-3]
	\arrow[from=1-4, to=1-5]
	\arrow["{f^{i+2}}", from=1-4, to=2-4]
	\arrow[from=2-1, to=2-2]
	\arrow["{\partial^i}", from=2-2, to=2-3]
	\arrow["{\partial^{i+1}}", from=2-3, to=2-4]
\end{tikzcd}\]
The $n$-th cohomology of a cochain complex $\{M^\bullet; \partial_M^\bullet\}$ is defined as:
\begin{equation*}
    H^n(M^\bullet;\partial_M^\bullet):=\frac{\ker\partial^i: M^i\to M^{i+1}}{\im\partial^{i-1}: M^{i-1}\to M^i}
\end{equation*}
We remark that there is nothing unexpected here from what we learned about chain complexes. Namely, if we reindex $\{M^\bullet;\partial_M^\bullet\}$, this defines a chain complex with $M_i'=M^{-i}$. This implies that the snake lemme holds! (with $\partial^i:H^i(C)\to H^{i+1}(A)).$
\begin{thm}
    There is a functor $D$ and a natural transformation  $\beta$ such that the following diagram of functors commute up to the natrual transformation $\beta$:
    \[\begin{tikzcd}
        {ch_R^{op}} & {coch_R} \\
        {Mod_R^{op}} & {Mod_R}
        \arrow["D", from=1-1, to=1-2]
        \arrow["{H_n^{op}}"', from=1-1, to=2-1]
        \arrow["{H^n}", from=1-2, to=2-2]
        \arrow["{\textcolor{blue}{\beta}}", from=2-1, to=1-2]
        \arrow["{\overline{D}}"', from=2-1, to=2-2]
    \end{tikzcd}\]
    where $\overline{D}(M)=Hom_R(M,R)$, and 
    \begin{equation*}
        D(\{M_\bullet;\partial_\bullet^M\}) \text{ is defined as } \{DM^\bullet; \partial^\bullet\}
    \end{equation*}
    where 
    \begin{equation*}
        DM^n:=Hom_R(M_n,R), \partial^n: DM^n\to DM^{n+1} \text{ is the map induced by } \partial_{n+1}: M_{n+1}\to M_n
    \end{equation*}
\end{thm}
We observe that $\partial^{n+1}\partial^n=0$ since $\partial_{n+2}\partial_{n+1}=0$.
\begin{prob}
    \textbf{HW(Q16):} Show that $D$ is a functor.
\end{prob}
Next we define the natural transformation $\beta$. We have $\beta: H^n(DM)\to Hom_R(H_n(M_\bullet),R)$, such that 
\begin{equation*}
    \beta: [\varphi]\mapsto \beta[\varphi]
\end{equation*}
let $[x]\in H_n(M_\bullet)$, where $\beta[\varphi]([x])=\varphi(x)$ (where $\varphi\in Hom_R(M_n,\R), x\in M_n$).
\begin{proof}
    We first need to show that $\beta$ is well-defined. If $x=\partial_{n+1}(y)$, then consider 
    \begin{equation*}
        \beta[\varphi][x]=\varphi(x)=\varphi(\partial_{n+1}(y))=\partial^n(\varphi)(y)=0, x\in\ker\varphi
    \end{equation*}
    Conversely, if $\varphi=\delta^{n-1}(\psi)$, we have 
    \begin{equation*}
        \beta[\varphi][x]=\varphi(x)=\delta^{n-1}\psi(x)=\psi(\partial_n(x))=0
    \end{equation*}
    It remains to check that $\beta$ commutes with morphisms in $ch_R^{op}$ (which we will do next time).
\end{proof}




\section{Lecture 4 9/9}
Today we prove the snake lemma. We will refer to this following diagram throughout the proof.
\[\begin{tikzcd}
	{A_{n+1}} & {B_{n+1}} & {C_{n+1}} \\
	{A_n} & {B_n} & {C_n} \\
	{A_{n-1}} & {B_{n-1}} & {C_{n-1}} \\
	{A_{n-2}} & {B_{n-2}} & {C_{n-2}}
	\arrow["{f_{n+1}}", from=1-1, to=1-2]
	\arrow["{\delta^A}"', from=1-1, to=2-1]
	\arrow["{g_{n+1}}", from=1-2, to=1-3]
	\arrow["{\delta^B}"', from=1-2, to=2-2]
	\arrow["{\delta^C}"', from=1-3, to=2-3]
	\arrow["{f_n}", from=2-1, to=2-2]
	\arrow["{\delta^A}"', from=2-1, to=3-1]
	\arrow["{g_n}", from=2-2, to=2-3]
	\arrow["{\delta^B}"', from=2-2, to=3-2]
	\arrow["{\delta^C}"', from=2-3, to=3-3]
	\arrow["{f_{n-1}}", from=3-1, to=3-2]
	\arrow["{\delta^A}"', from=3-1, to=4-1]
	\arrow["{g_{n-1}}", from=3-2, to=3-3]
	\arrow["{\delta^B}"', from=3-2, to=4-2]
	\arrow["{\delta^C}"', from=3-3, to=4-3]
	\arrow["{f_{n-2}}", from=4-1, to=4-2]
	\arrow["{g_{n-2}}", from=4-2, to=4-3]
\end{tikzcd}\]
\begin{proof}
    First we define the map $\delta_n: H_n(C)\to H_{n-1}(A)$. Let $[x]\in H_n(C)$, then $x\in\delta^C$, where $\delta^C:C_n\to C_{n-1}$. We define 
    \begin{equation*}
        \delta[x]=[y], y\in A_{n-1}
    \end{equation*}
    as follows: for $x\in C_n$, $g_n: B_n\to C_n$ is surjective, hence there exists $b\in B_n$ such that $g_n(b)=x$. Then consider $d=\delta^B(b)$, since the diagram commutes, we have 
    \begin{equation*}
        d\in\ker g_{n-1}\Rightarrow d\in\im f_{n-1}
    \end{equation*}
    Let $y\in A_{n-1}$ be this unique $y$ such that $f_{n-1}(y)=d$, where uniqueness is by $f_{n-1}$ is injective. This is indicated in the below diagram:
    \[\begin{tikzcd}
        {A_{n+1}} & {B_{n+1}} & {C_{n+1}} \\
        {A_n} & {\textcolor{red}{b}\in B_n} & {\textcolor{red}{x}\in C_n} \\
        {\textcolor{red}{y}\in A_{n-1}} & {\textcolor{red}{d}\in B_{n-1}} & {C_{n-1}} \\
        {A_{n-2}} & {B_{n-2}} & {C_{n-2}}
        \arrow["{f_{n+1}}", from=1-1, to=1-2]
        \arrow["{\delta^A}"', from=1-1, to=2-1]
        \arrow["{g_{n+1}}", from=1-2, to=1-3]
        \arrow["{\delta^B}"', from=1-2, to=2-2]
        \arrow["{\delta^C}"', from=1-3, to=2-3]
        \arrow["{f_n}", from=2-1, to=2-2]
        \arrow["{\delta^A}"', from=2-1, to=3-1]
        \arrow["{g_n}", from=2-2, to=2-3]
        \arrow["{\delta^B}"', from=2-2, to=3-2]
        \arrow["{\delta^C}"', from=2-3, to=3-3]
        \arrow["{f_{n-1}}", from=3-1, to=3-2]
        \arrow["{\delta^A}"', from=3-1, to=4-1]
        \arrow["{g_{n-1}}", from=3-2, to=3-3]
        \arrow["{\delta^B}"', from=3-2, to=4-2]
        \arrow["{\delta^C}"', from=3-3, to=4-3]
        \arrow["{f_{n-2}}", from=4-1, to=4-2]
        \arrow["{g_{n-2}}", from=4-2, to=4-3]
    \end{tikzcd}\]

    We first need to check that $[y]$ does not depend on the choice of $b$. Let $g_n(b_1)=g_n(b_2)=x$, then 
    \begin{equation*}
        g(b_1-b_2)=0\Rightarrow b_1-b_2=f_n(a), a\in A_n
    \end{equation*}
    let $y_1, y_2$ be those determined by $b_1, b_2$, then 
    \begin{equation*}
        f_{n-1}(y_1-y_2)=\delta^B(b_1-b_2)=\delta^B(f_n(a)), a\in A_n
    \end{equation*}
    Because the following diagram commutes,
    \[\begin{tikzcd}
        {\textcolor{red}{a}\in A_n} & {B_n} \\
        {A_{n-1}} & {B_{n-1}}
        \arrow["{f_n}", from=1-1, to=1-2]
        \arrow["{\delta^A}"', from=1-1, to=2-1]
        \arrow["{\delta^B}", from=1-2, to=2-2]
        \arrow["{f_{n-1}}"', from=2-1, to=2-2]
    \end{tikzcd}\]
    we then have 
    \begin{equation*}
        y_1-y_2=\delta^A(a)
    \end{equation*}
    i.e., $[y_1]=[y_2]$, as they only differ by $\im\delta$.

    \begin{prob}
        \textbf{HW(Q13):} Check that if $x\in\im\delta^C$, then $\delta_n[x]=0$.
    \end{prob}
    \textcolor{red}{the proof is not finished, too lazy to tex it up}
\end{proof}

Next we review the tensor products of $R$-modules. We first review $R$-bilinear maps 
\begin{defn}[bilinear maps]
    Let $M,N, P$ be $R$-modules, an $R$-bilinear map $f:M\times N\to P$ is a map such that 
    \begin{enumerate}
        \item $f$ is linear in both coordinates, we have $f(m_1+m_2,n)=f(m_1,n)+f(m_2,n)$, and similarly, $f(m,n_1+n_2)=f(m,n_1)+f(m,n_2)$.
        \item For all $r\in R$, we have $f(rm,n)=f(m,rn)=rf(m,n)$.
    \end{enumerate}
\end{defn}
Next we define tensor products.
\begin{defn}[tensor product]
    A tensor product of $M\times N$ is an $R$-module denoted by $M\otimes_R N$ such that 
    \begin{enumerate}
        \item $M\otimes_R N$ comes endowed with an $R$-bilinear map 
        \begin{equation*}
            M\times N\xrightarrow{\varphi}M\otimes_RN
        \end{equation*}
        \item given any other $R$-bilinear map $f: M\times N\to P$, there exists a unique $R$-module map $\psi$ such that the following diagram commutes:
        \[\begin{tikzcd}
            {M\times N} & {M\otimes_RN} \\
            P
            \arrow["\varphi", from=1-1, to=1-2]
            \arrow["f"', from=1-1, to=2-1]
            \arrow["\psi", from=1-2, to=2-1]
        \end{tikzcd}\]
    \end{enumerate}
\end{defn}
It is not clear that $M\otimes_RN$ exists or not. In fact, they exist!
\begin{thm}[$M\otimes_RN$ exists]
    Define $M\otimes_RN=R\langle M\times N\rangle/K$, where $R\langle M\times N\rangle$ is the free $R$-module on the set $M\times N$. We define $K$ as the submodule generated by the following four relations:
    \begin{enumerate}
        \item $\la(m_1+m_2,n)\ra-\la(m_1,n)\ra-\la(m_2,n)\ra$
        \item $\la(m,n_1+n_2)\ra-\la(m,n_1)\ra-\la(m,n_2)\ra$
        \item $r\la(m,n)\ra-\la(rm,n)\ra$
        \item $r\la(m,n)\ra-\la(m,rn)\ra$
    \end{enumerate}
    Moreover, the map $\varphi:M\times N\to M\otimes_RN$ given by 
    \begin{equation*}
        (m,n)\mapsto \la(m,n)\ra:=m\otimes_Rn
    \end{equation*}
\end{thm}

\section{Lecture 5 9/11}
We continue with the tensors of $R$-modules. Let $f:A\to B$ an an $R$-module map, let $N$ be some fixed $R$-module, then $f$ induces maps:  $f\otimes id: A\otimes_R N\to B\otimes_R N$, 
\begin{equation*}
    f\otimes id: a\otimes n\mapsto f(a)\otimes n
\end{equation*}
and $id\otimes f: N\otimes f: N\otimes_R A\to N\otimes_R B:$
\begin{equation*}
    id\otimes f: n\otimes a\mapsto n\otimes f(a)
\end{equation*}
\begin{prob}
    \textbf{HW(Q14(a)): } Show that the following maps induce functors:
    \begin{enumerate}
        \item $-\otimes_R N: Mod_R\to Mod_R$, where 
        \begin{equation*}
            A\mapsto A\otimes_R N, f\mapsto f\otimes id
        \end{equation*}
        \item $N\otimes_R -: Mod_R\to Mod_R$, where \begin{equation*}
            A\mapsto N\otimes_R A, f\mapsto id\otimes f
        \end{equation*}
    \end{enumerate}
\end{prob}


\begin{prob}
    \textbf{HW(Q14(b)):} Show that one has the following natural isomorphisms:
    \begin{enumerate}
        \item $0\otimes_R M\cong 0$, and $0\otimes_R -\cong F_0$ (recall the definition of $F_0$ as a functor).
        \item $R\otimes_R M\cong M$, and $R\otimes_R-\cong id$.
        \item $M\otimes_R N\cong N\otimes_R M$, and $M\otimes_R-\cong -\otimes_R M$.
        \item $M\otimes_R(N\otimes_R K)\cong (M\otimes_R N)\otimes_R K$.
        \item $(M\oplus N)\otimes_RK\cong (M\otimes_R K)\oplus(N\otimes_R K)$.
    \end{enumerate}
\end{prob}
For convenience, we introduce the following definition:
\begin{defn}[positively graded chain complex]
    A positively graded chain complex $\{M.;\partial.^M\}$ is a chain complex so that $M_i=0$ for all $i<0$. The category of positively graded chain complexed is denoted as $ch_R^+$.
\end{defn}

We have our first important theorem for $ch_R^+$.
\begin{thm}
    There exists a functor $\otimes_R$ and a natural transformation $X$ such that the following diagram of functors commutes up to some $X$:
    \[\begin{tikzcd}
        {ch_R^+\times ch+R^+} & {ch_R^+} \\
        {Mod_R\times Mod_R} & {Mod_R}
        \arrow["{\otimes_R}", from=1-1, to=1-2]
        \arrow["{H_i\times H_j}"', from=1-1, to=2-1]
        \arrow["{H_{i+j}}", from=1-2, to=2-2]
        \arrow["{\textcolor{blue}{X}}", Rightarrow, from=2-1, to=1-2]
        \arrow["{\otimes_R}"', from=2-1, to=2-2]
    \end{tikzcd}\]
    where $X: \otimes_R\circ(H_i\times H_j)\Rightarrow H_{i+j}\circ\otimes_R$ is a natural transformation.
\end{thm}
We note that the existence of $X$ means this diagram doesn't commute ``on the nose,'' but these two composition functors are the same up to some natural transformation. Before we given the proof, we recall that $Ob(C\times D)=Ob(C)\times Ob(D), mor((X,Y), (X',Y'))=mor(X,Y)\times mor(X',Y')$.
\begin{proof}
    We define $\otimes_R$ of positively graded chain complexes as follows: let $\{M.;\partial.^M\}, \{N.;\partial.^N\}$ be two PGCC. Define $\{M\otimes_RN.;\partial.^M\}$:
    \begin{equation*}
        (M\otimes_RN)=\bigoplus_{i+j=n}(M_i\otimes_RN_j)
    \end{equation*} 
    note that the RHS is always a finite sum. Moreover, $\partial^\otimes$ is defined as follows:
    \begin{equation*}
        \partial^\otimes: (M\otimes_RN)_n\to (M\otimes_RN)_{n-1} \text{ is defined on the component $M_i\otimes_R N_j$ (from the RHS) }
    \end{equation*}
    and 
    \begin{equation*}
        \partial^\otimes(m_i\otimes n_j):=\partial^M(m_i)\otimes n_j+(-1)^im_i\otimes\partial^N(n_j)
    \end{equation*}
    It is easy to check that $\partial^\otimes\circ\partial^\otimes=0$.

    Now we've show $ch_R^+\otimes_Rch_R^+$ is well-defined, it remains to define $X$, the natural transformation. We define 
    \begin{equation*}
        X: H_i(M.)\otimes_R H_j(N.)\to H_{i+j}(M.\otimes_R N.)
    \end{equation*}
    again, it suffices to define $X$ on elementary tensors.
    \begin{equation*}
        X: [\alpha]\otimes[\beta]\mapsto [\alpha\otimes\beta]
    \end{equation*}
    we need to check that 
    \begin{enumerate}
        \item $\partial^\otimes(\alpha\otimes\beta)=0$ if $\partial^M(\alpha)=0$ and $\partial^N(\beta)=0$.
        \item If $\alpha=\partial(r)i$, then notice that $\partial^\otimes(r\otimes\beta)=\alpha\otimes\beta$, similarly for $\beta$. This would show that $X$ is well-defined. 
    \end{enumerate}
    It is straightforward to check that $X$ commutes with morphisms in $ch_R^+\times ch_R^+$.
\end{proof}

Next we define cochain complexes and cohomologies. 
\begin{defn}[cochain]
    A cochain of $R$-modules $(M^\bullet, \partial_M^\bullet)$ is a sequence of $R$-module maps:
    \[\begin{tikzcd}
        \dots & {M^i} & {M^{i+1}} & {M^{i+2}} & \dots
        \arrow[from=1-1, to=1-2]
        \arrow["{\partial^i}", from=1-2, to=1-3]
        \arrow["{\partial^{i+1}}", from=1-3, to=1-4]
        \arrow[from=1-4, to=1-5]
    \end{tikzcd}\]
    such that $\partial\circ\partial=0$.
\end{defn}
Cochain complexes form a category, with morphisms $\{f^\bullet\}$ form a ladder: 
\[\begin{tikzcd}
	\dots & {M^i} & {M^{i+1}} & {M^{i+2}} & \dots \\
	\dots & {N^i} & {N^{i+1}} & {N^{i+2}} & \dots
	\arrow[from=1-1, to=1-2]
	\arrow["{\partial^i}", from=1-2, to=1-3]
	\arrow["{f^i}", from=1-2, to=2-2]
	\arrow["{\partial^{i+1}}", from=1-3, to=1-4]
	\arrow["{f^{i+1}}", from=1-3, to=2-3]
	\arrow[from=1-4, to=1-5]
	\arrow["{f^{i+2}}", from=1-4, to=2-4]
	\arrow[from=2-1, to=2-2]
	\arrow["{\partial^i}", from=2-2, to=2-3]
	\arrow["{\partial^{i+1}}", from=2-3, to=2-4]
\end{tikzcd}\]
The $n$-th cohomology of a cochain complex $\{M^\bullet; \partial_M^\bullet\}$ is defined as:
\begin{equation*}
    H^n(M^\bullet;\partial_M^\bullet):=\frac{\ker\partial^i: M^i\to M^{i+1}}{\im\partial^{i-1}: M^{i-1}\to M^i}
\end{equation*}
We remark that there is nothing unexpected here from what we learned about chain complexes. Namely, if we reindex $\{M^\bullet;\partial_M^\bullet\}$, this defines a chain complex with $M_i'=M^{-i}$. This implies that the snake lemme holds! (with $\partial^i:H^i(C)\to H^{i+1}(A)).$
\begin{thm}
    There is a functor $D$ and a natural transformation  $\beta$ such that the following diagram of functors commute up to the natrual transformation $\beta$:
    \[\begin{tikzcd}
        {ch_R^{op}} & {coch_R} \\
        {Mod_R^{op}} & {Mod_R}
        \arrow["D", from=1-1, to=1-2]
        \arrow["{H_n^{op}}"', from=1-1, to=2-1]
        \arrow["{H^n}", from=1-2, to=2-2]
        \arrow["{\textcolor{blue}{\beta}}", from=2-1, to=1-2]
        \arrow["{\overline{D}}"', from=2-1, to=2-2]
    \end{tikzcd}\]
    where $\overline{D}(M)=Hom_R(M,R)$, and 
    \begin{equation*}
        D(\{M_\bullet;\partial_\bullet^M\}) \text{ is defined as } \{DM^\bullet; \partial^\bullet\}
    \end{equation*}
    where 
    \begin{equation*}
        DM^n:=Hom_R(M_n,R), \partial^n: DM^n\to DM^{n+1} \text{ is the map induced by } \partial_{n+1}: M_{n+1}\to M_n
    \end{equation*}
\end{thm}
We observe that $\partial^{n+1}\partial^n=0$ since $\partial_{n+2}\partial_{n+1}=0$.
\begin{prob}
    \textbf{HW(Q15):} Show that $D$ is a functor.
\end{prob}
Next we define the natural transformation $\beta$. We have $\beta: H^n(DM)\to Hom_R(H_n(M_\bullet),R)$, such that 
\begin{equation*}
    \beta: [\varphi]\mapsto \beta[\varphi]
\end{equation*}
let $[x]\in H_n(M_\bullet)$, where $\beta[\varphi]([x])=\varphi(x)$ (where $\varphi\in Hom_R(M_n,\R), x\in M_n$).
\begin{proof}
    We first need to show that $\beta$ is well-defined. If $x=\partial_{n+1}(y)$, then consider 
    \begin{equation*}
        \beta[\varphi][x]=\varphi(x)=\varphi(\partial_{n+1}(y))=\partial^n(\varphi)(y)=0, x\in\ker\varphi
    \end{equation*}
    Conversely, if $\varphi=\delta^{n-1}(\psi)$, we have 
    \begin{equation*}
        \beta[\varphi][x]=\varphi(x)=\delta^{n-1}\psi(x)=\psi(\partial_n(x))=0
    \end{equation*}
    It remains to check that $\beta$ commutes with morphisms in $ch_R^{op}$ (which we will do next time).
\end{proof}




