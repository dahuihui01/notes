\chapter{Category Theory}
\textbf{Instructor:} Nitu Kitchro, 
\textbf{Office Hours}: Monday after class, \textbf{TA: } Anna Matsui
\section{Lecture 1 8/26}
\begin{defn}[Category]
    A category $\mathcal{C}$ consists of the following data:
    \begin{enumerate}
        \item A collection of objects denoted as Ob$(\mathcal{C})$
        \item Given two objects $X,Y\in$ Ob$(\mathcal{C})$, a collection of morphisms between $X,Y$, $f:X\to Y$, denoted as mor$_\CC(X,Y)$.
        \item (Composition) We have mor$_\CC(X,Y)\times mor_\CC(Y,Z)\to mor_\CC(X,Z)$ that satisfies associativity
        \begin{equation*}
            f\circ(g\circ h)=(f\circ g)\circ h
        \end{equation*}
        \item (Identity) There is a distinguished morphism for each $X$, $Id_\CC(X,X)$ such that given any $f\in mor(X,Y)$, we have $f\circ id_X=id_Y\circ f=f$.
    \end{enumerate}
\end{defn}
In this course, we will make the assumption that in all the categories that we work with, Ob$(\CC)$ need not be a set, but given any $X,Y\in Ob(\CC)$, mor$(X,Y)$ will always be a set. Now we talka bout some examples of categories.
\begin{example}[Sets]
    Let Ob$(Sets)$ be all the sets in the universe. Given $X,Y$ sets, mor$(X,Y)$ be all the set maps from $X$ to $Y$, and $id_X$ is the identity map.
\end{example}
\begin{example}[Top]
    Let Ob$(Top)$ be all the topological spaces, and mor$(X,Y)$ be all the continuous maps from $X$ to $Y$.
\end{example}
\begin{example}[Vect$_\F$]
    Let $\F$ be a field, and let Ob be all the $\F$-vector spaces. Then mor$(V,W)$ is all the $\F$-linear homomorphisms from $V$ to $W$, where $id_V$ is the identity homomorphism.
\end{example}
\begin{example}[Posets]
    Fix a poset $P$, let Ob$(P)$ be the collection of elements in $P$, and given $p,q$ we define 
    \begin{equation*}
        mor(p,q)=\begin{cases}
            *, \text{ if } q\leq p\\
            \emptyset, \text{ otherwise }
        \end{cases}
    \end{equation*}
\end{example}
\begin{prob}
    \textbf{HW(Q1): check this is a category}
\end{prob}
\begin{example}[Opposite category]
    Given a category $\CC$, there is another category called the opposite category, denoted as $\CC^{op}$, where 
    \begin{enumerate}
        \item The objects are the same as $\CC$
        \item Given $X,Y\in$ Ob$(C^{op})$, we have mor$_{op}(X,Y):=$ mor$_\CC(Y,X)$. 
        \item Moreover, given $f\in mor_{op}(X,Y), g\in mor_{op}(Y,Z)$, then $g\circ f$ in $C^{op}$ is $f\circ g: Z\to X$.
    \end{enumerate}
\end{example}
Naturally, we define isomorphisms now.
\begin{defn}[isomorphism]
    Given a category $\CC$, and a morphism $f\in mor_C(X,Y)$, we say $f$ is an isomorphism if there exists $g\in mor_C(Y,X)$ such that 
    \begin{equation*}
        f\circ g=Id_Y, g\circ f=Id_X
    \end{equation*}
\end{defn}
Now we introduce maps between categories.
\begin{defn}[functor]
    Given categories $\CC,\mathcal{D}$, a functor $F:C\to D$ is the following;
    \begin{enumerate}
        \item Given an object $X$ in $\mathcal{C}$, $F(X)$ is an object in $D$. 
        \item Given a morphism $f: X\to Y$, $F(f)$ is a functor $F(f): F(X)\to F(Y)$. Moreover, it satisfies the following:
        \begin{enumerate}
            \item $F(id_X)=id_{F(X)}$
            \item $F(f\circ g)=F(f)\circ F(g)$. Alternatively, we can rewrite this condition as the following: 
            \[\begin{tikzcd}
                {mor(X,Y)\times mor(Y,Z)} & {mor(X,Z)} \\
                {mor(F(X), F(Y))\times mor(F(Y),F(Z))} & {mor(F(X),F(Z))}
                \arrow[from=1-1, to=1-2]
                \arrow["{mor(F)\times mor(F)}", from=1-1, to=2-1]
                \arrow["{mor(F)}", from=1-2, to=2-2]
                \arrow[from=2-1, to=2-2]
            \end{tikzcd}\]
            such that this diagram commutes.
        \end{enumerate}
    \end{enumerate}
\end{defn}
\begin{prob}
    \textbf{HW(Q2): functors take isomorphisms to isomorphisms.}
\end{prob}
Now we talk about some examples of functors.
\begin{example}
    $F: Top\to Set$, where $X\mapsto X$, where the latter is a set, and $f\mapsto f$ as set maps.
\end{example}
\begin{example}
    Let $\F$ be a field, and $F: Sets\to \text{Vect}_\F$, where $X\mapsto \F\la X\ra$, where $\F\la X\ra$ is the free vector space over $\F$ on the set $X$.
\end{example}
\begin{prob}
    \textbf{HW(Q3): extend this to a functor by defining $mor(f)$ and show this is a functor.}
\end{prob}
\begin{example}
    Let $\F$ be a field, then the following is a functor, $F: Sets^{op}\to\text{Vect}_\F$, where
    \begin{equation*}
        h  F: X\mapsto Maps(X,\F)
    \end{equation*}
\end{example}
\begin{prob}
    \textbf{HW(Q4)}: show this extends to a functor by defining $F(f)$, and show it is a functor.
\end{prob}

\section{Lecture 2 8/28}
\begin{defn}[contravariant functor]
    Let $F:\CC\to\mathcal{D}$ is a contravariant functor from $\CC^{op}\to\mathcal{D}$, (equivalently, $\CC\to\mathcal{D}^{op}$).
\end{defn}
\begin{prob}
    \textbf{HW(Q5):} Show that the following functor $F$ from Vect$_\F$ to Vect$_\F$ extends to a contravariant functor, where 
    \begin{equation*}
        Ob_F: V\mapsto V^*=Hom(V,\F)
    \end{equation*}
    i.e., define the morphism function and show it is a contravariant functor.
\end{prob}
We remark that we can define a category of categories: let $Cat$ be the category of categories, with morphisms as functors, and note that objects or morphisms in this case are both not sets!
\begin{defn}[natural transformation]
    Given functors $F,G: \CC\to\mathcal{D}$, a natural transformation $T$ from $F$ to $G$ is the following: $T: F\Rightarrow G$:
    \begin{enumerate}
        \item given object $X\in Ob(\CC)$, $T(X)\in mor(F(X),G(X))$
        \item Given $f\in mor(X,Y)$, the following diagram commutes:
        \[\begin{tikzcd}
            {F(X)} & {F(Y)} \\
            {G(X)} & {G(Y)}
            \arrow["{F(f)}", from=1-1, to=1-2]
            \arrow["{T(X)}"', from=1-1, to=2-1]
            \arrow["{T(Y)}", from=1-2, to=2-2]
            \arrow["{G(f)}"', from=2-1, to=2-2]
        \end{tikzcd}\]
        where $mor_F, mor_G$ is the identification function on morphisms by functors $F,G$
    \end{enumerate}
    If for all $X$, $T(X)$ is an isomorphism, then this natural transformation is called a natural isomorphism.
\end{defn}
In other words, this natural transformation is how one takes a functor $F$ and turn it to another functor $G$. We will (in a homework) show there exists natural transofrmation between the following two functors.
\begin{example}
    Consider $F,G: Vect_\F\to Vect_\F$, define 
    \begin{equation*}
        F(V)=V\otimes_\F V/_{\la a\otimes b-b\otimes a\ra}=V\otimes_\F V/\Sigma_2, G(V)=(V\otimes_F V)^{\Sigma_2}=\{\alpha\in V\otimes_\F V: \sigma(\alpha)=\alpha\}
    \end{equation*}
    Both are vector spaces are fixed under ``swaps.'' Then a natural transformation can be defined as follows $T(V):$
    \begin{equation*}
        T(V): a\otimes b\mapsto a\otimes b+b\otimes a
    \end{equation*}
\end{example}
\begin{prob}
    \textbf{HW(Q6):} For the above $F,G$
    \begin{enumerate}
        \item Show that $T$ defines a natural transformation from $F$ to $G$. 
        \item Find conditions on $\F$ for $T$ being a natural isomorphism.
    \end{enumerate}
\end{prob}
Next we define limits and colimits.
    Let $\mathcal{C},\mathcal{D}$ be categories, $d$ be an object in $\mathcal{D}$, then we can define a functor $F_d: \mathcal{C}\to\mathcal{D}$ such that for any object $c$ in $\mathcal{C}$,
    \begin{equation*}
        F_d(c)=d, F_d(f)=Id_d
    \end{equation*}
    In other words, this is the ``constant functor'' on $\mathcal{D}$, i.e., every object is sent to $d$, and every morphism is sent to $id_d$.
\begin{defn}[colimit]
    Given any functor $F:\mathcal{C}\to\mathcal{D}$, the colimit of $F$, denoted as $\colim(F)$ is an object in $\mathcal{D}$ endowed with a natural transformation:
    \begin{equation*}
        \varphi_F:F\Rightarrow F_{\colim(F)}
    \end{equation*}
    such that given any other object $d$ in $D$ and a natural transformation 
    \begin{equation*}
        \varphi: F\Rightarrow F_d
    \end{equation*}
    there exists a unique morphism in $\mathcal{D}$, $f:\colim(F)\to d$ making the following diagram commute: for any $X,Y,g$:
    \[\begin{tikzcd}
        {F(X)} && {F(Y)} \\
        & {\text{colim}(F)} \\
        & d
        \arrow["{F(g)}", from=1-1, to=1-3]
        \arrow["{\varphi_F}", from=1-1, to=2-2]
        \arrow["\varphi"', curve={height=12pt}, from=1-1, to=3-2]
        \arrow["{\varphi_F}"', from=1-3, to=2-2]
        \arrow["\varphi", curve={height=-12pt}, from=1-3, to=3-2]
        \arrow["{\textcolor{red}{f}}", from=2-2, to=3-2]
    \end{tikzcd}\]
\end{defn}
Next we prove some facts about colimits and give an example, where $\colim(F)$ exists.
\begin{prop}
    If $\colim F$ exists, then $\colim F$ is unique up to isomorphisms.
\end{prop}
\begin{proof}
    Let $\colim(F), \colim(F)'$ be two colimits that satisfy the criteria. They are both objects in $\mathcal{D}$, then we get a morphism $f:\colim(F)\to\colim(F)'$, and likewise $g:\colim(F)\to\colim(G)'$, then
    \begin{equation*}
        f\circ g:\colim(F)'\to\colim(F)'
    \end{equation*} 
    is the only morphism, and is the identity morphism. Similarly for $g\circ f$.
\end{proof}
Next we demonstrate a fact via an example.
\begin{thm}
    Let $\mathcal{C}$ be a category where $Ob(\mathcal{C}), mor(X,Y)$ are all sets. Let $F: \mathcal{C}\to\text{Top}$ be any functor, then $\colim(F)$ exists.
\end{thm}
\begin{proof}
    Define $\colim(F):=\bigsqcup_{c}F(c)/\sim$, where $\sim$ is induced by the equivalence relation given by 
    \begin{equation*}
        y\sim F(f)y
    \end{equation*}
    where $y\in F(C_1), f:C_1\to C_2, F(f)x\in F(C_2)$. The natural transformation we endow on $F$ as $\varphi_F:F\Rightarrow F_{\colim(F)}$:
    \begin{equation*}
        \varphi_F: F(C)\mapsto \bigsqcup_{C\in Ob(C)}F(C)/\sim
    \end{equation*}
\end{proof}
\begin{prob}
    \textbf{HW(Q7):} Show that $\colim(F), \varphi_F$ is indeed a colimit. 
\end{prob}
We note that colimits also exist (the same argument goes through) if we replace $\text{Top}$ with groups, sets, but with slightly different constructions, replacing disjoint unions with products, etc.

\begin{defn}[limit]
    Given a functor $F: \mathcal{C}\to\mathcal{D}$, the limit of $F$, denoted as $\lim(F)$ is an object of $\mathcal{D}$, endowed with a natural transformation:
    \begin{equation*}
        \varphi_F: F_{\lim(F)}\Rightarrow F
    \end{equation*}
    such that given any other object $d\in Ob(\mathcal{D})$ and a natural transformation 
    \begin{equation*}
        \varphi: F_d\to F
    \end{equation*}
    there exists a unique $f: \lim F\to d$ such that the following diagram commutes:
    \[\begin{tikzcd}
        & {\lim F} \\
        & d \\
        {F(X)} && {F(Y)}
        \arrow["{\textcolor{red}{f}}", from=1-2, to=2-2]
        \arrow["{\varphi_F}"', curve={height=12pt}, from=1-2, to=3-1]
        \arrow["{\varphi_F}"', curve={height=-12pt}, from=1-2, to=3-3]
        \arrow["\varphi"', from=2-2, to=3-1]
        \arrow["\varphi", from=2-2, to=3-3]
        \arrow["{F(g)}"', from=3-1, to=3-3]
    \end{tikzcd}\]
\end{defn}
Just like colimits, limits are unique up to isomorphisms. 
\begin{prob}
    \textbf{HW(Q8):} Given $F:\mathcal{C}\to\mathcal{D}$, consider $F^{op}:\mathcal{C}^{op}\to\mathcal{D}^{op}$, then 
    \begin{equation*}
        \lim F=\colim F^{op}
    \end{equation*}
\end{prob}
The above problem is interpretation of diagrams and essentially we just reverse all the maps.

\section{Lecture 3 9/4}
Today we define (co)chain complexes: let $R$ be a commutative ring, let $Mod_R$ denote the category of $R$-modules and $R$-module maps.
\begin{defn}[chain complex]
    A chain complex of $R$-modules is a collection of $R$-modules and $R$-modules maps 
    \begin{equation*}
        \dots\to M_{i+1}\xrightarrow{\partial_{i+1}}M_i\xrightarrow{\partial_i}M_{i-1}\xrightarrow{\partial_{i-1}}\dots
    \end{equation*}
    such that $\partial_i\circ\partial_{i+1}=0$ for all $i$. In other words, the image of previous map is contained in the kernal of the subsequent map. In short, we have 
    \begin{equation*}
        \partial^2=0
    \end{equation*}
    We will denote a chain complex by $\{M.; \partial.^M\}$.
\end{defn}
Next we introduce morphisms between chain complexes.
\begin{defn}[morphism between complexes]
    Let $\{M.;\partial.^M\}, \{N.;\partial.^N\}$, a morphism $\{f.\}$ between chain complexes is a ``ladder'' such that the following commutes:
    \[\begin{tikzcd}
        \dots & {M_{i+1}} & {M_i} & {M_{i-1}} & \dots \\
        \dots & {N_{i+1}} & {N_i} & {N_{i-1}} & \dots
        \arrow[from=1-1, to=1-2]
        \arrow["{\partial_{i+1}^M}", from=1-2, to=1-3]
        \arrow["{\partial_i^M}", from=1-3, to=1-4]
        \arrow[from=1-4, to=1-5]
        \arrow[from=2-1, to=2-2]
        \arrow["{\partial_{i+1}^N}", from=2-2, to=2-3]
        \arrow["{\partial_{i}^N}", from=2-3, to=2-4]
        \arrow[from=2-4, to=2-5]
    \end{tikzcd}\]
    
\end{defn}




