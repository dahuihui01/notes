\chapter{Prep work}
We will start from the beginning and take baby steps. It's going to be okay.


An algebra is a vector space (with addition and scalar multiplication, usually over $\R, \C$), with an extra multiplication operation such that it is associative, and distributive. Then a normed algebra is an algebra with a sub-multiplicative norm, such that for all $a,b\in\mathcal{A}$, we have
\begin{equation*}
    \|ab\|\leq\|a\|\|b\|
\end{equation*}
\begin{comment}
    We don't know how to multiply two vectors if we are just given a vector space. Hence giving it a norm gives us the ability to multiply. Note such multiplication is only sub-multiplicative.
\end{comment}

A Banach algebra is a normed algebra that is complete under the metric induced by the norm. And we can form a Banach algebra by starting with a normed algebra and form its completion and by uniform continuity of addition and multiplicatoin extend to the completion of the algebra to form a Banach algebra.

We will begin with some important examples of Banach algebras.
Let $X$ be a compact topological space, and let $C(X)$ be the space of continuous functions, equip it with $\|\cdot\|_{L^\infty}$ norm, then $(C(X), \|\cdot\|_{L^\infty})$ is a Banach algebra. Similarly, if $X$ is only locally compact, then $C_b(X)$, the space of bounded continuous functions under the $\|\cdot\|_{L^\infty}$ norm is also a Banach algebra.
\subsection{Some Banach algebra examples}

Another important example is that let $X$ be a Banach space, and the space of all bounded/continuous operators on $X$, denoted by $\mathcal{B}(X)$ is a Banach algebra with the operator norm. Any closed subalgebra of $B(X)$ is also Banach.

If $X$ is a Hilbert space, then we also have the operation of taking adjoints, namely $\|T\|=\|T^*\|$.



\begin{definition}
    A $C^*$ algebra is a closed subalgebra of the space of bounded (equivalently) functions defined on a Hilbert space, $\mathcal{B}(\mathcal{H})$. 
\end{definition}
\begin{remark}
    The space of continuous/bdd operators on a Hilbert space, under the operator norm, then closed under the norm topology and taking adjoints of the operators. On wikipedia, C* algebra is defined to be a Banach algebra equipped with an involution that acts like a adjoint.
\end{remark}


One of the goals of this course is to develop the following theorem.
\begin{theorem}
    Let $\mathcal{A}$ be a commutative $C^*$-algebra of $\mathcal{B}(\mathcal{H})$, then $\mathcal{A}$ is isometrically and *-algebraically isomorphic to some $C(X)$, where $X$ is some locally compact space.
\end{theorem}


\begin{proposition}
    Multiplication is continuous in Banach algebras.
\end{proposition}
\begin{proof}
    Multiplication $\cdot: \mathcal{A}\times\mathcal{A}\to\mathcal{A}$, hence if we have $x_n, y_n$ such that $x_n\to x, y_n\to y$, then we have
    \begin{equation*}
        \|x_ny_n-xy\|\leq\|x_n-x\|\|y_n\|+\|x\|\|y_n-y\|<\epsilon
    \end{equation*}
    Hence multiplication is continuous.
\end{proof}

\begin{definition}[Unital Banach algebra and invertibility] 
    A Banach algebra (let's repeat, a complete vector space with addition, scalar multiplicatin, and multiplication such that the norm is sub-multiplicative) is called unital if there exists a multiplicative inverse.

    An element $a\in\mathcal{A}$ is called invertible if there exists an element $a^{-1}\in\mathcal{A}$ such that
    \begin{equation*}
        aa^{-1}=a^{-1}a=e
    \end{equation*}
\end{definition}
Regarding invertibility, we can determine whether an element is invertible by knowing a related element's norm.
\begin{proposition}
    Let $\mathcal{A}$ be a unital Banach algebra, and if $\|a\|<1$, then $(1-a)$ is invertible.
\end{proposition}
\begin{proof}
We would like to use the fact that every Cauchy sequence converges. Define
\begin{equation*}
    (1-a)^{-1}=\sum_{n=0}^\infty a^n
\end{equation*}
where $a^0=1$ by definition. We first show that this geometric series converges to an element in $\mathcal{A}$, and we will show that the quantity defined above is indeed the inverse of $(1-a)$.

Note that we define the partial sum $S_N=\sum_{n=0}^Na^n$, then
\begin{equation*}
    \|S_N-S_M\|\leq\sum_{M+1}^N\|a\|^n<\epsilon
\end{equation*}
Hence $\{S_N\}$ is a cauchy sequence, hence converges to some element which we denoted as $(1-a)^{-1}\in\mathcal{A}$. Now
\begin{equation*}
    (1-a)\cdot (1-a)^{-1}=(1-a)\cdot\lim_{N\to\infty}S_N=(1-a)\cdot\frac{1}{1-a}=1
\end{equation*}
Likewise for the other side. Notice our $(1-a)^{-1}$ is a defined quantity, while $\frac{1}{1-a}$ is the sum of geometric series.
\end{proof}
\qed

\begin{corollary}
    Let $\mathcal{A}$ be a unital Banach algebra, then if $\|(1-a)\|<1$, then we have, $a$ is invertible.
\end{corollary}

The implication of this corollary is interesting.
\begin{corollary} 
    The open ball of radius 1 around the identity element $1_\mathcal{A}$ consists of invertible elements.
\begin{equation*}
    \|1-a\|<1 \Rightarrow a\in B_1(1_\mathcal{A})
\end{equation*}
And we know $a$ is invertible.
\end{corollary}

\begin{proposition}
    The set of invertible elements of a unital Banach algebra is an open subset.
\end{proposition}
\begin{proof}
    We use the fact that $B_1(1_\mathcal{A})$ is an open set. Note that for any invertible element $d$, we define the map, for all $a\in\mathcal{A}$,
    \begin{equation*}
        L_d(a)=da
    \end{equation*}
    We observe this map is continuous, and by $d$ be invertible, the inverse is also continuous, hence a homeomorphism. Bijectivity follows from $da=db\Rightarrow a=b$, and for every $c\in\mathcal{A}$, we can find $a=d^{-1}c$ such that $L_d(a)=c$.

    Hence for every $d$ invertible, we have $d\cdot O$ an open ball of invertible elements, and taking all union of these open balls give us the set of invertible elements, which is an open set.
\end{proof}
\qed

\begin{proposition}
    For $f\in C(X)$, we have $\alpha$ is in the range of $f$ if and only if $(f-1\cdot\alpha)$ is invertible.
\end{proposition}
\begin{proof}
    Refer to the lecture notes. In function spaces, the word \textbf{invertible} means having trivial kernel, i.e. $f(x)=0$ implies $x=0$.
\end{proof}
\qed


\subsection{Algebra homomorphisms on $C(X)$}
\begin{definition}[Algebra homomorphism]
    An algebra homomorphism is a homomorphism between two algebras. For example, consider $X$ a compact space, and $C(X)$ the space of continuous functions, hence if we define the evalutation map as follows:
    \begin{equation*}
        \varphi_x(f)=f(x)
    \end{equation*}
    This is an algebra homomorphism between $C(X)$ and $(\C)$. Namely, the homomorphism property is justified as: (under both addition and multiplication)
    \begin{equation*}
        \varphi_x(f+g)=f+g(x)=f(x)+g(x)=\varphi_x(f)+\varphi_x(g)
    \end{equation*}
    \begin{equation*}
        \varphi_x(fg)=(fg)(x)=f(x)g(x)=\varphi_x(f)\varphi_x(g)
    \end{equation*}
    And of course, same thing follows for scalar multiplication. 
\end{definition}
\begin{remark}
    We need to check all three conditions to make sure such $\varphi$ preserves the structures between the algebras.
\end{remark}

An algebra homomorphism is called unital if if maps the (multiplicative identity) unity to unity. In the above example, a unital homomorphism would be $\varphi(1)=1$, where the left 1 is the constant 1 function, and the right 1 is the number.



Now we will introduce the proposition that every multiplicative linear functional on $C(X)$. Note we can use algebra homomorphism and multiplicative linear functional synonomously on $C(X)$, hence they entail the same information.
\begin{proposition}
    Let $\varphi$ be a multiplicative linear functional on $C(X)$, i.e. a nontrivial algebra homomorphism, then $\varphi(f)=f(x_0)$ for some $x_0\in X$. In other words, a multiplicative linear functional always takes this form.
\end{proposition}
\begin{proof}
    It suffices to show the following lemma: 
    \begin{lemma}
        There exists $x_0$ such that if $\varphi(f)=0$, then we have $f(x_0)=0$.
    \end{lemma}
    We will first show how the lemma implies $\varphi(f)=f(x_0)$. Consider the function $f-\varphi(f)\cdot 1$, then we know
    \begin{equation*}
        \varphi(f-\varphi(f)\cdot 1)=0
    \end{equation*}
    Then there exists $x_0$ such that $f(x_0)-\varphi(f)=0$, this gives $\varphi(f)=f(x_0)$.  

    Now we prove the lemma.
    \begin{proof}
        Our claim is that there exists $x_0$ such that if $\varphi(f)=0$, then we have $f(x_0)=0$. Assume the contrary, which states for all $x$, there exists an $f_x$ such that $\varphi(f_x)=0$, but $f(x)\neq 0$. We define a nonnegative function $g_x=f_x\overline{f_x}$. And by multiplicativity, we have $\varphi(g_x)=0$. We now note that because $g$ is continuous, in a small nbd of $x$, denoted by $O_x$, we have $g(y)>0$ for all $y\in O_x$.
        
        Now using compactness, we can write $X$ as a finite union of small neighborhoods $X=\bigcup_{j=1}^nO_{x_j}$, and define
        \begin{equation*}
            g=g_{x_1}+...+g_{x_n}
        \end{equation*}
        Then for each $y\in X$, $y\in O_{x_j}$ for some $j$, hence $g(y)>0$ for all $y\in X$. This implies that $g$ is invertible hence we have
        \begin{equation*}
            \varphi(g\cdot 1/g)=1
        \end{equation*}
        This contradicts with the fact that $\varphi(g)=0$. And we are done.
    \end{proof}
\end{proof}
\qed

Hence we have the following corollary.
\begin{corollary}
    Let $X$ be compact, and $C(X)$ the space of continuous functions, then $\varphi$ is a multiplicative linear functional (i.e. a algebra homomorphism with $\C$) if and only if it is a point evaluation.
\end{corollary}

\begin{definition}[$\widehat{\mathcal{A}}$]
    Given a unital commutative (or Banach) algebra, for example, $C(X)$ with $\|\cdot\|_{L^\infty}$, we define the set of unital homomorphisms, i.e., nonzero unital multiplicative linear functionals on $\mathcal{A}$ as $\widehat{\mathcal{A}}$.
\end{definition}
\begin{proposition}
    If $\mathcal{A}$ is a unital algebra, then for $\varphi\in\widehat{\mathcal{A}}$, we have $\|\varphi\|=1$
\end{proposition}
\begin{proof}
    We have
    \begin{equation*}
        \|\varphi\|=\sup\{|\varphi(f)|:\|f\|_{L^\infty}=1 \}
    \end{equation*}
    Because $|\varphi(f)|=|f(x_0)|$ for some $x_0$, we always have $\|\varphi\|\leq 1$, but with the unity, we have $|\varphi(e)|=1$, and taking the sup we have $\|\varphi\|=1$.
\end{proof}
\qed

\subsection{Spectrum}
We now define the spectrum of an element in a Banach algebra.
\begin{definition}[spectrum]
    Let $\mathcal{A}$ be a Banach algebra, fix $a\in\mathcal{A}$, we define the following set to be the spectrum of $a$, denoted by $\sigma(a)$.
    \begin{equation*}
        \sigma(a)=\{\lambda\in\mathbb{F}:a-\lambda\cdot 1_\mathcal{A} \text{ is not invertible } \}
    \end{equation*}
\end{definition}

We have a bound on the size of $\lambda$ given $\|a\|$.
\begin{proposition}
    For $\lambda\in\sigma(a)$, we have
    \begin{equation*}
        |\lambda|\leq\|a\|
    \end{equation*}
\end{proposition}
\begin{proof}
    Assume the contrary, we have $|\lambda|>\|a\|$, then $a/\lambda$ has norm $\|a/\lambda\|<1$. Thus, $(1-a/\lambda)$ is invertible.
    \begin{equation*}
        a-\lambda\cdot 1=-\lambda(1-a/\lambda)
    \end{equation*}
    Because the product of two invertible elements is again, invertible, we get that $\lambda\not\in\sigma(a)$. Hence a contradiction.
\end{proof}
\qed

\begin{proposition}
    Let $\mathcal{A}$ be a unital Banach algebra, and let $\varphi\in\widehat{\mathcal{A}}$, then we have
    \begin{equation*}
        \varphi(a)\in\sigma(a)
    \end{equation*}
\end{proposition}
\begin{proof}
    It suffices to show that $a-\varphi(a)\cdot 1$ is not invertible. Assuming that it is, denote its inverse by $(a-\varphi(a))^{-1}$, then
    \begin{equation*}
        \varphi\left((a-\varphi(a)1)\frac{1}{a-\varphi(a)}\right)=1
    \end{equation*}
    However, $\varphi(a-\varphi(a)\cdot 1)=0$. Hence a contradiction.
\end{proof}
\qed

\begin{remark}
    To prove an element $a\in\mathcal{A}$ is not invertible, it suffices to prove $\varphi(a)=0$.
\end{remark}

\begin{corollary}
    For the above, $|\varphi(a)|\leq\|a\|$, and again, $\|\varphi\|=1$.
\end{corollary}
\begin{remark}
    This is to say, every unital homomorphism  $\varphi\in\mathcal{A}$ is continuous.
\end{remark}

We now show that the spectrum of an element is always closed.
\begin{proposition}
    Let $a\in\mathcal{A}$, then $\sigma(a)$ is closed.
\end{proposition}
\begin{proof}
    We define a map $\phi:\mathbb{F}\to\mathcal{A}$ as 
    \begin{equation*}
        \phi(\lambda)=a-\lambda\cdot 1
    \end{equation*}
    The map is continuous, and we notice that the $\sigma(a)$ is the complement of the preimage of invertible elements under $\phi$, i.e.
    \begin{equation*}
        \sigma(a)=(\phi^{-1}(\text{ invertible }))^c
    \end{equation*}
    Using the fact that the set of invertible elements is open, we get $\sigma(a)$ is closed.
\end{proof}
\qed

\subsection{Weak-* topology}

We now do some topology. Fix $\mathcal{A}$, Recall the weak-* topology is defined on $\mathcal{A}'$ and it is the weakest topology such that the map $\psi\in\mathcal{A}'$,
\begin{equation*}
    \psi\mapsto \psi(a) \text{ continuous }
\end{equation*}

We first note that if $\varphi\in\widehat{\mathcal{A}}$, then $\|\varphi\|=1$. Hence $\widehat{\mathcal{A}}$ is a subset of the closed unit ball in $\mathcal{A}'$. Now with respect to the weak-* topology, we have some nice properties.
\begin{theorem}
    $\widehat{\mathcal{A}}$ is closed with respect to the weak-* topology.
\end{theorem}
\begin{proof}
    Let $\{\varphi_\lambda\}$ be a net that converges to some $\varphi$ in the weak-* topology, which is a linear functional, i.e. $\varphi\in\mathcal{A}'$. Weak-* convergence implies for all $a\in\mathcal{A}$, we have
    \begin{equation*}
        \varphi_\lambda(a)\to \varphi(a)
    \end{equation*}
    We show that $\varphi$ is multiplicative.
    \begin{equation*}
        \varphi(ab)=\lim\varphi_\lambda(ab)=\lim\varphi_\lambda(a)\lim\varphi_\lambda(b)=\varphi(a)\varphi(b)
    \end{equation*}
    Now  it remains to show that $\|\varphi\|=1$ to show that it is closed. It suffices to show $\varphi$ is unital.
    \begin{equation*}
        \varphi(1)=\lim\varphi_\lambda(1)=1
    \end{equation*}
    Hence $|\varphi(1)|\leq\|\varphi\|$, hence $\|\varphi\|=1$.
\end{proof}
\qed

Now we recall Alaoglu's theorem.
\begin{theorem}[Alaoglu's]
    The closed unit ball is compact in the weak-* topology.
\end{theorem}

Hence as an immediate corollary,
\begin{corollary}
    $\widehat{\mathcal{A}}$ is compact with respect to the weak-* topology.
\end{corollary}
\begin{proof}
    $\widehat{\mathcal{A}}$ is a closed subset of a compact set, hence is also compact.
\end{proof}
\qed

Let $S$ be a semiroup with unity $e$, and $l^1(S)$ with convolution is a Banach algebra, hence we denote $\mathcal{A}=l^1(S)$.
\begin{example}
    Let the positive integers including 0 be the semigroup $S$, then we have $f=\sum_{n\in S}f(n)\delta_n$.
\end{example}
Now we try to find out what $\widehat{\mathcal{A}}$ looks like. Recall $\widehat{\mathcal{A}}$ is the space of nonzero unital homomorphisms, $\varphi:S\to\mathbb{C}$. For any $a\in\mathcal{A}=l^1(S)$, we know that
\begin{equation*}
    \varphi(a)\in\sigma(a)
\end{equation*}
and we have $|\varphi(a)|\leq\|a\|$, hence we have $\|\varphi\|\leq 1$. If we view $\varphi$ as an element in $l^\infty$, then we have
\begin{equation*}
    \|\varphi\|_{l^\infty}\leq 1
\end{equation*}
Hence this is a unit disk in the space of homomorphisms from $l^1(S)$ to $\mathcal{C}$.



We now extend to the double dual of $\mathcal{A}$, which is $\mathcal{A}''$. For any $a\in\mathcal{A}$. we define 
\begin{equation*}
    \widehat{a}(\varphi)=\varphi(a)
\end{equation*}


Now we attempt to define a Banach algebra of functions on a semigroup. A semigroup is with associative product, but not necessarily an inverse.


\begin{example}
    For example, the set of natural numbers with 0, under addition is a semigroup. We will define $\mathbb{N}_{\geq 0}=S$.
\end{example}

We let $l^(\mathbb{N}_{\geq 0})$ denote the set of functions defined on $\mathbb{N}_{\geq 0}$ such that if $f\in C_c(S)$,
\begin{equation*}
    f(x)=\sum_{n\in S}f(n)\delta_n
\end{equation*}

We define $\delta_x\delta_y=\delta_{xy}$, and we thus have
\begin{equation*}
    \left(\sum_nf(n)\delta_n\right)\left(\sum_yg(y)\delta_y \right)=\sum_z\left(\sum_{xy=z}f(x)g(y) \right)\delta_z
\end{equation*}

\begin{example}
    If we consider polynomials of the form $\sum f(n)x^n$, then we note that 
    \begin{equation*}
        \left(\sum f(m)x^m \right)\left(\sum g(n)x^n \right)=\sum_p\left(\sum_{mn=p}f(m)g(n)x^p \right)
    \end{equation*}
    Hence naturally we have $\delta_m\delta_n=\delta_{mn}$, which agrees with $x^mx^n=x^{m+n}$.
\end{example}

It is also easy to check $\|f\ast g\|_{L^1}\leq\|f\|_1\|g\|_1$.



And we define $f\in l^1(S)$ if we have $\sum_{n\in S}|f(n)|<\infty$.

Then we note that $l^1(S)$ is a Banach algebra under the convolution defined as follows: let 
\begin{equation*}
    f=\sum_{n\in S}f(n)\delta_n, g=\sum_{n\in S}g(n)\delta_n
\end{equation*}

\begin{definition}[Convolution]
    We will define a convlution between two functiosn of the above form as 
    \begin{equation*}
        f\ast g(x)=\sum_{x=yz}f(y)g(z)
    \end{equation*}
\end{definition}

Then we have $\delta_e$ as our identity function, in this case $\delta_1$.
\begin{equation*}
    f\ast\delta_e(x)=\sum_{x=yz}f(y)\delta_1(z)=f(x)
\end{equation*}

Let's now discuss a specific example. Let $\mathcal{A}=l^1(S)$, and $\widehat{\mathcal{A}}$ is the set of unital homomorphisms from $\mathcal{A}\to\C$. Hence $\widehat{\mathcal{A}}\subset\mathcal{A}'$. And from previous knowledge, we know
\begin{equation*}
    \mathcal{A}'=l^\infty(S)
\end{equation*}

Hence let $\varphi\in\widehat{\mathcal{A}}$, and we view it as an element in $l^\infty(S)$, then we define a pairing between $\varphi$ and the $f\in l^1(S)$ that it acts on. We have
\begin{equation*}
    \langle f,\varphi \rangle =\varphi(f)=\sum_{x\in S}f(x)\varphi(x)
\end{equation*}

\subsection{On semigroups}
Let $S$ be a discrete commutative semigroup.
\begin{proposition}
    We have
    \begin{equation*}
        \widehat{\mathcal{A}} \text{ ``='' } Hom (S,\mathbb{D})
    \end{equation*}
    where $\D$ denotes the unit disk in the complex plane.
\end{proposition}
\begin{proof}
    In other words, a unital homomorphism acting on $l^1(S)$ can be viewed as a unital homomorphism that acts directly on the semigroup and mapping into $\D$.

    We note that $\varphi\in\widehat{\mathcal{A}}$, then $\varphi\in l^\infty(S)$, and we also have $\|\varphi\|_{l^\infty}=1$, hence $\|\varphi(s)\|\leq 1, s\in S$. For $\varphi$ being multiplicative, we have $\varphi(\delta_{xy})\varphi(\delta_x\delta_y)=\varphi(\delta_x)\varphi(\delta_y)$, hence
    \begin{equation*}
        \varphi(xy)=\varphi(x)\varphi(y)
    \end{equation*}

    We also have
    \begin{equation*}
        \varphi(x+y)=\varphi(x)+\varphi(y)
    \end{equation*}

    And
    \begin{equation*}
        \varphi(e)=1
    \end{equation*}

\end{proof}
\begin{remark}
    We are simply using the fact that every $\varphi\in\widehat{\mathcal{A}}$ can be viewed as an element of $l^\infty(S)$.
\end{remark}
\qed


\begin{proposition}
    For $S=\N$, There is a natural identification between $\widehat{l^1(S)}$ with the unit disk $\D$.
\end{proposition}
\begin{proof}
    We note that $\N$ is generated by 1, so $l^1(S)$ is generated by $\delta_1$, hence $\varphi\in \widehat{l^1(S)}$ is determined by $\varphi(\delta_1)$. Alternatively, if we view $\varphi\in l^\infty(S)$, then $\varphi$ is determined by its value on $\varphi(1)$. Let $\varphi(1)=z_0$. Note $z_0\in\D$, Then we we have, given $\varphi$ is multiplicative,
    \begin{equation*}
        \varphi(m)=z_0^m
    \end{equation*}
    Hence there is a natural identification from $\D$ to $\widehat{l^1(\mathbb{N})}$, taking an element in $z\in\D$, to a map $\varphi:\N\to\C, \varphi\in l^\infty(\mathbb{N})$, by the map
    \begin{equation*}
        z\mapsto \varphi(n)=z^n
    \end{equation*}
    The map is bijective and continuous.
\end{proof}
\qed

\begin{proposition}
    The unit disk $\D$ under the standard topology, coincides with the weak-* topology on $\D$ that is determined in the sense of $\widehat{l^1(S)}$. In other words,
    \begin{equation*}
        \D_{std}\cong \D_{weak-*}
    \end{equation*}
\end{proposition}
\begin{proof}
    We would like to show the map
    \begin{equation*}
        z\mapsto\varphi(f)=\sum_{n\in S}f(n)\varphi(n)
    \end{equation*}
    is continuous. We have noted the natural corespondence from $z\mapsto \varphi(n)=z^n$. And by definition of the pairing between $\varphi\in l^1(S), f\in l^1(S)$, we have
    \begin{equation*}
        z\mapsto \varphi(z)=z^m\mapsto \sum_{n\in S}f(n)z^n
    \end{equation*}
    The first map is continuous, and the second is also continuous, hence we have a continuous, bijective map between $\D$, which is a compact space, to $\widehat{l^1(S)}$, a Hausdorff space, hence
    \begin{equation*}
        \D_{std}\cong\D_{weak-*}
    \end{equation*}
\end{proof}
\qed

\subsection{On groups}
Let $G$ be a discrete commutative group. Everything above applies, however, we note that in this case $\varphi\in\widehat{l^1(G)}$ implies $|\varphi(x)|=1$ for all $x\in G$.
This is because $\|x\|=1, \forall x\in G$. This implies $|\varphi(x)|\leq 1$. Hence,
\begin{equation*}
    \|\varphi(e)\|=\|\varphi(x)\varphi(x^{-1})\|=1
\end{equation*}
This means $|\varphi(x)|=1, \forall x\in G$.

Previously, we had $\widehat{l^1(S)}$ ``='' $\D$, since $|\varphi(s)|\leq 1$, and now we have
\begin{proposition}
    For $G$ a commutative discrete group, we have
    \begin{equation*}
        \widehat{l^1(G)}\cong \mathbb{T}
    \end{equation*}
    where $\mathbb{T}=\{x\in\C: |x|=1\}$.
\end{proposition}

Just like $\D$, we have $\mathbb{T}$ as a compact topological group. Hence the standard topology on $\T$ coincides with the weak-* topology on $\T$, by the map $z\in\T$,
\begin{equation*}
    z\mapsto \sum_{n\in G}f(n)z^n
\end{equation*}

If we denote $z\in T$ as $z=e^{2\pi it}$,
then we would have
\begin{equation*}
    \sum_{n\in G}f(n)e^{2\pi int}
\end{equation*}

And this is the Fourier series!


\begin{definition}[Self-adjoint Algebras]
    A Banach algebra is called self-adjoint if for every $a\in\mathcal{A}$, we have $a^*\in\mathcal{A}$ as well.
\end{definition}

\begin{proposition}
    The Gelfand transformation is onto for $\mathcal{A}$ Banach algebras that are self-adjoint. It is also an isometry.
\end{proposition}

Our goal for the following few propositions is to establish the relationship between the spectral radius, Gelfand transform, and maximal ideals.

Let $\mathcal{A}$ be a commutative Banach algebra.
\begin{proposition}
    There is a natural correspondance between the multiplication functionals $\varphi$ on $\mathcal{A}$ and the set of maximal ideals in $\mathcal{A}$. 

    Namely, for every maximal ideal $\mathcal{M}$ in $\mathcal{A}$, we can find a $\varphi\in\widehat{\mathcal{A}}$ such that $\ker(\varphi)=\mathcal{M}$.
\end{proposition}
The proof uses algebra, and we did it in class, so we do not illustrate here. The important thing is the following result.

\begin{corollary}
    $a\in\mathcal{A}$ is invertible if and only if $\widehat{a}$ is invertible, where $\widehat{a}=\Gamma(a)$ is the Gelfand transform.
\end{corollary}
\begin{proof}
    We know if $a$ is invertible, then 
    \begin{equation*}
        \Gamma(aa^{-1})=\Gamma(a)\Gamma(a^{-1})=1
    \end{equation*}
    Hence $\Gamma(a^{-1})$ is the inverse of $\Gamma(a)=\widehat{a}$, hence is invertible.

    Now we want to show if $\widehat{a}$ is invertible, then $a$ is invertible. Suppose $a$ is not invertible, we show $\widehat{a}$ is not invertible. In other words, there exists $\varphi$ such that 
    \begin{equation*}
        \widehat{a}(\varphi)=\varphi(a)=0
    \end{equation*}
    Using the previous proposition, we notice that the set
    \begin{equation*}
        \{ab: b\in\mathcal{A}\} \text{ is a proper ideal of } \mathcal{A}
    \end{equation*}
    This is due to $a$ be not invertible, hence does not contain $1_\mathcal{A}$. And every proper ideal is contained in some maximal ideal $\mathcal{M}$, hence there exists $\varphi\in\widehat{\mathcal{A}}$ such that $\varphi(a)=0$, and we are therefore done.
\end{proof}
\qed

Now we connect the spectral radius with the Gelfand transform. Recall the definition of the spectral radius.
\begin{definition}[spectral radius]
    Let $a\in\mathcal{A}$, then we define
    \begin{equation*}
        r(a)=\sup\{|\lambda|: \lambda\in\sigma(a)\}
    \end{equation*}
\end{definition}
We now have the following claim.
\begin{proposition}
    We have
    \begin{equation*}
        r(a)=\|\widehat{a}\|_\infty
    \end{equation*}
\end{proposition}
\begin{proof}
    We already know that for every $a\in\mathcal{A}$, $\widehat{a}(\varphi)=\varphi(a)\in\sigma(a)$, hence $\|\widehat{a}\|_\infty\leq r(a)$.
    \begin{lemma}
        For $a\in\mathcal{A}$, we have
        \begin{equation*}
            \sigma(a)\subset Range(\widehat{a})
        \end{equation*}
    \end{lemma}
    \begin{proof}
        Suppose $\lambda$ is not in the range of $\widehat{a}$, then $\widehat{a}-\lambda(\varphi)\neq 0$ for all $\varphi$, hence
        \begin{equation*}
            \widehat{a}-\lambda \text{ is invertible} \Rightarrow a-\lambda \text{ is invertible }
        \end{equation*}
        Hence $\lambda\not\in\sigma(a)$. Hence this implies $\lambda\in\sigma(a)$ implies $\lambda=\varphi(a)$ for some $a$.
    \end{proof}
    \qed

    Hence $r(a)\leq\|\widehat{a}\|_\infty$. Hence $r(a)=\|\widehat{a}\|$.
\end{proof}
\qed

In class we saw if $\|a^2\|=\|a\|^2$, then 
\begin{equation*}
    r(a)=\|a\|
\end{equation*}
Now we connect this with the Gelfand transform.
\begin{proposition}
    The Gelfand transform is an isometry i.e. $\|\widehat{a}\|=\|a\|$ if and only if 
    \begin{equation*}
        \|a^2\|=\|a\|^2
    \end{equation*}
\end{proposition}
\begin{proof}
    We have $\|\widehat{a}\|=r(a)$, and by the previous remark, we already have one direction. Now we want to show if $r(a)=\|a\|$, then $\|a^2\|=\|a\|^2$.
    \begin{lemma}[Spectral mapping theorem]
        For $a\in\mathcal{A}$, we have
        \begin{equation*}
            \varphi(\sigma(a))=\sigma(\varphi(a))
        \end{equation*}
    \end{lemma}
    \textcolor{red}{what}
    Hence we have $r(a^2)=(r(a))^2$, then we have
    \begin{equation*}
        \|a^2\|=r(a^2)=(r(a))^2=\|a\|^2
    \end{equation*}
\end{proof}
\qed

Now we enter the realm of Hilbert spaces.

\begin{theorem}
    For $T\in\alb(\mathcal{H})$ if $\langle T\xi, \xi\rangle=0$ for all $\xi\in\mathcal{H}$, then we have $T=0$
\end{theorem}
\begin{remark}
    This is proved by polarization.
\end{remark}

\begin{proposition}
    By the same reasoning, if $\langle T\xi, \xi\rangle$ is real for all $\xi$, then $T=T^*$.
\end{proposition}
\begin{proof}
    \begin{equation*}
        \langle T\xi, \xi\rangle =\langle \xi, T^*\xi\rangle=\langle T^*\xi, \xi\rangle
    \end{equation*}
    By the previous theorem, we know $T=T^*$.
\end{proof}

\begin{proposition}
    If we have $\|T\xi\|\geq a\|\xi\|$, and similarly $\|T^*\xi\|\geq b\|\xi\|$, then we have $T$ is invertible.
\end{proposition}
\begin{proof}
    For $T\xi=0$, we have $\xi=0$, hence $T$ is injective. And similarly, $T^*$ is injective, and we have
    \begin{equation*}
        \ker T^*=(Range(T))^\perp=\{0\}
    \end{equation*}
    Hence we have $Range(T)$ is dense in $\mathcal{H}$. Thus, by $T$ is injective, we can define $T^{-1}$ on $Range(T)$. It now suffices to show that $T^{-1}$ is bounded on $Range(T)$, then it will extend. Let $\xi\in Range(T)$, we have
    \begin{equation*}
        \|\xi\|=\|TT^{-1}\xi\|\geq a\|T^{-1}\xi\|
    \end{equation*}
    Hence $T^{-1}$ is bounded on a dense subset of $\mathcal{H}$.
\end{proof}

We recall both big and small Gelfand-Naimark theorems.
\begin{theorem}[Little Gelfand-Naimark theorem]
    Let $\mathcal{A}$ be a unital commutative $C^*$-algebra, then it is isomorphic to $C(\widehat{\mathcal{A}})$, via the Gelfand transform $a\mapsto\widehat{a}$.
\end{theorem}

\begin{theorem}[Big Gelfand-Naimark theorem]
    Let $\mathcal{A}$ be a commutative $^*$-algebra, then there exists a *-representation $\pi$ of $\mathcal{A}$ on such that 
    \begin{equation*}
        \mathcal{A}\cong \{\pi(a): a\in\mathcal{A}\}
    \end{equation*}
    This is to say every $C^*$-algebra is isomorphic to another $C^*$-algebra of bounded operators on a Hilbert space.
\end{theorem}

\subsection{GNS construction}
Let $\mathcal{A}$ be a unital commutative $C^*$-algebra, and we next define a correspondance between the cyclic *-representations of $\mathcal{A}$ and the states on $\mathcal{A}$.
\begin{definition}[*-representation]
    Let $\mathcal{A}$ be a unital commutative $C^*$-algebra, and let $\pi$ be a non-degenerate representation on $\mathcal{A}$, such that $\pi$ takes involution on $\mathcal{A}$ to involution of operators.
\end{definition}
\begin{note}
    For 
\end{note}

\begin{definition}[Cyclic vector]
    For $\xi\in\mathcal{H}$, and $\xi\neq 0$, it is called a cyclic vector if the set
    \begin{equation*}
        \{\pi(a)\xi: a\in\mathcal{A}\}
    \end{equation*}
    is norm dense in $\mathcal{H}$. Then we call $\pi$ a cyclic representation.
\end{definition}
Not all Hilbert spaces have a cyclic representation of course, but they can be dissected into orthogonal subspaces that are $\pi$-cyclic.
\begin{proposition}
    Given a *-representation $\pi$ of $\mathcal{A}$ on $\mathcal{H}$, and take $\xi\neq 0, \xi\in\mathcal{H}$, and $K=\overline{\{\pi(a)\xi: a\in\mathcal{A}\}}$, then $K$ is called the $\pi$-cyclic subspace of $\mathcal{H}$. For any $\mathcal{H}$, there exists a family of orthogonal $\pi$-cyclic subspaces such that 
    \begin{equation*}
        \mathcal{H}=\bigoplus_\lambda K_\lambda
    \end{equation*} 
\end{proposition}
\begin{remark}
    Any nonzero vector in an irreducible representation is cyclic.
\end{remark}

\begin{definition}[state]
    Let $\varphi$ be a linear functional on $\mathcal{A}$, we say $\varphi$ is \textbf{positive} if for all $a\in\mathcal{A}$, we have
    \begin{equation*}
        \varphi(aa^*)\geq 0
    \end{equation*}
    Moreover, if $\varphi$ has norm 1, then we call them states.
\end{definition}

\begin{proposition}
    Let $\xi$ be a cyclic vector, and $\pi$ a *-representation (or any $\xi\neq 0$), and the map $a\mapsto \langle \pi(a)\xi, \xi\rangle $ is a state on $\mathcal{A}$.
\end{proposition}
\begin{proof}
    Denote this map as $\pi(a)=\langle \pi(a)\xi, \xi\rangle$.
    \begin{equation*}
        \varphi(a^*a)=\langle \pi(a^*a)\xi, \xi\rangle =\langle \pi(a)\xi, \pi(a)\xi\rangle \geq 0
    \end{equation*}
\end{proof}
\qed


Is it true that $\varphi(a)=\overline{\varphi(a^*)}$ for $\varphi$ a positive linear functional on $\mathcal{A}$.
If $\varphi(a)=\varphi(a^*)$, but if $\lambda\in\C$, we have $\varphi(\lambda a)=\lambda (\varphi(a))=\lambda\varphi(a^*)$, but $\varphi((\lambda a)^*)=\overline{\lambda}\varphi(a^*)$. Hence not equal.

It then suffices to show that $|\varphi(a)|=|\varphi(a^*)$.

\begin{definition}[Unitary equivalent representations]
    There exists two representations $\pi_1, \pi_2$ on $\mathcal{H}$, if there exists a unitary operator $U: \mathcal{H}\to\mathcal{H}$ such that for all $\xi\in\mathcal{H}$,
    \begin{equation*}
        U\pi_1(a)\xi=\pi_2(a)U\xi
    \end{equation*}
    i.e. the following diagram commutes:
    \begin{tikzcd}
        \mathcal{H}\arrow{d}{U}\arrow{r}{\pi_1} & \mathcal{H}\arrow{d}{U}\\
        \mathcal{H}\arrow{r}{\pi_2} &\mathcal{H}
    \end{tikzcd}
\end{definition}

