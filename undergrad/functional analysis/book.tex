This is for myself.

\section{Chapter 1: Hilbert Spaces}
-orthogonality

-riesz representation

We will state the Riesz representation theorem here, if $T:\mathcal{H}\to\mathbb{F}$ is a bounded linear functional, then there is a unique vector $h_0$ in $\mathcal{H}$ such that
\begin{equation*}
    T(h)=\langle h, h_0\rangle, \forall h\in\mathcal{H}
\end{equation*}
Moreover, we have
\begin{equation*}
    \|T\|=\|h_0\|
\end{equation*}


-orthonormal sets of vectors and bases
-isomorphic hilbert spaces and the fourier transform for the circle

We define an isomorphism between two Hilbert spaces as follows:
\begin{definition}
    Let $\mathcal{H}, \mathcal{K}$ be Hilbert spaces, then an isomorophism $U$ is a surjective isometry, i.e.
    \begin{equation*}
        \langle Uh, Ug\rangle=\langle h, g\rangle
    \end{equation*}
    Here we use $U$ to highlight such isomorphism is also called the unitary operator.
\end{definition}
We conclude this section by a few theorems regarding the Fourier transform.
\begin{proposition}
    $\{e^{2\pi inx}\}_{n\in\mathbb{Z}}$ is a basis for $L_\mathbb{C}^2[0,2\pi]$.
\end{proposition}
And we also have
\begin{theorem}
    The Fourier transform is a linear isometry from $L_\mathbb{C}^2[0,2\pi]$ to $l^2(\mathbb{Z})$.
\end{theorem}

Then we talk a little bit about the direct sum of hilbert spaces

\section{Chapter 2: Operators on Hilbert Space}
-We start with some properties of operators on the Hilbert space, such as the Schur's test.

\begin{proposition}[Schur's test]
    If we have a kernel $k(x,y)$, and such that
    \begin{equation*}
        \int|k(x,y)|d\mu(x)\leq c_1
    \end{equation*}
    \begin{equation*}
        \int|k(x,y)|d\mu(y)\leq c_2
    \end{equation*}
    Then if we define an operator $K:L^2(\mu)\to L^2(\mu)$ as
    \begin{equation*}
        T(f)(x)=\int k(x,y)f(y)d\mu(y)
    \end{equation*}
    Then we have $L^2$ boundedness of $T$, and that
    \begin{equation*}
        \|T\|\leq(c_1c_2)^{1/2}
    \end{equation*}
\end{proposition}


