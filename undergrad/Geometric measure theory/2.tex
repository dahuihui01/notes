\chapter{Work}
Maybe recover Wolff's bound on spherical measure using decoupling. 

duality, let $r\to\infty$, then $\mu$  could be large.
$\hat{\mu}\in L^2$, and extend it,  $\mu$ weighted restriction, just integrate with respect to $\mu$. Our eventual goal in $L^1$. Weighted restriction is decay of spherical mean, by duality.

-Weighted restriction in high dimension.

-take a convolution operator, convolve with a straight line, this turns into pointwise multiplication, and use the $L^2$ bound on this. (this can be viewed as a projection, then vary the direction of the line).

Hi, what is going on here
\chapter{Questions}

Can one think of rectifiable or unrectifiable as compressible or noncompressible?

-some exaples of rectifiable sets

-The measures of projections are big when it's a rectifiable set, and are small when it's a rectifiable.

-Unrectifiable means not rectifiable? What?


-the constant 4 in page 28

-high density (purely unrectifiable sets and the fact that their projections are quite small)


\textbf{Week 3}
-Notation issue: page 37, $\hat{\mu}\in L^2$ is well defined, but a measure $\mu\in L^2$ means it has an $L^2$ density?

-check whether Wolff's statement 9B1 is true in $\R^d$,

-check you definition of the $I_s(\mu)$ using spherical average, namely, 
\begin{equation*}
    I_s(\mu)=\int r^{s-d}(\sigma(\mu)(r))^2dr
\end{equation*}

-the cricial exponent $s-2$ in the spherical average, so bounding the spherical average by anything below is a gain.

-did not talk about how spherical averages of the decay of Fourier transform relates to the distance set.

-Fourier dimension before Theorem 6.28? But was only talking about it in Theorem 6.30.

-The discretized sum-product and projections theorems for Fourier dimension of the product set: what type of problem is this related to, and why do we care about Fourier dimension to begin with.

-can we talk about the Erdos ring conjecture.



\text{week 4}
The mechanism from lower minkowski to hausdorff always works?

-do we have difficulty bounding how tubes intersect (it was not easy in $\R^2$) in $\R^3$

-Wolff's notes uses two proofs of kakeya maximal function conjecture. And fefferman and bourgain showed that the restriction conjecture implies kakeya maximal function conjecture.
\begin{equation*}
    \|\widehat{fd\sigma}\|_{L^p}\lesssim \|f\|_{L^p}
\end{equation*}
one can show that this is equivalent to replacing the RHS with $\|f\|_{L^\infty}$.

-which direction is people mainly working towards.

-what are some difficulties using what we know about how lines intersect, guth and Katz used polynomial method to prove the joints conjecture in $\R^3$, basically saying we can't have too many joints, why can't be adapt these

-Hausdorff: just a fraction of the tube, but Minkowski is the entire tube.

-plane brush (by someone in Caltech), by Josh Zahl and Natz katz
-the $L^2$ method doesn't work
in $\R^3$, there are $2^{2j}$ times $1/2^{j}$ directions, then summing we do not get a log.

-consider the example where all tubes go through the same point, where in $\R^2$, we get $N^2$ divided by $N^2$, by in $\R^3$, it becomes $N^4$ divided by $N^3$.

-joints conjecture-multilinear conjecture; high-low method


\textbf{week 6}
\begin{enumerate}
    \item How much can heavy tubes cover
    \item train track
    \item Kevin Ren and Yuqiu Fu 
    \item Endpoint case of Kakeya
    \item Guth-Solomon-Wang well-spaced tubes
    \item Wang-Orponen-Shmerkin bootstrapping method
\end{enumerate}

-orponen's radial projection theorem, and Vinh's sum product, going from energy to sobolev norm, applying littlewood-paley, and coming back to the norm.

-Furstenburg set: the example is not regular, but what if the set is regular, can we improve on Furstenburg set.


When I talk:
\textbf{Nov.28}
-will begin with $L^2$ method
-Guth-Solomon-Wang, well-spaced tubes
-Furstenburg set in the plane



-assumed the set is AD-regular



Questions:
\begin{enumerate}
    \item a set that looks like t dimensional at low scale and s dimensional at high scale.
    \item page 11 of heavy lines
    \item The bound on sub-uniformly distributed set is not sharp, and not for AD-regular either.
    \item 
    \item next text: Mattila's book on Fourier analysis and Hausdorff dimension.


\end{enumerate}


\begin{enumerate}
    \item uniformly distributed sets and AD-regular sets; sub-uniformly distributed sets
    \item 
\end{enumerate}

