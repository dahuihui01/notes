\chapter{Singular Integrals}
\begin{definition}[Zero Average Functions on $S^{n-1}$]
    Let $\sigma$ denote the angular measure on $S^{n-1}$, then we say a function defined on $S^{n-1}$ has zero average if $y'\in S^{n-1}$
    \begin{equation*}
        \int_{S^{n-1}}\Omega(y')d\sigma(y')=0
    \end{equation*}
\end{definition}
And with this definition, we define singular integrals as the following operator.
\begin{proposition}
    For $\Omega:S^{n-1}\to\C$, if it has zero average, then
    \begin{equation*}
        \int_{\epsilon\leq|x|<1}\frac{\Omega(y')}{|y|^n}dy=0
    \end{equation*}
\end{proposition}
\begin{proof}
    We will demonstrate this using polar coordinates.
    \begin{equation*}
        \int_{\epsilon\leq|x|<1}\frac{\Omega(y')}{|y|^n}dy=\int_\epsilon^1 \frac{1}{r^n}\left(\int_{S^{n-1}}\Omega(y') d\sigma(y')\right)r^{n-1}dr=\int_{S^{n-1}}\Omega(y')d\sigma(y')\log(1/\epsilon)=0
    \end{equation*}
\end{proof}
\qed

\begin{definition}[Singular integrals]
    We call $T$ a singular integral is $T$ is an operator of the form: for $f\in\mathcal{S}$
    \begin{equation*}
        Tf(x)=\lim_{\epsilon\to 0}\int_{|y|>\epsilon}\frac{\Omega(y')}{|y|^n}f(x-y)dy
    \end{equation*}
    where $\Omega:S^{n-1}\to\C$, and $y'=\frac{y}{|y|}\in S^{n-1}$ and $\Omega\in L^1$, and with zero average, i.e.
    \begin{equation*}
            \int_{S^{n-1}}\Omega(y')d\sigma(y')=0
    \end{equation*}
\end{definition}
We would of course, want to make sense of the integral in the definition, i.e. does it converge for $f\in\mathcal{S}$.
\begin{proposition}
    $p.v. \frac{\Omega(x')}{|x|^n}$ is a tempered distribution.
\end{proposition}
\begin{proof}
    \begin{equation*}
        p.v.\frac{\Omega(x')}{|x|^n}(\phi)=\lim_{\epsilon\to 0}\int_{|x|>\epsilon}\frac{\Omega(x')}{|x|^n}\phi(x)dx=\int_{|x|<1}\frac{\Omega(x')}{|x|^n}(\phi(x)-\phi(0))+\int_{|x|>1}\frac{\Omega(x')}{|x|^n}\phi(x)dx
    \end{equation*}
    By $\phi\in\mathcal{S}$, we have that both 
    \begin{equation*}
        \frac{\phi(x)-\phi(0)}{|x|^n}\in\mathcal{S}, \frac{\phi(x)}{|x|^n}\in\mathcal{S}
    \end{equation*}
    Hence both integrals converge.
\end{proof}
\qed

\begin{corollary}[Necessary condition for singular integrals]
    If $p.v.\frac{\Omega(x')}{|x|^n}$ is a tempered distribution, then $\Omega$ has zero average.
\end{corollary}

\section{Fourier transform of the kernel}
\begin{definition}[Homogeneous functions]
    If $f$ is such that for any $x\in\R^n$ and any $\lambda>0$, we have
    \begin{equation*}
        f(\lambda x)=\lambda^af(x)
    \end{equation*}
    Then we say $f$ is of Homogeneous of degree $a$.
\end{definition}
Similarly, if we scale a test function such that
\begin{equation*}
    \phi_\lambda(x)=\lambda^{-n}\phi(\lambda^{-1}x)
\end{equation*}
And we integrate against a homogeneous of degree $a$ function $f(x)$, we have
\begin{equation*}
    \int\phi_\lambda(x)f(x)=\int \phi(x)f(\lambda x)dx=\lambda^a\int\phi(x)f(x)dx
\end{equation*}
\begin{definition}[Homogeneous distribution]
    A distribution $T$ is homogeneous of degree $a$, if for every test function $\phi$, we have
    \begin{equation*}
        T(\phi_\lambda)=\lambda^aT(\phi)
    \end{equation*}
\end{definition}
\begin{exercise}
$p.v.\frac{\Omega(x')}{|x|^n}$ is homogeneous of degree $-n$.
\begin{equation*}
    p.v.\frac{\Omega(x')}{|x|^n}(\phi_\lambda)=\lim_{\epsilon\to 0}\int_{|x|>\epsilon}\frac{\Omega(x')}{|x|^n}\phi_\lambda(x)dx=\lambda^{-n}p.v.\frac{\Omega(x')}{|x|^n}(\phi)
\end{equation*}
\end{exercise}

\begin{proposition}
    If $T$ is a tempered distribtuion, homogeneous of degree $a$, then its Fourier transform has degree $-n-a$.
\end{proposition}
\begin{proof}
    Let $\phi\in\mathcal{S}$, we have
    \begin{equation*}
        \hat{T}(\phi_\lambda)=T(\hat{\phi_\lambda})=T(\hat{\phi(\lambda\xi)})=\lambda^{-n}T(\lambda^n\phi(\lambda\xi))=\lambda^{-n}T(\phi_{\lambda^{-1}})=\lambda^{-n-a}T(\hat{\phi})=\lambda^{-n-a}\hat{T}(\phi)
    \end{equation*}
\end{proof}
\qed

\begin{theorem}
    The Fourier transform of our kernel $p.v.\frac{\Omega(x')}{|x|^n}$ is of homogeneous of degree 0, and
    \begin{equation*}
        \left(p.v.\frac{\Omega(x')}{|x|^n}\right)^{\widehat{\phantom{.}}}(\xi)=\int_{S^{n-1}}\Omega(u)\left[\log\left(\frac{1}{|u\cdot\xi'|}\right)-i\frac{\pi}{2}sgn(u\cdot\xi')\right]d\sigma(u)
    \end{equation*}
\end{theorem}



\section{Singular integrals with variable kernels}
Consider the following polynomial operator:
\begin{equation*}
    P(x,D)=\sum_{|a|=m}b_\alpha(x) D^\alpha
\end{equation*}
Note now $b_\alpha(x)$ is a variable function with $x$.

When $f\in\mathcal{S}$, we have
\begin{equation*}
    p(x,D)f(x)=\int_{\R^n}P(x,2\pi i\xi)\hat{f}(\xi)e^{2\pi ix\xi}d\xi
\end{equation*}
And let $\Gamma=(-\Delta)^{1/2}$, then we would like to find $T$ such that
\begin{equation*}
    P(x,D)f=T(\Lambda^mf)
\end{equation*}
\begin{align*}
    Tf(x)&=\int_{\R^n}\sigma(x,\xi)\hat{f}(\xi)e^{2\pi ix\cdot\xi}d\xi
\end{align*}
where
\begin{equation*}
    \sigma(x,\xi)=\frac{P(x,i\xi)}{|\xi|^m}
\end{equation*}
where it is homogeneous of degree 0, then 
\begin{equation*}
    Tf(x)=\int\sigma(x,\xi)e^{2\pi ix\cdot\xi}\int f(y)e^{-2\pi iy\cdot\xi}dyd\xi=\int_{\R^n}K(x,x-y)f(y)dy
\end{equation*}
where
\begin{equation*}
    K(x,z)=\int_{\R^n}\sigma(x,\xi)e^{2\pi iz\cdot\xi}d\xi
\end{equation*}
\begin{note}
    It looks like the inverse Fourier transform of a homogeneous function of degree 0, so we could apply the previous theories. (We won't be doing this rigorously).
\end{note}


\section{Lecture 10/27}
Today we will finish chapter 4. We will begin with a comment.
\begin{theorem}
    We have $T_m\in\mathcal{A}$ if and only if $m$ is nowhere-vanishing on $S^{n-1}$
\end{theorem}

We now continue our discussion of section 6, variable kernel.
\begin{equation*}
    K(x,z)=\int_{\R^n}\sigma(x,\xi)e^{2\pi iz\xi}d\xi
\end{equation*}
We have that $\sigma$ is homogeneous of  degree 0 in $\xi$, and 
\begin{equation*}
    \sigma(x,\cdot)\in C^\infty(S^{n-1})
\end{equation*}

By theorem 4.13, for every $x$, there exists $a(x)$ and $\Omega(x, \cdot)$ on $C^\infty(S^{n-1})$ with zero average such that
\begin{equation*}
    K(x,z)=a(x)\delta(z)+p.v.\frac{\Omega(x,z')}{|z|^n}
\end{equation*}
This motivates us to study the behavior:
\begin{equation*}
    Tf=\lim_{\epsilon\to 0}\int_{|y|>\epsilon}\frac{\Omega(x', y')}{|y|^n}f(x-y)dy
\end{equation*}
\begin{remark}
    Chapter 5 deals with the case where $\Omega$ is smooth.
\end{remark}

\begin{note}
    Convergence issues, leads us to study Hilbert transform (being bounded), and Hilbert transform naturally generalizes to higher dimensions of singular integrals, and now variable coefficients are natural in PDE's.
\end{note}
\begin{definition}[Psedodifferential operator]
    \begin{equation*}
        T_\sigma f(x)=\int\sigma(x,\xi)\hat{f}(\xi)e^{2\pi ix\cdot\xi}d\xi
    \end{equation*}
    is called a pseudo differential operator, where $\sigma$ is called a symbol. Special case: differential operator of variable coefficients.
\end{definition}
\begin{theorem}
    Let $\Omega(x,y)$ be a function that is homogeneous of degree 0 in $y$ such that
    \begin{enumerate}
        \item $\Omega(x,-y)=-\Omega(x,y)$
        \item $\Omega^*(u)=\sup_x|\Omega(x,u)|\in L^1(S^{n-1})$
    \end{enumerate}
    Then $T$ defined by 
    \begin{equation*}
         Tf=\lim_{\epsilon\to 0}\int_{|y|>\epsilon}\frac{\Omega(x', y')}{|y|^n}f(x-y)dy
    \end{equation*}
    is strong (p,p), $p<\infty$.
\end{theorem}
\begin{proof}
    Let $f\in\mathcal{S}$, then we have
    \begin{equation*}
        Tf(x)=\frac{\pi}{2}\int_{S^{n-1}}\Omega(x,u)H_uf(x)d\sigma(u)
    \end{equation*}
    Hence we have
    \begin{equation*}
        |Tf(x)|\leq\frac{\pi}{2}\int_{S^{n-1}}|\Omega^*(u)|H_uf(x)d\sigma(u)
    \end{equation*}
    And the conclusion follows from Minkowski's inequality.
    And we know that (from proposition 4.6), we have the operator
    \begin{equation*}
        T_\omega f(x)=\int_{S^{n-1}}T_uf(x)d\sigma(u)
    \end{equation*}
    is bounded on $L^p$. Hence we are done.
\end{proof}
\qed

\begin{note}
    If our second condition (2) is replaced with 
    \begin{equation*}
        \sup_x\left(\int_{S^{n-1}}|\Omega(x,u)|^pd\sigma(u)\right)^{\frac{1}{p}}=B_p<\infty
    \end{equation*}
    for $1<q<\infty$, then we have that $T$ is bounded on $L^p(\R^n)$, and $q'<p<\infty$.
\end{note}
\begin{proof}
    \begin{equation*}
        |Tf(x)|\leq\frac{\pi}{2}B_q\left(\int_{S^{n-1}}|H_uf(x)|^{q'}d\sigma(u)\right)^\frac{1}{q'}
    \end{equation*}
    Interchanging the order, we have that
    \begin{equation*}
        \int |Tf(x)|^{q'}dx\lesssim B_q^{q'}\int_{S^{n-1}}\int_{\R^n}|H_uf(x)|^{q'}dxd\sigma(u)\lesssim B_q^{q'}\|f\|_{L^{q'}}^{q'}
    \end{equation*}
    Since it holds for $q$, it also olds for $p'\in (1,q]$.
\end{proof}
\qed

\chapter{Singular integrals (II)}

\section{Calderon-Zygmund theorem}
\begin{theorem}[Calderon-Zygmund]
    Let $K\in\mathcal{S}'$ agreeing with a locally integral function on $\R^n\setminus\{0\}$ such that 
    \begin{enumerate}
        \item $|\hat{K}(\xi)\leq A$
        \item (Smoothness condition) We also have \begin{equation*}
            \int_{|x|>2|y|}|K(x-y)-K(x)|dx\leq B
        \end{equation*}
    \end{enumerate}
    Then we have 
    \begin{equation*}
        Tf=K\ast f
    \end{equation*}
    is weak $(1,1)$ and sstrong $(p,p)$, for $1<p<\infty$ with implied constants depend only on $n, p, A, B$.
\end{theorem}
\begin{proposition}[Better smoothness condition]
    (5.2) holds if we have
    \begin{equation*}
        |\nabla K(x)|\leq\frac{C}{|x|^{n+1}}
    \end{equation*}
    for all $x\neq 0$.
\end{proposition}
Hence Theorem 5.1 still holds if we replace the second condition with this one.


\section{Lecture Oct 30}
We begin talking about the theorem introduced last time, and show that $Tf$ is weak (1,1).

\begin{proof}
    We will only do a proof sketch. (5.1) condition implies that
    \begin{equation*}
        \|Tf\|_{L^2}\leq A\|f\|_{L^2}
    \end{equation*}
    Now it suffices to show that $T$ is weak (1,1), by interpolation and duality.

    Again, we will rely on Calderon-Zygmund decomposition to do this. Fix $\lambda>0$, form the CZ decomposition of $f$ at height $\lambda$, this gives that
    \begin{equation*}
        f=g+b, g\in L^2, \|g\|_{L^\infty}\lesssim \lambda
    \end{equation*}
    We will control $b$ exactly the way we did for the Hilbert transform in 1-dim. We use $L^2$ boundedness to control $Tg$. Let $b=\sum_jb_j$, and $supp(b_j)\subset Q_j$, we have
    \begin{equation*}
        \sum|Q_j|\lesssim \frac{\|f\|_{L^1}}{\lambda}
    \end{equation*}
    Let $Q_j^*=2\sqrt{n}Q_j$, on $\R^n\setminus(\cup_jQ_j^*)$,
    \begin{equation*}
        |Tb|\leq\sum_j|Tb_j|
    \end{equation*}
    And we have
    \begin{equation*}
        \int_{\R^n\setminus Q_j^*}|Tb_j(x)|dx=\int_{\R^n\setminus Q_j^*}\left|\int_{Q_j}K(x-y)b_j(y)dy\right|dx
    \end{equation*}
    Now we use the zero average, and the bound on $Q_j$
    \begin{align*}
        &=\int_{\R^n\setminus Q_j^*}\left|\int_{Q_j}(K(x-y)-K(x-c_j))b_j(y)dy\right|dx\\
        &=\int_{Q_j}|b_j(y)|dy\int_{\R^n\setminus Q_j^*}|K(x-y)-K(x-c_j)|dx\\
        &\leq B\int_{Q_j}|b_j(y)|dy\\
        &=B\|b_j\|_{L^1}
    \end{align*}
    Now we add everything together, we get

    \begin{equation*}
        \int_{\R^n(\cup_j Q_j^*)}|Tb(x)|dx\leq B\|b\|_{L^1}\lesssim B\|f\|_{L^1}
    \end{equation*}
\end{proof}
\qed


Now we ask the question whether we can relax the Homander condition, i.e. we interpret $t$ as the distance, and for $p.v. \frac{\Omega(x')}{|x|^n}$, define
\begin{equation*}
    \omega_\infty(t)=\sup_{|u_1-u_2|\leq t}|\Omega(u_1)-\Omega(u_2)|
\end{equation*}
where $u_1, u_2\in S^{n-1}$. In fact, we can replace the Homander condition with a Dini type condition. Note that $\omega_\infty$ is monotone with respect to $t$.

\begin{proposition}
    If $\Omega$ satisfies 
    \begin{equation*}
        \int_0^1\frac{\omega_\infty(t)}{t}<\infty
    \end{equation*}
    then the Homander condition holds, and $\Omega\in C^\alpha, \alpha>0$, 
\end{proposition}
\begin{proof}
    Note that we have
    \begin{equation*}
        \omega_\infty\xrightarrow{t\to 0}0
    \end{equation*}
    hence $\Omega$ is bounded.
    \begin{align*}
        |K(x-y)-K(x)|&=\left|\frac{\Omega(x-y)'}{x-y|^n}-\frac{\Omega(x')}{|x|^n}\right|\\
        &=\leq \frac{|\Omega(x-y)'-\Omega(x')}{|x-y|^n}+|\Omega(x')|\left|\frac{1}{|x-y|^n}-\frac{1}{|x|^n}\right|\\
    \end{align*}
    From the Mean value theorem, jwe get that when $|x|>2|y|$, the second term is controlled,
    \begin{equation*}
        \lesssim |\Omega(x')|\frac{|y|}{|x|^{n+1}}
    \end{equation*}
    Its integration on $\{|x|>2|y|\}$ is bounded, by some constant. Hence we drop the second term. Note that we have
    \begin{equation*}
        |(x-y)'-x'|\leq 4\frac{|y|}{|x|}
    \end{equation*}
    We have that
    \begin{align*}
        &=\int_{|x|>2|y|}\frac{\Omega(x-y)'-\Omega(x')}{|x-y|^n}dx\\
        &\lesssim \int_{|x|>2|y|}\frac{\omega_\infty\left(\frac{4|y|}{|x|}\right)}{|x|^n}dx\\
        &\lesssim \int_{2|y|}^\infty S^{-1}\omega_\infty\left(\frac{4|y|}{s}\right)ds\\
        &=\int_0^2\frac{\omega_\infty(t)}{t}dt<\infty
    \end{align*}
\end{proof}
\qed

\begin{corollary}
    If $\Omega$ is a function on $S^{n-1}$ with zero avergae, and satisfy (5.5), and then
    \begin{equation*}
        Tf(x)=p.v.\int_{\R^n}\frac{\Omega(y')}{|y|^n}f(x-y)dy
    \end{equation*}
    is strong (p,p), and $1<p<\infty$, and also weak (1,1).
\end{corollary}




\section{Lecture 11/1}
\subsection{Trucated integrals and the principal value}
We now ask the question when is $\widehat{K}\in C^\infty$.

One last comment on 5.5: $(2)$ and $(2')$ are equivalent:
\begin{equation*}
    \int_{|x|<a}|x||k(x)|dx\leq B'a
\end{equation*}
for all $a>0$.

\begin{proposition}
    Let $K\in L_{loc}^1(\R^n\setminus\{0\})$, define 
    \begin{equation*}
        K_{\epsilon, R}=K(x)\chi_{\{\xi<|x|<R\}}
    \end{equation*}
    If 
    \begin{enumerate}
        \item $\left|\int_{a<|x|<b}kdx\right|\leq A, \forall 0<a<b<\infty$
        \item $\int_{a<|x|<2a}|k|dx\leq B, \forall a>0$
        \item $\int_{|x|>2|y|}|K(x-y)-k(x)|dx\leq C. \forall y\in\R$.
    \end{enumerate}
    Then we have
    \begin{equation*}
        \|\widehat{K}_{\epsilon, R}\|_\infty\leq \text{constant}
    \end{equation*}
    And the constant is uniform in $\xi$ and $R$.
\end{proposition}

\begin{proof}
    Assume that $\epsilon<|\xi|^{-1}<R$, we have
    \begin{align*}
        \widehat{K_\epsilon, R}(\xi)&=\int_{\epsilon<|x|<R}K(x)e^{-2\pi ix\cdot\xi}dx\\
        &=\int_{\epsilon<|x|<|\xi|^{-1}}e^{-2\pi ix\cdot\xi}dx\\
        &+\int_{|\xi|^{-1}<|x|<R}K(x)e^{-2\pi ix\cdot\xi}dx
    \end{align*}
    Denote the above two integrals as I and II.
    \begin{align*}
        |I|&=\left|\int_{\epsilon<|x|<|\xi|^{-1}}K(x)dx\right|+\left|\int_{\epsilon<|x|<|\xi|^{-1}}K(x)(e^{-2\pi ix\cdot\xi}-1)dx\right|\\
        &\leq \left|\int_{\epsilon<|x|<|\xi|^{-1}}K(x)dx\right|+\int_{\epsilon<x<|\xi|^{-1}}|K(x)|\cdot 2\pi |x||\xi|dx\\
        &\leq A+2\pi|\xi|\cdot B'|\xi|^{-1}\\
        &= A+2\pi B'
    \end{align*}
    And moreover, let $z=\frac{1}{2}\frac{\xi}{|\xi|^2}e^{2\pi iz\cdot\xi}=-1$, we get
    \begin{align*}
        II&=-\int_{|\xi|^{-1}<|x-z|<R}K(x-z)e^{-2\pi ix\cdot\xi}dx\\
        &=\frac{1}{2}\int_{|\xi|^{-1}<|x|<R}K(x)e^{-2\pi ix\cdot\xi}dx-\frac{1}{2}\int_{|\xi|^{-1}<|x-z|<R}K(x-z)e^{-2\pi x\cdot\xi}dx
    \end{align*}
    Hence we have
    \begin{align*}
        |II|&\leq\int_{|\xi|^{-1}<|x|}|K(x)-K(x-z)|dx\\
        &+\int_{\frac{1}{2}|\xi|^{-1}<|x|<\frac{3}{2}|\xi|^{-1}}|K(x)|dx\\
        &+\int_{R-\frac{1}{2}|\xi|^{-1}<|x|<R+\frac{1}{2}|\xi|^{-1}}K(X)dx\\
        &\lesssim C+B
    \end{align*}
\end{proof}
\qed

\begin{corollary}
    If $K$ satisfies the hypotheses of Proposition 5.5, then we have
    \begin{equation*}
        \|K_{\epsilon, R}\ast f\|_{L^p}\lesssim \|f\|_{L^p}
    \end{equation*}
    and $|\{x\in\R^n: |K_{\epsilon, R}\ast f(x)|>\lambda\}|\lesssim\frac{\\f\|_{L^1}}{\lambda}$.
\end{corollary}
\begin{proof}
    The $p=2$ case follows from Proposition 5.5, and the second and third conditions imply the Homander condition of $K_{\epsilon, R}$.
\end{proof}
\qed

\begin{remark}
    When $K=\frac{\Omega(x')}{|x|^n}$, (1) and (2) imply that
    \begin{equation*}
        |\omega\in L^1(S^{n-1})
    \end{equation*}
    and $\Omega$ has zero average. We want to define $T$ by
    \begin{equation*}
        Tf(x)=\lim_{\epsilon\to 0, R\to\infty} K_{\epsilon, R}\ast f(x), f\in\mathcal{S}
    \end{equation*}
    And this is good for $R\to\infty$, due to (2), and we will address $\epsilon\to 0$ next time.
\end{remark}


\textcolor{red}{there is a lecture  missing}

\section{Lecture Nov 6}
We let $\Delta=\{(x,x):x\in\R^n\}$ is the diagonal of $\R^n$.

\begin{theorem}
    Let $T$ be bounded on $L^2(\R^n)$, if $K$ is a function on $\R^n\times\R^n\setminus\Delta$, such that 
    \begin{equation*}
        Tf(x)=\int_{\R^n}K(x,y)f(y)dy, x\not\in supp(f)
    \end{equation*}
    for $f\in L^2$, compactly supported and 
    \begin{equation*}
        \int_{|x-y|>2|y-z|}|K(x,y)-K(x,z)|dx\leq C
    \end{equation*}
    \begin{equation*}
        \int_{|x-y|>2|x-w|}|K(x,y)-K(x,y)|dy\leq C
    \end{equation*}
    Then $T$ is weak (1,1) and strong (p,p), $1<p<\infty$.
\end{theorem}
\begin{proof}
    Entirely parallel to the proof of 5.1.
\end{proof}

\begin{definition}
    We say a kernel $K:\R^n\times\R^n\setminus\Delta$ is a standard kernel if there exists $\delta>0$ such that 
    \begin{equation*}
        |K(x,y)|\leq \frac{C}{|x-y|^n}
    \end{equation*}
    \begin{equation*}
        |K(x,y)-K(x,z)|\leq C\frac{|y-z|^\delta}{|x-y|^{n+\delta}}
    \end{equation*}
    \begin{equation*}
        |K(x,y)-K(w,y)|\leq C\frac{|x-w|^\delta}{|x-y}^{n+\delta}
    \end{equation*}
    for $|x-y|>2|x-w|$.
\end{definition}
There are some important examples and motivation: Cauchy integral along a Lipschitz curve. Let $A: \R\to\R$ be Lipschitz, $\Gamma=(t, A(t))$, for $f\in\mathcal{S}(\R)$, define 
\begin{equation*}
    C_\Gamma f(z)=\frac{1}{2\pi i}\int_{-\infty}^\infty \frac{f(t)(1+iA'(t))}{t+iA(t)^{-z}}dx
\end{equation*}
boundary value problem motivated.

\begin{definition}
    $T$ is a generalized $C-Z$ operator if 
    \begin{enumerate}
        \item $T$ is bounded on $L^2(\R^n)$
        \item there exists a standard kernel $K$ such that 
            \begin{equation*}
            Tf(x)=\int_{\R^n}K(x,y)f(y)dy, \forall f\in L^2, \text{compactly supported}, \forall x\not\in supp(f)
            \end{equation*}
    \end{enumerate}
\end{definition}
\begin{theorem}
    The above $T$ is  weak (1,1), and strong $(p,p)$
\end{theorem}
\begin{proof}
    By Theorem 5.10.
\end{proof}

\subsection{C-Z Singular integrals}
For a $C-Z$ operator $T$, does 
\begin{equation*}
    Tf(x)=\lim_{\epsilon\to 0}\int_{|x-y|<\epsilon}K(x,y)f(y)dy
\end{equation*}
Does this always exist for $f\in\mathcal{S}$. The answer is that: not always. The following is a counterexample.
\begin{example}
    $p.v. |x|^{-n-it}\ast f$
\end{example}
And  define 
\begin{equation*}
    T_\epsilon f(x)=\int_{|x-y|>\epsilon}K(x,y)f(y)dy
\end{equation*}

\begin{proposition}
    The limit 
    \begin{equation*}
        \lim_{\epsilon\to 0}T_\epsilon f(x) \text{ always exists}
    \end{equation*}
    for a.e. $x$, for all $f\in C_c^\infty$ if and only if 
    \begin{equation*}
        \lim_{\epsilon\to 0}\int_{\epsilon<|x-y|<1}K(x,y)dy \text{ exists }
    \end{equation*}
    for almost every $x$.
\end{proposition}
\begin{proof}
    The proof is identical of that 5.7.
\end{proof}

\begin{proposition}
    If two $C-Z$ operators are associated with the same standard kernel, then their differences is a pointwise multiplication by an $L^\infty$ function.
\end{proposition}

\begin{definition}
    A $C-Z$ singular integral is a $C-Z$ operator $T$ such that
    \begin{equation*}
        Tf(x)=\lim_{\epsilon\to 0}T_\epsilon f(x)
    \end{equation*}
    for all $f\in\mathcal{S}$.
\end{definition}



\section{Lecture Nov 8}
\begin{theorem}
    If $T$ is a C-Z operator, then $T^*$ is strong $(p,p), 1<p<\infty$, and weak (1,1).
\end{theorem}
\begin{proof}
    Pointwise convergence of $T_\epsilon f$ for $f\in L^p$ and also $L^1$ will then follow. Note that the key here is the generalization of Cotlar's inequality.

    \begin{lemma}[5.15 Cotlar]
        If $T$ is a $C-Z$ operator, then for any $0<v\leq 1$, and $f\in C_c^\infty$, we have 
        \begin{equation*}
            T^*f(x)\leq c_v (M(|Tf|^v)(x))^\frac{1}{v}+Mf(x)
        \end{equation*}
    \end{lemma}
    We first show that the theorem follows from the lemma. If we take $v=1$, we get strong (p,p) for $p>1$. For weak (1,1), we choose arbitrarily $0<v<1$, 
    \begin{equation*}
        |\{x: |T^*f(x)|>\lambda\}|\leq |\{x: |Mf(x)|\geq \frac{\lambda}{2c_v}\}|+|\{x: M(|Tf|^v)(x)\geq(\frac{\lambda}{2c_v})^v\}|
    \end{equation*}
    First term is bounded by $c_v \frac{\|f\|_{L^1}}{\lambda}$. And by section 2.6, we get that
    \begin{note}
        Now we would like to replace $M$ with the dyadic maximal to get a good bound on $|E|$
    \end{note}
    \begin{equation*}
        |\{x: M(|Tf|^v)(x)\geq(\frac{\lambda}{2c_v})^v\}|\leq 2^n|\{x: M_d(|Tf|^v)(x)>4^{-n}\left(\frac{\lambda}{2c_v}\right)^v\}|
    \end{equation*}
    And for $f\in C_c^\infty$, we get 
    \begin{equation*}
        E:=|\{x: M_d(|Tf|^v)>\tilde{\lambda}^v\}|
    \end{equation*}
    is finite, and 
    \begin{equation*}
        |E|\lesssim \frac{1}{\tilde{\lambda}^v}\int_E |Tf(y)|^v dy\lesssim_v\frac{1}{\tilde{\lambda}^v}|E|^{1-v}\|f\|_{L^1}^v
    \end{equation*}
    where the last inequality follows from the lemma below, and the fact that $T$ is indeed weak (1,1).
    \begin{lemma}[Kolmogorov]
        Given a weak (1,1) operator $S$, and $0<v<1$, and $E\subset\R^n$, and $|E|<\infty$, we get 
        \begin{equation*}
            \int_E |Sf(x)|^vdx\lesssim |E|^{1-v}\|f\|_{L^1}^v
        \end{equation*}
    \end{lemma}
    We then get 
    \begin{equation*}
        |E|\lesssim \|f\|_{L^1}/\tilde{\lambda}
    \end{equation*}
\end{proof}
\qed

Now it remains to prove the two lemmas.
\begin{proof}[second lemma]
    \begin{align*}
        \int_E |Sf|^vdx&=v\int_0^\infty\lambda^{v-1}|\{x\in E: |Sf(x)|>\lambda\}|d\lambda\\
        &\lesssim v\int_0^\infty \lambda^{v-1}\min\{|E|, \frac{\|f\|_{L^1}}{\lambda}\}d\lambda\\
        &\lesssim \|f\|_{L^1}^v|E|^{1-v}
    \end{align*}
\end{proof}
\qed

Now we prove the previous lemma. 
\begin{proof}[first lemma]
    WLOG, assume $x=0$. You take two disks, one contained in the other one with radius $Q=\epsilon/2$, and the larger one with $2Q=\epsilon$. We take $f=f_1+f_2$, where $f_1=f\chi_{2Q}$, and $f_2=f(1-\chi_{2Q})$. We will show that 
    \begin{equation*}
        |T_\epsilon f(0)|\lesssim \left(M(|Tf|^v)(0)\right)^\frac{1}{v}Mf(0)
    \end{equation*}
    We note that 
    \begin{equation*}
        Tf_2(0)=T_\epsilon f(0)
    \end{equation*}
    For $z\in Q$, 
    \begin{align*}
        |Tf_2(z)-Tf_2(0)|&=\left|\int_{|y|>\epsilon}K(z,y)-K(0,y)f(y)dy\right|\\
        &\lesssim |z|^\delta\int_{|y|>\epsilon}\frac{f(y)}{|y|^{n+\delta}}dy\\
        &\lesssim Mf(0)
    \end{align*}
    However, we have 
    \begin{equation*}
        Tf_2(z)=Tf(z)-Tf_1(z)
    \end{equation*}
    This gives that 
    \begin{equation*}
        |T_\epsilon f(0)|\leq CMf(0)+|Tf(z)|+|Tf_1(z)|, \forall z\in Q
    \end{equation*}
    WLOG, assume $|T_\epsilon f(0)|>0$, fix $\lambda\in (0, |T_\epsilon f(0)|)$, and let 
    \begin{equation*}
        Q_1=\{z\in Q: Tf(z)>\frac{\lambda}{3}\}, Q_2=\{z\in Q: Tf_1(z)>\frac{\lambda}{3}\}
    \end{equation*}
    We have either $CMf(0)>\frac{\lambda}{3}$, or $Q=Q_1\cup Q_2$.
    The rest is an exercise.
\end{proof}
\qed





\section{Chapter 6}
We know that $Hf\in L^1$ if and only if $\int f=0$. Hence we classify these functions.
\begin{definition}[atom]
    An atom is a function $a$ such that $a$ is supported on a cube $Q$, and 
    \begin{equation*}
        \int_Q a=0, \|a\|_\infty\leq\frac{1}{|Q|}
    \end{equation*}
\end{definition}
\begin{remark}
    We also have $\|a\|_{L^1}\leq 1$.
    \begin{equation*}
        \int|a|\leq \int_Q \|a\|_\infty\leq 1
    \end{equation*}
\end{remark}
\begin{proposition}
    Let $T$ be an operator that is bounded on $L^2$, then we have 
    \begin{equation*}
        \|Ta\|_{L^1}<\infty
    \end{equation*}
\end{proposition}
\begin{proof}
    Cauchy-Schwarz.
\end{proof}
\begin{definition}[atomic $H^1$]
    We have the atomic $H^1$ as $H_a^1$:  
    \begin{equation*}
        H_a^1=\{\sum_{j}\lambda_ja_j: \sum|\lambda_j|<\infty\}
    \end{equation*}
\end{definition}
\begin{remark}
    The atomic $H^1$ space is embedded in $L^1$.
\end{remark}




\newpage
\section{Last Lecture}
If we define the norm $\|\|_{H^s}$ as follows:
\begin{equation*}
    \|f\|_{H^s}^2=\int(1+|\xi|^2)^s|\hat{f}(\xi)|^2d\xi
\end{equation*}
\begin{proposition}
    If $f\in H^s$, and $s>n/2$, then $\hat{f}\in L^1$.
\end{proposition}
\begin{proof}
    \begin{equation*}
        \int |\hat{f}(\xi)|d\xi\leq \left(\int |\hat{f}(\xi)|^2(1+|\xi|^2)^sd\xi\right)^{1/2}\left(\int (1+|\xi|^2)^{-s}\right)^{1/2}<\infty
    \end{equation*}
\end{proof}

This is theorem 8.10 in the book.
\begin{theorem}[Homander multiplier theorem]
    Let function $m$ and $s>n/2$ such that 
    \begin{equation*}
        \sup_j\|e^j\psi\|_{H^s}<\infty
    \end{equation*}
    with $\psi\geq 0$, radial, and $c^\infty$, supported on $\{\frac{1}{2}\leq|\xi|\leq 2\}$, and $\sum_j|\psi(2^{-j}\xi)|^2=1$, for all $\xi\neq 0$. Then the operator $T$ such that 
    \begin{equation*}
        \widehat{Tf}=m\hat{f}
    \end{equation*}
    is bounded on $L^p(\R^n)$, with $1<p<\infty$.
\end{theorem}
\begin{corollary}
    If $k>n/2$, $m\in C^k$ away from the origina, and 
    \begin{equation*}
        \sup_{R>0}R^{|\beta|}\left(\frac{1}{R^n}\int_{R<|\xi|<2R}|\partial^\beta m(\xi)|^2d\xi\right)^{1/2}, \forall |\xi|\leq k
    \end{equation*}
    then $m$ is a multiplier on $L^p$, with $1<p<\infty$. In particular, this is so when
    \begin{equation*}
        |D^\beta m(\xi)|<|\xi|^{-|\beta|}, \forall |\beta|\leq k
    \end{equation*}
\end{corollary}
Now we prove the theorem.
\begin{lemma}
    If $m\in H^s$, and $s>n/2$, and $\lambda>0$. Let 
    \begin{equation*}
        \widehat{T_\lambda f}(\xi)=m(\lambda\xi)\hat{f}(\xi)
    \end{equation*}
    Then 
    \begin{equation*}
        \int|T_\lambda f(x)|^2 u(x)dx\lesssim \int |f(x)|^2Mu(x)dx
    \end{equation*}
\end{lemma}
\begin{proof}
    Let $K=\check{m}$, 
    \begin{equation*}
        R(x)=(1+|x|^2)^{s/2}K(x)\in L^2
    \end{equation*}
    Kernel of $T_\lambda$ is $\lambda^{-n}K(\lambda^{-1}x)$, we have 
    \begin{equation*}
        \int |T_\lambda f|^2u=\int\left( \int \frac{\lambda^{-n}R^(\lambda^{-1}(x-y))}{(1+\|\lambda^{-1}(x-y)|^2)^{s/2}}f(y)dy\right)^2u(x)dx
    \end{equation*}
    And by Cauchy-Schwarz, we have 
    this is bounded by 
    \begin{equation*}
        \lesssim \|R\|_{L^2}^2\int |f(y)|^2Mu(y)dy
    \end{equation*}
    since $\lambda^{-n}/(1+\lambda^{-1}|x-y|^2)^s$ is radial and decreasing, and has uniform $L^1$ norm.
\end{proof}
\qed

\begin{proof}[of the theorem]
    Let $\tilde{\psi}_j=\tilde{\psi}(2^{-j})$. And first $p>2$, and prove the theorem, and $p<2$ is by duality.
\end{proof}

We talk a bit about disk multiplier.
\begin{equation*}
    \widehat{S_1f}=\chi_{B_1}\hat{f}
\end{equation*}
is unbounded on $L^p$ unless $p=2$. Bochner -Riesz gives us
\begin{equation*}
    \widehat{S_1^\delta f}=(1-|\xi|^2)_+^\delta\hat{f}, \delta>0
\end{equation*}
There is a conjecture as follows.
$S_1^\delta$ is strong $(p,p)$, for all 
\begin{equation*}
    \frac{2n}{n+1+2\delta}<p<\frac{2n}{n-1-2\delta}, \forall\delta>0
\end{equation*}
