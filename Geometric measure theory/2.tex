\chapter{Work}
Maybe recover Wolff's bound on spherical measure using decoupling. 

duality, let $r\to\infty$, then $\mu$  could be large.
$\hat{\mu}\in L^2$, and extend it,  $\mu$ weighted restriction, just integrate with respect to $\mu$. Our eventual goal in $L^1$. Weighted restriction is decay of spherical mean, by duality.

-Weighted restriction in high dimension.

-take a convolution operator, convolve with a straight line, this turns into pointwise multiplication, and use the $L^2$ bound on this. (this can be viewed as a projection, then vary the direction of the line).

Hi, what is going on here
\chapter{Questions}

Can one think of rectifiable or unrectifiable as compressible or noncompressible?

-some exaples of rectifiable sets

-The measures of projections are big when it's a rectifiable set, and are small when it's a rectifiable.

-Unrectifiable means not rectifiable? What?


-the constant 4 in page 28

-high density (purely unrectifiable sets and the fact that their projections are quite small)


\textbf{Week 3}
-Notation issue: page 37, $\hat{\mu}\in L^2$ is well defined, but a measure $\mu\in L^2$ means it has an $L^2$ density?

-check whether Wolff's statement 9B1 is true in $\R^d$,

-check you definition of the $I_s(\mu)$ using spherical average, namely, 
\begin{equation*}
    I_s(\mu)=\int r^{s-d}(\sigma(\mu)(r))^2dr
\end{equation*}

-the cricial exponent $s-2$ in the spherical average, so bounding the spherical average by anything below is a gain.

-did not talk about how spherical averages of the decay of Fourier transform relates to the distance set.

-Fourier dimension before Theorem 6.28? But was only talking about it in Theorem 6.30.

-The discretized sum-product and projections theorems for Fourier dimension of the product set: what type of problem is this related to, and why do we care about Fourier dimension to begin with.

-can we talk about the Erdos ring conjecture.




Exercise 6.25. 
Example 6.27.

