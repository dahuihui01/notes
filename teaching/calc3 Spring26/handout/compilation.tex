\documentclass[openany]{book}

\usepackage[margin=1in]{geometry}
\usepackage{amsmath,amsfonts,amsthm, amssymb}
\usepackage{yhmath}
\usepackage{mathrsfs}
\usepackage{mathtools}
\usepackage{xcolor}
\usepackage{graphicx}
\usepackage{comment}
\usepackage{tikz-cd}
\usepackage{quiver}
\usepackage{hyperref,cleveref}
\renewcommand{\familydefault}{ppl}
\newcommand{\tr}{\text{tr}}
\newcommand{\R}{\mathbb{R}}
\newcommand{\E}{\mathbb{E}}
\newcommand{\Z}{\mathbb{Z}}
\newcommand{\C}{\mathbb{C}}
\newcommand{\F}{\mathbb{F}}
\newcommand{\la}{\langle}
\newcommand{\ra}{\rangle}
\newcommand{\colim}{\text{colim}}
\DeclareMathOperator{\im}{im}
\DeclareMathOperator{\disc}{disc}
\let\oldemptyset\emptyset
\let\emptyset\varnothing
\newcommand{\tor}{\text{Tor}}
\newcommand{\id}{\text{id}}
\newcommand{\ext}{\text{Ext}}
\newcommand{\ptop}{\text{PTop}}
\newcommand{\pt}{\text{pt}}
\newcommand{\ach}{\text{Ach}}
\newcommand{\Q}{\mathbb{Q}}
\newcommand{\gal}{\text{Gal}}
\newcommand{\fraccomma}{\genfrac{}{}{0pt}{}{}{,}}
\newcommand{\diverg}{\operatorname{div}}
\newcommand{\curl}{\operatorname{curl}}
\definecolor{wikipediadarkblue}{rgb}{0.023, 0.270, 0.676}
\hypersetup{
    colorlinks,
    citecolor=black,
    filecolor=black,
    linkcolor=wikipediadarkblue,
    urlcolor=red
}
\usepackage{pgfplots}
\pgfplotsset{compat=newest}


\input{hui_r.tex}

\title{Calc III Section Notes with Answers
\\ 
\vspace{0.4cm}
\large Spring 2026}




\date{\today}
\author{Hui Sun}


\begin{document}

\maketitle
\tableofcontents

% \tableofcontents
\newpage

\chapter{The Geometry of Euclidean Spaces}


\section*{\centering Week 1 (1/19-23)}


% \begin{center}
%     \Large Calc III-Week 1 )
% \end{center}

\renewcommand\thesection{\arabic{section}}
\noindent
\textbf{Logistics}


\begin{itemize}
    \item TA: Hui.
    \item Email: hsun95@jh.edu.
    \item Drop-in Hours: Tuesday 2-3 PM, 4-5 pm, Krieger 211.
    \item Biweekly Quizzes: 10\%.
    \item Attendance: 5\%. (If you can't make it, email me).
\end{itemize}


\begin{defn}[standard basis of $\R^3$]
    The following vectors 
    \begin{equation*}
        i=(1,0,0), j=(0,1,0), k=(0,0,1)
    \end{equation*}
    are called the \textbf{standard basis} vectors of $\R^3$, and for any vector $a=(a_1,a_2,a_3)\in\R^3$, we can write 
    \begin{equation*}
        a=a_1i+a_2j+a_3k
    \end{equation*}
\end{defn}



\begin{defn}[dot product]
    Let $v=(v_1,v_2,v_3), w=(w_1,w_2, w_3)\in\R^3$, the \textbf{dot product} $v\cdot w$ is given by 
    \begin{equation*}
        v\cdot w=v_1w_1+v_2w_2+v_3w_3
    \end{equation*}
    Alternatively,
    \begin{equation*}
        v\cdot w=\|v\|\|w\|\cos\theta
    \end{equation*}
    where 
    \begin{equation*}
        \theta=\arccos\left(\frac{v\cdot w}{\|v\|\|w\|}\right)
    \end{equation*}
\end{defn}

\begin{defn}[length of vector]
    Let $v=(v_1,v_2,v_3)\in\R^3$, the \textbf{length} or \textbf{norm} of $v$, denoted as $\|v\|$, is 
    \begin{equation*}
        \|v\|=\sqrt{v_1^2+v_2^2+v_3^2}=\sqrt{v\dot v}
    \end{equation*}
\end{defn}

\begin{defn}[linear combination]
    Let $v, w\in\R^3$, a \textbf{linear combination} of $v,w$ is a sum 
    \begin{equation*}
        av+bw
    \end{equation*}
    for some $a,b\in\R$. One can generalize this definition to $n$ vectors: let $v_1, v_2\dots, v_n\in\R^3$, a linear combination of these vectors is a finite sum 
    \begin{equation*}
        a_1v_1+a_2v_2+\dots+a_nv_n
    \end{equation*}
    for some $a_i\in\R, 1\leq i\leq n$. 
\end{defn}


\begin{prop}[properties of the dot product]
    Let $a,b,c\in\R^n$, then 
    \begin{enumerate}
        \item[(a)] Nonnegativity: $a\cdot a\geq 0$, and $a\cdot a=0$ if and only if $a=0$.
        \item[(b)] Scalar multiplication: let $\lambda\in\R$, then 
        \begin{equation*}
            \lambda(a\cdot b)=\lambda a\cdot b=a\cdot \lambda b
        \end{equation*}
        \item[(c)] Distributivity:
        \begin{equation*}
            a\cdot(b+c)=a\cdot b+a\cdot c, \quad (a+b)\cdot c=a\cdot c+b\cdot c
        \end{equation*}
        \item[(d)] Symmetry: $a\cdot b=b\cdot a$.
    \end{enumerate}
\end{prop}


% \noindent
% \textbf{Icebreaking Activity}
% \begin{itemize}
%     \item In a group of three or four: 
%     \begin{enumerate}
%         \item Learn each other names, year, pronouns.
%         \item Find something in common and different among you and share with the entire class.
%         \item Play Buzz if you have time, with prime $7$: say the number if it doens't contain or is not divisible by $7$, say buzz otherwise.
%     \end{enumerate}
% \end{itemize}


\begin{prob}
    Draw the following vectors in $\R^2$:
    \begin{equation*}
        u=(1,2), \quad v=(3,-2)
    \end{equation*}
    Compute $u+v, u-v$, and draw them in the plane.
\end{prob}
\begin{proof}
    \begin{equation*}
        u+v=(4,0), \quad u-v=(-2, 4)
    \end{equation*}
\end{proof}

\begin{prob}
    Consider the following vectors in $\R^3$:
    \begin{equation*}
        u=(1,2,3), \quad, v=(-2, 1, 4)
    \end{equation*}
    \begin{enumerate}
        \item Compute their norms.
        \item  Two vectors $a,b\in\R^3$ are called \textbf{orthognal} if $a\cdot b=0$. Are $u,v$ orthogonal? If not, find a nonzero vector orthogonal to $u$.
    \end{enumerate}
\end{prob}
\begin{proof}
    \begin{enumerate}
        \item \begin{equation*}
            \|u\|=(u\cdot u)^\frac{1}{2}=\sqrt{14}, \quad \|v\|=\sqrt{21}
        \end{equation*}
        \item We check 
        \begin{equation*}
            u\cdot v=-2+2+12=12\neq 0
        \end{equation*}
        thus not orthogonal. A vector that is orthognal to $u$: $(-3,0,1)$. Note that this vector is \textbf{not} unique! For example, $(-1,-1,1)$ is another such vector.
    \end{enumerate}
\end{proof}


\begin{prob}
    Can you express $w=(1,2)$ as a linear combination of $v_1,v_2$ for difference choices of $v_1,v_2$?
    \begin{enumerate}
        \item $v_1=(1,1), v_2=(-2,-2)$.
        \item $v_1=(2,1), v_2=(-1, 0)$.
    \end{enumerate}
\end{prob}
\begin{proof}
    \begin{enumerate}
        \item We first note that $(1,1), (-2,-2)$ lie on the same line through the origin. Hence, any linear combination of $v_1,v_2$ will stay in this line, i.e., of the form $(a,a)$, for some $a\in\R$. Therefore, it is impossible to write $w=(1,2)$ as a linear combination of $v_1,v_2$.
        \item Suppose $w=a_1v_1+a_2v_2$ for some $a_1,a_2\in\R$, then 
        \begin{equation*}
            \begin{cases}
                2a_1-a_2=1\\
                a_1=2
            \end{cases}\Rightarrow \begin{cases}
                a_1=2\\
                a_2=3
            \end{cases}
        \end{equation*}
        Thus we can write $w$ as a linear combination of $v_1,v_2$:
        \begin{equation*}
            w=2v_1+3v_2
        \end{equation*}
    \end{enumerate}
\end{proof}





\begin{prob}
    Let $u,v\in\R^3$, suppose that $u,v$ are orthongal, show that 
    \begin{equation*}
        \|u+v\|^2=\|u\|^2+\|v\|^2
    \end{equation*}
    Bonus: is the converse true? (meaning assuming $ \|u+v\|^2=\|u\|^2+\|v\|^2$, is it true that $u\cdot v=0$?)
\end{prob}
\begin{proof}
    We have 
    \begin{align*}
        \|u+v\|^2&=(u+v)\cdot(u+v)\\
        &=u\cdot u+u\cdot v+v\cdot u+v\cdot v\\
        &=\|u\|^2+\|v\|^2
    \end{align*}
    because $u\cdot v=v\cdot u=0$. The converse is also true: we know by definition that
    \begin{equation*}
        \|u+v\|^2=\|u\|^2+\|v\|^2+2u\cdot v
    \end{equation*}
    given the assumption, we also have 
    \begin{equation*}
        \|u+v\|^2=\|u\|^2+\|v\|^2
    \end{equation*}
    Thus equating them we get 
    \begin{equation*}
        \|u\|^2+\|v\|^2+2u\cdot v=\|u\|^2+\|v\|^2\Rightarrow u\cdot v=0
    \end{equation*}
\end{proof}




\newpage

\section*{\centering Week 2 (1/26-30)}
Topics: determinant, cross product.

\begin{defn}[determinant]
    Let $A=\begin{pmatrix}
        a&b\\
        c&d
    \end{pmatrix}$ be a $2\times 2$ matrix, the \textbf{determinant} of $A$ is given by 
    \begin{equation*}
        \det(A)=ad-bc
    \end{equation*}
    Let $A=\begin{pmatrix}
        a_1&a_2&a_3\\
        b_1&b_2&b_3\\
        c_1&c_2&c_3
    \end{pmatrix}$ be a $3\times 3$ matrix, the \textbf{determinant} of $A$ is given by
    \begin{equation*}
        \det(A)=a_1(b_2c_3-b_3c_2)-a_2(b_1c_3-b_3c_1)+a_3(b_1c_2-b_2c_1)
    \end{equation*}
\end{defn}


\begin{defn}[cross product]
    Let $a,b\in\R^3$, write $a=(a_1,a_2,a_3), b=(b_1,b_2,b_3)$, then the \textbf{cross product}
    \begin{equation*}
        a\times b=\det\begin{bmatrix}
            i&j&k\\
            a_1&a_2&a_3\\
            b_1&b_2&b_3
        \end{bmatrix}
    \end{equation*}
    where $i,j,k$ are the standard vectors in $\R^3$.
\end{defn}


\begin{prop}[properties of the cross product]
    We have the following properties regarding the cross product: let $a,b\in\R^3$,
    \begin{enumerate}
        \item $a\times a=0$.
        \item $a\times b=-b\times a$.
        \item $(a+b)\times c=a\times c+b\times c$, and $a\times (b+c)=a\times b+a\times c$.
        \item $(\alpha a)\times b=\alpha(a\times b)$ for any $a\in\R$.
        \item $a\times b$ is perpendicular to vectors $a,b$.
        \item The length of the cross product is the area of the parallelogram spanned by $a,b$:
        \begin{equation*}
            \|a\times b\|=\|a\|\|b\|\sin\theta
        \end{equation*}
        where $0\leq\theta\leq\pi$ is the angle between them. 
        \item $a\times b=0$ iff $a,b$ are parallel or either $a$ or $b$ are $0$.
        \item The cross product is \textbf{not} associative! For example, compute 
        \begin{equation*}
            (i\times i)\times j, \quad i\times (i\times j)
        \end{equation*}
    \end{enumerate}
\end{prop}


\begin{prop}[determinant and linear combination]\label{det}
    Let $A=\begin{pmatrix}
        a_1&a_2&a_3\\
        b_1&b_2&b_3\\
        c_1&c_2&c_3
    \end{pmatrix}$ be a $3\times 3$ matrix, let 
    \begin{equation*}
        a=(a_1,a_2, a_3), b=(b_1,b_2,b_3), c=(c_1,c_2,c_3)
    \end{equation*}
    If any of $a,b,$ or $c$ is a linear combination of the other two vectors, then $\det(A)=0$. (Relevant topic: linear independence).
\end{prop}



\begin{prob}
    Let $\vec{u}=(1,2,3), \vec{v}=(0,1,1)$ be vectors in $\R^3$, compute the area of the parallelogram spanned by these two vectors.
\end{prob}
\begin{proof}
    \begin{equation*}
        u\times v=\begin{bmatrix}
            i&j&k\\
            1&2&3\\
            0&1&1
        \end{bmatrix}=-i-j+k=(-1,-1,1)
    \end{equation*}
    Thus the area of the parallelogram is 
    \begin{equation*}
        \|u\times v\|=\sqrt{3}
    \end{equation*}
\end{proof}


\begin{prob}
    Compute the determinant of the following matrix $A$:
    \begin{equation*}
        A=\begin{pmatrix}
            1&0&2\\
            2&1&-1\\
            3&1&1
        \end{pmatrix}
    \end{equation*}
\end{prob}
\begin{proof}
    Notice that the third row vector $(3,1,1)$ is the sum of the two row vectors above, hence by Proposition \ref{det}, we know we must have $\det(A)=0$.
\end{proof}
\newpage

\section*{\centering Week 2 (Section Activity)}
First, introduce yourself to one another:
\begin{enumerate}
    \item Name.
    \item Pronouns.
    \item Year.
    \item Major.
    \item If you could have a noncat, nondog pet, what would it be?
\end{enumerate}
Next, work on and \textbf{discuss} the following problems!
\begin{prob}
    Let $u=(1,2,-1)$, find a nonzero vector that is orthogonal to $u$.
\end{prob}
\begin{proof}
    There are many choices, for example, $v=(1,0,1)$.
\end{proof}

\begin{prob}
    Let $u=(1,0), v=(2,-1)$, 
    \begin{enumerate}
        \item Using the dot product, what is the angle between $u,v$? (You do not need to simplify your answer).
        \item Can you find a nonzero vector in $\R^2$ that is orthogonal to both of $u,v$? 
        \item Can you find nonzero vectors that are orthogonal to any basis of $\R^2$? (Hint: take $u=(1,0)$, can you draw the set of vectors orthogonal to $u$? A basis of $\R^2$ contains two noncolinear vectors).
    \end{enumerate}
\end{prob}
\begin{proof}
    \begin{enumerate}
        \item We have 
        \begin{equation*}
            \cos\theta=\frac{u\cdot v}{\|u\|\|v\|}=\frac{2}{\sqrt{5}}
        \end{equation*}
        Hence 
        \begin{equation*}
            \theta=\arccos\left(\frac{2}{\sqrt{5}}\right)
        \end{equation*}
        \item Suppose there exists a nonzero vector $w=(w_1,w_2)\in\R^2$ such that $w\cdot u=w\cdot v=0$, then this implies
        \begin{equation*}
            w\cdot u=w_1=0, \quad w\cdot v=2w_1-w_2=0\Rightarrow w_1=w_2=0
        \end{equation*}
        Thus the only vector orthogonal to $u,v$ is the zero vector, hence the answer is no.
        \item Let $\{a,b\}$ be a basis of $\R^2$, where $a=(a_1,a_2), b=(b_1,b_2)$, we know they are noncolinear, hence 
        \begin{equation*}
            \frac{a_1}{b_1}\neq\frac{a_2}{b_2}
        \end{equation*}
        Then suppose $w$ is orthogonal to both $a,b$. Then (one can draw a picture), $w$ should be of the form 
        \begin{equation*}
            w=(\lambda a_2,-\lambda a_1)=(\zeta b_2, -\zeta b_1)
        \end{equation*}
        for some $\lambda,\zeta\in\R$. But this implies 
        \begin{equation*}
            \frac{a_1}{b_1}=\frac{a_2}{b_2}=\frac{\zeta}{\lambda}
        \end{equation*}
        which is a contradiction to the above. This concludes the proof that no nonzero vector can be orthogonal to a basis in $\R^2$.
    \end{enumerate}
\end{proof}

\begin{prob}
    Let 
    \begin{equation*}
        A=\begin{pmatrix}
            1&2&0\\
            -1&3&1\\
            0&0&2
        \end{pmatrix}
    \end{equation*}
    Compute the determinant of $A$.
\end{prob}
\begin{proof}
    We have $\det(A)=10$.
\end{proof}

\begin{prob}
    Construct a nonzero $3\times 3$ matrix $A$ such that $\det(A)=0$.
\end{prob}
\begin{proof}
    \begin{equation*}
        A=\begin{pmatrix}
            1&1&1\\
            2&2&2\\
            3&3&3
        \end{pmatrix}
    \end{equation*}
\end{proof}


\newpage 
\section*{\centering Week 3 (2/2-2/6)}
Topics: parametric equations, multivariable functions, and level sets.

\begin{defn}[matrix addition, multiplication]
    Let $A, B$ be $m\times n$ matrices, $C$ be $n\times k$ matrix as follows 
    \begin{equation*}
        A=\begin{pmatrix}
            a_{11}&\dots& a_{1n}\\
            \vdots&\ddots&\vdots\\
            a_{m1}&\dots&a_{mn}
        \end{pmatrix}, \quad B=\begin{pmatrix}
            b_{11}&\dots&b_{1n}\\
            \vdots&\ddots&\vdots\\
            b_{m1}&\dots&b_{mn}
        \end{pmatrix}
    \end{equation*}
    \begin{equation*}
        C=\begin{pmatrix}
            c_{11}&\dots&c_{1k}\\
            \vdots&\ddots&\vdots\\
            c_{n1}&\dots&c_{nk}
        \end{pmatrix}
    \end{equation*}
    then \textbf{matrix addition} $A+B$ defined as 
    \begin{equation*}
        A+B\coloneq\begin{pmatrix}
            a_{11}+b_{11}&\dots& a_{1n}+b_{1n}\\
            \vdots&\ddots&\vdots\\
            a_{m1}+b_{m1}&\dots&a_{mn}+b_{mn}
        \end{pmatrix}
    \end{equation*}
    and \textbf{matrix multiplication} is defined as 
    \begin{equation*}
        AC=\begin{pmatrix}
            \sum_{j=1}^na_{1j}c_{j1}&\dots&\sum_{j=1}^na_{1j}c_{jk}\\
            \vdots&\ddots&\vdots\\
            \sum_{j=1}^na_{mj}c_{j1}&\dots&\sum_{j=1}^na_{mj}c_{jk}
        \end{pmatrix}
    \end{equation*}
    \begin{remark}
        Given matrices $A,B$, for the matrix multiplication $AB$ to be well-defined,
        \begin{equation*}
            \text{number of columns of $A$}=\text{number of rows of $B$}
        \end{equation*}
    \end{remark}
\end{defn}

\begin{defn}[matrix as linear transformation]
    Let $A$ be an $m\times n$ matrix 
    \begin{equation*}
        A=\begin{pmatrix}
            a_{11}&\dots& a_{1n}\\
            \vdots&\ddots&\vdots\\
            a_{m1}&\dots&a_{mn}
        \end{pmatrix}
    \end{equation*}
    and let $x\in\R^n$, where $x=(x_1,\dots, x_n)$. Then $A$ applied to $x$ as a linear transformation is given by 
    \begin{equation*}
        Ax=\begin{pmatrix}
            a_{11}&\dots& a_{1n}\\
            \vdots&\ddots&\vdots\\
            a_{m1}&\dots&a_{mn}
        \end{pmatrix}\begin{pmatrix}
            x_1\\
            \vdots\\
            x_n
        \end{pmatrix}=\begin{pmatrix}
            \sum_{j=1}^na_{1j}x_j\\
            \vdots\\
            \sum_{j=1}^na_{mj}x_j
        \end{pmatrix}
    \end{equation*}
    where 
    \begin{equation*}
        \sum_{j=1}^na_{1j}x_j=a_{11}x_1+a_{12}x_2+\dots+a_{1n}x_n, \quad\dots
    \end{equation*}
\end{defn}


\begin{defn}[image, graph]
    The \textbf{image} of a function $f:U\subset \R^n\to\R^m$ is a subset of $\R^m$,
    \begin{equation*}
        \text{Image}(f)=\{f(x)\in\R^m: x\in U\}
    \end{equation*}
    and the \textbf{graph} of $f$ is a subset of $\R^{n+m}$,
    \begin{equation*}
        \text{Graph}(f)=\{(x,f(x)): x\in U\}
    \end{equation*}
\end{defn}


\begin{defn}[level set]
    Let $f:U\subset\R^n\to\R^m$, and $c\in\R$ be some constant. Then the \textbf{level set} of $f$ at $c$ is the set $\mathcal{L}_c$
    \begin{equation*}
        \mathcal{L}_c\coloneq\{x\in U: f(x)=c\}\subset\R^n
    \end{equation*}
\end{defn}

\begin{defn}[Equation of a line]\label{line}
    A \textbf{line} $l$ in $\R^3$ through the tip of $a=(a_1,a_2,a_3)$ pointing in the direction of a vector $v=(v_1,v_2,v_3)$ is given by 
    \begin{equation*}
        l(t)=a+tv=(a_1+tv_1, a_2+tv_2, a_3+tv_3)
    \end{equation*}
    where $t\in\R$. Alternatively, a line passing through two points $P=(x_1,y_1,z_1), Q=(x_2,y_2,z_2)$ is given by 
    \begin{equation*}
        l(t)=(x(t), y(t), z(t))
    \end{equation*}
    where 
    \begin{equation*}
        \begin{cases}
            x(t)=x_1+(x_2-x_1)t\\
            y(t)=y_1+(y_2-y_1)t\\
            z(t)=z_1+(z_2-z_1)t
        \end{cases}
    \end{equation*}
\end{defn}


\begin{defn}[Plane in $\R^3$]\label{plane}
    If a plane ${P}$ passes through some point $(x_0,y_0,z_0)$, and $n=(A,B,C)$ is a vector orthogonal to the plane, then the plane ${P}$ is given by the equation:
     \begin{equation*}
         A(x-x_0)+B(y-y_0)+C(z-z_0)=0
     \end{equation*}
     \begin{remark}
        A point ${P}$ in the plane and a normal vector to the plane uniquely determine a plane in $\R^3$. Equivalently, three noncolinear points uniquely determine a plane.
     \end{remark}
 \end{defn}
 \begin{prob}
    Compute the plane containing all three points:
    \begin{equation*}
        (1,0,2), \quad (2, -1, 0), \quad (-1, 2, 3)
    \end{equation*}
 \end{prob}
 \begin{proof}
    Let $A=(1,0,2), B=(2, -1, 0), C=(-1, 2, 3)$, then consider two vectors in this plane 
    \begin{equation*}
        AB=(1,-1,-2), AC=(-2,2,1)
    \end{equation*}
    Then taking their cross product we find a normal vector to this plane:
    \begin{equation*}
        AB\times AC=\begin{bmatrix}
            i&j&k\\
            1&-1&-2\\
            -2&2&1
        \end{bmatrix}=3i+3j+0k=(3,3,0)
    \end{equation*}
    Thus using the definition above, and point $A$, we know the formula is given by 
    \begin{equation*}
        3(x-1)+3(y)=0
    \end{equation*}
    One can simplify this to 
    \begin{equation*}
        x+y-1=0
    \end{equation*}
\end{proof}

 \begin{prob}
    \begin{enumerate}
        \item[(a)] Find the equation of the line through \( (1, 1, 0) \) in the direction of \( 2\mathbf{i} - \mathbf{k} \).  
        \item[(b)] Find the equation of the line passing through \( (0, 1, 1) \) and \( (0, 1, 0) \).  
        \item[(c)] Find an equation for the plane perpendicular to the vector \( (-1, 1, -1) \) and passing through the point \( (1, 1, 1) \).
    \end{enumerate}
\end{prob}
\begin{proof}
    \begin{enumerate}
        \item[(a)] The equation is given by 
        \begin{equation*}
            l_1(t)=(1,1,0)+t(2, 0, -1), \quad t\in\R
        \end{equation*}
        \item[(b)] The equation is given by 
        \begin{equation*}
            l_2(t)=(0,1,1)+t(0, 0, -1), \quad t\in\R
        \end{equation*}
        \item[(c)] The equation of the plane is given by 
        \begin{equation*}
            (x-1, y-1, z-1)\cdot (-1, 1, -1)=0
        \end{equation*}
        simplifying we get 
        \begin{equation*}
            x-y+z=1
        \end{equation*}
    \end{enumerate}
\end{proof}


\newpage 
\section*{\centering Week 4 (2/9-2/13)}
Topic: limits.






\begin{defn}[limit]
    Let $f: A\subset\R^n\to\R^m$, where $A$ is open, let $x_0$ be in $A$ or be a boundary point of $A$ and $N$ be a neighborhood of a point $b\in\R^m$. Now let $x$ approach $x_0$, $f$ is said to be \textbf{eventually in $N$} if there exists a neighborhood $U$ of $x_0$ such that 
    \begin{equation*}
        \text{ if } x\in U, \text{ then } f(x)\in N
    \end{equation*}
    If $f$ is eventually in $N$ for \textit{any} neighborhood $N$ around $b$, then the \textbf{limit} of $f$ as $x\to x_0$ exists, denoted as 
    \begin{equation*}
        \lim_{x\to x_0}f(x)=b
    \end{equation*}
    Alternatively, if the limit exists, then $\lim_{x\to x'}f(x)=b$ is when $x=(x_1, x_2, \dots, x_n)\to x'=(x_1',x_2',\dots, x_n')$ from \textbf{all directions}, and $f(x)$ approaches $b=(b_1,\dots, b_m)$.
\end{defn}



\begin{defn}[continuity]
    Let $f:A\subset\R^n\to\R^m$ is said to be \textbf{continuous} at $x_0\in A$ if 
    \begin{equation*}
        \lim_{x\to x_0}f(x)=f(x_0)
    \end{equation*}
    And $f$ is called continuous if $f$ is continuous at every $x_0\in A$.
\end{defn}

\begin{example}
    The limit doesn't need to exist! For example, let 
    \begin{equation*}
        H(x)=\begin{cases}
            1, x\geq 0\\
            -1, x<0
        \end{cases}
    \end{equation*}
    Note the limit doesn't exist at $x=0$.
\end{example}




\begin{prob}
    For the following functions, find their (1) image, (2) graph, (3) \textbf{draw their level sets}.
    \begin{enumerate}
        \item  Let $f: \R\to\R$, and $f(x)=x^2+1$.
        \item Let $g: \R^2\to \R$, and $g(x,y)=x^2+y^2$.
    \end{enumerate}
\end{prob}
\begin{proof}
    \begin{enumerate}
        \item $\text{Image}(f)=\{x^2+1: x\in\R\}$, and $\text{Graph}(f)=\{(x,x^2+1):x\in\R\}$.
        \item $\text{Image}(g)=\{x^2+y^2, (x,y)\in\R^2\}$, and $\text{Graph}(g)=\{(x,y,x^2+y^2):(x,y)\in\R^2\}$.
    \end{enumerate}
\end{proof}

\begin{prob}
    Compute the following limits:
    \begin{enumerate}
        \item \begin{equation*}
            \lim_{(x,y)\to (0,0)}\frac{\sin xy}{y}
        \end{equation*}
       ( Hint: try writing $\frac{\sin xy}{y}=\frac{\sin xy}{xy}\cdot x$, and recall $\lim_{t\to 0}\frac{\sin t}{t}=1$).
        \item \begin{equation*}
            \lim_{(x,y)\to(0,0)}\frac{e^{xy}-1}{y}
        \end{equation*}
        \item \begin{equation*}
            \lim_{(x,y)\to(0,0)}\frac{(x-y)^2}{x^2+y^2}
        \end{equation*}
    \end{enumerate}
\end{prob}
\begin{proof}
    \begin{enumerate}
        \item Following the hint, we see 
        \begin{equation*}
            \lim_{(x,y)\to(0,0)}\frac{\sin xy}{y}=\lim_{(x,y)\to(0,0)}\frac{\sin xy}{xy}x=\lim_{x\to 0}x=0
        \end{equation*}
        \item This one uses the exact same trick: \begin{equation*}
            \lim_{(x,y)\to(0,0)}\frac{e^{xy}-1}{xy}\cdot y=0
        \end{equation*}
        \item First letting $x\to 0$ along $y=0$, we see the limit is $1$; letting $x=y\to 0$, we see the limit is $0$, thus the limit doesn't exist!
    \end{enumerate}
\end{proof}


\begin{prob}
    Compute the limit of the following functions:
    \begin{enumerate}
        \item \begin{equation*}
            \lim_{(x,y)\to(0,0)}\frac{x}{x+y} 
        \end{equation*}
        \item \begin{equation*}
            \lim_{(x,y)\to(0,0)}\frac{xy}{x+y}
        \end{equation*}
        (Hint: try considering $y=x^2-x$ and $y=x$)
        \item \begin{equation*}
            \lim_{(x,y)\to (0,0)}\frac{\sin(xy)}{x+y}
        \end{equation*}
        % \item $f(x,y)=\frac{x}{x+y}$ as $(x,y)\to (0,0)$.
        % \item $g(x,y)=\frac{xy}{x+y}$ as $(x,y)\to (0,0)$.
        % \item $h(x,y)=\frac{\sin(xy)}{x+y}$ as $(x,y)\to (0,0)$. Recall that $\lim_{x\to 0}\frac{\sin(x)}{x}=1$.
    \end{enumerate}
\end{prob}
\begin{proof}
    \begin{enumerate}
        \item First fix $x=0$, let $y\to 0$, then the limit is $0$; now fix $y=0$, let $x\to 0$, the limit is $1$. The limit doesn't exist!
        \item Consider $y=x^2-x$, (as $x\to0, y\to 0$), then 
        \begin{equation*}
            \lim_{(x,y)\to(0,0)}\frac{xy}{x+y}=\lim_{x\to 0}\frac{x^3-x^2}{x^2}=\lim_{x\to 0}x-1=-1
        \end{equation*}
        and consider $y=x$, we see the limit is $0$, thus the limit doesn't exist!
        \begin{warn}
            2 does not follow from 1! A student suggests a proof: $\lim_{(x,y)\to(0,0)}=\frac{x}{x+y}\cdot y$, and by 1, the limit $\frac{x}{x+y}$ doesn't exist, this implies the limit of $\frac{xy}{x+y}$ also doesn't exist. This argument is not correct! Consider the following counterexample: $\lim_{y\to 0}\frac{1}{y}$ doesn't exist, but the limit
            \begin{equation*}
                \lim_{y\to 0}\frac{1}{y}\cdot y=1
            \end{equation*}
            exists! More concretely, if you multiply by any function that doesn't tend to $0$, the argument follows, but it doesn't work when the function tends to $0$!
        \end{warn}
        \item We see that 
        \begin{equation*}
            \lim_{(x,y)\to(0,0)}\frac{\sin(xy)}{xy}\frac{xy}{x+y}
        \end{equation*}
        Note that the limit of $\sin(xy)/(xy)=1$, but the second one doesn't exist, thus the limit doesn't exist!
    \end{enumerate}
\end{proof}


\noindent
\begin{idea}
How to find a a limit $\lim_{x\to x_0}f(x)$:
\begin{itemize}
    \item Step 1: Guess what the limit should be.
    \item Step 2: Try from approaching $x_0$ from different directions.
    \item Step 3: Try to replace terms with expressions you are familiar with.
\end{itemize}
\end{idea}














\end{document}