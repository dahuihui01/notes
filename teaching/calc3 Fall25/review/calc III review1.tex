\documentclass[openany]{book}

\usepackage[margin=1in]{geometry}
\usepackage{amsmath,amsfonts,amsthm, amssymb}
\usepackage{yhmath}
\usepackage{mathrsfs}
\usepackage{mathtools}
\usepackage{xcolor}
\usepackage{graphicx}
\usepackage{comment}
\usepackage{tikz-cd}
\usepackage{quiver}
\usepackage{hyperref,cleveref}
\renewcommand{\familydefault}{ppl}
\newcommand{\tr}{\text{tr}}
\newcommand{\R}{\mathbb{R}}
\newcommand{\E}{\mathbb{E}}
\newcommand{\Z}{\mathbb{Z}}
\newcommand{\C}{\mathbb{C}}
\newcommand{\F}{\mathbb{F}}
\newcommand{\la}{\langle}
\newcommand{\ra}{\rangle}
\newcommand{\colim}{\text{colim}}
\DeclareMathOperator{\im}{im}
\DeclareMathOperator{\disc}{disc}
\let\oldemptyset\emptyset
\let\emptyset\varnothing
\newcommand{\tor}{\text{Tor}}
\newcommand{\id}{\text{id}}
\newcommand{\ext}{\text{Ext}}
\newcommand{\ptop}{\text{PTop}}
\newcommand{\pt}{\text{pt}}
\newcommand{\ach}{\text{Ach}}
\newcommand{\Q}{\mathbb{Q}}
\newcommand{\gal}{\text{Gal}}
\newcommand{\fraccomma}{\genfrac{}{}{0pt}{}{}{,}}
\definecolor{wikipediadarkblue}{rgb}{0.023, 0.270, 0.676}
\hypersetup{
    colorlinks,
    citecolor=black,
    filecolor=black,
    linkcolor=wikipediadarkblue,
    urlcolor=red
}



\input{hui_r.tex}

\title{Calc III Midterm Review
\\ 
\vspace{0.4cm}
\large Fall 2025}



\author{(This document only contains materials before the midterm;\\
please email hsun95@jh.edu if you see typos)}
\date{\today}


\begin{document}

\maketitle

\tableofcontents
\newpage


\chapter{Definition review}



\begin{defn}[standard basis in $\R^3$]
    The vectors 
    \begin{equation*}
        i=(1,0,0), j=(0,1,0), k=(0,0,1)
    \end{equation*}
    are called the \textbf{standard basis} vectors of $\R^3$, and for any vector $a=(a_1,a_2,a_3)\in\R^3$, we can write 
    \begin{equation*}
        a=a_1i+a_2j+a_3k
    \end{equation*}
\end{defn}



\begin{defn}[Equation of a line]\label{line}
    A \textbf{line} $l$ in $\R^3$ through the tip of $a=(a_1,a_2,a_3)$ pointing in the direction of a vector $v=(v_1,v_2,v_3)$ is given by 
    \begin{equation*}
        l(t)=a+tv=(a_1+tv_1, a_2+tv_2, a_3+tv_3)
    \end{equation*}
    where $t\in\R$. Alternatively, a line passing through two points $P=(x_1,y_1,z_1), Q=(x_2,y_2,z_2)$ is given by 
    \begin{equation*}
        l(t)=(x(t), y(t), z(t))
    \end{equation*}
    where 
    \begin{equation*}
        \begin{cases}
            x(t)=x_1+(x_2-x_1)t\\
            y(t)=y_1+(y_2-y_1)t\\
            z(t)=z_1+(z_2-z_1)t
        \end{cases}
    \end{equation*}
\end{defn}



\begin{defn}[inner product, dot product]
    Let $a,b\in\R^3$, the \textbf{dot product}, also called the inner product, of $a,b$ is 
    \begin{equation*}
        a\cdot b=a_1b_1+a_2b_2+a_3b_3
    \end{equation*}
    where $a=(a_1,a_2,a_3), b=(b_1,b_2,b_3)$. The \textbf{norm}, also called the length, of $a$ is 
    \begin{equation*}
        \|a\|=(a\cdot a)^\frac{1}{2}
    \end{equation*}
    A vector of norm $1$ is called a \textbf{unit vector}. Given any $u\in\R^3$, we can find the unit vector $\frac{u}{\|u\|}$ pointing in the same direction as $u$, this is called ``normalizing'' $u$.
\end{defn}


\begin{defn}[orthogonal projection]
    The \textbf{orthogonal projection} of vector $v$ onto another vector $a$ is
    \begin{equation*}
        \text{Proj}_av=\frac{a\cdot v}{a\cdot a}a
    \end{equation*}
    For example, the orthogonal projection of $(1,1,0)$ onto $(1,1,1)$ is 
    \begin{equation*}
        \left(\frac{2}{3}, \frac{2}{3},\frac{2}{3}\right)
    \end{equation*}
\end{defn}


\begin{defn}[orthogonal]
    Let $a,b\in\R^n$, then $a,b$ are called \textbf{orthogonal} or perpendicular iff 
    \begin{equation*}
        a\cdot b=0
    \end{equation*}
\end{defn}


\begin{defn}[determinant]
    The \textbf{determinant} of a $2\times 2$ matrix is given by 
    \begin{equation*}
        \det\begin{bmatrix}
            a&b\\
            c&d
        \end{bmatrix}=ad-bc
    \end{equation*}
    and the determinant of a $3\times 3$ matrix is given by 
    \begin{equation*}
        \det\begin{bmatrix}
            a_{11}&a_{12}&a_{13}\\
            a_{21}&a_{22}&a_{23}\\
            a_{31}&a_{32}&a_{33}\\
        \end{bmatrix}=a_{11}(a_{22}a_{33}-a_{23}a_{32})-a_{12}(a_{21}a_{33}-a_{23}a_{31})+a_{13}(a_{21}a_{32}-a_{22}a_{31})
    \end{equation*}
\end{defn}



\begin{defn}[cross product]
    Let $a,b\in\R^3$, write $a=(a_1,a_2,a_3), b=(b_1,b_2,b_3)$, then the \textbf{cross product}
    \begin{equation*}
        a\times b=\det\begin{bmatrix}
            i&j&k\\
            a_1&a_2&a_3\\
            b_1&b_2&b_3
        \end{bmatrix}
    \end{equation*}
    where $i,j,k$ are the standard vectors in $\R^3$.
\end{defn}

\begin{defn}[Plane in $\R^3$]\label{plane}
   If a plane ${P}$ passes through some point $(x_0,y_0,z_0)$, and $n=(A,B,C)$ is a vector orthogonal to the plane, then the plane ${P}$ is given by the equation:
    \begin{equation*}
        A(x-x_0)+B(y-y_0)+C(z-z_0)=0
    \end{equation*}
    (Notice that a point in ${P}$ and a normal vector to ${P}$ uniquely define a plane in $\R^3$.)
\end{defn}


\begin{defn}[image, graph]
    The \textbf{image} of a function $f: U\subset\R^n\to\R^m$ is a subset of $\R^m$,
    \begin{equation*}
        \text{Image}(f)=\{f(x)\in\R^m: x\in U\}
    \end{equation*}
    and the \textbf{graph} of $f$ is a subset of $\R^{n+m}$,
    \begin{equation*}
        \text{Graph}(f)=\{(x,f(x)): x\in U\}
    \end{equation*}
\end{defn}


\begin{defn}[level set]
    Let $f:U\subset\R^n\to\R^m$, and $c\in\R$ be some constant. Then the \textbf{level set} of $f$ at $c$ is the set 
    \begin{equation*}
        \{x\in U: f(x)=c\}\subset\R^n
    \end{equation*}
\end{defn}

\begin{defn}[open set, closed set, neighborhood, boundary]
    Let $U\subset\R^n$, we say $U$ is an \textbf{open set} if for every $x_0\in U$, there exists some $r>0$ such that $D_r(x_0)\subset U$, where $D_r(x_0)$ is the open disk of radius $r$ centered at $x_0$:
    \begin{equation*}
        D_r(x_0)=\{x\in\R^n: \|x-x_0\|<r\}
    \end{equation*}
    Some examples of open sets: $\R$, $D_{1}((0,0))$, $(1,2)\subset\R$.
    A \textbf{neighborhood} of $x_0\in\R^n$ is an open set containing the point $x_0$. A point $x\in\R^n$ is called a \textbf{boundary point} of $A$ if \textit{every} neighborhood of $x$ contains at least one point in $A$ and at least one point not in $A$. A set is \textbf{closed} if it contains all its boundary points. Example of closed set: level sets of a continuous function $f$.
\end{defn}


\begin{defn}[limit]
    Let $f: A\subset\R^n\to\R^m$, where $A$ is open, let $x_0$ be in $A$ or be a boundary point of $A$ and $N$ be a neighborhood of a point $b\in\R^m$. Now let $x$ approach $x_0$, $f$ is said to be \textbf{eventually in $N$} if there exists a neighborhood $U$ of $x_0$ such that 
    \begin{equation*}
        \text{ if } x\in U, \text{ then } f(x)\in N
    \end{equation*}
    If $f$ is eventually in $N$ for \textit{any} neighborhood $N$ around $b$, then the \textbf{limit} of $f$ as $x\to x_0$ exists, denoted as 
    \begin{equation*}
        \lim_{x\to x_0}f(x)=b
    \end{equation*}
\end{defn}

\begin{defn}[continuous]
    Let $f:A\subset\R^n\to\R^m$ and $x_0\in A$, then $f$ is \textbf{continuous at} \boldmath{${x_0}$} \unboldmath if 
    \begin{equation*}
        \lim_{x\to x_0}f(x)=f(x_0)
    \end{equation*}
\end{defn}


\begin{defn}[partial derivative]
    Let $f:U\subset\R^n\to\R$, where $U$ is open. Then the \textbf{partial derivative} with respect to $x_i$ is defined by 
    \begin{equation*}
        \frac{\partial f}{\partial x_i}(x_1, \dots, x_n)=\lim_{h\to 0}\frac{f(x+he_i)-f(x)}{h}
    \end{equation*}
    where $e_i=(0,\dots, 1,\dots, 0)$ with $1$ in the $i$th coordinate. 
\end{defn}



\begin{defn}[differentiability in two variables]
    Let $f:\R^2\to\R$, then $f$ is \textbf{differentiable} at $(x_0,y_0)$ if 
    \begin{enumerate}
        \item[(1)] $\frac{\partial f}{\partial x},\frac{\partial f}{\partial y}$ exist at $(x_0,y_0)$
        \item[(2)] 
        \begin{equation*}
            \lim_{(x,y)\to(x_0,y_0)}\frac{f(x,y)-f(x_0,y_0)-\left[\frac{\partial f}{\partial x}(x_0,y_0)\right](x-x_0)-\left[\frac{\partial f}{\partial y}(x_0,y_0)\right](y-y_0)}{\|(x,y)-(x_0,y_0)\|}=0
        \end{equation*}
    \end{enumerate}
    The derivative of $f$ at $(x_0,y_0)$ is the $1\times 2$ matrix 
    \begin{equation*}
        \begin{bmatrix}\frac{\partial f}{\partial x}(x_0,y_0)&\frac{\partial f}{\partial y}(x_0,y_0)\end{bmatrix}
    \end{equation*}
    Moreover, the \textbf{tangent plane} of the graph of $f$ at $(x_0,y_0, f(x_0,y_0))$ is given by 
    \begin{equation*}
        z=f(x_0,y_0)+\left[\frac{\partial f}{\partial x}(x_0,y_0)\right](x-x_0)+\left[\frac{\partial f}{\partial y}(x_0,y_0)\right](y-y_0)
    \end{equation*}
\end{defn}

\begin{defn}[differentiability in the general setting]
    Let $f:U\subset\R^n\to\R^m$, then $f$ is differentiable at $x_0\in U$ if
    \begin{enumerate}
        \item[(1)] the partial derivatives $\frac{\partial f_i}{\partial x_j}$ exist for all $1\leq i\leq m, 1\leq j\leq n$. 
        \item[(2)]
        \begin{equation*}
            \lim_{x\to x_0}\frac{\|f(x)-f(x_0)-T(x-x_0)\|}{\|x-x_0\|}=0
        \end{equation*}
        where $T=Df(x_0)$ is the $m\times n$ matrix 
        \begin{equation*}
            Df(x_0)=
            \begin{bmatrix}
            \frac{\partial f_1}{\partial x_1}(x_0) & \cdots & \frac{\partial f_1}{\partial x_n}(x_0) \\
            \vdots & \ddots & \vdots \\
            \frac{\partial f_m}{\partial x_1}(x_0) & \cdots & \frac{\partial f_m}{\partial x_n}(x_0)
            \end{bmatrix}
        \end{equation*}
    \end{enumerate}
    The derivative of $f$ at $x_0$ is the $m\times n$ matrix $Df(x_0)$.
\end{defn}

\begin{defn}[gradient]
    Let $f:U\subset\R^n\to\R$, the \textbf{gradient} $\nabla f(x)$ is a special case of the general case above when $m=1$, i.e., it is a $1\times n$ matrix 
    \begin{equation*}
        Df(x)=\begin{bmatrix}
            \frac{\partial f}{\partial x_1}&\dots&\frac{\partial f}{\partial x_n}
        \end{bmatrix}
    \end{equation*}
\end{defn}


\begin{defn}[path and curve]
    A \textbf{path} in $\R^n$ is a map $c:[a,b]\to\R^n$, and the image of $c$ is called a \textbf{curve}. We say the path $c$ parametrizes the curve. 

    \noindent For example, $c(t)=(\cos t, \sin t)$ is a path, and the unit circle is a curve.
\end{defn}


\begin{defn}[velocity of a path]
    Let $c:[a,b]\to\R^n$ be a path, and we can write $c(t)=(c_1(t), \dots, c_n(t))$. If $c$ is differentiable, then we define the \textbf{velocity} of $c$ at any $t_0\in [a,b]$ as 
    \begin{equation*}
        c'(t_0)=\left(c_1'(t_0), \dots, c_n'(t_0)\right)
    \end{equation*}
    The velocity vector of $c$ at $t_0$ is also a \textbf{tangent} vector to $c$ at $t_0$. The \textbf{speed} of the path $c$ at $t_0$ is the length of the velocity vector $\|c'(t_0)\|$.
\end{defn}

\begin{defn}[tangent line to a path]\label{tangent path}
    Let $c:[a,b]\to\R^n$ be a path, if $c'(t_0)\neq 0$, then the \textbf{tangent line} at $x_0$ is given by 
    \begin{equation*}
        l(t)=c(t_0)+c'(t_0)(t-t_0)
    \end{equation*}
\end{defn}




\begin{defn}[directional derivative]
    Let $f:\R^3\to\R$, be differentiable, then the \textbf{directional derivative} at $x_0\in\R^3$ in the direction of a \textit{unit vector} $v$ is given by 
    \begin{equation*}
        \nabla f(x_0)\cdot v=\left[\frac{\partial f}{\partial x_1}(x_0)\right]v_1+\left[\frac{\partial f}{\partial x_2}(x_0)\right]v_2+\left[\frac{\partial f}{\partial x_3}(x_0)\right]v_3
    \end{equation*}
    where $v=(v_1,v_2,v_3)$.
\end{defn}
\begin{warn}
    Make sure you normalize any given direction $v$! This formula works for unit vectors.
\end{warn}



\begin{defn}[First order Taylor expansion]
    Let $f:U\subset\R^n\to\R$ be differentiable at $a\in U$, then 
    \begin{equation*}
        f(x)=f(a)+\sum_{i=1}^n\frac{\partial f}{\partial x_i}(a)(x_i-a_i)+R_1(a,x)
    \end{equation*}
    where 
    \begin{equation*}
        \frac{R_1(a,x)}{\|x-a\|}\to 0 \text{ as } x\to a
    \end{equation*}
\end{defn}

\begin{defn}[Second order Taylor expansion]\label{taylor}
    Let $f:U\subset\R^n\to\R$ be twice continuously differentiable at $a\in U$, then 
    \begin{equation*}
        f(x)=f(a)+\sum_{i=1}^n\frac{\partial f}{\partial x_i}(a)(x_i-a_i)+\frac{1}{2}\sum_{i,j=1}^n\frac{\partial^2 f}{\partial x_i\partial x_j}(a)(x_i-a_i)(x_j-a_j)+R_2(a,x)
    \end{equation*}
    where 
    \begin{equation*}
        \frac{R_2(a,x)}{\|x-a\|}\to 0 \text{ as } x\to a
    \end{equation*}
\end{defn}



\begin{defn}[critical point]
    Let $f:U\subset\R^n\to\R$, a point $x_0\in U$ is a \textbf{critical point} of $f$ if either $f$ is not differentiable at $x_0$, or $Df(x_0)=0$. A critical point that is not a local extremum is called a saddle point.
\end{defn}

\begin{defn}[quadratic function]
    A function $g:\R^n\to\R$ is called a \textbf{quadratic function} if it is given by 
    \begin{equation*}
        g(h_1,\dots, h_n)=\sum_{i,j=1}^na_{ij}h_ih_j
    \end{equation*}
    where $(a_{ij})$ is an $n\times n$ matrix. We can also write $g$ as follows:
    \begin{equation*}
        g(h_1,\dots,h_n)=[h_1,\dots,h_n]\begin{bmatrix}
            a_{11}&\dots &a_{1n}\\
            \vdots&\ddots &\vdots\\
            a_{n_1}&\dots&a_{nn}
        \end{bmatrix}\begin{bmatrix}
            h_1\\
            \vdots\\
            h_n
        \end{bmatrix}
    \end{equation*}

\end{defn}

\begin{defn}[Hessian matrix]
    Let $f:U\subset\R^n\to\R$, and suppose all the second-order partial derivatives $\frac{\partial^2 f}{\partial x_i\partial x_j}$ exist, then the Hessian matrix of $f$ is the $n\times n$ matrix given by 
    \begin{equation*}
        Hf=\begin{bmatrix}
            \frac{\partial^2 f}{\partial x_1\partial x_1}&\dots&\frac{\partial^2f}{\partial x_1\partial x_n}\\
            \vdots&\ddots&\vdots\\
            \frac{\partial^2f}{\partial x_nx_1}&\dots&\frac{\partial^2f}{\partial x_n\partial x_n}
        \end{bmatrix}
    \end{equation*}
    The Hessian as a quadratic function is defined by 
    \begin{equation*}
        Hf(x)(h)=\frac{1}{2}\begin{bmatrix}
            h_1&\dots&h_n
        \end{bmatrix}Hf(x)\begin{bmatrix}
            h_1\\
            \dots\\
            h_n
        \end{bmatrix}
    \end{equation*}
    where $h=(h_1,\dots,h_n)$.
\end{defn}

\begin{defn}[degenerate/nondegenerate points]
    Let $f:U\subset\R^2\to\R$ be of $C^2$, let $(x_0,y_0)$ be a critical point. We define the \textbf{discriminant}, \boldmath $\mathcal{D}$,\unboldmath of the Hessian by
    \begin{equation*}
        \mathcal{D}=\det (Hf)=\left(\frac{\partial^2f}{\partial x^2}\right)\left(\frac{\partial^2f}{\partial y^2}\right)-\left(\frac{\partial^2f}{\partial x\partial y}\right)^2
    \end{equation*}
    If $\mathcal{D}\neq 0$, the critical point $(x_0,y_0)$ is called \textbf{nondegenerate}; if $\mathcal{D}=0$, the point $(x_0,y_0)$ is called \textbf{degenerate}.
\end{defn}

\begin{defn}[positive, negative-definite]
    A quadratic function $g:\R^n\to\R$ is called \textbf{positive-definite} if $g(h)\geq 0$ for all $h\in\R^n$ and $g(h)=0$ implies $h=0$. Similarly, $g$ is \textbf{negative-definite} if $g(h)\leq 0$ for all $h\in\R^n$ and $g(h)=0$ implies $h=0$. 
\end{defn}


\begin{defn}[global extremum]
    Let $f:A\to\R$ be a function defined on $A\subset\R^2$ or $A\subset\R^3$. A point $x_0\in A$ is said to be an \textbf{absolute maximum} if $f(x_0)\geq f(x)$ for all $x\in A$. Similarly, $x_0$ is an \textbf{absolute minimum} if $f(x_0)\leq f(x)$ for all $x\in A$. 
\end{defn}

\begin{defn}[bounded set]
    A set $A\subset\R^n$ is said to be \textbf{bounded} if there is a number $M>0$ such that $\|x\|\leq M$ for all $x\in A$. 
\end{defn}










































\newpage

\chapter{Theorem Review}
\begin{prop}[dot product]\label{dot}
    Let $a,b\in\R^3$, and let $\theta$ be the angle between $a,b$, where $0\leq\theta\leq\pi$, then 
    \begin{equation*}
        a\cdot b=\|a\|\|b\|\cos\theta
    \end{equation*}
\end{prop}




\begin{prop}[properties of the dot product]
    Let $a,b,c\in\R^n$, then 
    \begin{enumerate}
        \item[(a)] Nonnegativity: $a\cdot a\geq 0$, and $a\cdot a=0$ if and only if $a=0$.
        \item[(b)] Scalar multiplication: let $\lambda\in\R$, then 
        \begin{equation*}
            \lambda(a\cdot b)=\lambda a\cdot b=a\cdot \lambda b
        \end{equation*}
        \item[(c)] Distributivity:
        \begin{equation*}
            a\cdot(b+c)=a\cdot b+a\cdot c, \quad (a+b)\cdot c=a\cdot c+b\cdot c
        \end{equation*}
        \item[(d)] Symmetry: $a\cdot b=b\cdot a$.
    \end{enumerate}
\end{prop}



\begin{prop}[Cauchy-Schwarz]
    Let $a,b\in\R^n$, then $a\cdot b\in\R$, 
    \begin{equation*}
        |a\cdot b|\leq\|a\|\|b\|
    \end{equation*}
    where the left hand side is the absolute value of $a\cdot b$, and the right hand side is multiplication of two nonnegative real numbers.
\end{prop}


\begin{prop}[triangle inequality]\label{traingle}
    Let $a,b\in\R^n$, then 
    \begin{equation*}
        \|a+b\|\leq\|a\|+\|b\|
    \end{equation*}
\end{prop}



\begin{prop}[cross product]
    We have the following properties regarding the cross product: let $a,b\in\R^3$,
    \begin{enumerate}
        \item $a\times b$ is perpendicular to vectors $a,b$.
        \item The length of the cross product is the area of the parallelogram:
        \begin{equation*}
            \|a\times b\|=\|a\|\|b\|\sin\theta
        \end{equation*}
        where $0\leq\theta\leq\pi$ is the angle between them. 
        \item $a\times b=-b\times a$, $(a+b)\times c=a\times c+b\times c$, and $a\times (b+c)=a\times b+a\times c$. Moreover, $a\times b=0$ iff $a,b$ are parallel or either $a$ or $b$ are $0$.
        \item The cross product is \textbf{not} associative! For example, compute 
        \begin{equation*}
            (i\times i)\times j, \quad i\times (i\times j)
        \end{equation*}
    \end{enumerate}
\end{prop}



\begin{prop}[limits]
    Here are some properties of limits: let $f: U_1\subset\R^n\to\R^m, g: U_2\subset\R^n\to\R^m$,
    \begin{enumerate}
        \item[(a)] (Uniquessness):  \begin{equation*}
            \text{ If } \lim_{x\to x_0}f(x)=b_1, \quad \lim_{x\to x_0}f(x)=b_2
        \end{equation*}
        then we must have 
        \begin{equation*}
            b_1=b_2
        \end{equation*}
        \item[(b)] (Scalar multiplication): Let $c\in\R$, if $\lim_{x\to x_0}f(x)=b_1$, then 
        \begin{equation*}
            \lim_{x\to x_0}cf(x)=cb_1
        \end{equation*}
        \item[(c)] (Addition): Let $f$ be as in (b), and $\lim_{x\to x_0}g(x)=b_2$, then 
        \begin{equation*}
            \lim_{x\to x_0}(f+g)(x)=b_1+b_2
        \end{equation*}
        \item[(d)] (Component): Write $f(x)=(f_1(x),\dots, f_n(x))$, if $\lim_{x\to x_0}f(x)=b=(b_1,\dots, b_n)$, then 
        \begin{equation*}
            \lim_{x\to x_0}f_i(x)=b_i
        \end{equation*}
        for all $i=1,\dots, m$.
    \end{enumerate}
    The same set of properties hold for continuity.
\end{prop}

\begin{prop}[continuity of compositions]
    Let $g: A\subset\R^n\to\R^m$, and $f: B\subset\R^m\to\R^p$, and $g(A)\subset B$. If $g$ is continuous at $x_0$, $f$ is continuous at $g(x_0)$, then $f\circ g$ is continuous at $x_0$.
\end{prop}

\begin{prop}[differentiability implies continuity]
    Let $f:U\subset\R^n\to\R^m$. If $f$ is differentiable at $x_0\in U$, then $f$ is continuous at $x_0$.
\end{prop}

\begin{prop}[differentiability]
    Let $f:U\subset\R^n\to\R^m$. Suppose $\partial f_i/\partial x_j$ exists for all $i,j$ and are continuous in a neighborhood of $x_0\in U$, then $f$ is differentiable at $x_0$.
\end{prop}



\begin{prop}[properties of derivatives]
    Let $f:U\subset\R^n\to\R^m$ be differentiable at $x_0$, then the derivative of $f$ at $x_0$ is an $m\times n$ matrix $Df(x_0)=\left(\frac{\partial f_i}{\partial x_j}\right)_{ij}$. The derivative follows the same properties as derivative for single variable functions:
    \begin{enumerate}
        \item Let $c\in\R$, then 
        \begin{equation*}
            D(cf)(x_0)=cDf(x_0) \tag {multiplication of a matrix by constant $c$}
        \end{equation*}
        \item Let $g: U\subset\R^n\to\R^m$ also be differentiable at $x_0$, then 
        \begin{equation*}
            D(f+g)(x_0)=Df(x_0)+Dg(x_0) \tag{sum of two matrices}
        \end{equation*}
        \item Let $h_1: U\subset\R^n\to\R, h_2: U\subset\R^n\to\R$,then 
        \begin{equation*}
            D(h_1h_2)(x_0)=Dh_1(x_0)h_2(x_0)+h_1(x_0)Dh_2(x_0) \tag{product rule}
        \end{equation*}
        and if $h_2\neq 0$ on $U$.
        \begin{equation*}
            D(h_1/h_2)(x_0)=\frac{Dh_1(x_0)h_2(x_0)-h_1(x_0)Dh_2(x_0)}{h_2^2(x_0)} \tag{quotient rule}
        \end{equation*}
        \item Let $g: U\subset\R^n\to\R^m, f:V\subset\R^m\to\R^p$ such that $g(U)\subset V$, then 
        \begin{equation*}
            D(f\circ g)(x_0)=Df(g(x_0))Dg(x_0) \tag{chain rule}
        \end{equation*}
    \end{enumerate}
\end{prop}



\begin{prop}[fastest rate of change]
    Suppose that $\nabla f(x_0)\neq 0$, then the direction for which $f$ increases the fastest at $x_0$ is along $\nabla f(x_0)$.
\end{prop}


\begin{prop}[gradient is normal, tangent plane]\label{tangent plane}
    Let $f:\R^3\to\R$ be differentiable, let $S$ be a level surface of $f$, i.e., $S$ is a surface described by 
    \begin{equation*}
        f(x,y,z)=k
    \end{equation*}
    were $k$ is some constant. Let $(x_0,y_0,z_0)\in S$, then
    \begin{equation*}
        \nabla f(x_0,y_0,z_0) \text{ is \textbf{normal} to the level surface at } (x_0,y_0,z_0)
    \end{equation*}
    This means if $c(t)$ is a path in $S$, and $v(0)=(x_0,y_0,z_0)$, and if $v$ is a tangent vector to $c(t)$ at $t=0$, then 
    \begin{equation*}
        \nabla f(x_0,y_0, z_0)\cdot v=0
    \end{equation*}
    Moreover, if $\nabla f(x_0,y_0,z_0)\neq 0$, the \textbf{tangent plane} of $S$ at $(x_0,y_0,z_0)$ is given by 
    \begin{equation*}
        \nabla f(x_0,y_0,z_0)\cdot (x-x_0, y-y_0, z-z_0)=0
    \end{equation*}
\end{prop}

\begin{prop}[Equality of mixed partials]
    If $f(x,y)$ is twice continuously differentiable, then 
    \begin{equation*}
        \frac{\partial^2 f}{\partial x\partial y}=\frac{\partial^2f}{\partial y\partial x}
    \end{equation*}
\end{prop}


\begin{prop}[extremums are critical points]
    Let $f:U\subset\R^n\to\R$ be differentiable, where $U$ is open. If $x_0$ is a local extremum, then $Df(x_0)=0$. 
\end{prop}


\begin{prop}[extremum]
    Let $f\colon U\subset\R^n\to\R$ be in $C^3$, and $x_0$ is a critical point of $f$. If the Hessian $Hf(x_0)$ is positive-definite, then $x_0$ is a local minimum of $f$; if $Hf(x_0)$ is negative-definite, then $x_0$ is a local maximum.
\end{prop}




\begin{prop}[local minimum]
    Let $f(x,y)$ be of $C^2$, and $U$ is open in $\R^2$. A point $(x_0,y_0)$ is a strict local \textbf{minimum} of $f$ if the following conditions hold:
    \begin{enumerate}
        \item \begin{equation*}
            \frac{\partial f}{\partial x}(x_0,y_0)=\frac{\partial f}{\partial y}(x_0,y_0)=0
        \end{equation*}
        \item \begin{equation*}
            \mathcal{D}(x_0,y_0)>0
        \end{equation*}
        where $\mathcal{D}$ is the \textbf{discriminant} of the Hessian, defined by 
        \begin{equation*}
            \mathcal{D}=\det (Hf)=\left(\frac{\partial^2 f}{\partial x^2}\right)\left(\frac{\partial^2 f}{\partial y^2}\right)-\left(\frac{\partial^2 f}{\partial x\partial y}\right)^2
        \end{equation*}
        where $Hf$ is the $2\times 2$ Hessian matrix.
        \item  \begin{equation*}
            \frac{\partial^2f}{\partial x^2}(x_0,y_0)>0
        \end{equation*}
    \end{enumerate}
\end{prop}


\begin{prop}[local maximum]
    Let $f(x,y)$ be of $C^2$, and $U$ is open in $\R^2$. A point $(x_0,y_0)$ is a strict local \textbf{maximum} of $f$ if the following conditions hold:
    \begin{enumerate}
        \item \begin{equation*}
            \frac{\partial f}{\partial x}(x_0,y_0)=\frac{\partial f}{\partial y}(x_0,y_0)=0
        \end{equation*}
        \item \begin{equation*}
            \mathcal{D}(x_0,y_0)>0
        \end{equation*}
        where $\mathcal{D}$ is the {discriminant} of the Hessian, defined above.
        \item  \begin{equation*}
            \frac{\partial^2f}{\partial x^2}(x_0,y_0)<0
        \end{equation*}
    \end{enumerate}
\end{prop}

\begin{prop}[saddle points]
    Let $f(x,y):U\subset\R^2\to \R$ be of  $C^2$, if $\frac{\partial f}{\partial x}(x_0,y_0)=\frac{\partial f}{\partial y}(x_0,y_0)=0$, and $\mathcal{D}(x_0,y_0)<0$, where $\mathcal{D}$ is the discriminant, then the critical point $(x_0,y_0)$ is a saddle point, i.e., neither a maximum or a minimum.
\end{prop}


\begin{prop}[continuous functions attain extremum on closed bounded sets]
    Let $f:D\to\R$ be continuous, where $D$ is closed and bounded in $\R^n$. Then $f$ assumes its absolute maximum and absolute minimum values at some point $x_0, x_1\in D$.
\end{prop}


















\chapter{Practice Problems}


\begin{prob}
    Find the equation of the line passing through $(1,0,2)$ in the direction $(2,-1,3)$.
\end{prob}
% \begin{proof}
%     By definition \ref{line}, the line is given by 
%     \begin{equation*}
%         l(t)=(1+2t, -t, 2+3t)
%     \end{equation*}
% \end{proof}

\begin{prob}
    In which direction does the line 
    \begin{equation*}
        l(t)=(3-2t, 2+5t, 1+t)
    \end{equation*}
    point?
\end{prob}
% \begin{proof}
%     In the direction of the vector $(-2, 5, 1)$.
% \end{proof}


\begin{prob}
    Do the following two lines intersect?
    \begin{equation*}
        l_1(t)=(1+2t, 2+t, 3-t), \quad l_2(s)=(3-s, 4-s, 2+s)
    \end{equation*}
\end{prob}
% \begin{proof}
%     For them to intersect, we must have $t,s$ such that 
%     \begin{equation*}
%         \begin{cases}
%             1+2t=3-s \quad (1)\\
%             2+t=4-s \quad (2)\\
%             3-t=2+s \quad (3)
%         \end{cases}
%     \end{equation*}
%     $(2)-(1)$ gives $-t+1=1$, which implies $t=0, s=2$, but this does not satisfy $(3)$, hence these two lines do not intersect!
% \end{proof}


\begin{prob}
    Do the following points lie on the same line?
    \begin{equation*}
        A=(1,0,1),\quad B=(2,1,1), \quad C=(0,-1,1)
    \end{equation*}
\end{prob}
% \begin{proof}
%     We can find the unique line passing through $A,B$ by the equation given in\ref{line}
%     \begin{equation*}
%         l(t)=(1,0,1)+(1,1,0)t
%     \end{equation*}
%     then for $C$ to lie on this line, there must exists some $t$ such that 
%     \begin{equation*}
%         \begin{cases}
%             1+t=0\\
%             t=-1\\
%             1=1
%         \end{cases}
%     \end{equation*}
%     and $t=-1$ satisfies. This means all three points lie on the same line!
% \end{proof}


\begin{prob}
    Find the angle between two vectors $(1,2,0), (3,1,1)$.
\end{prob}
% \begin{proof}
%     By Proposition \ref{dot} 
%     \begin{equation*}
%         \cos\theta=\frac{a\cdot b}{\|a\|\|b\|}=\frac{5}{\sqrt{5}\sqrt{11}}=\sqrt{\frac{5}{11}}
%     \end{equation*}
%     hence 
%     \begin{equation*}
%         \theta=\arccos\left(\sqrt{\frac{5}{11}}\right)
%     \end{equation*}
% \end{proof}

\begin{prob}
    Let $b=(2,1,3)$ and $P$ be the plane through the origin given by $x+y+2z=0$. 
    \begin{enumerate}
        \item[(a)] Find two distinct vectors $v_1,v_2$ that are orthogonal in $P$.
        \item[(b)] Find the projection of $b$ onto the plane $P$, namely, 
        \begin{equation*}
            \text{Proj}_{v_1}b+\text{Proj}_{v_2}b
        \end{equation*}
    \end{enumerate}
\end{prob}
% \begin{proof}
%     \begin{enumerate}
%         \item[(a)] We can let $v_1=(1,-1,0), v_2=(1,1,-1)$. One can verify that $v_1,v_2\in P$ and $v_1\cdot v_2=0$.
%         \item[(b)] The projection is given by
%         \begin{align*}
%             \text{Proj}_{v_1}b+\text{Proj}_{v_2}b&=\frac{v_1\cdot b}{v_1\cdot v_1}v_1+\frac{v_2\cdot b}{v_2\cdot v_2}v_2\\
%             &=\frac{1}{2}(1,-1,0)+0\\
%             &=\left(\frac{1}{2}, -\frac{1}{2},0\right)
%         \end{align*}
%     \end{enumerate}
% \end{proof}


\begin{prob}
    Find a unit vector orthognal to both vectors $a=(1,2,-1), b=(2,3,-1)$.
\end{prob}
% \begin{proof}
%     The cross product is orthognal to both of the vectors:
%     \begin{equation*}
%         a\times b=\det\begin{bmatrix}
%             i&j&k\\
%             1&2&-1\\
%             2&3&-1
%         \end{bmatrix}=(1,-1, -1)
%     \end{equation*}
% \end{proof}


\begin{prob}
    Find the equation of the plane containing all three points below:
    \begin{equation*}
        P=(2,1,-1), \quad Q=(1,0,-2), \quad T=(3,2,1)
    \end{equation*}
\end{prob}
% \begin{proof}
%     We can find two vectors in this plane:
%     \begin{equation*}
%         \text{PQ}=Q-P=(-1,-1,-1), \text{ PT}=T-Q=(1,1,2)
%     \end{equation*}
%     then we can find a normal vector $n$ to the plane by taking the cross product:
%     \begin{equation*}
%         n=\text{PQ}\times\text{PT}=\det\begin{bmatrix}
%             i&j&k\\
%             -1&-1&-1\\
%             1&1&2
%         \end{bmatrix}=(-1,1,0)
%     \end{equation*}
%     Then by Definition \ref{plane}, using point $Q$, we see the plane can be written as 
%     \begin{equation*}
%         -1(x-1)+y=0
%     \end{equation*}
%     simplifying we get $x-y=1$.
% \end{proof}

\begin{prob}
    \begin{enumerate}
        \item[(a)] Find an equation for the line that passes through the point $(1,1,0)$ and is perpendicular to the plane $3x+y-2z+1=0$.
        \item[(b)] Find an equation for the plane that contains the line 
        \begin{equation*}
            l(t)=(-1+t, 2+2t, 1+3t)
        \end{equation*}
        and is perpendicular to the plane 
        \begin{equation*}
            2x+y-z+1=0
        \end{equation*}
    \end{enumerate}
\end{prob}
% \begin{proof}
%     \begin{enumerate}
%         \item[(a)] A normal vector to the plane $3x+y-2z+1$ is $(3,1,-2)$, since the line is perpendicular to the plane, the line is parallel along the direction $(3,1,-2)$. Now the line passes through $(1,1,0)$, thus we have the equation for the line 
%         \begin{equation*}
%             l(t)=(1,1,0)+t(3,1,-2)
%         \end{equation*}
%         \item[(b)] A normal vector to $2x+y-z$ is $n=(2,1,-1)$, and since our plane is perpendicular to this, it is parallel to the vector $n$. Thus a normal vector to our plane must be orthogonal to both $n$ and $(1,2,3)$, where the latter is given by the line in the plane. Thus taking the cross product:
%         \begin{equation*}
%             n_1=n\times (1,2,3)=(5,-7,3)
%         \end{equation*}
%         Hence the equation for the plane is given by:
%         \begin{equation*}
%             5(x+1)-7(y-2)+3(z-1)=0
%         \end{equation*}
%         simplifying we get $5x-7y+3z+16=0$.
%     \end{enumerate}
% \end{proof}

\begin{prob}
    Compute the area of the parallelogram spanned by the vectors $(1,1,0), (0,2,1)$.
\end{prob}
% \begin{proof}
%     Since we know 
%     \begin{equation*}
%         \|u\times v\|=\|u\|\|v\|\sin\theta
%     \end{equation*}
%     the length of the cross product is exactly the area of the parallelogram, thus computing
%     \begin{equation*}
%         \|(1,1,0)\times(0,2,1)\|=\|(1,-1,2)\|=\sqrt{6}
%     \end{equation*}
% \end{proof}


\begin{prob}
    Use the triangle inequality \ref{traingle} to show the reverse triangle inequality:
    \begin{equation*}
        \bigg|\|a\|-\|b\|\bigg|\leq\|a-b\|
    \end{equation*}
\end{prob}
% \begin{proof}
%     We know by traingle inequality 
%     \begin{align*}
%         \|a\|&=\|(a-b)+b\|\\
%         &\leq \|a-b\|+\|b\|
%     \end{align*}
%     rearranging, we get $\|a\|-\|b\|\leq\|a-b\|$. Similarly 
%     \begin{equation*}
%         \|b\|-\|a\|\leq \|a-b\|
%     \end{equation*}
%     Together this implies 
%     \begin{equation*}
%         \bigg|\|a\|-\|b\|\bigg|\leq\|a-b\|
%     \end{equation*}
% \end{proof}






\begin{prob}
    Compute the following limits if they exist; if the limits don't exist, please explain why.
    \begin{enumerate}
        \item \begin{equation*}
            \lim_{(x,y)\to(2,1)}\frac{x^2+y^2-2xy}{x-y}
        \end{equation*}
        \item \begin{equation*}
            \lim_{(x,y)\to(0,0)}\frac{\cos x-1}{x^2+y^2}
        \end{equation*}
        \item \begin{equation*}
            \lim_{(x,y)\to (0,0)}\frac{(x-y)^2}{(x+y)^2}
        \end{equation*}
        \item \begin{equation*}
            \lim_{(x,y)\to(0,0)}\frac{\sin 2x-2x+y}{x^3+y}
        \end{equation*}
        \item \begin{equation*}
            \lim_{(x,y,z)\to(0,0,0)}\frac{2x^2y\cos z}{x^2+y^2}
        \end{equation*}
        \item \begin{equation*}
            \lim_{(x,y)\to(2,1)}\frac{x^2-2xy}{x^2-4y^2}
        \end{equation*}
        \item \begin{equation*}
            \lim_{(x,y)\to(0,0)}\frac{x^2-y^6}{xy^3}
        \end{equation*}
    \end{enumerate}
\end{prob}
% \begin{proof}
%     \begin{enumerate}
%         \item \begin{equation*}
%             \lim_{(x,y)\to(2,1)}\frac{x^2+y^2-2xy}{x-y}=\lim_{(x,y)\to(2,1)}\frac{(x-y)^2}{x-y}=\lim_{(x,y)\to(2,1)}x-y=1
%         \end{equation*}
%         \item The limit doesn't exist, \begin{equation*}
%             \lim_{(x,y)\to(0,0)}\frac{\cos x-1}{x^2+y^2}
%         \end{equation*}
%         Consider the path $x=0, y\to 0$, we have 
%         \begin{equation*}
%             \lim_{x=0,y\to 0}\frac{0}{y^2}=0
%         \end{equation*}
%         Consider the path $y=0, x\to 0$, 
%         \begin{equation*}
%             \lim_{y=0, x\to 0}\frac{\cos x-1}{x^2}=\lim_{x\to 0}\frac{-\sin x}{2x}=\lim_{x\to 0}\frac{-\cos x}{2}=-\frac{1}{2}
%         \end{equation*}
%         \item The limit doesn't exist, \begin{equation*}
%             \lim_{(x,y)\to (0,0)}\frac{(x-y)^2}{(x+y)^2}
%         \end{equation*}
%         Consider the path $x=0, y\to 0$, 
%         \begin{equation*}
%             \lim_{x=0,y\to 0}\frac{y^2}{y^2}=1
%         \end{equation*}
%         Consider the path $y=x\to 0$, 
%         \begin{equation*}
%             \lim_{x=y\to 0}\frac{0}{4x^2}=0
%         \end{equation*}
%         \item The limit doesn't exist, \begin{equation*}
%             \lim_{(x,y)\to(0,0)}\frac{\sin 2x-2x+y}{x^3+y}
%         \end{equation*}
%         Consider the path $x=0, y\to 0$,
%         \begin{equation*}
%             \lim_{x=0, y\to 0}\frac{y}{y}=1
%         \end{equation*}
%         Consider the path $y=0, x\to 0$, 
%         \begin{align*}
%             \lim_{y=0, x\to 0}\frac{\sin 2x-2x}{x^3}&=\lim_{x\to 0}\frac{2\cos 2x-2}{3x^2}\\
%             &=\lim_{x\to 0}\frac{-4\sin 2x}{6x}\\
%             &=\lim_{x\to 0}\frac{-8\cos 2x}{6}\\
%             &=-\frac{4}{3}
%         \end{align*}
%         \item \begin{equation*}
%             \lim_{(x,y,z)\to(0,0,0)}\frac{2x^2y\cos z}{x^2+y^2}
%         \end{equation*}
%         Writing $x=r\cos\theta, y=r\sin\theta$ in polar coordinates, we can rewrite this as 
%         \begin{equation*}
%             \left|\frac{2r^3\cos^2\theta\sin\theta\cos z}{r^2}\right|=|2r\cos^2\theta\sin\theta\cos z|\leq 2r\to 0
%         \end{equation*}
%         as $(x,y,z)\to(0,0,0)$. Thus the limit is $0$.
%         \item \begin{equation*}
%             \lim_{(x,y)\to(2,1)}\frac{x^2-2xy}{x^2-4y^2}
%         \end{equation*}
%         We factor:
%         \begin{equation*}
%             \lim_{(x,y)\to(2,1)}\frac{x^2-2xy}{x^2-4y^2}=\lim_{(x,y)\to(2,1)}\frac{(x-2y)x}{(x+2y)(x-2y)}=\lim_{(x,y)\to(2,1)}\frac{x}{x+2y}=\frac{1}{2}
%         \end{equation*}
%         \item The limit doesn't exist, \begin{equation*}
%             \lim_{(x,y)\to(0,0)}\frac{x^2-y^6}{xy^3}
%         \end{equation*}
%         Consider $x=y\to 0$, then 
%         \begin{equation*}
%             \lim_{x=y\to 0}\frac{x^2-x^6}{x^4}=\lim_{x\to 0}\frac{1-x^4}{x^2}=\infty
%         \end{equation*}
%         Consider $x=y^3\to 0$, then 
%         \begin{equation*}
%             \lim_{x=y^3\to 0}\frac{0}{y^6}=0
%         \end{equation*}
%     \end{enumerate}
% \end{proof}


\begin{prob}
    \begin{enumerate}
        \item[(a)] Show that $f:\R\to\R$,
        \begin{equation*}
            f(x)=(1-x)^8+\cos(1+x^3)
        \end{equation*}
        is continuous.
        \item[(b)] Show $f:\R\to\R$, 
        \begin{equation*}
            f(x)=\frac{x^2e^x}{2-\sin x}
        \end{equation*}
        is continuous.
    \end{enumerate}
\end{prob}
% \begin{proof}
%     \begin{enumerate}
%         \item[(a)] $(1-x)^8$ is a polynomial, thus continuous, and $\cos x, 1+x^3$ are both continuous, thus the composition $\cos(1+x^3)$ is also continuous. Thus adding continuous functions gives another continuous function.
%         \item[(b)] $x^2e^x, 2-\sin x$ are both continuous, and $\frac{x^2e^x}{2-\sin x}$ is continuous if $2-\sin x\neq 0$ for all $x$. This is indeed true because $-1\leq\sin x\leq 1$, thus $1\leq 2-\sin x\leq 3$.
%     \end{enumerate}
% \end{proof}


\begin{prob}
    Compute all the partial derivatives.
    \begin{enumerate}
        \item $w=e^{xy}\log(x^2+y^2)$.
        \item $w=\cos(ye^{xy})\sin x$.
    \end{enumerate}
\end{prob}
% \begin{proof}
%     \begin{enumerate}
%         \item \begin{equation*}
%             \frac{\partial w}{\partial x}=ye^{xy}\ln(x^2+y^2)+e^{xy}\frac{2x}{x^2+y^2}
%         \end{equation*}
%         and 
%         \begin{equation*}
%             \frac{\partial w}{\partial y}=xe^{xy}\ln(x^2+y^2)+e^{xy}\frac{2y}{x^2+y^2}
%         \end{equation*}
%         \item \begin{equation*}
%             \frac{\partial w}{\partial x}=-y^2e^{xy}\sin (ye^{xy})\sin x+\cos(ye^{xy})\cos x
%         \end{equation*}
%         and 
%         \begin{equation*}
%             \frac{\partial w}{\partial y}=-(1+xy)e^{xy}\sin(ye^{xy})\sin x
%         \end{equation*}
%     \end{enumerate}
% \end{proof}


\begin{prob}
    Compute the gradient of $h(x,y,z)=(x+z)e^{x-y}$ at $(1,1,0)$.
\end{prob}
% \begin{proof}
%     The gradient is 
%     \begin{align*}
%         \nabla h(x,y,z)&=\begin{bmatrix}
%             \frac{\partial h}{\partial x}&\frac{\partial h}{\partial y}&\frac{\partial h}{\partial z}
%         \end{bmatrix}\\
%         &=\begin{bmatrix}
%             e^{x-y}(1+x+z)&-(x+z)e^{x-y}&e^{x-y}
%         \end{bmatrix}
%     \end{align*}
%     Thus 
%     \begin{equation*}
%         \nabla h(1,1,0)=\begin{bmatrix}
%             2&-1&1
%         \end{bmatrix}
%     \end{equation*}
% \end{proof}





\begin{prob}
    Determine the velocity vector of the given path:
    \begin{equation*}
        c(t)=(\cos 2t, 3t^2-t, -t)
    \end{equation*}
\end{prob}
% \begin{proof}
%     It is given by 
%     \begin{equation*}
%         c'(t)=(-2\sin 2t, 6t-1, -1)
%     \end{equation*}
% \end{proof}


\begin{prob}
    Find the tangent line to the given path at $t=0$
    \begin{equation*}
        c(t)=(e^t\sin t, 2t, -t^3)
    \end{equation*}
\end{prob}
% \begin{proof}
%     By the equation in Definition \ref{tangent path}, we have 
%     \begin{equation*}
%         c'(t)=(e^t\sin t+e^t\cos t, 2, -3t^2)
%     \end{equation*}
%     and $c(0)=(0,0,0), c'(0)=(1,2,0)$. Thus the tangent line is given by 
%     \begin{equation*}
%         l(t)=(t,2t,0)
%     \end{equation*}
% \end{proof}



\begin{prob}
    Compute the derivatives.
    \begin{enumerate}
        \item Let
        \begin{equation*}
            f(u,v)=u^2v+2v, \quad u=-x^2+y, v=x+y
        \end{equation*}
        Compute $\frac{\partial f}{\partial x},\frac{\partial f}{\partial y}$.
        \item Let 
        \begin{equation*}
            g(u,v)=(e^u, u+\sin v), \quad f(x,y,z)=(x^2, yz)
        \end{equation*}
        Compute $D(g\circ f)$ at $(0,1,0)$.
        \item Let $f:\R^3\to\R$ and $c(t)=\R\to\R^3$. Suppose $c(0)=(1,2,0)$, and 
        \begin{equation*}
            \nabla f(1,2,0)=(0,0,1), \quad c'(0)=(2,1,1)
        \end{equation*}
        Compute $\frac{d(f\circ c)}{dt}$ at $t=0$.
    \end{enumerate}
\end{prob}
% \begin{proof}
%     \begin{enumerate}
%         \item We have 
%         \begin{equation*}
%             \frac{\partial f}{\partial x}=\frac{\partial f}{\partial u}\frac{\partial u}{\partial x}+\frac{\partial f}{\partial v}\frac{\partial v}{\partial x}=-4xuv+u^2+2
%         \end{equation*}
%         and 
%         \begin{equation*}
%             \frac{\partial f}{\partial y}=\frac{\partial f}{\partial u}\frac{\partial u}{\partial y}+\frac{\partial f}{\partial v}\frac{\partial v}{\partial y}=2uv+u^2+2
%         \end{equation*}
%         (You might want to replace $u,v$ with $x,y$, but I am lazy).
%         \item We have 
%         \begin{equation*}
%             D(g\circ f)(0,1,0)=Dg(f(0,1,0))Df(0,1,0)
%         \end{equation*}
%         where $f(0,1,0)=(0,0)$
%         \begin{equation*}
%             Dg(u,v)=\begin{bmatrix}
%                 e^u&0\\
%                 1&\cos v
%             \end{bmatrix}, \quad, Df(x,y,z)=\begin{bmatrix}
%                 2x&0&0\\
%                 0&z&y
%             \end{bmatrix}
%         \end{equation*}
%         Thus 
%         \begin{align*}
%             D(g\circ f)(0,1,0)&=Dg(0,0)Df(0,1,0)\\
%             &=\begin{bmatrix}
%                 1&0\\
%                 1&1
%             \end{bmatrix}
%             \begin{bmatrix}
%                 0&0&0\\
%                 0&0&1
%             \end{bmatrix}\\
%             &=\begin{bmatrix}
%                 0&0&0\\
%                 0&0&1
%             \end{bmatrix}
%         \end{align*}
%         \item We have 
%         \begin{equation*}
%             \frac{d(f\circ c)}{dt}(0)=\nabla f(1,2,0)c'(0)=\begin{bmatrix}
%                 0 &0&1
%             \end{bmatrix}\begin{bmatrix}
%                 2\\
%                 1\\
%                 1
%             \end{bmatrix}=1
%         \end{equation*}
%     \end{enumerate}
% \end{proof}



\begin{prob}
    Determine the directional derivative of 
    \begin{equation*}
        f(x,y,z)=x^3y-xyz
    \end{equation*}
    at $(1,1,0)$ along $v=(0,-1,1)$.
\end{prob}
% \begin{proof}
%     First we compute 
%     \begin{equation*}
%         \nabla f(x,y,z)=(3x^2y-yz, x^3-xz, -xy)
%     \end{equation*}
%     Thus 
%     \begin{equation*}
%         \nabla f(1,1,0)=(3, 1, -1)
%     \end{equation*}
%     Recall the directional derivative is given by 
%     \begin{equation*}
%         \nabla f(1,1,0)\cdot \frac{v}{\|v\|}=-\frac{2}{\sqrt{2}}
%     \end{equation*}
%     We need to make sure that the direction vector is a unit vector!
% \end{proof}





\begin{prob}
    Find a unit vector normal to the surface
    \[xe^y+ye^z+ze^x=e+1\]
    at the point $(0,1,1)$. % ye^z + z = 0
\end{prob}
% \begin{proof}
%     This is a level set for the multivariate function $f(x,y,z)=xe^y+ye^z+ze^x$. We compute the gradient
%     \[\nabla f(x,y,z)=(e^y+ze^x,e^z+xe^y,e^x+ye^z).\]
%     hence $\nabla f(0,1,1)=(e+1,e,e+1)$, and this vector is normal to the surface. To make this a unit vector, we normalize to get
%     \[\frac{\nabla f(0,1,1)}{\left\|\nabla f(0,1,1)\right\|}=\frac1{\sqrt{3e^2+4e+2}}(e+1,e,e+1),\]
% \end{proof}






\begin{prob}
    Find the tangent plane of the level surface of $f(x,y,z)=\ln(x+y)-2xz=\ln(3)+2$ at $(1,2,-1)$.
\end{prob}
% \begin{proof}
%     By the equation given in Proposition \ref{tangent plane}, the point $(1,2)$ lies on the level set 
%     \begin{equation*}
%         f(x,y)=2-\ln 3
%     \end{equation*}
%     In order to find a normal vector to the tangent plane, we compute the gradient of $f$ at $(1,2,-1)$:
%     \begin{equation*}
%         \nabla f(x,y)=\left(\frac{1}{x+y}-2z, \frac{1}{x+y}, -2x\right)
%     \end{equation*}
%     and $\nabla f(1,2, -1)=\left(\frac{7}{3}, \frac{1}{3}, -2\right)$, thus the tangent plane is given by 
%     \begin{equation*}
%         \frac{7}{3}(x-1)+\frac{1}{3}(y-2)-2(z+1)=0
%     \end{equation*}
%     simplifying we get $7x+y-6z-15=0$.
% \end{proof}


\begin{prob}
    A function $f\colon\R^n\to\R$ is called an \textit{even} function if $f(x)=f(-x)$ for every $x$ in $\R^n$. If $f$ is differentiable and even, find $\nabla f$ at the origin.
\end{prob}
% \begin{proof}
%     We claim that $\nabla f(0,\dots,0)=0$. It suffices to show that $\nabla f(0,\dots, 0)\cdot v=\nabla f(0, \dots, 0)\cdot (-v)$ for any vector $v\in\R^n$. Because this implies $2\nabla f(0,\dots, 0)\cdot v=0$ for every $v\in\R^n$, so $Df(0,\dots, 0)=0$.
%     We know that
%     \[\nabla f(0,\dots, 0)\cdot v=\frac d{dt}f(tv)\bigg|_{t=0}, \quad \nabla f(0,\dots, 0)(-v)=\frac d{dt}f(-tv)\bigg|_{t=0}\]
%     But $f(tv)=f(-tv)$ since $f$ is even, thus 
%     \begin{equation*}
%         \nabla f(0,\dots, 0)\cdot v=\nabla f(0,\dots, 0)\cdot (-v)
%     \end{equation*}
%     as desired.
% \end{proof}



\begin{prob}
    Consider the function
    \[f(x,y)=\frac1{\log(x^2+y)}.\]
    Verify by hand that $f_{xy}=f_{yx}$.
\end{prob}
% \begin{proof}
%     We compute these separately.
%     \begin{equation*}
%         f_x=\frac{2x}{x^2+y}, \quad f_{xy}=-\frac{2x}{\left(x^2+y\right)^2}
%     \end{equation*}
%     and 
%     \begin{equation*}
%         f_y=\frac1{x^2+y}, \quad f_{yx}=-\frac{2x}{\left(x^2+y\right)^2}
%     \end{equation*}
% \end{proof}


\begin{prob}
    Consider the function $f(x,y,z)=\left(x^2+y^2+z^2\right)^{-1/2}$. Show that
    \[f_{xx}+f_{yy}+f_{zz}=0.\]
\end{prob}
% \begin{proof}
%     Note
%     \[f_x=-\frac12\cdot\frac{2x}{\left(x^2+y^2+z^2\right)^{3/2}}=-\frac{x}{\left(x^2+y^2+z^2\right)^{3/2}},\]
%     so
%     \[f_{xx}=-\frac{\left(x^2+y^2+z^2\right)^{3/2}-x\cdot\frac32\left(x^2+y^2+z^2\right)^{1/2}\cdot2x}{\left(x^2+y^2+z^2\right)^{3}},\]
%     which is
%     \[f_{xx}=-\frac{x^2+y^2+z^2-3x^2}{\left(x^2+y^2+z^2\right)^{5/2}},\]
%     or
%     \[f_{xx}=-\frac{-2x^2+y^2+z^2}{\left(x^2+y^2+z^2\right)^{5/2}}.\]
%     By symmetry,
%     \[f_{yy}=-\frac{x^2-2y^2+z^2}{\left(x^2+y^2+z^2\right)^{5/2}},\]
%     and
%     \[f_{zz}=-\frac{x^2+y^2-2z^2}{\left(x^2+y^2+z^2\right)^{5/2}},\]
%     so we see that $f_{xx}+f_{yy}+f_{zz}=0$.
% \end{proof}


\begin{prob}
    Find the second-order Taylor expansion for the function 
    \begin{equation*}
        f(x,y)=x^2+2xy
    \end{equation*}
    at $(1,1)$.
\end{prob}
% \begin{proof}
%     First $f(1,1)=3$, then we find all first-order and second-order partial derivatives:
%     \begin{equation*}
%         f_x=2x+2y, f_y=2x, f_{xx}=2, f_{xy}=2, f_{yy}=0
%     \end{equation*}
%     Thus by formula in Definition \ref{taylor}, we have 
%     \begin{align*}
%         f(x,y)&=3+4(x-1)+2(y-1)+\frac{1}{2}2(x-1)^2+\frac{1}{2}2(x-1)(y-1)+\frac{1}{2}2(x-1)(y-1)+R_2((1,1),(x,y))\\
%         &=3+4(x-1)+2(y-1)+(x-1)^2+2(x-1)(y-1)+R_2((1,1),(x,y))
%     \end{align*}
%     where 
%     \begin{equation*}
%         \frac{R_2((1,1),(x,y))}{\|(x-1, y-1)\|}\to 0
%     \end{equation*}
%     as $(x,y)\to (1,1)$.
% \end{proof}


\begin{prob}
    Find and classify all critical points of the following function:
    \begin{enumerate}
        \item \begin{equation*}
            f(x,y)=e^x\cos y
        \end{equation*}
        \item \begin{equation*}
            g(x,y)=(2x^2+x)(3y+1)
        \end{equation*}
    \end{enumerate}
\end{prob}
% \begin{proof}
%     \begin{enumerate}
%         \item The critical point of $f$ requires 
%         \begin{equation*}
%             f_x=f_y=0
%         \end{equation*}
%         This gives 
%         \begin{equation*}
%             f_x=e^x\cos y=0\Rightarrow y=\frac{\pi}{2}+k\pi, k\in\Z
%         \end{equation*}
%         similarly,
%         \begin{equation*}
%             f_y=-e^x\sin y=0\Rightarrow y=\pi+n\in\Z
%         \end{equation*}
%         We see that there is no such $y$ that makes $f_x=f_y=0$ simultaneously. Hence there are no critical points.
%         \item We again compute $x,y$ such that $g_x=g_y=0$.
%         \begin{equation*}
%             g_x=(4x+1)(3y+1)\Rightarrow x=-\frac{1}{4}, y=-\frac{1}{3}
%         \end{equation*}
%         and 
%         \begin{equation*}
%             g_y=(2x^2+x)3=0\Rightarrow x=0 \text{ or } x=-\frac{1}{2}
%         \end{equation*}
%         Thus the points $(x,y)$ such that $g_x=g_y=0$ are 
%         \begin{equation*}
%             \left(0,-\frac{1}{3}\right), \quad \left(-\frac{1}{2}, -\frac{1}{3}\right)
%         \end{equation*}
%         Now we classify them by first computing their Hessians:
%         \begin{equation*}
%             g_{xx}=4(3y+1), \quad g_{xy}=3(4x+1), \quad g_{yy}=0
%         \end{equation*}
%         Thus 
%         \begin{equation*}
%             \disc Hf=\det(Hf)=g_{xx}g_{yy}-g_{xy}^2=-9(4x+1)^2
%         \end{equation*}
%         Then we see that $x=0, -\frac{1}{2}$ both result in $\disc Hf<0$, which means 
%         \begin{equation*}
%             \left(0,-\frac{1}{3}\right), \quad \left(-\frac{1}{2}, -\frac{1}{3}\right)
%         \end{equation*}
%         are both saddle points.
%     \end{enumerate}
% \end{proof}


\begin{prob}
    Show that $(0,0)$ is a critical point of
    \begin{equation*}
        f(x,y)=x^2y-2x^2-y^2
    \end{equation*}
    and is it a local maximum, local minimum, or a saddle point?
\end{prob}
% \begin{proof}
%     We have 
%     \begin{equation*}
%         f_x=2xy-4x, \quad f_y=x^2-2y
%     \end{equation*}
%     and we see $f_x(0,0)=f_y(0,0)=0$, thus $(0,0)$ is a critical point. Now we compute the discriminant:
%     \begin{equation*}
%         f_{xx}=2y-4, \quad f_{xy}=2x, \quad f_{yy}=-2
%     \end{equation*}
%     Then 
%     \begin{equation*}
%         \disc Hf=f_{xx}f_{yy}-f_{xy}^2=-2(2y-4)-4x^2
%     \end{equation*}
%     Hence $\disc Hf(0,0)=8>0$, and $f_{xx}(0,0)=-4$ imply that $(0,0)$ is a local maximum.
% \end{proof}













\chapter{Answer Key}

\begin{prob}
    Find the equation of the line passing through $(1,0,2)$ in the direction $(2,-1,3)$.
\end{prob}
\begin{proof}
    By definition \ref{line}, the line is given by 
    \begin{equation*}
        l(t)=(1+2t, -t, 2+3t)
    \end{equation*}
\end{proof}

\begin{prob}
    In which direction does the line 
    \begin{equation*}
        l(t)=(3-2t, 2+5t, 1+t)
    \end{equation*}
    point?
\end{prob}
\begin{proof}
    In the direction of the vector $(-2, 5, 1)$.
\end{proof}


\begin{prob}
    Do the following two lines intersect?
    \begin{equation*}
        l_1(t)=(1+2t, 2+t, 3-t), \quad l_2(s)=(3-s, 4-s, 2+s)
    \end{equation*}
\end{prob}
\begin{proof}
    For them to intersect, we must have $t,s$ such that 
    \begin{equation*}
        \begin{cases}
            1+2t=3-s \quad (1)\\
            2+t=4-s \quad (2)\\
            3-t=2+s \quad (3)
        \end{cases}
    \end{equation*}
    $(2)-(1)$ gives $-t+1=1$, which implies $t=0, s=2$, but this does not satisfy $(3)$, hence these two lines do not intersect!
\end{proof}


\begin{prob}
    Do the following points lie on the same line?
    \begin{equation*}
        A=(1,0,1),\quad B=(2,1,1), \quad C=(0,-1,1)
    \end{equation*}
\end{prob}
\begin{proof}
    We can find the unique line passing through $A,B$ by the equation given in\ref{line}
    \begin{equation*}
        l(t)=(1,0,1)+(1,1,0)t
    \end{equation*}
    then for $C$ to lie on this line, there must exists some $t$ such that 
    \begin{equation*}
        \begin{cases}
            1+t=0\\
            t=-1\\
            1=1
        \end{cases}
    \end{equation*}
    and $t=-1$ satisfies. This means all three points lie on the same line!
\end{proof}


\begin{prob}
    Find the angle between two vectors $(1,2,0), (3,1,1)$.
\end{prob}
\begin{proof}
    By Proposition \ref{dot} 
    \begin{equation*}
        \cos\theta=\frac{a\cdot b}{\|a\|\|b\|}=\frac{5}{\sqrt{5}\sqrt{11}}=\sqrt{\frac{5}{11}}
    \end{equation*}
    hence 
    \begin{equation*}
        \theta=\arccos\left(\sqrt{\frac{5}{11}}\right)
    \end{equation*}
\end{proof}

\begin{prob}
    Let $b=(2,1,3)$ and $P$ be the plane through the origin given by $x+y+2z=0$. 
    \begin{enumerate}
        \item[(a)] Find two distinct vectors $v_1,v_2$ that are orthogonal in $P$.
        \item[(b)] Find the projection of $b$ onto the plane $P$, namely, 
        \begin{equation*}
            \text{Proj}_{v_1}b+\text{Proj}_{v_2}b
        \end{equation*}
    \end{enumerate}
\end{prob}
\begin{proof}
    \begin{enumerate}
        \item[(a)] We can let $v_1=(1,-1,0), v_2=(1,1,-1)$. One can verify that $v_1,v_2\in P$ and $v_1\cdot v_2=0$.
        \item[(b)] The projection is given by
        \begin{align*}
            \text{Proj}_{v_1}b+\text{Proj}_{v_2}b&=\frac{v_1\cdot b}{v_1\cdot v_1}v_1+\frac{v_2\cdot b}{v_2\cdot v_2}v_2\\
            &=\frac{1}{2}(1,-1,0)+0\\
            &=\left(\frac{1}{2}, -\frac{1}{2}, 0\right)
        \end{align*}
    \end{enumerate}
\end{proof}


\begin{prob}
    Find a unit vector orthognal to both vectors $a=(1,2,-1), b=(2,3,-1)$.
\end{prob}
\begin{proof}
    The cross product is orthognal to both of the vectors:
    \begin{equation*}
        a\times b=\det\begin{bmatrix}
            i&j&k\\
            1&2&-1\\
            2&3&-1
        \end{bmatrix}=(1,-1, -1)
    \end{equation*}
    Then we normalize it:
    \begin{equation*}
        \frac{a\times b}{\|a\times b\|}=\left(\frac{1}{\sqrt{3}}, -\frac{1}{\sqrt{3}}, -\frac{1}{\sqrt{3}}\right)
    \end{equation*}
\end{proof}


\begin{prob}
    Find the equation of the plane containing all three points below:
    \begin{equation*}
        P=(2,1,-1), \quad Q=(1,0,-2), \quad T=(3,2,1)
    \end{equation*}
\end{prob}
\begin{proof}
    We can find two vectors in this plane:
    \begin{equation*}
        \text{PQ}=Q-P=(-1,-1,-1), \quad \text{ PT}=T-P=(1,1,2)
    \end{equation*}
    then we can find a normal vector $n$ to the plane by taking the cross product:
    \begin{equation*}
        n=\text{PQ}\times\text{PT}=\det\begin{bmatrix}
            i&j&k\\
            -1&-1&-1\\
            1&1&2
        \end{bmatrix}=(-1,1,0)
    \end{equation*}
    Then by Definition \ref{plane}, using point $Q$, we see the plane can be written as 
    \begin{equation*}
        -1(x-1)+y=0
    \end{equation*}
    simplifying we get $x-y=1$.
\end{proof}

\begin{prob}
    \begin{enumerate}
        \item[(a)] Find an equation for the line that passes through the point $(1,1,0)$ and is perpendicular to the plane $3x+y-2z+1=0$.
        \item[(b)] Find an equation for the plane that contains the line 
        \begin{equation*}
            l(t)=(-1+t, 2+2t, 1+3t)
        \end{equation*}
        and is perpendicular to the plane 
        \begin{equation*}
            2x+y-z+1=0
        \end{equation*}
    \end{enumerate}
\end{prob}
\begin{proof}
    \begin{enumerate}
        \item[(a)] A normal vector to the plane $3x+y-2z+1$ is $(3,1,-2)$, since the line is perpendicular to the plane, the line is parallel along the direction $(3,1,-2)$. Now the line passes through $(1,1,0)$, thus we have the equation for the line 
        \begin{equation*}
            l(t)=(1,1,0)+t(3,1,-2)
        \end{equation*}
        \item[(b)] A normal vector to $2x+y-z$ is $n=(2,1,-1)$, and since our plane is perpendicular to this, it is parallel to the vector $n$. Thus a normal vector to our plane must be orthogonal to both $n$ and $(1,2,3)$, where the latter is given by the line in the plane. Thus taking the cross product:
        \begin{equation*}
            n_1=n\times (1,2,3)=(5,-7,3)
        \end{equation*}
        Hence the equation for the plane is given by:
        \begin{equation*}
            5(x+1)-7(y-2)+3(z-1)=0
        \end{equation*}
        simplifying we get $5x-7y+3z+16=0$.
    \end{enumerate}
\end{proof}

\begin{prob}
    Compute the area of the parallelogram spanned by the vectors $(1,1,0), (0,2,1)$.
\end{prob}
\begin{proof}
    Since we know 
    \begin{equation*}
        \|u\times v\|=\|u\|\|v\|\sin\theta
    \end{equation*}
    the length of the cross product is exactly the area of the parallelogram, thus computing
    \begin{equation*}
        \|(1,1,0)\times(0,2,1)\|=\|(1,-1,2)\|=\sqrt{6}
    \end{equation*}
\end{proof}


\begin{prob}
    Use the triangle inequality \ref{traingle} to show the reverse triangle inequality:
    \begin{equation*}
        \bigg|\|a\|-\|b\|\bigg|\leq\|a-b\|
    \end{equation*}
\end{prob}
\begin{proof}
    We know by traingle inequality 
    \begin{align*}
        \|a\|&=\|(a-b)+b\|\\
        &\leq \|a-b\|+\|b\|
    \end{align*}
    rearranging, we get $\|a\|-\|b\|\leq\|a-b\|$. Similarly 
    \begin{equation*}
        \|b\|-\|a\|\leq \|a-b\|
    \end{equation*}
    Together this implies 
    \begin{equation*}
        \bigg|\|a\|-\|b\|\bigg|\leq\|a-b\|
    \end{equation*}
\end{proof}






\begin{prob}
    Compute the following limits if they exist; if the limits don't exist, please explain why.
    \begin{enumerate}
        \item \begin{equation*}
            \lim_{(x,y)\to(2,1)}\frac{x^2+y^2-2xy}{x-y}
        \end{equation*}
        \item \begin{equation*}
            \lim_{(x,y)\to(0,0)}\frac{\cos x-1}{x^2+y^2}
        \end{equation*}
        \item \begin{equation*}
            \lim_{(x,y)\to (0,0)}\frac{(x-y)^2}{(x+y)^2}
        \end{equation*}
        \item \begin{equation*}
            \lim_{(x,y)\to(0,0)}\frac{\sin 2x-2x+y}{x^3+y}
        \end{equation*}
        \item \begin{equation*}
            \lim_{(x,y,z)\to(0,0,0)}\frac{2x^2y\cos z}{x^2+y^2}
        \end{equation*}
        \item \begin{equation*}
            \lim_{(x,y)\to(2,1)}\frac{x^2-2xy}{x^2-4y^2}
        \end{equation*}
        \item \begin{equation*}
            \lim_{(x,y)\to(0,0)}\frac{x^2-y^6}{xy^3}
        \end{equation*}
    \end{enumerate}
\end{prob}
\begin{proof}
    \begin{enumerate}
        \item \begin{equation*}
            \lim_{(x,y)\to(2,1)}\frac{x^2+y^2-2xy}{x-y}=\lim_{(x,y)\to(2,1)}\frac{(x-y)^2}{x-y}=\lim_{(x,y)\to(2,1)}x-y=1
        \end{equation*}
        \item The limit doesn't exist, \begin{equation*}
            \lim_{(x,y)\to(0,0)}\frac{\cos x-1}{x^2+y^2}
        \end{equation*}
        Consider the path $x=0, y\to 0$, we have 
        \begin{equation*}
            \lim_{x=0,y\to 0}\frac{0}{y^2}=0
        \end{equation*}
        Consider the path $y=0, x\to 0$, 
        \begin{equation*}
            \lim_{y=0, x\to 0}\frac{\cos x-1}{x^2}=\lim_{x\to 0}\frac{-\sin x}{2x}=\lim_{x\to 0}\frac{-\cos x}{2}=-\frac{1}{2}
        \end{equation*}
        \item The limit doesn't exist, \begin{equation*}
            \lim_{(x,y)\to (0,0)}\frac{(x-y)^2}{(x+y)^2}
        \end{equation*}
        Consider the path $x=0, y\to 0$, 
        \begin{equation*}
            \lim_{x=0,y\to 0}\frac{y^2}{y^2}=1
        \end{equation*}
        Consider the path $y=x\to 0$, 
        \begin{equation*}
            \lim_{x=y\to 0}\frac{0}{4x^2}=0
        \end{equation*}
        \item The limit doesn't exist, \begin{equation*}
            \lim_{(x,y)\to(0,0)}\frac{\sin 2x-2x+y}{x^3+y}
        \end{equation*}
        Consider the path $x=0, y\to 0$,
        \begin{equation*}
            \lim_{x=0, y\to 0}\frac{y}{y}=1
        \end{equation*}
        Consider the path $y=0, x\to 0$, 
        \begin{align*}
            \lim_{y=0, x\to 0}\frac{\sin 2x-2x}{x^3}&=\lim_{x\to 0}\frac{2\cos 2x-2}{3x^2}\\
            &=\lim_{x\to 0}\frac{-4\sin 2x}{6x}\\
            &=\lim_{x\to 0}\frac{-8\cos 2x}{6}\\
            &=-\frac{4}{3}
        \end{align*}
        \item \begin{equation*}
            \lim_{(x,y,z)\to(0,0,0)}\frac{2x^2y\cos z}{x^2+y^2}
        \end{equation*}
        Writing $x=r\cos\theta, y=r\sin\theta$ in polar coordinates, we can rewrite this as 
        \begin{equation*}
            \left|\frac{2r^3\cos^2\theta\sin\theta\cos z}{r^2}\right|=|2r\cos^2\theta\sin\theta\cos z|\leq 2r\to 0
        \end{equation*}
        as $(x,y,z)\to(0,0,0)$. Thus the limit is $0$.
        \item \begin{equation*}
            \lim_{(x,y)\to(2,1)}\frac{x^2-2xy}{x^2-4y^2}
        \end{equation*}
        We factor:
        \begin{equation*}
            \lim_{(x,y)\to(2,1)}\frac{x^2-2xy}{x^2-4y^2}=\lim_{(x,y)\to(2,1)}\frac{(x-2y)x}{(x+2y)(x-2y)}=\lim_{(x,y)\to(2,1)}\frac{x}{x+2y}=\frac{1}{2}
        \end{equation*}
        \item The limit doesn't exist, \begin{equation*}
            \lim_{(x,y)\to(0,0)}\frac{x^2-y^6}{xy^3}
        \end{equation*}
        Consider $x=y\to 0$, then 
        \begin{equation*}
            \lim_{x=y\to 0}\frac{x^2-x^6}{x^4}=\lim_{x\to 0}\frac{1-x^4}{x^2}=\infty
        \end{equation*}
        Consider $x=y^3\to 0$, then 
        \begin{equation*}
            \lim_{x=y^3\to 0}\frac{0}{y^6}=0
        \end{equation*}
    \end{enumerate}
\end{proof}


\begin{prob}
    \begin{enumerate}
        \item[(a)] Show that $f:\R\to\R$,
        \begin{equation*}
            f(x)=(1-x)^8+\cos(1+x^3)
        \end{equation*}
        is continuous.
        \item[(b)] Show $f:\R\to\R$, 
        \begin{equation*}
            f(x)=\frac{x^2e^x}{2-\sin x}
        \end{equation*}
        is continuous.
    \end{enumerate}
\end{prob}
\begin{proof}
    \begin{enumerate}
        \item[(a)] $(1-x)^8$ is a polynomial, thus continuous, and $\cos x, 1+x^3$ are both continuous, thus the composition $\cos(1+x^3)$ is also continuous. Thus adding continuous functions gives another continuous function.
        \item[(b)] $x^2e^x, 2-\sin x$ are both continuous, and $\frac{x^2e^x}{2-\sin x}$ is continuous if $2-\sin x\neq 0$ for all $x$. This is indeed true because $-1\leq\sin x\leq 1$, thus $1\leq 2-\sin x\leq 3$.
    \end{enumerate}
\end{proof}


\begin{prob}
    Compute all the partial derivatives.
    \begin{enumerate}
        \item $w=e^{xy}\log(x^2+y^2)$.
        \item $w=\cos(ye^{xy})\sin x$.
    \end{enumerate}
\end{prob}
\begin{proof}
    \begin{enumerate}
        \item \begin{equation*}
            \frac{\partial w}{\partial x}=ye^{xy}\ln(x^2+y^2)+e^{xy}\frac{2x}{x^2+y^2}
        \end{equation*}
        and 
        \begin{equation*}
            \frac{\partial w}{\partial y}=xe^{xy}\ln(x^2+y^2)+e^{xy}\frac{2y}{x^2+y^2}
        \end{equation*}
        \item \begin{equation*}
            \frac{\partial w}{\partial x}=-y^2e^{xy}\sin (ye^{xy})\sin x+\cos(ye^{xy})\cos x
        \end{equation*}
        and 
        \begin{equation*}
            \frac{\partial w}{\partial y}=-(1+xy)e^{xy}\sin(ye^{xy})\sin x
        \end{equation*}
    \end{enumerate}
\end{proof}


\begin{prob}
    Compute the gradient of $h(x,y,z)=(x+z)e^{x-y}$ at $(1,1,0)$.
\end{prob}
\begin{proof}
    The gradient is 
    \begin{align*}
        \nabla h(x,y,z)&=\begin{bmatrix}
            \frac{\partial h}{\partial x}&\frac{\partial h}{\partial y}&\frac{\partial h}{\partial z}
        \end{bmatrix}\\
        &=\begin{bmatrix}
            e^{x-y}(1+x+z)&-(x+z)e^{x-y}&e^{x-y}
        \end{bmatrix}
    \end{align*}
    Thus 
    \begin{equation*}
        \nabla h(1,1,0)=\begin{bmatrix}
            2&-1&1
        \end{bmatrix}
    \end{equation*}
\end{proof}





\begin{prob}
    Determine the velocity vector of the given path:
    \begin{equation*}
        c(t)=(\cos 2t, 3t^2-t, -t)
    \end{equation*}
\end{prob}
\begin{proof}
    It is given by 
    \begin{equation*}
        c'(t)=(-2\sin 2t, 6t-1, -1)
    \end{equation*}
\end{proof}


\begin{prob}
    Find the tangent line to the given path at $t=0$
    \begin{equation*}
        c(t)=(e^t\sin t, 2t, -t^3)
    \end{equation*}
\end{prob}
\begin{proof}
    By the equation in Definition \ref{tangent path}, we have 
    \begin{equation*}
        c'(t)=(e^t\sin t+e^t\cos t, 2, -3t^2)
    \end{equation*}
    and $c(0)=(0,0,0), c'(0)=(1,2,0)$. Thus the tangent line is given by 
    \begin{equation*}
        l(t)=(t,2t,0)
    \end{equation*}
\end{proof}



\begin{prob}
    Compute the derivatives.
    \begin{enumerate}
        \item Let
        \begin{equation*}
            f(u,v)=u^2v+2v, \quad u=-x^2+y, v=x+y
        \end{equation*}
        Compute $\frac{\partial f}{\partial x},\frac{\partial f}{\partial y}$.
        \item Let 
        \begin{equation*}
            g(u,v)=(e^u, u+\sin v), \quad f(x,y,z)=(x^2, yz)
        \end{equation*}
        Compute $D(g\circ f)$ at $(0,1,0)$.
        \item Let $f:\R^3\to\R$ and $c(t)=\R\to\R^3$. Suppose $c(0)=(1,2,0)$, and 
        \begin{equation*}
            \nabla f(1,2,0)=(0,0,1), \quad c'(0)=(2,1,1)
        \end{equation*}
        Compute $\frac{d(f\circ c)}{dt}$ at $t=0$.
    \end{enumerate}
\end{prob}
\begin{proof}
    \begin{enumerate}
        \item We have 
        \begin{equation*}
            \frac{\partial f}{\partial x}=\frac{\partial f}{\partial u}\frac{\partial u}{\partial x}+\frac{\partial f}{\partial v}\frac{\partial v}{\partial x}=-4xuv+u^2+2
        \end{equation*}
        and 
        \begin{equation*}
            \frac{\partial f}{\partial y}=\frac{\partial f}{\partial u}\frac{\partial u}{\partial y}+\frac{\partial f}{\partial v}\frac{\partial v}{\partial y}=2uv+u^2+2
        \end{equation*}
        (You might want to replace $u,v$ with $x,y$, but I am lazy).
        \item We have 
        \begin{equation*}
            D(g\circ f)(0,1,0)=Dg(f(0,1,0))Df(0,1,0)
        \end{equation*}
        where $f(0,1,0)=(0,0)$
        \begin{equation*}
            Dg(u,v)=\begin{bmatrix}
                e^u&0\\
                1&\cos v
            \end{bmatrix}, \quad, Df(x,y,z)=\begin{bmatrix}
                2x&0&0\\
                0&z&y
            \end{bmatrix}
        \end{equation*}
        Thus 
        \begin{align*}
            D(g\circ f)(0,1,0)&=Dg(0,0)Df(0,1,0)\\
            &=\begin{bmatrix}
                1&0\\
                1&1
            \end{bmatrix}
            \begin{bmatrix}
                0&0&0\\
                0&0&1
            \end{bmatrix}\\
            &=\begin{bmatrix}
                0&0&0\\
                0&0&1
            \end{bmatrix}
        \end{align*}
        \item We have 
        \begin{equation*}
            \frac{d(f\circ c)}{dt}(0)=\nabla f(1,2,0)c'(0)=\begin{bmatrix}
                0 &0&1
            \end{bmatrix}\begin{bmatrix}
                2\\
                1\\
                1
            \end{bmatrix}=1
        \end{equation*}
    \end{enumerate}
\end{proof}



\begin{prob}
    Determine the directional derivative of 
    \begin{equation*}
        f(x,y,z)=x^3y-xyz
    \end{equation*}
    at $(1,1,0)$ along $v=(0,-1,1)$.
\end{prob}
\begin{proof}
    First we compute 
    \begin{equation*}
        \nabla f(x,y,z)=(3x^2y-yz, x^3-xz, -xy)
    \end{equation*}
    Thus 
    \begin{equation*}
        \nabla f(1,1,0)=(3, 1, -1)
    \end{equation*}
    Recall the directional derivative is given by 
    \begin{equation*}
        \nabla f(1,1,0)\cdot \frac{v}{\|v\|}=-\frac{2}{\sqrt{2}}
    \end{equation*}
    We need to make sure that the direction vector is a unit vector!
\end{proof}





\begin{prob}
    Find a unit vector normal to the surface
    \[xe^y+ye^z+ze^x=e+1\]
    at the point $(0,1,1)$. % ye^z + z = 0
\end{prob}
\begin{proof}
    This is a level set for the multivariate function $f(x,y,z)=xe^y+ye^z+ze^x$. We compute the gradient
    \[\nabla f(x,y,z)=(e^y+ze^x,e^z+xe^y,e^x+ye^z).\]
    hence $\nabla f(0,1,1)=(e+1,e,e+1)$, and this vector is normal to the surface. To make this a unit vector, we normalize to get
    \[\frac{\nabla f(0,1,1)}{\left\|\nabla f(0,1,1)\right\|}=\frac1{\sqrt{3e^2+4e+2}}(e+1,e,e+1),\]
\end{proof}






\begin{prob}
    Find the tangent plane of the level surface of $f(x,y,z)=\ln(x+y)-2xz=\ln(3)+2$ at $(1,2,-1)$.
\end{prob}
\begin{proof}
    By the equation given in Proposition \ref{tangent plane}, we first compute a normal vector to the tangent plane, which is the gradient of $f$ at $(1,2,-1)$:
    \begin{equation*}
        \nabla f(x,y,z)=\left(\frac{1}{x+y}-2z, \frac{1}{x+y}, -2x\right)
    \end{equation*}
    and $\nabla f(1,2, -1)=\left(\frac{7}{3}, \frac{1}{3}, -2\right)$, thus the tangent plane is given by 
    \begin{equation*}
        \frac{7}{3}(x-1)+\frac{1}{3}(y-2)-2(z+1)=0
    \end{equation*}
    simplifying we get $7x+y-6z-15=0$.
\end{proof}


\begin{prob}
    A function $f\colon\R^n\to\R$ is called an \textit{even} function if $f(x)=f(-x)$ for every $x$ in $\R^n$. If $f$ is differentiable and even, find $\nabla f$ at the origin.
\end{prob}
\begin{proof}
    We claim that $\nabla f(0,\dots,0)=0$. It suffices to show that $\nabla f(0,\dots, 0)\cdot v=\nabla f(0, \dots, 0)\cdot (-v)$ for any vector $v\in\R^n$. Because this implies $2\nabla f(0,\dots, 0)\cdot v=0$ for every $v\in\R^n$, so $Df(0,\dots, 0)=0$.
    We know that
    \[\nabla f(0,\dots, 0)\cdot v=\frac d{dt}f(tv)\bigg|_{t=0}, \quad \nabla f(0,\dots, 0)(-v)=\frac d{dt}f(-tv)\bigg|_{t=0}\]
    But $f(tv)=f(-tv)$ since $f$ is even, thus 
    \begin{equation*}
        \nabla f(0,\dots, 0)\cdot v=\nabla f(0,\dots, 0)\cdot (-v)
    \end{equation*}
    as desired.
\end{proof}



\begin{prob}
    Consider the function
    \[f(x,y)=\frac1{\log(x^2+y)}.\]
    Verify by hand that $f_{xy}=f_{yx}$.
\end{prob}
\begin{proof}
    We compute these separately.
    \begin{equation*}
        f_x=\frac{2x}{x^2+y}, \quad f_{xy}=-\frac{2x}{\left(x^2+y\right)^2}
    \end{equation*}
    and 
    \begin{equation*}
        f_y=\frac1{x^2+y}, \quad f_{yx}=-\frac{2x}{\left(x^2+y\right)^2}
    \end{equation*}
\end{proof}


\begin{prob}
    Consider the function $f(x,y,z)=\left(x^2+y^2+z^2\right)^{-1/2}$. Show that
    \[f_{xx}+f_{yy}+f_{zz}=0.\]
\end{prob}
\begin{proof}
    Note
    \[f_x=-\frac12\cdot\frac{2x}{\left(x^2+y^2+z^2\right)^{3/2}}=-\frac{x}{\left(x^2+y^2+z^2\right)^{3/2}},\]
    so
    \[f_{xx}=-\frac{\left(x^2+y^2+z^2\right)^{3/2}-x\cdot\frac32\left(x^2+y^2+z^2\right)^{1/2}\cdot2x}{\left(x^2+y^2+z^2\right)^{3}},\]
    which is
    \[f_{xx}=-\frac{x^2+y^2+z^2-3x^2}{\left(x^2+y^2+z^2\right)^{5/2}},\]
    or
    \[f_{xx}=-\frac{-2x^2+y^2+z^2}{\left(x^2+y^2+z^2\right)^{5/2}}.\]
    By symmetry,
    \[f_{yy}=-\frac{x^2-2y^2+z^2}{\left(x^2+y^2+z^2\right)^{5/2}},\]
    and
    \[f_{zz}=-\frac{x^2+y^2-2z^2}{\left(x^2+y^2+z^2\right)^{5/2}},\]
    so we see that $f_{xx}+f_{yy}+f_{zz}=0$.
\end{proof}


\begin{prob}
    Find the second-order Taylor expansion for the function 
    \begin{equation*}
        f(x,y)=x^2+2xy
    \end{equation*}
    at $(1,1)$.
\end{prob}
\begin{proof}
    First $f(1,1)=3$, then we find all first-order and second-order partial derivatives:
    \begin{equation*}
        f_x=2x+2y, f_y=2x, f_{xx}=2, f_{xy}=2, f_{yy}=0
    \end{equation*}
    Thus by formula in Definition \ref{taylor}, we have 
    \begin{align*}
        f(x,y)&=3+4(x-1)+2(y-1)+\frac{1}{2}2(x-1)^2+\frac{1}{2}2(x-1)(y-1)+\frac{1}{2}2(x-1)(y-1)+R_2((1,1),(x,y))\\
        &=3+4(x-1)+2(y-1)+(x-1)^2+2(x-1)(y-1)+R_2((1,1),(x,y))
    \end{align*}
    where 
    \begin{equation*}
        \frac{R_2((1,1),(x,y))}{\|(x-1, y-1)\|}\to 0
    \end{equation*}
    as $(x,y)\to (1,1)$.
\end{proof}


\begin{prob}
    Find and classify all critical points of the following function:
    \begin{enumerate}
        \item \begin{equation*}
            f(x,y)=e^x\cos y
        \end{equation*}
        \item \begin{equation*}
            g(x,y)=(2x^2+x)(3y+1)
        \end{equation*}
    \end{enumerate}
\end{prob}
\begin{proof}
    \begin{enumerate}
        \item The critical point of $f$ requires 
        \begin{equation*}
            f_x=f_y=0
        \end{equation*}
        This gives 
        \begin{equation*}
            f_x=e^x\cos y=0\Rightarrow y=\frac{\pi}{2}+k\pi, k\in\Z
        \end{equation*}
        similarly,
        \begin{equation*}
            f_y=-e^x\sin y=0\Rightarrow y=n\pi, n\in\Z
        \end{equation*}
        We see that there is no such $y$ that makes $f_x=f_y=0$ simultaneously. Hence there are no critical points.
        \item We again compute $x,y$ such that $g_x=g_y=0$.
        \begin{equation*}
            g_x=(4x+1)(3y+1)\Rightarrow x=-\frac{1}{4}, y=-\frac{1}{3}
        \end{equation*}
        and 
        \begin{equation*}
            g_y=(2x^2+x)3=0\Rightarrow x=0 \text{ or } x=-\frac{1}{2}
        \end{equation*}
        Thus the points $(x,y)$ such that $g_x=g_y=0$ are 
        \begin{equation*}
            \left(0,-\frac{1}{3}\right), \quad \left(-\frac{1}{2}, -\frac{1}{3}\right)
        \end{equation*}
        Now we classify them by first computing their Hessians:
        \begin{equation*}
            g_{xx}=4(3y+1), \quad g_{xy}=3(4x+1), \quad g_{yy}=0
        \end{equation*}
        Thus 
        \begin{equation*}
            \mathcal{D}=\det(Hf)=g_{xx}g_{yy}-g_{xy}^2=-9(4x+1)^2
        \end{equation*}
        Then we see that $x=0, -\frac{1}{2}$ both result in $\mathcal{D}<0$, which means 
        \begin{equation*}
            \left(0,-\frac{1}{3}\right), \quad \left(-\frac{1}{2}, -\frac{1}{3}\right)
        \end{equation*}
        are both saddle points.
    \end{enumerate}
\end{proof}


\begin{prob}
    Show that $(0,0)$ is a critical point of
    \begin{equation*}
        f(x,y)=x^2y-2x^2-y^2
    \end{equation*}
    and is it a local maximum, local minimum, or a saddle point?
\end{prob}
\begin{proof}
    We have 
    \begin{equation*}
        f_x=2xy-4x, \quad f_y=x^2-2y
    \end{equation*}
    and we see $f_x(0,0)=f_y(0,0)=0$, thus $(0,0)$ is a critical point. Now we compute the discriminant:
    \begin{equation*}
        f_{xx}=2y-4, \quad f_{xy}=2x, \quad f_{yy}=-2
    \end{equation*}
    Then 
    \begin{equation*}
        \mathcal{D}=f_{xx}f_{yy}-f_{xy}^2=-2(2y-4)-4x^2
    \end{equation*}
    Hence $\mathcal{D}(0,0)=8>0$, and $f_{xx}(0,0)=-4$ imply that $(0,0)$ is a local maximum.
\end{proof}




\chapter{Tips}

\begin{enumerate}
    \item When asked to find the limit: 
    \begin{enumerate}
        \item[Step 1:] Factor out common factor, for example,
        \begin{equation*}
            \frac{x^2-2xy}{x^2-4y^2}=\frac{(x-2y)x}{(x-2y)(x+2y)}=\frac{x}{x+2y}
        \end{equation*}
        \item[Step 2:] Try the following four paths: take $(x,y)\to (0,0)$ as an example,
        \begin{enumerate}
            \item $x=0, y\to 0$.
            \item $y=0, x\to 0$.
            \item $x=y\to 0$.
            \item $x=-y\to 0$.
        \end{enumerate}
        \item[Step 3:] Try to put into expressions that you are familiar with, for example, 
        \begin{equation*}
            \lim_{(x,y)\to(0,0)}\frac{\sin xy}{x}=\lim_{(x,y)\to(0,0)}\frac{\sin xy}{xy}y
        \end{equation*}
        and use the fact that $\lim_{t\to 0}\frac{\sin t}{t}=1$.
    \end{enumerate}
   
    If any two paths give different limits, then the limit doesn't exist. Step 2: 
    \item When asked to find a directional derivative of $f$ along $v$ : make sure you normalize $v$ as $\frac{v}{\|v\|}$.
    \item When asked to find an equation for a plane: identify a normal vector by
    \begin{enumerate}
        \item taking the cross product of two vectors in the plane \ref{plane}.
        \item computing the gradient if the plane is the tangent plane to a level surface \ref{tangent plane}.
    \end{enumerate}
    \item Let $f:U\subset\R^n\to\R^m$ be differentiable, then $Df$ is an $m\times n$ matrix. Let $A$ be an $m\times n$ matrix and $B$ be a $k\times p$ matrix, then the matrix multiplication $AB$ only makes sense when $n=k$.
\end{enumerate}























\end{document}