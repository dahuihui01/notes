\documentclass[openany]{book}

\usepackage[margin=1in]{geometry}
\usepackage{amsmath,amsfonts,amsthm, amssymb}
\usepackage{yhmath}
\usepackage{mathrsfs}
\usepackage{mathtools}
\usepackage{xcolor}
\usepackage{graphicx}
\usepackage{comment}
\usepackage{tikz-cd}
\usepackage{quiver}
\usepackage{hyperref,cleveref}
\renewcommand{\familydefault}{ppl}
\newcommand{\tr}{\text{tr}}
\newcommand{\R}{\mathbb{R}}
\newcommand{\E}{\mathbb{E}}
\newcommand{\Z}{\mathbb{Z}}
\newcommand{\C}{\mathbb{C}}
\newcommand{\F}{\mathbb{F}}
\newcommand{\la}{\langle}
\newcommand{\ra}{\rangle}
\newcommand{\colim}{\text{colim}}
\DeclareMathOperator{\im}{im}
\DeclareMathOperator{\disc}{disc}
\let\oldemptyset\emptyset
\let\emptyset\varnothing
\newcommand{\tor}{\text{Tor}}
\newcommand{\id}{\text{id}}
\newcommand{\ext}{\text{Ext}}
\newcommand{\ptop}{\text{PTop}}
\newcommand{\pt}{\text{pt}}
\newcommand{\ach}{\text{Ach}}
\newcommand{\Q}{\mathbb{Q}}
\newcommand{\gal}{\text{Gal}}
\newcommand{\fraccomma}{\genfrac{}{}{0pt}{}{}{,}}
\newcommand{\diverg}{\operatorname{div}}
\newcommand{\curl}{\operatorname{curl}}
\definecolor{wikipediadarkblue}{rgb}{0.023, 0.270, 0.676}
\definecolor{wikipediadarkblue}{rgb}{0.023, 0.270, 0.676}
\hypersetup{
    colorlinks,
    citecolor=black,
    filecolor=black,
    linkcolor=wikipediadarkblue,
    urlcolor=red
}



\input{hui_r.tex}

\title{Calc III Final Review
\\ 
\vspace{0.4cm}
\large Fall 2025}



\author{(Please email hsun95@jh.edu if you see typos.)}
\date{\today}


\begin{document}

\maketitle

\tableofcontents
\newpage


\chapter{Definition Review}


\begin{defn}[standard basis in $\R^3$]
    The vectors 
    \begin{equation*}
        i=(1,0,0), j=(0,1,0), k=(0,0,1)
    \end{equation*}
    are called the \textbf{standard basis} vectors of $\R^3$, and for any vector $a=(a_1,a_2,a_3)\in\R^3$, we can write 
    \begin{equation*}
        a=a_1i+a_2j+a_3k
    \end{equation*}
\end{defn}



\begin{defn}[Equation of a line]\label{line}
    A \textbf{line} $l$ in $\R^3$ through the tip of $a=(a_1,a_2,a_3)$ pointing in the direction of a vector $v=(v_1,v_2,v_3)$ is given by 
    \begin{equation*}
        l(t)=a+tv=(a_1+tv_1, a_2+tv_2, a_3+tv_3)
    \end{equation*}
    where $t\in\R$. Alternatively, a line passing through two points $P=(x_1,y_1,z_1), Q=(x_2,y_2,z_2)$ is given by 
    \begin{equation*}
        l(t)=(x(t), y(t), z(t))
    \end{equation*}
    where 
    \begin{equation*}
        \begin{cases}
            x(t)=x_1+(x_2-x_1)t\\
            y(t)=y_1+(y_2-y_1)t\\
            z(t)=z_1+(z_2-z_1)t
        \end{cases}
    \end{equation*}
\end{defn}



\begin{defn}[inner product, dot product]
    Let $a,b\in\R^3$, the \textbf{dot product}, also called the inner product, of $a,b$ is 
    \begin{equation*}
        a\cdot b=a_1b_1+a_2b_2+a_3b_3
    \end{equation*}
    where $a=(a_1,a_2,a_3), b=(b_1,b_2,b_3)$. The \textbf{norm}, also called the length, of $a$ is 
    \begin{equation*}
        \|a\|=(a\cdot a)^\frac{1}{2}
    \end{equation*}
    A vector of norm $1$ is called a \textbf{unit vector}. Given any $u\in\R^3$, we can find the unit vector $\frac{u}{\|u\|}$ pointing in the same direction as $u$, this is called ``normalizing'' $u$.
\end{defn}


\begin{defn}[orthogonal projection]
    The \textbf{orthogonal projection} of vector $v$ onto another vector $a$ is
    \begin{equation*}
        \text{Proj}_av=\frac{a\cdot v}{a\cdot a}a
    \end{equation*}
    For example, the orthogonal projection of $(1,1,0)$ onto $(1,1,1)$ is 
    \begin{equation*}
        \left(\frac{2}{3}, \frac{2}{3},\frac{2}{3}\right)
    \end{equation*}
\end{defn}


\begin{defn}[orthogonal]
    Let $a,b\in\R^n$, then $a,b$ are called \textbf{orthogonal} or perpendicular iff 
    \begin{equation*}
        a\cdot b=0
    \end{equation*}
\end{defn}


\begin{defn}[determinant]
    The \textbf{determinant} of a $2\times 2$ matrix is given by 
    \begin{equation*}
        \det\begin{bmatrix}
            a&b\\
            c&d
        \end{bmatrix}=ad-bc
    \end{equation*}
    and the determinant of a $3\times 3$ matrix is given by 
    \begin{equation*}
        \det\begin{bmatrix}
            a_{11}&a_{12}&a_{13}\\
            a_{21}&a_{22}&a_{23}\\
            a_{31}&a_{32}&a_{33}\\
        \end{bmatrix}=a_{11}(a_{22}a_{33}-a_{23}a_{32})-a_{12}(a_{21}a_{33}-a_{23}a_{31})+a_{13}(a_{21}a_{32}-a_{22}a_{31})
    \end{equation*}
\end{defn}



\begin{defn}[cross product]
    Let $a,b\in\R^3$, write $a=(a_1,a_2,a_3), b=(b_1,b_2,b_3)$, then the \textbf{cross product}
    \begin{equation*}
        a\times b=\det\begin{bmatrix}
            i&j&k\\
            a_1&a_2&a_3\\
            b_1&b_2&b_3
        \end{bmatrix}
    \end{equation*}
    where $i,j,k$ are the standard vectors in $\R^3$.
\end{defn}

\begin{defn}[Plane in $\R^3$]\label{plane}
   If a plane ${P}$ passes through some point $(x_0,y_0,z_0)$, and $n=(A,B,C)$ is a vector orthogonal to the plane, then the plane ${P}$ is given by the equation:
    \begin{equation*}
        A(x-x_0)+B(y-y_0)+C(z-z_0)=0
    \end{equation*}
    (Notice that a point in ${P}$ and a normal vector to ${P}$ uniquely define a plane in $\R^3$.)
\end{defn}


\begin{defn}[image, graph]
    The \textbf{image} of a function $f: U\subset\R^n\to\R^m$ is a subset of $\R^m$,
    \begin{equation*}
        \text{Image}(f)=\{f(x)\in\R^m: x\in U\}
    \end{equation*}
    and the \textbf{graph} of $f$ is a subset of $\R^{n+m}$,
    \begin{equation*}
        \text{Graph}(f)=\{(x,f(x)): x\in U\}
    \end{equation*}
\end{defn}


\begin{defn}[level set]
    Let $f:U\subset\R^n\to\R^m$, and $c\in\R$ be some constant. Then the \textbf{level set} of $f$ at $c$ is the set 
    \begin{equation*}
        \{x\in U: f(x)=c\}\subset\R^n
    \end{equation*}
\end{defn}

\begin{defn}[open set, closed set, neighborhood, boundary]
    Let $U\subset\R^n$, we say $U$ is an \textbf{open set} if for every $x_0\in U$, there exists some $r>0$ such that $D_r(x_0)\subset U$, where $D_r(x_0)$ is the open disk of radius $r$ centered at $x_0$:
    \begin{equation*}
        D_r(x_0)=\{x\in\R^n: \|x-x_0\|<r\}
    \end{equation*}
    Some examples of open sets: $\R$, $D_{1}((0,0))$, $(1,2)\subset\R$.
    A \textbf{neighborhood} of $x_0\in\R^n$ is an open set containing the point $x_0$. A point $x\in\R^n$ is called a \textbf{boundary point} of $A$ if \textit{every} neighborhood of $x$ contains at least one point in $A$ and at least one point not in $A$. A set is \textbf{closed} if it contains all its boundary points. Example of closed set: level sets of a continuous function $f$.
\end{defn}


\begin{defn}[limit]
    Let $f: A\subset\R^n\to\R^m$, where $A$ is open, let $x_0$ be in $A$ or be a boundary point of $A$ and $N$ be a neighborhood of a point $b\in\R^m$. Now let $x$ approach $x_0$, $f$ is said to be \textbf{eventually in $N$} if there exists a neighborhood $U$ of $x_0$ such that 
    \begin{equation*}
        \text{ if } x\in U, \text{ then } f(x)\in N
    \end{equation*}
    If $f$ is eventually in $N$ for \textit{any} neighborhood $N$ around $b$, then the \textbf{limit} of $f$ as $x\to x_0$ exists, denoted as 
    \begin{equation*}
        \lim_{x\to x_0}f(x)=b
    \end{equation*}
\end{defn}

\begin{defn}[continuous]
    Let $f:A\subset\R^n\to\R^m$ and $x_0\in A$, then $f$ is \textbf{continuous at} \boldmath{${x_0}$} \unboldmath if 
    \begin{equation*}
        \lim_{x\to x_0}f(x)=f(x_0)
    \end{equation*}
\end{defn}


\begin{defn}[partial derivative]
    Let $f:U\subset\R^n\to\R$, where $U$ is open. Then the \textbf{partial derivative} with respect to $x_i$ is defined by 
    \begin{equation*}
        \frac{\partial f}{\partial x_i}(x_1, \dots, x_n)=\lim_{h\to 0}\frac{f(x+he_i)-f(x)}{h}
    \end{equation*}
    where $e_i=(0,\dots, 1,\dots, 0)$ with $1$ in the $i$th coordinate. 
\end{defn}



\begin{defn}[differentiability in two variables]
    Let $f:\R^2\to\R$, then $f$ is \textbf{differentiable} at $(x_0,y_0)$ if 
    \begin{enumerate}
        \item[(1)] $\frac{\partial f}{\partial x},\frac{\partial f}{\partial y}$ exist at $(x_0,y_0)$
        \item[(2)] 
        \begin{equation*}
            \lim_{(x,y)\to(x_0,y_0)}\frac{f(x,y)-f(x_0,y_0)-\left[\frac{\partial f}{\partial x}(x_0,y_0)\right](x-x_0)-\left[\frac{\partial f}{\partial y}(x_0,y_0)\right](y-y_0)}{\|(x,y)-(x_0,y_0)\|}=0
        \end{equation*}
    \end{enumerate}
    The derivative of $f$ at $(x_0,y_0)$ is the $1\times 2$ matrix 
    \begin{equation*}
        \begin{bmatrix}\frac{\partial f}{\partial x}(x_0,y_0)&\frac{\partial f}{\partial y}(x_0,y_0)\end{bmatrix}
    \end{equation*}
    Moreover, the \textbf{tangent plane} of the graph of $f$ at $(x_0,y_0, f(x_0,y_0))$ is given by 
    \begin{equation*}
        z=f(x_0,y_0)+\left[\frac{\partial f}{\partial x}(x_0,y_0)\right](x-x_0)+\left[\frac{\partial f}{\partial y}(x_0,y_0)\right](y-y_0)
    \end{equation*}
\end{defn}

\begin{defn}[differentiability in the general setting]
    Let $f:U\subset\R^n\to\R^m$, then $f$ is differentiable at $x_0\in U$ if
    \begin{enumerate}
        \item[(1)] the partial derivatives $\frac{\partial f_i}{\partial x_j}$ exist for all $1\leq i\leq m, 1\leq j\leq n$. 
        \item[(2)]
        \begin{equation*}
            \lim_{x\to x_0}\frac{\|f(x)-f(x_0)-T(x-x_0)\|}{\|x-x_0\|}=0
        \end{equation*}
        where $T=Df(x_0)$ is the $m\times n$ matrix 
        \begin{equation*}
            Df(x_0)=
            \begin{bmatrix}
            \frac{\partial f_1}{\partial x_1}(x_0) & \cdots & \frac{\partial f_1}{\partial x_n}(x_0) \\
            \vdots & \ddots & \vdots \\
            \frac{\partial f_m}{\partial x_1}(x_0) & \cdots & \frac{\partial f_m}{\partial x_n}(x_0)
            \end{bmatrix}
        \end{equation*}
    \end{enumerate}
    The derivative of $f$ at $x_0$ is the $m\times n$ matrix $Df(x_0)$.
\end{defn}

\begin{defn}[gradient]
    Let $f:U\subset\R^n\to\R$, the \textbf{gradient} $\nabla f(x)$ is a special case of the general case above when $m=1$, i.e., it is a $1\times n$ matrix 
    \begin{equation*}
        Df(x)=\begin{bmatrix}
            \frac{\partial f}{\partial x_1}&\dots&\frac{\partial f}{\partial x_n}
        \end{bmatrix}
    \end{equation*}
\end{defn}


\begin{defn}[path and curve]
    A \textbf{path} in $\R^n$ is a map $c:[a,b]\to\R^n$, and the image of $c$ is called a \textbf{curve}. We say the path $c$ parametrizes the curve. 

    \noindent For example, $c(t)=(\cos t, \sin t)$ is a path, and the unit circle is a curve.
\end{defn}


\begin{defn}[velocity of a path]
    Let $c:[a,b]\to\R^n$ be a path, and we can write $c(t)=(c_1(t), \dots, c_n(t))$. If $c$ is differentiable, then we define the \textbf{velocity} of $c$ at any $t_0\in [a,b]$ as 
    \begin{equation*}
        c'(t_0)=\left(c_1'(t_0), \dots, c_n'(t_0)\right)
    \end{equation*}
    The velocity vector of $c$ at $t_0$ is also a \textbf{tangent} vector to $c$ at $t_0$. The \textbf{speed} of the path $c$ at $t_0$ is the length of the velocity vector $\|c'(t_0)\|$.
\end{defn}

\begin{defn}[tangent line to a path]\label{tangent path}
    Let $c:[a,b]\to\R^n$ be a path, if $c'(t_0)\neq 0$, then the \textbf{tangent line} at $x_0$ is given by 
    \begin{equation*}
        l(t)=c(t_0)+c'(t_0)(t-t_0)
    \end{equation*}
\end{defn}




\begin{defn}[directional derivative]
    Let $f:\R^3\to\R$, be differentiable, then the \textbf{directional derivative} at $x_0\in\R^3$ in the direction of a \textit{unit vector} $v$ is given by 
    \begin{equation*}
        \nabla f(x_0)\cdot v=\left[\frac{\partial f}{\partial x_1}(x_0)\right]v_1+\left[\frac{\partial f}{\partial x_2}(x_0)\right]v_2+\left[\frac{\partial f}{\partial x_3}(x_0)\right]v_3
    \end{equation*}
    where $v=(v_1,v_2,v_3)$.
\end{defn}
\begin{warn}
    Make sure you normalize any given direction $v$! This formula works for unit vectors.
\end{warn}



\begin{defn}[First order Taylor expansion]
    Let $f:U\subset\R^n\to\R$ be differentiable at $a\in U$, then 
    \begin{equation*}
        f(x)=f(a)+\sum_{i=1}^n\frac{\partial f}{\partial x_i}(a)(x_i-a_i)+R_1(a,x)
    \end{equation*}
    where 
    \begin{equation*}
        \frac{R_1(a,x)}{\|x-a\|}\to 0 \text{ as } x\to a
    \end{equation*}
\end{defn}

\begin{defn}[Second order Taylor expansion]\label{taylor}
    Let $f:U\subset\R^n\to\R$ be twice continuously differentiable at $a\in U$, then 
    \begin{equation*}
        f(x)=f(a)+\sum_{i=1}^n\frac{\partial f}{\partial x_i}(a)(x_i-a_i)+\frac{1}{2}\sum_{i,j=1}^n\frac{\partial^2 f}{\partial x_i\partial x_j}(a)(x_i-a_i)(x_j-a_j)+R_2(a,x)
    \end{equation*}
    where 
    \begin{equation*}
        \frac{R_2(a,x)}{\|x-a\|}\to 0 \text{ as } x\to a
    \end{equation*}
\end{defn}



\begin{defn}[critical point]
    Let $f:U\subset\R^n\to\R$, a point $x_0\in U$ is a \textbf{critical point} of $f$ if either $f$ is not differentiable at $x_0$, or $Df(x_0)=0$. A critical point that is not a local extremum is called a saddle point.
\end{defn}

\begin{defn}[quadratic function]
    A function $g:\R^n\to\R$ is called a \textbf{quadratic function} if it is given by 
    \begin{equation*}
        g(h_1,\dots, h_n)=\sum_{i,j=1}^na_{ij}h_ih_j
    \end{equation*}
    where $(a_{ij})$ is an $n\times n$ matrix. We can also write $g$ as follows:
    \begin{equation*}
        g(h_1,\dots,h_n)=[h_1,\dots,h_n]\begin{bmatrix}
            a_{11}&\dots &a_{1n}\\
            \vdots&\ddots &\vdots\\
            a_{n_1}&\dots&a_{nn}
        \end{bmatrix}\begin{bmatrix}
            h_1\\
            \vdots\\
            h_n
        \end{bmatrix}
    \end{equation*}

\end{defn}

\begin{defn}[Hessian matrix]
    Let $f:U\subset\R^n\to\R$, and suppose all the second-order partial derivatives $\frac{\partial^2 f}{\partial x_i\partial x_j}$ exist, then the Hessian matrix of $f$ is the $n\times n$ matrix given by 
    \begin{equation*}
        Hf=\begin{bmatrix}
            \frac{\partial^2 f}{\partial x_1\partial x_1}&\dots&\frac{\partial^2f}{\partial x_1\partial x_n}\\
            \vdots&\ddots&\vdots\\
            \frac{\partial^2f}{\partial x_nx_1}&\dots&\frac{\partial^2f}{\partial x_n\partial x_n}
        \end{bmatrix}
    \end{equation*}
    The Hessian as a quadratic function is defined by 
    \begin{equation*}
        Hf(x)(h)=\frac{1}{2}\begin{bmatrix}
            h_1&\dots&h_n
        \end{bmatrix}Hf(x)\begin{bmatrix}
            h_1\\
            \dots\\
            h_n
        \end{bmatrix}
    \end{equation*}
    where $h=(h_1,\dots,h_n)$.
\end{defn}

\begin{defn}[degenerate/nondegenerate points]
    Let $f:U\subset\R^2\to\R$ be of $C^2$, let $(x_0,y_0)$ be a critical point. We define the \textbf{discriminant}, \boldmath $\mathcal{D}$,\unboldmath of the Hessian by
    \begin{equation*}
        \mathcal{D}=\det (Hf)=\left(\frac{\partial^2f}{\partial x^2}\right)\left(\frac{\partial^2f}{\partial y^2}\right)-\left(\frac{\partial^2f}{\partial x\partial y}\right)^2
    \end{equation*}
    If $\mathcal{D}\neq 0$, the critical point $(x_0,y_0)$ is called \textbf{nondegenerate}; if $\mathcal{D}=0$, the point $(x_0,y_0)$ is called \textbf{degenerate}.
\end{defn}

\begin{defn}[positive, negative-definite]
    A quadratic function $g:\R^n\to\R$ is called \textbf{positive-definite} if $g(h)\geq 0$ for all $h\in\R^n$ and $g(h)=0$ implies $h=0$. Similarly, $g$ is \textbf{negative-definite} if $g(h)\leq 0$ for all $h\in\R^n$ and $g(h)=0$ implies $h=0$. 
\end{defn}


\begin{defn}[global extremum]
    Let $f:A\to\R$ be a function defined on $A\subset\R^2$ or $A\subset\R^3$. A point $x_0\in A$ is said to be an \textbf{absolute maximum} if $f(x_0)\geq f(x)$ for all $x\in A$. Similarly, $x_0$ is an \textbf{absolute minimum} if $f(x_0)\leq f(x)$ for all $x\in A$. 
\end{defn}

\begin{defn}[bounded set]
    A set $A\subset\R^n$ is said to be \textbf{bounded} if there is a number $M>0$ such that $\|x\|\leq M$ for all $x\in A$. 
\end{defn}

\begin{defn}[acceleration]
    Let $c(t)$ be a path, the \textbf{acceleration} $a(t)$ of $c(t)$ is 
    \begin{equation*}
        a(t)=c''(t)
    \end{equation*}
\end{defn}



\begin{defn}[arc length]
    Let $c(t)=(x(t), y(t), z(t))$ be a path, then the length of the path in $\R^3$ from $t_0\leq t\leq t_1$ is 
    \begin{align*}
        L_{t_0\to t_1}(c)&=\int_{t_0}^{t_1}\left(x'(t)^2+y'(t)^2+z'(t)^2\right)^\frac{1}{2}dt\\
        &=\int_{t_0}^{t_1}\|c'(t)\|dt
    \end{align*}
    More generally, if $c(t)=(x_1(t), \dots, x_n(t))$ is a path in $\R^n$, then 
    \begin{equation*}
        L_{t_0\to t_1}(c)=\int_{t_0}^{t_1}\left(\sum_{i=1}^nx_i'(t)^2\right)^\frac{1}{2}dt
    \end{equation*}
\end{defn}


\begin{defn}[vector field]
    A vector field is a function $F:A\subset\R^n\to\R^n$ that assigns $x\in\R^n$ to another vector $F(x)\in\R^n$.
\end{defn}


\begin{defn}[flow line]
    If $F$ is a vector field, a \textbf{flow line} for $F$ is a path $c(t)$ such that 
    \begin{equation*}
        c'(t)=F(c(t))
    \end{equation*}
    Intuitively speaking, flow lines are the ``streamlines'' threading through vector fields.
\end{defn}



\begin{defn}[divergence]
    Let ${F}$ be a vector field in $\R^3$ $F=(F_1,F_2,F_3)$, the divergence of ${F}$ is the \textbf{scalar field} (assigns one number to an given point $(x, y, z)$), 
    \begin{equation*}
        \diverg F\coloneq\nabla\cdot F=\frac{\partial F_1}{\partial x}+\frac{\partial F_2}{\partial y}+\frac{\partial F_3}{\partial z}
    \end{equation*}
    More generally, if $F=(F_1,\dots, F_n)$ is a vector field on $\R^n$, its divergence is 
    \begin{equation*}
        \diverg F=\sum_{i=1}^n\frac{\partial F_i}{\partial x_i}=\frac{\partial F_1}{\partial x_1}+\dots+\frac{\partial F_n}{\partial x_n}
    \end{equation*}
\end{defn}


\begin{defn}[curl]
    Let $F$ be a vector field in $\R^3$, writing $F=(F_1,F_2,F_3)$, the \textbf{curl} of $F$ is the vector field 
    \begin{equation*}
        \curl F\coloneq\nabla\times F=\det\begin{bmatrix}
            i&j&k\\
            \frac{\partial}{\partial x}&\frac{\partial}{\partial y}&\frac{\partial}{\partial z}\\
            F_1&F_2&F_3
        \end{bmatrix}
    \end{equation*}
    If $\curl F=0$, then we say the vector field is \textbf{irrotational}.
\end{defn}



\begin{defn}
    We say a region $D\subset\R^2$ is \boldmath$y$\unboldmath-\textbf{simple} if there are continuous functions $\phi_1,\phi_2$ such that $D$ is the set of pionts $(x,y)$ satisfying
    \begin{equation*}
        x\in[a,b], \quad  \phi_1(x)\leq y\leq\phi_2(x)
    \end{equation*}
    Similarly, we define $D$ to be \boldmath$x$\textbf{-simple} \unboldmath if there are continuous $\psi_1, \psi_2$ such that $D$ is the set of points $(x,y)$ satisfying 
    \begin{equation*}
        y\in[c,d], \quad \psi_1(y)\leq x\leq\psi_2(y)
    \end{equation*}
    A \textbf{simple} region is one that is both $x$- and $y$-simple.
\end{defn}



\begin{defn}[injective, surjective]
    Let $T:\R^2\to\R^2$ be a map, we say $T$ is \textbf{injective}, or \textbf{one-to-one}, on $D^*$, if for $x,y\in D^*$
    \begin{equation*}
        Tx=Ty
    \end{equation*}
    implies
    \begin{equation*}
        x=y.
    \end{equation*}
    We say $T$ is \textbf{surjective}, or \textbf{onto} $D$, if for all $y\in D$, there exists $x$ in the domain of $T$ such that 
    \begin{equation*}
        Tx=y
    \end{equation*}
    If $T$ is both injective and surjective, then we say $T$ is \textbf{bijective}.
\end{defn}


\begin{defn}[Jacobian Determinant]
    Let $T:D^*\subset\R^2\to\R^2$ be of $C^1$ defined by 
    \begin{equation*}
        T:\begin{pmatrix}
            u\\
            v
        \end{pmatrix}\mapsto \begin{pmatrix}
            x(u,v)\\
            y(u,v)
        \end{pmatrix}
    \end{equation*}
    The \textbf{Jacobian determinant} of $T$, denoted as $\frac{\partial(x,y)}{\partial(u,v)}$ is the determinant of the matrix $DT(u,v)$:
    \begin{equation*}
        \frac{\partial(x,y)}{\partial(u,v)}=\det\left|\begin{matrix}
            \frac{\partial x}{\partial u}& \frac{\partial x}{\partial v}\\
            \frac{\partial y}{\partial u}&\frac{\partial y}{\partial v}
        \end{matrix}\right|
    \end{equation*}
\end{defn}




\begin{defn}[path integral]
    Let $c:[a,b]\to\R^3$ be a path of $C^1$ and $f:\R^3\to\R$ is such that $f\circ c$ is continuous on $[a,b]$, The \textbf{path integral} of $f(x,y,z)$ along the path $c$ is given by 
    \begin{align*}
        \int_c fds&=\int_a^bf(c(t))\|c'(t)\|dt\\
        &=\int_a^bf(x(t),y(t), z(t))\|c'(t)\|dt
    \end{align*}
\end{defn}



\begin{defn}[line integral]
    Let $F$ be a vector field on $\R^3$ that is continuous on the $C^1$ path $c:[a,b]\to\R^3$, where $c(t)=(x(t), y(t), z(t))$. We define $\int_c F\cdot ds$, the \textbf{line integral} of $F$ along $c$ by the following
    \begin{align*}
        \int_c F\cdot ds&=\int_a^bF(c(t))\cdot c'(t)dt\\
        &=\int_a^b \left(F_1\frac{dx}{dt}+F_2\frac{dy}{dt}+F_3\frac{dz}{dt}\right)dt\\
        &\coloneq\int_c F_1dx+F_2dy+F_3dz
    \end{align*}
    the expression $F_1dx+F_2dy+F_3dz$ is called the \textbf{differential form}. 
    
    For example, the work done by a force field $F$ on a particle moving along a path $c$ is given by 
    \begin{equation*}
        \text{ work done by } F=\int_a^bF(c(t))\cdot c'(t)dt
    \end{equation*}
\end{defn}

% \begin{defn}[differential form]
%     We also write the line integral above as 
%     \begin{equation*}
%         \int_cF\cdot ds=\int_c F_1dx+F_2dy+F_3dz
%     \end{equation*}
%     where $F=(F_1,F_2,F_3)$. 
% \end{defn}


\begin{defn}[reparametrization]
    Let $h: I\to I_1$ be a $C^1$ real-valued bijective function. Let $c: I_1\to\R^3$ be a piecewise $C^1$ path. Then we call the composition 
    \begin{equation*}
        p=c\circ h: I\to\R^3
    \end{equation*}
    a \textbf{reparametrization} of $c$.

    For example, let $c:[0,1]\to\R^3$ be a $C^1$ path, then consider $h: [0,1]\to [0,1]$, where $h(t)=1-t$. Then the path 
    \begin{equation*}
        c_{\text{op}}=c\circ h(t)=c(1-t)
    \end{equation*}
    is the same path in the opposite direction.
\end{defn}



\begin{defn}[parametrization of surface]
    Let $S$ be a surface in $\R^3$, a \textbf{surface parametrization} is a map $\Phi:D\subset\R^2\to\R^3$, where 
    \begin{equation*}
        \Phi(u,v)=(x(u,v), y(u,v), z(u,v))
    \end{equation*}
\end{defn}


\begin{defn}[regular surface, tangent plane]
    Let $\Phi(u,v)$ be a parametrization of a surface $S\subset\R^3$.
    We say $S$ is \textbf{regular} at $\Phi(u_0,v_0)$ if 
    \begin{equation*}
        T_u\times T_v\neq 0 \text{ at } (u_0,v_0)
    \end{equation*}
    where 
    \begin{equation*}
        T_u=\frac{\partial\Phi}{\partial u}, \quad T_v=\frac{\partial\Phi}{\partial v}
    \end{equation*}
    If $S$ is regular at $\Phi(u_0,v_0)$, then we can find the tangent plane by first finding a normal vector to the surface at this point: $n=T_u\times T_v$, then the tangent plane at $(x_0, y_0, z_0)=\Phi(u_0, v_0)$ is given by 
    \begin{equation*}
        (x-x_0, y-y_0, z-z_0)\cdot n=0
    \end{equation*}
\end{defn}

\begin{defn}[surface area]
    Let $S\subset\R^3$ be a parametrized surface, then the \textbf{surface area} $A(S)$ of $S$ is given by 
    \begin{align*}
        A(S)&=\iint_D\|T_u\times T_v\|dudv\\
        &=\iint_D\left(\left[\frac{\partial(x,y)}{\partial(u,v)}\right]^2+\left[\frac{\partial(y,z)}{\partial(u,v)}\right]^2+\left[\frac{\partial(x,z)}{\partial(u,v)}\right]^2\right)^{1/2}dudv
    \end{align*}
    where $\|T_u\times T_v\|$ is the norm of $T_u\times T_v$, and 
    \begin{equation*}
        \frac{\partial(x,y)}{\partial(u,v)}=\det\left|\begin{matrix}
            \frac{\partial x}{\partial u}&\frac{\partial x}{\partial v}\\
            \frac{\partial y}{\partial u}&\frac{\partial y}{\partial v}
        \end{matrix}\right|, \quad \frac{\partial(y,z)}{\partial(u,v)}=\det\left|\begin{matrix}
            \frac{\partial y}{\partial u}&\frac{\partial y}{\partial v}\\
            \frac{\partial z}{\partial u}&\frac{\partial z}{\partial v}
        \end{matrix}\right|, \quad \frac{\partial(x,z)}{\partial(u,v)}=\det\left|\begin{matrix}
            \frac{\partial x}{\partial u}&\frac{\partial x}{\partial v}\\
            \frac{\partial z}{\partial u}&\frac{\partial z}{\partial v}
        \end{matrix}\right|
    \end{equation*}
\end{defn}


\begin{defn}[integral over a surface]
    Let $f:\R^3\to\R$ be continuous, i.e., $f$ is a scalar-valued continuous function defined on a parametrized surface $S$ by $\Phi:D\to S\subset\R^3$, we define the integral of $f$ over $S$ as 
    \begin{equation*}
        \iint_SfdS=\iint_Df(\Phi(u,v))\|T_u\times T_v\|dudv
    \end{equation*}
    A special case is when we take $S$ as the graph of some function $g(x,y)$. Then we have 
    \begin{equation*}
        \iint fdS=\iint_D\frac{f(x,y,g(x,y))}{\cos\theta}dxdy
    \end{equation*}
    where $\theta$ is the angle between the unit vector $k$ at $(x,y,g(x,y))$ and the normal vector to the surface. (Recall that the normal vector of a graph is given by $n=-\frac{\partial g}{\partial x}i-\frac{\partial g}{\partial y}j+k)$. More specifically, 
    \begin{equation*}
        \frac{1}{\cos\theta}=\sqrt{\left(\frac{\partial g}{\partial x}\right)^2+\left(\frac{\partial g}{\partial y}\right)^2+1}
    \end{equation*}
\end{defn}


\begin{defn}[flux]
    Let $F$ be a vector field defined on $S$, parametrized by $\Phi$. The surface integral of $F$, also called \textbf{flux}, over $S$, denoted by 
    \begin{equation*}
        \iint_\Phi F\cdot dS
    \end{equation*}
    is defined by 
    \begin{equation*}
        \iint_\Phi F\cdot dS=\iint_DF\cdot(T_u\times T_v)dudv
    \end{equation*}
\end{defn}



\begin{defn}[oriented surface]
    An oriented surface is a two-sided surface with one side as the \textbf{outside (positive)} and one side as the \textbf{inside (negative)}. Let $\Phi:D\to\R^3$ be a parametrization of an oriented surface $S$, then the parametrization $\Phi$ is said to be orientation-preserving if 
    \begin{equation*}
        \frac{T_u\times T_v}{\|T_u\times T_v\|}=n(\Phi(u,v)) 
    \end{equation*} 
    at all $(u,v)\in D$ for which $S$ is smooth at $\Phi(u,v)$, where $n(\Phi(u,v))$ is the unit normal vector to $S$ at $(u,v)$ pointing away from the positive side of $S$ ($n$ is given).
\end{defn}


\begin{defn}[potential function]
    Let $F$ be a conservative vector field, then $F=\nabla f$, for some continuously differentiable function real-valued $f$.
\end{defn}











\chapter{Theorem Review}

\begin{prop}[dot product]\label{dot}
    Let $a,b\in\R^3$, and let $\theta$ be the angle between $a,b$, where $0\leq\theta\leq\pi$, then 
    \begin{equation*}
        a\cdot b=\|a\|\|b\|\cos\theta
    \end{equation*}
\end{prop}




\begin{prop}[properties of the dot product]
    Let $a,b,c\in\R^n$, then 
    \begin{enumerate}
        \item[(a)] Nonnegativity: $a\cdot a\geq 0$, and $a\cdot a=0$ if and only if $a=0$.
        \item[(b)] Scalar multiplication: let $\lambda\in\R$, then 
        \begin{equation*}
            \lambda(a\cdot b)=\lambda a\cdot b=a\cdot \lambda b
        \end{equation*}
        \item[(c)] Distributivity:
        \begin{equation*}
            a\cdot(b+c)=a\cdot b+a\cdot c, \quad (a+b)\cdot c=a\cdot c+b\cdot c
        \end{equation*}
        \item[(d)] Symmetry: $a\cdot b=b\cdot a$.
    \end{enumerate}
\end{prop}



\begin{prop}[Cauchy-Schwarz]
    Let $a,b\in\R^n$, then $a\cdot b\in\R$, 
    \begin{equation*}
        |a\cdot b|\leq\|a\|\|b\|
    \end{equation*}
    where the left hand side is the absolute value of $a\cdot b$, and the right hand side is multiplication of two nonnegative real numbers.
\end{prop}


\begin{prop}[triangle inequality]\label{traingle}
    Let $a,b\in\R^n$, then 
    \begin{equation*}
        \|a+b\|\leq\|a\|+\|b\|
    \end{equation*}
\end{prop}



\begin{prop}[cross product]
    We have the following properties regarding the cross product: let $a,b\in\R^3$,
    \begin{enumerate}
        \item $a\times b$ is perpendicular to vectors $a,b$.
        \item The length of the cross product is the area of the parallelogram:
        \begin{equation*}
            \|a\times b\|=\|a\|\|b\|\sin\theta
        \end{equation*}
        where $0\leq\theta\leq\pi$ is the angle between them. 
        \item $a\times b=-b\times a$, $(a+b)\times c=a\times c+b\times c$, and $a\times (b+c)=a\times b+a\times c$. Moreover, $a\times b=0$ iff $a,b$ are parallel or either $a$ or $b$ are $0$.
        \item The cross product is \textbf{not} associative! For example, compute 
        \begin{equation*}
            (i\times i)\times j, \quad i\times (i\times j)
        \end{equation*}
    \end{enumerate}
\end{prop}



\begin{prop}[limits]
    Here are some properties of limits: let $f: U_1\subset\R^n\to\R^m, g: U_2\subset\R^n\to\R^m$,
    \begin{enumerate}
        \item[(a)] (Uniquessness):  \begin{equation*}
            \text{ If } \lim_{x\to x_0}f(x)=b_1, \quad \lim_{x\to x_0}f(x)=b_2
        \end{equation*}
        then we must have 
        \begin{equation*}
            b_1=b_2
        \end{equation*}
        \item[(b)] (Scalar multiplication): Let $c\in\R$, if $\lim_{x\to x_0}f(x)=b_1$, then 
        \begin{equation*}
            \lim_{x\to x_0}cf(x)=cb_1
        \end{equation*}
        \item[(c)] (Addition): Let $f$ be as in (b), and $\lim_{x\to x_0}g(x)=b_2$, then 
        \begin{equation*}
            \lim_{x\to x_0}(f+g)(x)=b_1+b_2
        \end{equation*}
        \item[(d)] (Component): Write $f(x)=(f_1(x),\dots, f_n(x))$, if $\lim_{x\to x_0}f(x)=b=(b_1,\dots, b_n)$, then 
        \begin{equation*}
            \lim_{x\to x_0}f_i(x)=b_i
        \end{equation*}
        for all $i=1,\dots, m$.
    \end{enumerate}
    The same set of properties hold for continuity.
\end{prop}

\begin{prop}[continuity of compositions]
    Let $g: A\subset\R^n\to\R^m$, and $f: B\subset\R^m\to\R^p$, and $g(A)\subset B$. If $g$ is continuous at $x_0$, $f$ is continuous at $g(x_0)$, then $f\circ g$ is continuous at $x_0$.
\end{prop}

\begin{prop}[differentiability implies continuity]
    Let $f:U\subset\R^n\to\R^m$. If $f$ is differentiable at $x_0\in U$, then $f$ is continuous at $x_0$.
\end{prop}

\begin{prop}[differentiability]
    Let $f:U\subset\R^n\to\R^m$. Suppose $\partial f_i/\partial x_j$ exists for all $i,j$ and are continuous in a neighborhood of $x_0\in U$, then $f$ is differentiable at $x_0$.
\end{prop}



\begin{prop}[properties of derivatives]
    Let $f:U\subset\R^n\to\R^m$ be differentiable at $x_0$, then the derivative of $f$ at $x_0$ is an $m\times n$ matrix $Df(x_0)=\left(\frac{\partial f_i}{\partial x_j}\right)_{ij}$. The derivative follows the same properties as derivative for single variable functions:
    \begin{enumerate}
        \item Let $c\in\R$, then 
        \begin{equation*}
            D(cf)(x_0)=cDf(x_0) \tag {multiplication of a matrix by constant $c$}
        \end{equation*}
        \item Let $g: U\subset\R^n\to\R^m$ also be differentiable at $x_0$, then 
        \begin{equation*}
            D(f+g)(x_0)=Df(x_0)+Dg(x_0) \tag{sum of two matrices}
        \end{equation*}
        \item Let $h_1: U\subset\R^n\to\R, h_2: U\subset\R^n\to\R$,then 
        \begin{equation*}
            D(h_1h_2)(x_0)=Dh_1(x_0)h_2(x_0)+h_1(x_0)Dh_2(x_0) \tag{product rule}
        \end{equation*}
        and if $h_2\neq 0$ on $U$.
        \begin{equation*}
            D(h_1/h_2)(x_0)=\frac{Dh_1(x_0)h_2(x_0)-h_1(x_0)Dh_2(x_0)}{h_2^2(x_0)} \tag{quotient rule}
        \end{equation*}
        \item Let $g: U\subset\R^n\to\R^m, f:V\subset\R^m\to\R^p$ such that $g(U)\subset V$, then 
        \begin{equation*}
            D(f\circ g)(x_0)=Df(g(x_0))Dg(x_0) \tag{chain rule}
        \end{equation*}
    \end{enumerate}
\end{prop}



\begin{prop}[fastest rate of change]
    Suppose that $\nabla f(x_0)\neq 0$, then the direction for which $f$ increases the fastest at $x_0$ is along $\nabla f(x_0)$.
\end{prop}


\begin{prop}[gradient is normal, tangent plane]\label{tangent plane}
    Let $f:\R^3\to\R$ be differentiable, let $S$ be a level surface of $f$, i.e., $S$ is a surface described by 
    \begin{equation*}
        f(x,y,z)=k
    \end{equation*}
    were $k$ is some constant. Let $(x_0,y_0,z_0)\in S$, then
    \begin{equation*}
        \nabla f(x_0,y_0,z_0) \text{ is \textbf{normal} to the level surface at } (x_0,y_0,z_0)
    \end{equation*}
    This means if $c(t)$ is a path in $S$, and $v(0)=(x_0,y_0,z_0)$, and if $v$ is a tangent vector to $c(t)$ at $t=0$, then 
    \begin{equation*}
        \nabla f(x_0,y_0, z_0)\cdot v=0
    \end{equation*}
    Moreover, if $\nabla f(x_0,y_0,z_0)\neq 0$, the \textbf{tangent plane} of $S$ at $(x_0,y_0,z_0)$ is given by 
    \begin{equation*}
        \nabla f(x_0,y_0,z_0)\cdot (x-x_0, y-y_0, z-z_0)=0
    \end{equation*}
\end{prop}

\begin{prop}[Equality of mixed partials]
    If $f(x,y)$ is twice continuously differentiable, then 
    \begin{equation*}
        \frac{\partial^2 f}{\partial x\partial y}=\frac{\partial^2f}{\partial y\partial x}
    \end{equation*}
\end{prop}


\begin{prop}[extremums are critical points]
    Let $f:U\subset\R^n\to\R$ be differentiable, where $U$ is open. If $x_0$ is a local extremum, then $Df(x_0)=0$. 
\end{prop}


\begin{prop}[extremum]
    Let $f\colon U\subset\R^n\to\R$ be in $C^3$, and $x_0$ is a critical point of $f$. If the Hessian $Hf(x_0)$ is positive-definite, then $x_0$ is a local minimum of $f$; if $Hf(x_0)$ is negative-definite, then $x_0$ is a local maximum.
\end{prop}




\begin{prop}[local minimum]
    Let $f(x,y)$ be of $C^2$, and $U$ is open in $\R^2$. A point $(x_0,y_0)$ is a strict local \textbf{minimum} of $f$ if the following conditions hold:
    \begin{enumerate}
        \item \begin{equation*}
            \frac{\partial f}{\partial x}(x_0,y_0)=\frac{\partial f}{\partial y}(x_0,y_0)=0
        \end{equation*}
        \item \begin{equation*}
            \mathcal{D}(x_0,y_0)>0
        \end{equation*}
        where $\mathcal{D}$ is the \textbf{discriminant} of the Hessian, defined by 
        \begin{equation*}
            \mathcal{D}=\det (Hf)=\left(\frac{\partial^2 f}{\partial x^2}\right)\left(\frac{\partial^2 f}{\partial y^2}\right)-\left(\frac{\partial^2 f}{\partial x\partial y}\right)^2
        \end{equation*}
        where $Hf$ is the $2\times 2$ Hessian matrix.
        \item  \begin{equation*}
            \frac{\partial^2f}{\partial x^2}(x_0,y_0)>0
        \end{equation*}
    \end{enumerate}
\end{prop}


\begin{prop}[local maximum]
    Let $f(x,y)$ be of $C^2$, and $U$ is open in $\R^2$. A point $(x_0,y_0)$ is a strict local \textbf{maximum} of $f$ if the following conditions hold:
    \begin{enumerate}
        \item \begin{equation*}
            \frac{\partial f}{\partial x}(x_0,y_0)=\frac{\partial f}{\partial y}(x_0,y_0)=0
        \end{equation*}
        \item \begin{equation*}
            \mathcal{D}(x_0,y_0)>0
        \end{equation*}
        where $\mathcal{D}$ is the {discriminant} of the Hessian, defined above.
        \item  \begin{equation*}
            \frac{\partial^2f}{\partial x^2}(x_0,y_0)<0
        \end{equation*}
    \end{enumerate}
\end{prop}

\begin{prop}[saddle points]
    Let $f(x,y):U\subset\R^2\to \R$ be of  $C^2$, if $\frac{\partial f}{\partial x}(x_0,y_0)=\frac{\partial f}{\partial y}(x_0,y_0)=0$, and $\mathcal{D}(x_0,y_0)<0$, where $\mathcal{D}$ is the discriminant, then the critical point $(x_0,y_0)$ is a saddle point, i.e., neither a maximum or a minimum.
\end{prop}


\begin{prop}[continuous functions attain extremum on closed bounded sets]
    Let $f:D\to\R$ be continuous, where $D$ is closed and bounded in $\R^n$. Then $f$ assumes its absolute maximum and absolute minimum values at some point $x_0, x_1\in D$.
\end{prop}









\begin{prop}
    Let $f$ be constrained to a surface $S$, if $f$ has a max or a min at $x_0$, then $\nabla f(x_0)$ is perpendicular to $S$ at $x_0$.    
\end{prop}

\begin{prop}[Lagrange]
    Suppose that $f:U\subset\R^n\to\R$ and $g:U\subset\R^n\to\R$ are $C^1$ functions. Let $x_0\in U$ and $g(x_0)=c$, and let $S$ be the level set for $g$ at $c$, i.e., $S=\{x: g(x)=c\}$. Assume $\nabla g(x_0)\neq 0$, then if $f$ has a local maximum or minimum on $S$ at $x_0$, then there exists some real number $\lambda$ such that
    \begin{equation*}
        \nabla f(x_0)=\lambda \nabla g(x_0)
    \end{equation*}
\end{prop}


\begin{prop}[Bordered Hessian]
    Let $f: U\subset\R^2\to\R$ and $g:U\subset\R^2\to\R$ be smooth functions. Let $x_0\in U, g(x_0)=c$, and let $S$ be the level curve of $g$ with value $c$. Assume that $\nabla g(x_0)\neq 0$ and that there exists a real number $\lambda$ such that 
    \begin{equation*}
        \nabla f(x_0)=\lambda\nabla g(x_0)
    \end{equation*}
    Let $h=f-\lambda g$ and the bordered Hessian determinant is defined by 
    \begin{equation*}
        |\bar{H}|=\det\left|\begin{matrix}
            0&-\frac{\partial g}{\partial x}&-\frac{\partial g}{\partial y}\\
            -\frac{\partial g}{\partial x} &\frac{\partial^2h}{\partial x^2}&\frac{\partial^2h}{\partial x\partial y}\\
            -\frac{\partial g}{\partial y} &\frac{\partial^2h}{\partial x\partial y}&\frac{\partial^2h}{\partial y^2}
        \end{matrix} \right|
    \end{equation*}
    For $f$ restricted to the curve $S$, 
    \begin{enumerate}
        \item If  $|\bar{H}|>0$, then $x_0$ is a local max.
        \item If $|\bar{H}|<0$, then $x_0$ is a local min.
        \item If $|\bar{H}|=0$, then it is inconclusive.
    \end{enumerate}
\end{prop}


\begin{prop}[Newton's Second Law]
    Let $F$ be the force acting on a particle of mass $m$, then 
    \begin{equation*}
        F=ma
    \end{equation*}
    where $a$ is the acceleration.
\end{prop}



\begin{prop}[gradient is irrotational]
    Let $f\in C^2$, viewing $\nabla f$ as a vector field, then
    \begin{equation*}
        \nabla\times(\nabla f)=0
    \end{equation*}    
\end{prop}

\begin{prop}[divergence of a curl vanishes]
    For any $C^2$ vector field $F$, 
    \begin{equation*}
        \nabla\cdot(\nabla\times F)=0
    \end{equation*}
\end{prop}



\begin{prop}[Fubini's Theorem for rectangles]
    Let $f$ be a continuous function on a rectangular domain $R=[a,b]\times[c,d]$, then 
    \begin{equation*}
        \int_a^b\int_c^df(x,y)dydx=\int_c^d\int_a^bf(x,y)dxdy
    \end{equation*}
\end{prop}

\begin{prop}[Fubini's Theorem for general regions]
    Suppose $D$ is a set of points $(x,y)$ such that $y\in [c,d]$ and $\psi_1(y)\leq x\leq\psi_2(y)$, and similarly for $x\in [a,b]$, $\varphi_1(x)\leq y\leq\varphi_2(x)$. If $f$ is continuous on $D$, then 
    \begin{equation*}
        \iint_Df(x,y)dA=\int_a^b\int_{\varphi_1(x)}^{\varphi_2(x)}f(x,y)dydx=\int_c^d\int_{\psi_1(y)}^{\psi_2(y)}f(x,y)dxdy
    \end{equation*}
\end{prop}


\begin{prop}
    We have the following identities regarding divergence and curl: 
    \begin{enumerate}
        \item $\nabla(f+g)=\nabla f+\nabla g$.
        \item $\nabla (cf)=c\nabla f$, for constant $c$.
        \item $\nabla (fg)=f\nabla g+g\nabla f$.
        \item $\nabla (f/g)=(g\nabla f-f\nabla g)/g^2$, at points $x$ where $g(x)\neq 0$.
        \item $\diverg (F+G)=\diverg F+\diverg G$.
        \item $\curl(F+G)=\curl F+\curl G$.
        \item $\diverg(fF)=f\diverg F+F\cdot\nabla f$.
        \item $\diverg (F\times G)=G\cdot\curl F-F\cdot\curl G$.
        \item $\diverg\curl F=0$.
        \item $\curl(fF)=f\curl F+\nabla f\times F$.
        \item $\curl\nabla f=0$.
        \item $\nabla^2(fg)=f\nabla^2g+g\nabla^2f+2(\nabla f\cdot \nabla g)$.
        \item $\diverg(\nabla f\times\nabla g)=0$.
        \item $\diverg(f\nabla g-g\nabla f)=f\nabla^2g-g\nabla^2f$.
    \end{enumerate}
\end{prop}



\begin{prop}[integrability]
    For different assumptions on $f$, we have the following integrability results:
    \begin{enumerate}
        \item Let $f$ be continuous and defined on a closed rectangle $R$, then $f$ is integrable over $R$.
        \item Let $f:R\to\R$ be a bounded function on $R$ and suppose the set of points where $f$ is discontinuous lies on a finite union of graphs of continuous functions, then $f$ is integrable over $R$.
    \end{enumerate}
\end{prop}


\begin{prop}[Fubini's Theorem for rectangles]
    For different assumptions on $f$, we have the following Fubini's theorem results:
    \begin{enumerate}
        \item   Let $f$ be a continuous function on a rectangular domain $R=[a,b]\times[c,d]$, then 
        \begin{equation*}
            \int_a^b\int_c^df(x,y)dydx=\int_c^d\int_a^bf(x,y)dxdy=\iint_Rf(x,y)dA
        \end{equation*}
        \item Let $f$ be bounded with domain $R=[a,b]\times[c,d]$ and the discontinuities of $f$ lie on a finite union of graphs of continuous functions. If the integral $\int_a^bfdy$ exists for each $x\in[a,b]$, then 
        \begin{equation*}
            \int_a^b\left(\int_c^df(x,y)dy\right)dx
        \end{equation*}
        exists and 
        \begin{equation*}
            \int_a^b\int_c^df(x,y)dydx=\iint_Rf(x,y)dA
        \end{equation*}
        Similar results hold if $\int_a^bfdx$ exists for each $y\in[c,d]$. If both hold simultaneously, then 
        \begin{equation*}
            \int_a^b\left(\int_c^df(x,y)dy\right)dx=\int_c^d\left(\int_a^bf(x,y)dx\right)dy=\iint_Rf(x,y)dA
        \end{equation*}
    \end{enumerate}

  
\end{prop}



\begin{prop}[Fubini's Theorem for general regions]
    Suppose $D$ is a set of points $(x,y)$ such that $y\in [c,d]$ and $\psi_1(y)\leq x\leq\psi_2(y)$, and similarly for $x\in [a,b]$, $\varphi_1(x)\leq y\leq\varphi_2(x)$. If $f$ is continuous on $D$, then 
    \begin{equation*}
        \iint_Df(x,y)dA=\int_a^b\int_{\varphi_1(x)}^{\varphi_2(x)}f(x,y)dydx=\int_c^d\int_{\psi_1(y)}^{\psi_2(y)}f(x,y)dxdy
    \end{equation*}
\end{prop}



\begin{prop}[simple-regions]
    If $D$ is a $x$-simple region with $y\in[c,d], \psi_1(y)\leq x\leq\psi_2(y)$, and if $f$ is continuous on $D$, then 
    \begin{equation*}
        \iint_Df(x,y)dA=\int_c^d\left(\int_{\psi_1(y)}^{\psi_2(y)}f(x,y)dx\right)dy
    \end{equation*}
    Similarly, if $D$ is $y$-simple, then
    \begin{equation*}
        \iint_Df(x,y)dA=\int_a^b\int_{\phi_1(x)}^{\phi_2(x)}f(x,y)dydx
    \end{equation*}
    If $D$ is simple, then the two expressions above are equal.
\end{prop}
For example, the area of a $x$-simple region $D$ can be computed as 
    \begin{equation*}
        \iint_DdA=\int_c^d\psi_2(y)-\psi_1(y)dy
    \end{equation*}

\begin{prop}
    Let $A$ be $2\times 2$ matrix with $\det(A)\neq 0$ and let $T:\R^2\to\R^2$ be the linear map $Tx=Ax$. Then $T$ transforms parallelograms into parallelograms and vertices into vertices.
\end{prop}

\begin{prop}
    Let $T:\R^n\to\R^n$ be a linear map, i.e., there exists $n\times n$ matrix $A$ such that $Tx=Ax$, then $T$ is injective iff surjective iff $\det(A)\neq 0$.
\end{prop}


\begin{thm}[change of variables formula]
    Let $D,D^*$ be elementary regions in $\R^2$, suppose $T:D^*\to D$ is both one-to-one and onto. Then for any integral function $f:D\to\R$, the \textbf{change of variable formula} states
    \begin{equation*}
        \iint_Df(x,y)dxdy=\iint_{D^*}f(x(u,v),y(u,v))\left|\det(J)\right|dudv
    \end{equation*}
    where 
    \begin{equation*}
        \det(J)=\left|\frac{\partial(x,y)}{\partial(u,v)}\right|
    \end{equation*}
    is the Jacobian determinant. 
\end{thm}

\begin{prop}[change of variables-polar coordinates]
    As a corollary to the theorem above, we have the following change of variables formula for polar coordinates:
    \begin{equation*}
        \iint_Df(x,y)dxdy=\iint_{D^*}f(r\cos\theta, r\sin\theta)rdrd\theta
    \end{equation*}
\end{prop}



\begin{prop}[change of variables-triple]
    Let $W,W^*$ be elementary regions in $\R^3$, and suppose $T:W^*\to W$ is bijective. Then the change of varialbes formula for triple integrals states:
    \begin{equation*}
        \iiint_Wf(x,y,z)dxdydz=\iiint_{W^*}f(x(u,v,w), y(u,v,w), z(u,v,w))\mid\det(J)\mid dudvdw
    \end{equation*}
    where 
    \begin{equation*}
        \det(J)=\det\left|\begin{matrix}
            \frac{\partial x}{\partial u}&\frac{\partial x}{\partial v}&\frac{\partial x}{\partial w}\\
            \frac{\partial y}{\partial u}&\frac{\partial y}{\partial v}&\frac{\partial y}{\partial w}\\
            \frac{\partial z}{\partial u}&\frac{\partial z}{\partial v}&\frac{\partial z}{\partial w}
        \end{matrix}\right|
    \end{equation*}
    is the Jacobian determinant.
\end{prop}


\begin{prop}[change of variables-triple cylindrical]
    As a corollary to the above, we have the following change of variables formula for cylindrical coordinates:
    \begin{equation*}
        \iiint_Wf(x,y,z)dxdydz=\iiint_{W^*}f(r\cos\theta, r\sin\theta, z)rdrd\theta dz
    \end{equation*}
    Recall cylinder cooridnates is setting up the following 
    \begin{equation*}
        x=r\cos\theta, y=r\sin\theta, z=z
    \end{equation*}
\end{prop}

\begin{prop}[change of variables-triple spherical]
    As a corollary to the above, we have the following change of variables formula for spherical coordinates:
    \begin{equation*}
        \iiint_Wf(x,y,z)dxdydz=\iiint_{W^*}f(\rho\sin\phi\cos\theta, \rho\sin\phi\sin\theta, \rho\cos\phi)\rho^2\sin\phi d\rho d\theta d\phi
    \end{equation*}
    Recall the spherical cooridnates is setting up the following 
    \begin{equation*}
        x=\rho\sin\phi\cos\theta, y=\rho\sin\phi\sin\theta, z=\rho\cos\phi
    \end{equation*}


\end{prop}






\begin{prop}[reparametrization for path integrals]
    Let $c$ be a $C^1$ path and $c'$ be any reparametrization of $c$, and let $f$ be a continuous function on the image of $c$, then 
    \begin{equation*}
        \int_cf(x,y,z)ds=\int_{c'}f(x,y,z)ds
    \end{equation*}
\end{prop}

\begin{prop}[reparametrization for line integrals]
    Let $F$ be a vector feld continuous on the $C^1$ path $c:[a,b]\to\R^3$, and let $c':[a',b']\to\R^3$ be a reparametrization of $c$. If the reparametrization $c'$ is orientation-preserving, then 
    \begin{equation*}
        \int_{c'}F\cdot ds=\int_c F\cdot ds
    \end{equation*}
    If $c'$ is orientation-reversing, then 
    \begin{equation*}
        \int_{c'}F\cdot ds=-\int_cF\cdot ds
    \end{equation*}
\end{prop}

\begin{prop}[fundamental theorem of line integrals]
    Suppose $f:\R^3\to\R$ is of $C^1$ and that $c:[a,b]\to\R^3$ is piecewise $C^1$. Then 
    \begin{equation*}
        \int_c\nabla f\cdot ds=f(c(b))-f(c(a))
    \end{equation*}
\end{prop}


\begin{prop}[surface integral of vector fields and orientations]
    Let $S$ be an oriented surface and let $\Phi_1, \Phi_2$ be two regular orientation-preserving parametrizations, with $F$ a continuous vector field defined on $S$. Then 
    \begin{equation*}
        \iint_{\Phi_1}F\cdot dS=\iint_{\Phi_2}F\cdot dS
    \end{equation*}
    If $\Phi_1$ is orientation-preserving and $\Phi_2$ is orientation-reversing, then 
    \begin{equation*}
        \iint_{\Phi_1}F\cdot dS=-\iint_{\Phi_2}F\cdot dS
    \end{equation*}
    If $f$ is a real-valued continuous function defined on $S$, and $\Phi_1, \Phi_2$ are parametrizations of $S$, then 
    \begin{equation*}
        \iint_{\Phi_1}fdS=\iint_{\Phi_2}fdS
    \end{equation*}
\end{prop}


\begin{prop}
    The surface integral of $F$ over a surface $S$ is equal to the integral of the normal component of $F$ over $S$: let $S$ be an oriented smooth surface $S$ and an orientation-preserving parametrization $\Phi$ of, then we denote $\iint_SF\cdot dS=\iint_\Phi F\cdot dS$, and
    \begin{equation*}
        \iint_SF\cdot dS=\iint_S\left(F\cdot n\right)dS
    \end{equation*}
\end{prop}


\begin{prop}
    Let $S$ be the graph of a function $g(x,y)$, then 
    \begin{equation*}
        \iint_SF\cdot dS=\iint_DF\cdot (T_x\times T_y)dxdy=\iint_D\left(F_1\left(-\frac{\partial g}{\partial x}\right)+F_2\left(-\frac{\partial g}{\partial y}\right)+F_3\right)dxdy
    \end{equation*}
\end{prop}

\begin{thm}[Green's theorem]
    Let $F(x,y)=(P(x,y), Q(x,y))$ be a continuously differentiable vector field. For a simple region $D\subset\R^2$ with $\partial D=C$ as its positively oriented boundary, we have 
    \begin{equation*}
        \int_{\partial D}F\cdot ds=\iint_D\left(\frac{\partial Q}{\partial x}-\frac{\partial P}{\partial y}\right)dxdy
    \end{equation*}
\end{thm}


\begin{thm}[Green's theorem (curl form)]
    Let $D\subset\R^2$ be a region to which Green's theorem applies, let $\partial D$ be its positively oriented boundary and let $F(P,Q)$ be a $C^1$ vector field on $D$. Then 
    \begin{equation*}
        \int_{\partial D}F\cdot ds=\iint_D\curl F\cdot k dA=\iint_D(\nabla\times F)\cdot kdA
    \end{equation*}
\end{thm}

\begin{prop}
    Let $D\subset\R^2$ be a region where Green's theorem applies and let $\partial D$ be its boundary. Let $n$ denote the outward unit normal to $\partial D$. If $c:[a,b]\to\R^2$, $t\mapsto c(t)=(x(t), y(t))$ is a positively oriented parametrization of $\partial D$, $n$ is given by 
    \begin{equation*}
        n=\frac{(y'(t), -x'(t))}{\sqrt{x'(t)^2+y'(t)^2}}
    \end{equation*}
    Let $F=(P,Q)$ be a $C^1$ vector field on $D$. Then 
    \begin{equation*}
        \int_{\partial D}F\cdot nds=\iint_D\text{div} FdA
    \end{equation*}
\end{prop}



\begin{prop}[area of a region]
    If $C$ is a simple closed curve that bounds a region to which Green's theorem applies, then the area of the region $D$ bounded by $C=\partial D$ is 
    \begin{equation*}
        A=\frac{1}{2}\int_{\partial D}xdy-ydx
    \end{equation*}
\end{prop}


\begin{thm}[Stokes' theorem]
    Let $S$ be the oriented surface defined by a $C^2$ function $z=f(x,y)$, where $(x,y)\in D$, a region to which Green's theorem applies, and let $F$ be a $C^1$ vector field on $S$. Then if $\partial S$ denotes the oriented boundary curve of $S$, then 
    \begin{equation*}
        \iint_S(\nabla\times F)\cdot dS=\int_{\partial S}F\cdot ds
    \end{equation*}
    More generally, let $S$ be an oriented surface defined by a one-to-one parametrization $\Phi: D\subset\R^2\to S$, where $D$ is a region to which Green's theorem applies. Let $\partial S$ denote the oriented boundary of $S$ and let $F$ be a $C^1$ vector field on $S$. Then 
    \begin{equation*}
        \iint_S(\nabla\times F)\cdot dS=\int_{\partial S}F\cdot ds
    \end{equation*}
\end{thm}



\begin{prop}
    Let $F(x,y)=P(x,y)i+Q(x,y)j$ be a vector field in $\R^2$, $F$ is conservative if and only if 
    \begin{equation*}
        \frac{\partial Q}{\partial x}=\frac{\partial P}{\partial y}
    \end{equation*}
\end{prop}


\begin{prop}
    Let $F$ be a continuously differentiable vector field in $\R^3$ with $\nabla\cdot F=0$, i.e., the divergence of $F$ is $0$. Then there exists a twice continuously differentiable vector field $G$ where 
    \begin{equation*}
        F=\curl(G)=\nabla\times G
    \end{equation*}
\end{prop}


\begin{thm}[Gauus' Theorem]
    Let $D$ be a bounded solid region in $\R^3$ whose boundary $\partial D$ is a closed oriented surface (normal vector pointing outward), then 
    \begin{equation*}
        \iint_{\partial D}F\cdot dS=\iiint_D(\nabla\cdot F)dV
    \end{equation*}
    (This is a relation between the flux and the integral of divergence).
\end{thm}






\chapter{Practice Problems}








\section{Chapter 1}



\begin{prob}[1]
    Let \( \mathbf{v} = 3\mathbf{i} + 4\mathbf{j} + 5\mathbf{k} \) and \( \mathbf{w} = \mathbf{i} - \mathbf{j} + \mathbf{k} \). Compute  
    \( \mathbf{v} + \mathbf{w},\; 3\mathbf{v},\; 6\mathbf{v} + 8\mathbf{w},\; -2\mathbf{v},\; \mathbf{v} \cdot \mathbf{w},\; \mathbf{v} \times \mathbf{w} \).  
    Interpret each operation geometrically by graphing the vectors.
\end{prob}
% \begin{proof}
%     \begin{enumerate}
%         \item \begin{equation*}
%             v+w=(4, 3, 6)
%         \end{equation*}
%         \item \begin{equation*}
%             3v=(9, 12, 15)
%         \end{equation*}
%         \item \begin{equation*}
%             6v+8w=(26, 16, 38)
%         \end{equation*}
%         \item \begin{equation*}
%             -2v=(-6, -8, -10)
%         \end{equation*}
%         \item \begin{equation*}
%             v\dot w=3-4+5=4
%         \end{equation*}
%         \item \begin{equation*}
%             v\times w=\det\left|\begin{matrix}
%                 i&j&k\\
%                 3&4&5\\
%                 1&-1&1
%             \end{matrix}\right|=(9, 2, -7)
%         \end{equation*}
%     \end{enumerate}
% \end{proof}

\begin{prob}[4]
    \begin{enumerate}
        \item[(a)] Find the equation of the line through \( (0, 1, 0) \) in the direction of \( 3\mathbf{i} + \mathbf{k} \).  
        \item[(b)] Find the equation of the line passing through \( (0, 1, 1) \) and \( (0, 1, 0) \).  
        \item[(c)] Find an equation for the plane perpendicular to the vector \( (-1, 1, -1) \) and passing through the point \( (1, 1, 1) \).
    \end{enumerate}
\end{prob}
% \begin{proof}
%     \begin{enumerate}
%         \item[(a)] The equation is given by 
%         \begin{equation*}
%             l_1(t)=(0,1,0)+t(3, 0, 1)
%         \end{equation*}
%         \item[(b)] The equation is given by 
%         \begin{equation*}
%             l_2(t)=(0,1,1)+t(0, 0, -1)
%         \end{equation*}
%         \item[(c)] The equation of the plane is given by 
%         \begin{equation*}
%             (x-1, y-1, z-1)\cdot (-1, 1, -1)=0
%         \end{equation*}
%         simplifying we get 
%         \begin{equation*}
%             x-y+z=1
%         \end{equation*}
%     \end{enumerate}
% \end{proof}



\begin{prob}[12]
    Show that three vectors \( \mathbf{a}, \mathbf{b}, \mathbf{c} \) lie in the same plane through the origin if and only if there are three scalars \( \alpha, \beta, \gamma \), not all zero, such that
    \[
        \alpha \mathbf{a} + \beta \mathbf{b} + \gamma \mathbf{c} = \mathbf{0}.
    \]
\end{prob}


\begin{prob}[27]
    \begin{enumerate}
        \item[(a)] Prove that the area of the triangle in the plane with vertices \((x_1, y_1)\), \((x_2, y_2)\), \((x_3, y_3)\) is the absolute value of  
        \[
            \frac{1}{2} \begin{vmatrix}
            1 & 1 & 1 \\
            x_1 & x_2 & x_3 \\
            y_1 & y_2 & y_3
            \end{vmatrix}.
        \]
        \item[(b)] Find the area of the triangle with vertices \((1, 2)\), \((0, 1)\), \((-1, 1)\).
    \end{enumerate}
\end{prob}
% \begin{proof}
%     \begin{enumerate}
%         \item[(a)] The area of the triangle is the half of the area of the paralellogram spanned by the vertices. (Please write out details in your proof).
%         \item[(b)] Using the result in (a), we get the area is $\frac{\sqrt{11}}{2}$.
%     \end{enumerate}
% \end{proof}


\begin{prob}[28]
    Convert the following points from Cartesian to cylindrical and spherical coordinates and plot:  

    \begin{itemize}
        \item[(a)] \((0, 3, 4)\)
        \item[(b)] \((-\sqrt{2}, 1, 0)\)
        \item[(c)] \((0, 0, 0)\)
        \item[(d)] \((-1, 0, 1)\)
        \item[(e)] \((-2\sqrt{3}, -2, 3)\)
    \end{itemize}
\end{prob}
% \begin{proof}
%     The cylindrical cooridnate is the $3$-tuple $(r,\theta, z)$ such that 
%     \begin{equation*}
%         x=r\cos\theta, y=r\sin\theta, z=z
%     \end{equation*}
%     and the spherical coordinate is such that $(\rho,\theta,\phi)$
%     \begin{equation*}
%         x=\rho\sin\phi\cos\theta, y=\rho\sin\phi\sin\theta, z=\rho\cos\phi
%     \end{equation*}
%     For example, for (a), you get 
%     \begin{equation*}
%         (r,\theta, z)=(3, \pi/2, 4)
%     \end{equation*}
%     and 
%     \begin{equation*}
%         (\rho,\theta,\phi)=(5, \arccos 4/5, \pi/2)
%     \end{equation*}
%     and so on $\dots$
% \end{proof}




\begin{prob}[36]
    Find the volume of the parallelepiped spanned by the vectors  
\[
\mathbf{u} = (1, 0, 1), \quad \mathbf{v} = (1, 1, 1), \quad \text{and} \quad \mathbf{w} = (-3, 2, 0).
\]
\end{prob}
% \begin{proof}
%     It is given by the following:
%     \begin{equation*}
%         |u\cdot(v\times w)|=3
%     \end{equation*}
% \end{proof}


\begin{prob}[52]
    Find a unit vector that has the following property:
    parallel to the line \( x = 3t + 1, \; y = 16t - 2, \; z = -(t + 2) \)
\end{prob}
% \begin{proof}
%     The direction of the line is along the vector 
%     \begin{equation*}
%         v=(3, 16, -1)
%     \end{equation*}
%     Normalizing, we get 
%     \begin{equation*}
%         \left(\frac{3}{\sqrt{266}}, \frac{16}{\sqrt{266}}, -\frac{1}{\sqrt{266}}\right)
%     \end{equation*}
% \end{proof}



\begin{prob}[53]
    Find a unit vector that has the following property: 
    orthogonal to the plane \( x - 6y + z = 12 \).
\end{prob}
% \begin{proof}
%     \begin{equation*}
%         n=\frac{(1, -6, 1)}{\|(1, -6, 1)\|}=\left(\frac{1}{\sqrt{38}}, -\frac{6}{\sqrt{38}}, \frac{1}{\sqrt{38}}\right)
%     \end{equation*}
% \end{proof}


\begin{prob}[54]
    Find a unit vector that has the following property:
    parallel to both the planes \( 8x + y + z = 1 \) and \( x - y - z = 0 \).
\end{prob}
% \begin{proof}
%     It suffices to find normal vectors $n_1, n_2$ to two planes and take the cross product:
%     \begin{equation*}
%         n_1=(8,1,1), \quad n_2=(1,-1, -1)
%     \end{equation*}
%     and 
%     \begin{equation*}
%         n_1\times n_2=(0, 9, -9)
%     \end{equation*}
%     Normalizing we get 
%     \begin{equation*}
%         n=\left(0, \frac{1}{\sqrt{2}}, -\frac{1}{\sqrt{2}}\right)
%     \end{equation*}
% \end{proof}


\section{Chapter 2}

\begin{prob}[2]
    Describe some appropriate level surfaces and sections of the graphs of:

    \begin{enumerate}
        \item[(a)] \( f(x, y, z) = 2x^2 + y^2 + z^2 \)
        \item[(b)] \( f(x, y, z) = x^2 \)
        \item[(c)] \( f(x, y, z) = xyz \)
    \end{enumerate}
\end{prob}
% \begin{proof}
%     Good luck with this one.
% \end{proof}

\begin{prob}[3]
    Compute the derivative \( Df(\mathbf{x}) \) of each of the following functions:

    \begin{enumerate}
        \item[(a)] \( f(x, y) = (x^2y, e^{-xy}) \)
        \item[(b)] \( f(x) = (x, x) \)
        \item[(c)] \( f(x, y, z) = e^x + e^y + e^z \)
        \item[(d)] \( f(x, y, z) = (x, y, z) \)
    \end{enumerate}
\end{prob}
% \begin{proof}
%     \begin{enumerate}
%         \item[(a)] The derivative matrix is 
%         \begin{equation*}
%             Df(x,y)=\begin{bmatrix}
%                 2xy&x^2\\
%                 -ye^{-xy}&-xe^{-xy}
%             \end{bmatrix}
%         \end{equation*}
%         \item[(b)] The derivative matrix is 
%         \begin{equation*}
%             Df(x)=\begin{bmatrix}
%                 1\\
%                 1
%             \end{bmatrix}
%         \end{equation*}
%         \item[(c)] The derivative matrix is 
%         \begin{equation*}
%             \begin{bmatrix}
%                 e^x&e^y&e^z
%             \end{bmatrix}
%         \end{equation*}
%         \item[(d)] The derivative matrix is
%         \begin{equation*}
%             \begin{bmatrix}
%                 1&0&0\\
%                 0&1&0\\
%                 0&0&1
%             \end{bmatrix}
%         \end{equation*}
%     \end{enumerate}
% \end{proof}

\begin{prob}[5]
    Let \( f(u, v) = (\cos u, v + \sin u) \) and  
    \( g(x, y, z) = (x^2 + \pi y^2, xz) \). Compute \( D(f \circ g) \) at \((0, 1, 1)\) using the chain rule.
\end{prob}
% \begin{proof}
%     Using the chain rule, we get 
%     \begin{align*}
%         D(f\circ g)(0,1,1)&=Df(g(x_0))Dg(x_0)\\
%         &=Df(\pi, 0)Dg(0,1,1)\\
%         &=\begin{bmatrix}
%             0&0\\
%             -1&1
%         \end{bmatrix}\begin{bmatrix}
%             0&2\pi &0\\
%             1&0&0
%         \end{bmatrix}\\
%         &=\begin{bmatrix}
%             0&0&0\\
%             1&-2\pi&0
%         \end{bmatrix}
%     \end{align*}
% \end{proof}

\begin{prob}[20]
    Let \( f: \mathbb{R}^2 \rightarrow \mathbb{R}^4 \) and \( g: \mathbb{R}^2 \rightarrow \mathbb{R}^2 \) be given by
    \[
    f(x, y) = (x^2 - y^2, 0, \sin(xy), 1)
    \]
    and
    \[
    g(x, y) = (ye^{x^2}, xe^{y^2}).
    \]
    Compute \( D(f \circ g)(1, 2) \).
\end{prob}
% \begin{proof}
%     We use chain rule again just like the above:
%     \begin{align*}
%         D(f\circ g)(1,2)&=Df(2e, e^4)Dg(1,2)
%     \end{align*}
%     and you compute :)
% \end{proof}


\begin{prob}[24]
    Find the plane tangent to the surface \( z = x^2 + y^2 \) at the point \( (1, -2, 5) \). Explain the geometric significance, for this surface, of the gradient of \( f(x, y) = x^2 + y^2 \) (see Exercise 23).
\end{prob}
% \begin{proof}
%     The surface is $x^2+y^2-z=0$, let $g=x^2+y^2-z$, and the gradient is normal to the surface:
%     \begin{equation*}
%         \nabla g=(2x, 2y, -1)
%     \end{equation*}
%     Thus the plane tangent to the surface is 
%     \begin{equation*}
%         (2, -4, -1)\cdot (x-1, y+2, z-5)=0
%     \end{equation*}
%     simplifying we get 
%     \begin{equation*}
%         2x-4y-z-5=0
%     \end{equation*}
% \end{proof}


\begin{prob}[26]
    Find the directional derivative of the given function at the given point and in the direction of the given vector.

    \begin{enumerate}
        \item[(a)] \( f(x, y, z) = e^x \cos(yz), \quad p_0 = (0, 0, 0), \quad \mathbf{v} = (2, 1, -2) \)
        \item[(b)] \( f(x, y, z) = xy + yz + zx, \quad p_0 = (1, 1, 2), \quad \mathbf{v} = (10, -1, 2) \)
    \end{enumerate}
\end{prob}
% \begin{proof}
%     The directional derivative is given by 
%     \begin{equation*}
%         D_vf(p_0)=\nabla f(p_0)\cdot\frac{v}{\|v\|}
%     \end{equation*}
%     I will not compute them here, but it is important that you normalize the direction vector $v$!
% \end{proof}


\begin{prob}[30]
    Find the direction in which the function \( w = x^2 + xy \) increases most rapidly at the point \((-1, 1)\). What is the magnitude of \( \nabla w \) at this point? Interpret this magnitude geometrically.
\end{prob}
% \begin{proof}
%     The gradient is the direction that $w$ increases most rapidly, i.e., 
%     \begin{equation*}
%         \nabla w=(2x+y, x)
%     \end{equation*}
%     evaluated at $(-1,1)$, we get 
%     \begin{equation*}
%         \nabla w=(-1, -1), \quad |\nabla w|=\sqrt{2}
%     \end{equation*}
%     This magnitude is the maximum rate of change of $w$ at this point.
% \end{proof}

\begin{prob}[40]
    Compute an equation for the plane tangent to the graph of
    \[
    f(x, y) = \frac{e^x}{x^2 + y^2}
    \]
    at \( x = 1, y = 2 \).
\end{prob}
% \begin{proof}
%     The tangent plane to the graph of $f$ is given by 
%     \begin{equation*}
%         z=f(1,2)+\frac{\partial f}{\partial x}(1,2)(x-1)+\frac{\partial f}{\partial y}(1,2)(y-2)
%     \end{equation*}
% \end{proof}


\begin{prob}[59]
    \begin{enumerate}
        \item[(a)] In what direction is the directional derivative of  
        \[
        f(x, y) = \frac{x^2 - y^2}{x^2 + y^2}
        \]
        at \((1, 1)\) equal to zero?  
        \item[(b)] How about at an arbitrary point \((x_0, y_0)\) in the first quadrant?  
        \item[(c)] Describe the level curves of \(f\). In particular, discuss them in terms of the result of part (b).
    \end{enumerate}
\end{prob}
% \begin{proof}
%     \begin{enumerate}
%         \item[(a)] We first compute $\nabla f(1,1)=(1,-1)$, thus a directional vector such as $(1,1)$ would make the directional derivative of $f$ zero (by taking the dot product).
%         \item[(b)] You would do the same process as in (a) to obtain a general form for each $(x_0, y_0)$.
%         \item[(c)] The level curves of $f$ are straight lines through the origin.
%     \end{enumerate}
% \end{proof}





\section{Chapter 3}
\begin{prob}[3]
    Let \( f(x, y) = x^2 - y^2 - xy + 5 \). Find all critical points of \( f \) and determine whether they are local minima, local maxima, or saddle points.
\end{prob}

\begin{prob}[5]
    Find the second-order Taylor polynomial for  
    \[
    f(x, y) = y^2 e^{-x^2} \text{ at } (1, 1).
    \]
\end{prob}

\begin{prob}[15]
    \( f(x, y) = x^2 - 2xy + 2y^2 \), subject to \( x^2 + y^2 = 1 \).
\end{prob}

\begin{prob}[17]
    \( f(x, y) = \cos(x^2 - y^2) \), subject to \( x^2 + y^2 = 1 \).
\end{prob}

\begin{prob}[24]
    Determine all values of \( k \) for which the function \( g(x, y, z) = x^2 + kxy + kxz + ky^2 + kz^2 \) has a local minimum at \( (0, 0, 0) \).
\end{prob}

\begin{prob}[30]
    Find the maximum and minimum of \( f(x, y) = xy - y + x - 1 \) on the set \( x^2 + y^2 \leq 2 \).
\end{prob}

\begin{prob}[38]
    Consider the surface \( S \) given by
    \[
    x^2z + x \sin y + ye^{x-1} = 1.
    \]
    \begin{enumerate}
        \item[(a)] Find the equation of the tangent plane to \( S \) at the point \( (1, 0, 1) \).
        \item[(b)] Is it possible to solve the equation defining \( S \) for the variable \( y \) as a function of the variables \( x \) and \( z \) near \( (1, 0, 1) \)? Why?
        \item[(c)] Find \( \frac{\partial y}{\partial x} \) at \( (1, 0, 1) \).
    \end{enumerate}
\end{prob}





\section{Chapter 4}
\begin{prob}[5]
    Calculate the tangent and acceleration vectors for the helix \( c(t) = (\cos t, \sin t, t) \) at \( t = \pi/4 \).
\end{prob}

\begin{prob}[8]
    \begin{enumerate}
        \item[(a)]  Let \( c(t) \) be a path with \( \|c(t)\| = \text{constant} \); that is, the curve lies on a sphere. Show that \( c'(t) \) is orthogonal to \( c(t) \).
        \item[(b)] Let \( c \) be a path whose speed is never zero. Show that \( c \) has constant speed if and only if the acceleration vector \( c'' \) is always perpendicular to the velocity vector \( c' \).
    \end{enumerate}
\end{prob}

\begin{prob}[10]
    Let \( \mathbf{F}(x, y, z) = (\sin(xz), e^{xy}, x^2y^3z^5) \).

    \begin{enumerate}
        \item[(a)] Find the divergence of \( \mathbf{F} \).
        \item[(b)] Find the curl of \( \mathbf{F} \).
    \end{enumerate}
\end{prob}

\begin{prob}[13]
    Express the arc length of the curve \( x^2 = y^3 = z^5 \) between \( x = 1 \) and \( x = 4 \) as an integral, using a suitable parametrization.
\end{prob}

\begin{prob}[34]
    Let a particle of mass \( m \) move along the elliptical helix  
    \[
    c(t) = (4 \cos t, \sin t, t).
    \]  

    \begin{enumerate}
        \item[(a)] Find the equation of the tangent line to the helix at \( t = \pi / 4 \).  
        \item[(b)] Find the force acting on the particle at time \( t = \pi / 4 \).  
        \item[(c)] Write an expression (in terms of an integral) for the arc length of the curve \( c(t) \) between \( t = 0 \) and \( t = \pi / 4 \).
    \end{enumerate}
\end{prob}

\begin{prob}[36]
    \begin{enumerate}
        \item[(a)] Write in parametric form the curve that is the intersection of the surfaces \( x^2 + y^2 + z^2 = 3 \) and \( y = 1 \).
        \item[(b)]  Find the equation of the line tangent to this curve at \((1, 1, 1)\).
        \item[(c)] Write an integral expression for the arc length of this curve. What is the value of this integral?
    \end{enumerate}
\end{prob}


\section{Chapter 5}

\begin{prob}[3]
    Evaluate the integral
    \[
    \int_{0}^{1} \int_{\sqrt{x}}^{1} (x+y)^{2} \, dy \, dx.
    \]
\end{prob}

\begin{prob}[9]
    Evaluate the integral
    \[
    \int_{0}^{1} \int_{0}^{x} \int_{0}^{y} (y+z) \, dz \, dy \, dx.
    \]
\end{prob}

\begin{prob}[11]
    Evaluate
    \[
    \int_{0}^{1} \int_{0}^{(\arcsin y)/y} y \cos(xy) \, dx \, dy.
    \]
\end{prob}

\begin{prob}[16]
    Find
    \[
    \iint_{D} y[1-\cos(\pi x/4)] \, dx \, dy,
    \]
    where \( D \) is the region in Figure 5.R.1.
\end{prob}

\begin{prob}[21]
    Evaluate the integral
    \[
    \int_{0}^{1} \int_{0}^{x^2} (x^2 + xy - y^2) \, dy \, dx.
    \]
\end{prob}

\begin{prob}[23]
    Evaluate the integral
    \[
    \int_{0}^{1} \int_{x^2}^{x} (x + y)^2 \, dy \, dx.
    \]
\end{prob}

\begin{prob}[33]
    Prove:
    \[
    \int_{0}^{x} \left[ \int_{0}^{t} F(u) \, du \right] dt = \int_{0}^{x} (x - u) F(u) \, du.
    \]
\end{prob}


\section{Chapter 6}



\begin{prob}[1]
    \begin{enumerate}
        \item[(a)]  Find a linear transformation taking the square \( S = [0, 1] \times [0, 1] \) to the parallelogram \( P \) with vertices \((0, 0), (2, 0), (1, 2), (3, 2)\).  
        \item[(b)]  Write down a change of variables formula appropriate to the transformation you found in part (a).
    \end{enumerate}
\end{prob}
% \begin{proof}
%     \begin{enumerate}
%         \item[(a)]  The matrix is given by 
%         \begin{equation*}
%             T=\begin{pmatrix}
%                 2&1\\
%                 0&2
%             \end{pmatrix}
%         \end{equation*}
%         \item[(b)] Writing the linear transformation as $T(u,v)=(2u+v, 2v)\coloneq(x,y)$, we fine 
%         \begin{equation*}
%             \det(J)=\det\left|\begin{matrix}
%                 \frac{\partial x}{\partial u}&\frac{\partial x}{\partial v}\\
%                 \frac{\partial y}{\partial u}&\frac{\partial y}{\partial v}
%             \end{matrix}\right|=4
%         \end{equation*}
%         Thus the change of variable formula is 
%         \begin{equation*}
%             dxdy=4dudv
%         \end{equation*}
%     \end{enumerate}
% \end{proof}

\begin{prob}[4]
    In parts (a) to (d), make the indicated change of variables. (Do not evaluate.)

    \begin{enumerate}
        \item[(a)] \(\displaystyle \int_{0}^{1} \int_{-1}^{1} \int_{-\sqrt{1-y^2}}^{\sqrt{1-y^2}} (x^2 + y^2)^{1/2} \, dx \, dy \, dz\), cylindrical coordinates
        \item[(b)] \(\displaystyle \int_{-1}^{1} \int_{-\sqrt{1-y^2}}^{\sqrt{1-y^2}} \int_{-\sqrt{4-x^2-y^2}}^{\sqrt{4-x^2-y^2}} xyz \, dz \, dy \, dx\), cylindrical coordinates
        \item[(c)] \(\displaystyle \int_{-\sqrt{2}}^{\sqrt{2}} \int_{-\sqrt{2-y^2}}^{\sqrt{2-y^2}} \int_{\sqrt{x^2+y^2}}^{\sqrt{4-x^2-y^2}} z^2 \, dz \, dx \, dy\), spherical coordinates
        \item[(d)] \(\displaystyle \int_{0}^{1} \int_{0}^{\pi/4} \int_{0}^{2\pi} \rho^3 \sin 2\phi \, d\theta \, d\phi \, d\rho\), rectangular coordinates
    \end{enumerate}
\end{prob}
% \begin{proof}
%     \begin{enumerate}
%         \item[(a)] 
%         \begin{equation*}
%             \int_0^1\int_0^{2\pi}\int_{0}^1r^2drd\theta dz
%         \end{equation*}
%         \item[(b)]
%         \begin{equation*}
%             \int_0^{2\pi}\int_0^1\int_{\sqrt{4-r^2}}^{\sqrt{4-r^2}}r^3\sin\theta\cos\theta zdzdrd\theta
%         \end{equation*}
%         \item[(c)] Spherical coordinates has $(x,y,z)=(\rho\sin\varphi\cos\theta,\rho\sin\varphi\sin\theta,\rho\cos\varphi)$, so
%         \[dx\,dy\,dz=\rho^2\sin\varphi\,d\rho\,d\theta\,d\varphi.\]
%         Now, the region lives above the circle $x^2+y^2\le2$ as contained in a large ellipse $x^2+y^2+z^2\le4$ but outside $z^2=x^2+y^2$. Thus, the integral is
%         \[\int_0^{2\pi}\int_0^{\pi/4}\int_0^4\rho^4\sin\varphi\cos^2\varphi\,d\rho\,d\varphi\,d\theta.\]
%         \item[(d)] As in (c), we note that $z\,dx\,dy\,dz=\rho^3\sin\varphi\cos\varphi\,d\rho\,d\varphi\,d\theta$. The region here is analogous to the region in (c), so we receive the integral
%         \[\int_{-1}^{1} \int_{-\sqrt{1-y^2}}^{\sqrt{1-y^2}} \int_{\sqrt{x^2+y^2}}^{\sqrt{1-x^2-y^2}} 2z \, dz \, dx \, dy.\]
%     \end{enumerate}
% \end{proof}

\begin{prob}[8]
    Let \( C_1 \) and \( C_2 \) be two cylinders of infinite extent, of diameter 2, and with axes on the \( x \) and \( y \) axes, respectively. Find the volume of their intersection, \( C_1 \cap C_2 \).
\end{prob}
% \begin{proof}
%     The two cylinders are given by the inequalities $x^2+z^2\le1$ and $y^2+z^2\le1$. We will integrate our volume against $z\in[-1,+1]$. Now, for each $z$, the plane at this value of $z$ intersects $C_1\cap C_2$ by the inequalities
%     \[\begin{cases}
%         \left|x\right|\le\sqrt{1-z^2}, \\
%         \left|y\right|\le\sqrt{1-z^2}.
%     \end{cases}\]
%     Thus, our volume is
%     \[\int_{-1}^1\int_{-\sqrt{1-z^2}}^{\sqrt{1-z^2}}\int_{-\sqrt{1-z^2}}^{\sqrt{1-z^2}}\,dx\,dy\,dz.\]
%     Evaluating the inner integrals reveals this is
%     \[4\int_{-1}^1\left(1-z^2\right)\,dz=8\int_0^1\left(1-z^2\right)\,dz.\]
%     This evaluates to $8\left(1-\frac13\right)=\frac{16}3$.
% \end{proof}

\begin{prob}[22]
    Evaluate
    \[
    \iint_{B} e^{-x^2-y^2} \, dx \, dy,
    \]
    where \( B \) consists of those \( (x, y) \) satisfying \( x^2 + y^2 \leq 1 \) and \( y \leq 0 \).
\end{prob}
% \begin{proof}
%     Using polar cooridnates, we get 
%     \begin{align*}
%         \iint_{B} e^{-x^2-y^2} \, dx \, dy&=\int_{\pi}^{2\pi}\int_0^1e^{-r^2}rdrd\theta\\
%         &=\frac{\pi}{2}\left(1-\frac{1}{e}\right)
%     \end{align*}
% \end{proof}

\begin{prob}[24]
    Evaluate
    \[
    \iiint_{D} (x^2 + y^2 + z^2) xyz \, dx \, dy \, dz
    \]
    over each of the following regions.

    \begin{enumerate}
        \item[(a)] The sphere \( D = \{(x, y, z) \mid x^2 + y^2 + z^2 \leq R^2\} \)
        \item[(b)] The hemisphere \( D = \{(x, y, z) \mid x^2 + y^2 + z^2 \leq R^2 \text{ and } z \geq 0\} \)
        \item[(c)] The octant \( D = \{(x, y, z) \mid x \geq 0, y \geq 0, z \geq 0, \text{ and } x^2 + y^2 + z^2 \leq R^2\} \)
    \end{enumerate}
\end{prob}
% \begin{proof}
%     We will convert to spherical coordinates. Note
%     \begin{align*}
%         \left(x^2+y^2+z^2\right)xyz\,dx\,dy\,dz &= \rho^2\cdot\rho^3\sin^2\varphi\cos\theta\sin\theta\cos\varphi\cdot\rho^2\sin\varphi\,d\rho\,d\theta\,d\varphi \\
%         &= \rho^7\cdot\sin^3\varphi\cos\varphi\cdot\sin\theta\cos\theta\,d\rho\,d\theta\,d\varphi.
%     \end{align*}
%     \begin{enumerate}
%         \item[(a)] The integral is
%         \begin{align*}
%             & \int_0^R\int_0^{2\pi}\int_0^\pi\rho^7\cdot\sin^3\varphi\cos\varphi\cdot\sin\theta\cos\theta\,d\varphi\,d\theta\,d\rho \\
%             ={}& \int_0^R\rho^7\,d\rho\int_0^{2\pi}\sin\theta\cos\theta\,d\theta\int_0^\pi\sin^3\varphi\cos\varphi\,d\varphi \\
%             ={}& \cdot\frac{R^8}8\cdot\frac12\sin^2\theta\bigg|_0^{2\pi}\cdot\frac14\sin^4\varphi\bigg|_0^\pi,
%         \end{align*}
%         which vanishes because the $\theta$ factor vanishes.
%         \item[(b)] Repeating the manipulations in (a), the integral is
%         \[\cdot\frac{R^8}8\cdot\frac12\sin^2\theta\bigg|_0^{2\pi}\cdot\frac14\sin^4\varphi\bigg|_0^{\pi/2}.\]
%         This again vanishes because the $\theta$ factor vanishes.
%         \item[(c)] Repeating the manipulations in (a), the integral is
%         \[\cdot\frac{R^8}8\cdot\frac12\sin^2\theta\bigg|_0^{\pi/2}\cdot\frac14\sin^4\varphi\bigg|_0^{\pi/2}.\]
%         This evaluates to $\frac18\cdot R^3\cdot\frac12\cdot\frac14=\frac1{64}R^8$.
%         \qedhere
%     \end{enumerate}
% \end{proof}



\section{Chapter 7}

\begin{prob}[3]
    Compute each of the following line integrals:
    \begin{enumerate}
        \item[(a)] 
        \[
        \int_C (\sin \pi x) \, dy - (\cos \pi y) \, dz,
        \]
        where \( C \) is the triangle whose vertices are \((1, 0, 0)\), \((0, 1, 0)\), and \((0, 0, 1)\), in that order.
    
        \item[(b)]
        \[
        \int_C (\sin z) \, dx + (\cos z) \, dy - (xy)^{1/3} \, dz,
        \]
        where \( C \) is the path \( c(\theta) = (\cos^3 \theta, \sin^3 \theta, \theta) \), \( 0 \leq \theta \leq \frac{7\pi}{2} \).
    \end{enumerate}
\end{prob}

\begin{prob}[5]
    Find the work done by the force  
    \[
    \mathbf{F}(x, y) = (x^2 - y^2)\,\mathbf{i} + 2xy\,\mathbf{j}
    \]
    in moving a particle counterclockwise around the square with corners \((0, 0)\), \((a, 0)\), \((a, a)\), \((0, a)\), \(a > 0\).
\end{prob}
% \begin{proof}
%     By Green's theorem, we have 
%     \begin{align*}
%         \int_CF\cdot ds&=\iint_D\frac{\partial Q}{\partial x}-\frac{\partial P}{\partial y}dA\\
%         &=\int_0^a\int_0^a4ydxdy\\
%         &=2a^3
%     \end{align*}
% \end{proof}

\begin{prob}[7]
Find a parametrization for each of the following surfaces:

\begin{enumerate}
    \item[(a)] \( x^2 + y^2 + z^2 - 4x - 6y = 12 \)
    \item[(b)] \( 2x^2 + y^2 + z^2 - 8x = 1 \)
    \item[(c)] \( 4x^2 + 9y^2 - 2z^2 = 8 \)
\end{enumerate}
\end{prob}
% \begin{proof}
%     \begin{enumerate}
%         \item[(a)] We would first complete some squares 
%         \begin{equation*}
%             (x^2-4x+4)+(y^2-6y+9)+z^2=25
%         \end{equation*}
%         Thus the parametrization can be given by 
%         \begin{equation*}
%             \Phi(\phi,\theta)=(2+5\sin\phi\cos\theta, 3+5\sin\phi\sin\theta, 5\cos\phi), 0\leq\theta\leq 2\pi, 0\leq\phi\leq\pi
%         \end{equation*}
%         \item[(b)] You would do the same as in (a).
%         \item[(c)] This one is kinda complicated. 
%     \end{enumerate}
% \end{proof}

\begin{prob}[12]
Find \(\displaystyle \iint_S f \, dS\) in each of the following cases:

\begin{enumerate}
    \item[(a)] \( f(x, y, z) = x \); \( S \) is the part of the plane  
          \( x + y + z = 1 \) in the positive octant defined by  
          \( x \geq 0, \; y \geq 0, \; z \geq 0 \)
    \item[(b)] \( f(x, y, z) = x^2 \); \( S \) is the part of the plane \( x = z \) inside the cylinder \( x^2 + y^2 = 1 \)
    \item[(c)] \( f(x, y, z) = x \); \( S \) is the part of the cylinder \( x^2 + y^2 = 2x \) with \( 0 \leq z \leq \sqrt{x^2 + y^2} \)
\end{enumerate}
\end{prob}
% \begin{proof}
%     \begin{enumerate}
%         \item[(a)] We parametrize the surface by  
%         \begin{equation*}
%             \Phi(x,y)=(x,y, 1-x-y), 0\leq x\leq 1, 0\leq y\leq 1-x
%         \end{equation*}
%         Then one would compute 
%         \begin{equation*}
%             \iint_SfdS=\int_0^1\int_0^{1-x}\|T_x\times T_y\|dydx
%         \end{equation*}
%         \item[(b)] One can parametrize the surface as 
%         \begin{equation*}
%             \Phi(x,y)=(x,y,x), x^2+y^2\leq 1
%         \end{equation*}
%         and use the same formula.
%         \item[(c)] One can parametrize the surface as follows 
%         \begin{equation*}
%             \Phi(\theta,z)=(1+\cos\theta, \sin\theta, z), 0\leq\theta\leq 2\pi, 0\leq z\leq \sqrt{2(1+\cos\theta)}
%         \end{equation*}
%         and use the same formula.
%     \end{enumerate}
% \end{proof}



\begin{prob}[16]
    A paraboloid of revolution \( S \) is parametrized by 
    \[
    \Phi(u, v) = (u \cos v, u \sin v, u^2), \quad 0 \leq u \leq 2, \; 0 \leq v \leq 2\pi.
    \]
    
    \begin{enumerate}
        \item[(a)] Find an equation in \( x, y, \) and \( z \) describing the surface.
        \item[(b)] What are the geometric meanings of the parameters \( u \) and \( v \)?
        \item[(c)] Find a unit vector orthogonal to the surface at \( \Phi(u, v) \).
        \item[(d)] Find the equation for the tangent plane at \( \Phi(u_0, v_0) = (1, 1, 2) \) and express your answer in the following two ways:
        
        \begin{enumerate}
            \item[(i)] parametrized by \( u \) and \( v \); and
            \item[(ii)] in terms of \( x, y, \) and \( z \).
        \end{enumerate}
        
        \item[(e)] Find the area of \( S \).
    \end{enumerate}
\end{prob}
% \begin{proof}
%     \begin{enumerate}
%         \item[(a)] $z=x^2+y^2$.
%         \item[(b)] $u$ is the radius of the paraboloid and also related to the height, and $v$ is the angle.
%         \item[(c)] You would compute 
%         \begin{equation*}
%             n=\frac{\partial\Phi}{\partial u}\times\frac{\partial\Phi}{\partial v}
%         \end{equation*}
%         \item[(d)] You would use the formula for the tangent plane using the normal vector found in (c).
%         \item[(e)] By the area formula, it is given by 
%         \begin{equation*}
%             \text{Area}=\int_0^2\int_0^{2\pi}\|n\| dvdu
%         \end{equation*}
%     \end{enumerate}
% \end{proof}

\begin{prob}[26]
    Calculate \( \iint_S \mathbf{F} \cdot d\mathbf{S} \), where \( \mathbf{F}(x, y, z) = (x, y, -y) \) and \( S \) is the cylindrical surface defined by \( x^2 + y^2 = 1 \), \( 0 \leq z \leq 1 \), with normal pointing out of the cylinder.
\end{prob}
% \begin{proof}
%     We parametrize the surface as follows:
%     \begin{equation*}
%         \Phi(\theta,z)=(\cos\theta, \sin\theta, z), 0\leq\theta\leq 2\pi, 0\leq z\leq 1
%     \end{equation*}
%     Then we compute $T_\theta\times T_z$,
%     \begin{equation*}
%         T_\theta\times T_z=(\cos\theta, \sin\theta, 0)
%     \end{equation*}
%     By the formula for surface integrals of vector fields, we get 
%     \begin{align*}
%         \iint_SF\cdot dS&=\int_0^{2\pi}\int_0^1F(\Phi(r,\theta))\cdot T_\theta\times T_z dzd\theta\\
%         &=\int_0^{2\pi}\int_0^1(\cos\theta, \sin\theta, -\sin\theta)\cdot(\cos\theta, \sin\theta, 0)dzd\theta\\
%         &=2\pi
%     \end{align*}
% \end{proof}

\begin{prob}[27]
    Let \( S \) be the portion of the cylinder \( x^2 + y^2 = 4 \) between the planes \( z = 0 \) and \( z = x + 3 \). Compute the following:
    
    \begin{enumerate}
        \item[(a)] \( \iint_S x^2 \, dS \)
        \item[(b)] \( \iint_S y^2 \, dS \)
        \item[(c)] \( \iint_S z^2 \, dS \)
    \end{enumerate}
\end{prob}
% \begin{proof}
%     We parametrize the portion of the cylinder as follows:
%     \begin{equation*}
%         (2\cos\theta, 2\sin\theta, z), \quad 0\leq\theta\leq 2\pi, 0\leq z\leq 3+2\cos\theta
%     \end{equation*}
%     Thus computing $T_\theta, T_z$ and taking the cross product $T_\theta\times T_z$, we have 
%     \begin{equation*}
%         \|T_\theta\times T_z\|=2
%     \end{equation*}
%     Thus by the formula of surface integral of scalar-valued functions, we get 
%     \begin{enumerate}
%         \item[(a)] 
%         \begin{align*}
%             \iint_Sx^2dS&=\iint_D4\cos^2\theta\|T_\theta\times T_z\| dA\\
%             &=\int_0^{2\pi}\int_{0}^{3+2\cos\theta}8\cos^2\theta dzd\theta\\
%             &=24\pi
%         \end{align*}
%         One would do the same for (b) and (c).
%     \end{enumerate}
% \end{proof}



\section{Chapter 8}

\begin{prob}[3]
    Let \( \mathbf{F} = x^2 y \, \mathbf{i} + z^8 \, \mathbf{j} - 2xyz \, \mathbf{k} \). Evaluate the integral of \( \mathbf{F} \) over the surface of the unit cube.
\end{prob}
% \begin{proof}
%     By Gauss' divergence theorem, we know
%     \begin{align*}
%         \int_SF\cdot dS=\int_{V}\nabla\cdot FdV
%     \end{align*}
%     where 
%     \begin{equation*}
%         \nabla\cdot F=2xy+0-2xy=0
%     \end{equation*}
%     hence 
%     \begin{equation*}
%         \int_SF\cdot dS=\iiint_{V}\nabla\cdot FdV=0
%     \end{equation*}
% \end{proof}

\begin{prob}[4]
    Verify Green's theorem for the line integral
    \[
    \int_C x^2 y \, dx + y \, dy,
    \]
    when \( C \) is the boundary of the region between the curves \( y = x \) and \( y = x^3 \), \( 0 \leq x \leq 1 \).
\end{prob}
% \begin{proof}
%     By Green's theorem, we have 
%     \begin{align*}
%         \int_Cx^2ydx+ydy&=\int_D\frac{\partial Q}{\partial x}-\frac{\partial P}{\partial y}dA\\
%         &=\int_0^1\int_{x^3}^x-x^2dydx\\
%         &=-\frac{1}{12}
%     \end{align*}
% \end{proof}

\begin{prob}[7]
    \begin{enumerate}
        \item[(a)] Show that  
        \[
        \mathbf{F} = 6xy(\cos z) \, \mathbf{i} + 3x^2(\cos z) \, \mathbf{j} - 3x^2 y(\sin z) \, \mathbf{k}
        \]
        is conservative (see Section 8.3). 
        \item[(b)] Find \( f \) such that \( \mathbf{F} = \nabla f \).  
        \item[(c)] Evaluate the integral of \( \mathbf{F} \) along the curve \( x = \cos^3 \theta, \; y = \sin^3 \theta, \; z = 0, \; 0 \leq \theta \leq \pi/2 \).
    \end{enumerate}
\end{prob}
% \begin{proof}
%     \begin{enumerate}
%         \item[(a)] The idea would be to show $\nabla\times F=0$.
%         \item[(b)] First integrating $F_1$ with respect to $x$ to get
%         \begin{equation*}
%             f=3x^2y\cos z
%         \end{equation*}
%         then verify that $\partial f/\partial y=F_2, \partial f/\partial z=F_3$.
%         \item[(c)] Since $F$ is conservative, the integral of $F$ over a curve only depends on the endpoints, which are $(1,0,0), (0,1,0)$.
%         \begin{equation*}
%             \int_cF\cdot ds=f(0,1,0)-f(1,0,0)=0
%         \end{equation*}
%     \end{enumerate}
% \end{proof}


\begin{prob}[12]
    Show that the fields \( \mathbf{F} \) in (a) and (b) are conservative and find a function \( f \) such that \( \mathbf{F} = \nabla f \).  

    \begin{enumerate}
        \item[(a)] \( \mathbf{F} = (y^2 e^x) \, \mathbf{i} + (2y e^x) \, \mathbf{j} \)
        \item[(b)] \( \mathbf{F} = (\sin y) \, \mathbf{i} + (x \cos y) \, \mathbf{j} + (e^x) \, \mathbf{k} \)
    \end{enumerate}
\end{prob}
% \begin{proof}
%     \begin{enumerate}
%         \item[(a)] We see that 
%         \begin{equation*}
%             \frac{\partial Q}{\partial x}=\frac{\partial P}{\partial y}=2ye^x
%         \end{equation*}
%         thus $F$ is conservative. We find 
%         \begin{equation*}
%             f(x,y)=y^2e^x
%         \end{equation*}
%         \item[(b)] This is a typo, $F$ is not conservative, and you can show $\nabla\times F\neq 0$.
%     \end{enumerate}
% \end{proof}

\begin{prob}[13]
    \begin{enumerate}
        \item[(a)] Let \( f(x, y, z) = 3xy e^z \). Compute \( \nabla f \).  
        \item[(b)] Let \( \mathbf{c}(t) = (3 \cos^3 t, \sin^2 t, e^t), \; 0 \leq t \leq \pi \). Evaluate
        \[
        \int_C \nabla f \cdot d\mathbf{s}.
        \]
        \item[(c)] Verify directly Stokes' theorem for gradient vector fields \( \mathbf{F} = \nabla f \).
    \end{enumerate}
\end{prob}
% \begin{proof}
%     \begin{enumerate}
%         \item[(a)] \begin{equation*}
%             \nabla f(x,y,z)=\left(3ye^z, 3xe^z, 3xye^z\right)
%         \end{equation*} 
%         \item[(b)] \begin{equation*}
%             \int_C\nabla f\cdot ds=f(-3, 0, e^\pi)-f(3, 0, 1)=0
%         \end{equation*}
%         \item[(c)] Stokes's theorem states that 
%         \begin{equation*}
%             \int_CF\cdot ds=\int_S\left(\nabla\times F\right)\cdot ndS
%         \end{equation*}
%         where $S$ is the surface for which $C$ is the boundary of. Thus we get 
%         \begin{equation*}
%             \int_CF\cdot ds=\int_S\left(\nabla\times F\right)\cdot ndS=\int_S\left(\nabla\times\nabla f\right)\cdot ndS=0
%         \end{equation*}
%     \end{enumerate}
% \end{proof}

\begin{prob}[20]
    Which of the following are conservative fields on \( \mathbb{R}^3 \)?  
    For those that are, find a function \( f \) such that \( \mathbf{F} = \nabla f \).  

    \begin{enumerate}
        \item[(a)] \( \mathbf{F}(x, y, z) = 3x^2 y \, \mathbf{i} + x^3 \, \mathbf{j} + 5 \, \mathbf{k} \)
        \item[(b)] \( \mathbf{F}(x, y, z) = (x + z) \, \mathbf{i} - (y + z) \, \mathbf{j} + (x - y) \, \mathbf{k} \)
        \item[(c)] \( \mathbf{F}(x, y, z) = 2xy^3 \, \mathbf{i} + x^2 z^3 \, \mathbf{j} + 3x^2 y z^2 \, \mathbf{k} \)
    \end{enumerate}
\end{prob}
% \begin{proof}
%     \begin{enumerate}
%         \item[(a)] It is conservative, one choice of the potential function is
%         \begin{equation*}
%             f=x^3y+5z
%         \end{equation*}
%         \item[(b)] It is conservative, one choice of the potential function is 
%         \begin{equation*}
%             f=\frac{1}{2}x^2-\frac{1}{2}y^2+xz-yz
%         \end{equation*}
%         \item[(c)] It is not conservative.
%     \end{enumerate}
% \end{proof}



\begin{prob}[21]
    Consider the following two vector fields in \( \mathbb{R}^3 \):  

    \begin{enumerate}
        \item[(i)] \( \mathbf{F}(x, y, z) = y^2 \, \mathbf{i} - z^2 \, \mathbf{j} + x^2 \, \mathbf{k} \)
        \item[(ii)] \( \mathbf{G}(x, y, z) = (x^3 - 3xy^2) \, \mathbf{i} + (y^3 - 3x^2 y) \, \mathbf{j} + z \, \mathbf{k} \)
    \end{enumerate}

    \begin{enumerate}
        \item[(a)] Which of these fields (if any) are conservative on \( \mathbb{R}^3 \)? (That is, which are gradient fields?) Give reasons for your answer.  

        \item[(b)] Find a potential for the fields that are conservative.  

        \item[(c)] Let \( \alpha \) be the path that goes from \( (0, 0, 0) \) to \( (1, 1, 1) \) by following edges of the cube \( 0 \leq x \leq 1 \),  
        \( 0 \leq y \leq 1 \), \( 0 \leq z \leq 1 \) from \( (0, 0, 0) \) to \( (0, 0, 1) \) to \( (0, 1, 1) \) to \( (1, 1, 1) \).  
        Let \( \beta \) be the path from \( (0, 0, 0) \) to \( (1, 1, 1) \) directly along the diagonal of the cube.  
        Find the values of the line integrals  
        \[
        \int_{\alpha} \mathbf{F} \cdot d\mathbf{s}, \quad 
        \int_{\alpha} \mathbf{G} \cdot d\mathbf{s}, \quad 
        \int_{\beta} \mathbf{F} \cdot d\mathbf{s}, \quad 
        \int_{\beta} \mathbf{G} \cdot d\mathbf{s}.
        \]
    \end{enumerate}
\end{prob}
% \begin{proof}
%     \begin{enumerate}
%         \item[(a)] $F$ is not conservative, $G$ is conservative. One can verify this by checking
%         \begin{equation*}
%             \nabla\times F\neq 0, \quad \nabla\times G=0
%         \end{equation*}
%         \item[(b)] Potential for $G$ can be given by 
%         \begin{equation*}
%             g(x,y,z)=\frac{1}{4}x^4+\frac{1}{4}y^4-\frac{3}{2}x^2y^2+\frac{1}{2}z^2
%         \end{equation*}
%         \item[(c)] Because $G$ is conservative, we know the line integral only depends on the endpoints: 
%         \begin{equation*}
%             \int_\alpha G\cdot ds=\int_\beta G\cdot ds=g(1,1,1)-g(0,0,0)=-\frac{1}{2}
%         \end{equation*}
%         For $F$, one would have to parametrize each line segment and evaluate, for example, we parametrize $\beta$ by 
%         \begin{equation*}
%             \beta(t)=(t,t,t), 0\leq t\leq 1
%         \end{equation*}
%         and 
%         \begin{equation*}
%             \int_\beta F\cdot ds=\int_0^1(t^2, -t^2, t^2)\cdot \beta'(t)=\int_0^1(t^2, -t^2, t^2)\cdot(1,1,1)dt=\frac{1}{3}
%         \end{equation*}
%         Similarly, you do this for every segment of $\alpha$ and get 
%         \begin{equation*}
%             \int_\alpha F\cdot ds=0
%         \end{equation*}
%     \end{enumerate}
% \end{proof}




\chapter{Answer Key}




\section{Chapter 1}


\begin{prob}[1]
    Let \( \mathbf{v} = 3\mathbf{i} + 4\mathbf{j} + 5\mathbf{k} \) and \( \mathbf{w} = \mathbf{i} - \mathbf{j} + \mathbf{k} \). Compute  
    \( \mathbf{v} + \mathbf{w},\; 3\mathbf{v},\; 6\mathbf{v} + 8\mathbf{w},\; -2\mathbf{v},\; \mathbf{v} \cdot \mathbf{w},\; \mathbf{v} \times \mathbf{w} \).  
    Interpret each operation geometrically by graphing the vectors.
\end{prob}
\begin{proof}
    \begin{enumerate}
        \item \begin{equation*}
            v+w=(4, 3, 6)
        \end{equation*}
        \item \begin{equation*}
            3v=(9, 12, 15)
        \end{equation*}
        \item \begin{equation*}
            6v+8w=(26, 16, 38)
        \end{equation*}
        \item \begin{equation*}
            -2v=(-6, -8, -10)
        \end{equation*}
        \item \begin{equation*}
            v\dot w=3-4+5=4
        \end{equation*}
        \item \begin{equation*}
            v\times w=\det\left|\begin{matrix}
                i&j&k\\
                3&4&5\\
                1&-1&1
            \end{matrix}\right|=(9, 2, -7)
        \end{equation*}
    \end{enumerate}
\end{proof}

\begin{prob}[4]
    \begin{enumerate}
        \item[(a)] Find the equation of the line through \( (0, 1, 0) \) in the direction of \( 3\mathbf{i} + \mathbf{k} \).  
        \item[(b)] Find the equation of the line passing through \( (0, 1, 1) \) and \( (0, 1, 0) \).  
        \item[(c)] Find an equation for the plane perpendicular to the vector \( (-1, 1, -1) \) and passing through the point \( (1, 1, 1) \).
    \end{enumerate}
\end{prob}
\begin{proof}
    \begin{enumerate}
        \item[(a)] The equation is given by 
        \begin{equation*}
            l_1(t)=(0,1,0)+t(3, 0, 1)
        \end{equation*}
        \item[(b)] The equation is given by 
        \begin{equation*}
            l_2(t)=(0,1,1)+t(0, 0, -1)
        \end{equation*}
        \item[(c)] The equation of the plane is given by 
        \begin{equation*}
            (x-1, y-1, z-1)\cdot (-1, 1, -1)=0
        \end{equation*}
        simplifying we get 
        \begin{equation*}
            x-y+z=1
        \end{equation*}
    \end{enumerate}
\end{proof}



\begin{prob}[12]
    Show that three vectors \( \mathbf{a}, \mathbf{b}, \mathbf{c} \) lie in the same plane through the origin if and only if there are three scalars \( \alpha, \beta, \gamma \), not all zero, such that
    \[
        \alpha \mathbf{a} + \beta \mathbf{b} + \gamma \mathbf{c} = \mathbf{0}.
    \]
\end{prob}
\begin{proof}
    In one direction, suppose we can write $\alpha\mathbf a+\beta\mathbf b+\gamma\mathbf c=\mathbf 0$, where at least one of $\alpha$, $\beta$, or $\gamma$ is nonzero. Without loss of generality, we take $\gamma$ nonzero; dividing out by $\gamma$ and rearranging allows us to write
    \[\mathbf c=\alpha'\mathbf a+\beta'\mathbf b,\]
    so it follows that $\mathbf c$ is in the plane (going through the origin) spanned by $\mathbf a$ and $\mathbf b$.

    Conversely, suppose that $\mathbf a$, $\mathbf b$, and $\mathbf c$ all lie in the same plane through the origin. Then we need to find three scalars $(\alpha,\beta,\gamma)$ (not all zero) for which $\alpha\mathbf a+\beta\mathbf b+\gamma\mathbf c=0$. We work in cases.
    \begin{itemize}
        \item If $\mathbf a=\mathbf b=\mathbf c=0$, then we can take $(\alpha,\beta,\gamma)=(1,1,1)$.
        \item Suppose the three vectors $\mathbf a$, $\mathbf b$, and $\mathbf c$ lie in the same plane but at least one is nonzero. Without loss of generality, say that $\mathbf a$ is nonzero. Then there is $\alpha$ for which $\mathbf b=-\alpha\mathbf a$, so $(\alpha,\beta,\gamma)=(\alpha,1,0)$ works.
        \item Lastly, suppose that two of the vectors in $\{\mathbf a,\mathbf b,\mathbf c\}$ are linearly independent; without loss of generality, suppose that $\mathbf a$ and $\mathbf b$ are linearly independent. Then $\mathbf c$ lives in the plane spanned by $\mathbf a$ and $\mathbf b$, so we may find scalars $\alpha$ and $\beta$ with $\mathbf c=-\alpha\mathbf a-\beta\mathbf b$, so $(\alpha,\beta,\gamma)=(\alpha,\beta,1)$ will work.
        \qedhere
    \end{itemize}
\end{proof}



\begin{prob}[27]
    \begin{enumerate}
        \item[(a)] Prove that the area of the triangle in the plane with vertices \((x_1, y_1)\), \((x_2, y_2)\), \((x_3, y_3)\) is the absolute value of  
        \[
            \frac{1}{2} \begin{vmatrix}
            1 & 1 & 1 \\
            x_1 & x_2 & x_3 \\
            y_1 & y_2 & y_3
            \end{vmatrix}.
        \]
        \item[(b)] Find the area of the triangle with vertices \((1, 2)\), \((0, 1)\), \((-1, 1)\).
    \end{enumerate}
\end{prob}
\begin{proof}
    \begin{enumerate}
        \item[(a)] The area of the triangle is the half of the area of the paralellogram spanned by the vertices. (Please write out details in your proof).
        \item[(b)] Using the result in (a), we get the area is $\frac{\sqrt{11}}{2}$.
    \end{enumerate}
\end{proof}


\begin{prob}[28]
    Convert the following points from Cartesian to cylindrical and spherical coordinates and plot:  

    \begin{itemize}
        \item[(a)] \((0, 3, 4)\)
        \item[(b)] \((-\sqrt{2}, 1, 0)\)
        \item[(c)] \((0, 0, 0)\)
        \item[(d)] \((-1, 0, 1)\)
        \item[(e)] \((-2\sqrt{3}, -2, 3)\)
    \end{itemize}
\end{prob}
\begin{proof}
    The cylindrical cooridnate is the $3$-tuple $(r,\theta, z)$ such that 
    \begin{equation*}
        x=r\cos\theta, y=r\sin\theta, z=z
    \end{equation*}
    and the spherical coordinate is such that $(\rho,\theta,\phi)$
    \begin{equation*}
        x=\rho\sin\phi\cos\theta, y=\rho\sin\phi\sin\theta, z=\rho\cos\phi
    \end{equation*}
    For example, for (a), you get 
    \begin{equation*}
        (r,\theta, z)=(3, \pi/2, 4)
    \end{equation*}
    and 
    \begin{equation*}
        (\rho,\theta,\phi)=(5, \arccos 4/5, \pi/2)
    \end{equation*}
    and so on $\dots$
\end{proof}




\begin{prob}[36]
    Find the volume of the parallelepiped spanned by the vectors  
\[
\mathbf{u} = (1, 0, 1), \quad \mathbf{v} = (1, 1, 1), \quad \text{and} \quad \mathbf{w} = (-3, 2, 0).
\]
\end{prob}
\begin{proof}
    It is given by the following:
    \begin{equation*}
        |u\cdot(v\times w)|=3
    \end{equation*}
\end{proof}


\begin{prob}[52]
    Find a unit vector that has the following property:
    parallel to the line \( x = 3t + 1, \; y = 16t - 2, \; z = -(t + 2) \)
\end{prob}
\begin{proof}
    The direction of the line is along the vector 
    \begin{equation*}
        v=(3, 16, -1)
    \end{equation*}
    Normalizing, we get 
    \begin{equation*}
        \left(\frac{3}{\sqrt{266}}, \frac{16}{\sqrt{266}}, -\frac{1}{\sqrt{266}}\right)
    \end{equation*}
\end{proof}



\begin{prob}[53]
    Find a unit vector that has the following property: 
    orthogonal to the plane \( x - 6y + z = 12 \).
\end{prob}
\begin{proof}
    \begin{equation*}
        n=\frac{(1, -6, 1)}{\|(1, -6, 1)\|}=\left(\frac{1}{\sqrt{38}}, -\frac{6}{\sqrt{38}}, \frac{1}{\sqrt{38}}\right)
    \end{equation*}
\end{proof}


\begin{prob}[54]
    Find a unit vector that has the following property:
    parallel to both the planes \( 8x + y + z = 1 \) and \( x - y - z = 0 \).
\end{prob}
\begin{proof}
    It suffices to find normal vectors $n_1, n_2$ to two planes and take the cross product:
    \begin{equation*}
        n_1=(8,1,1), \quad n_2=(1,-1, -1)
    \end{equation*}
    and 
    \begin{equation*}
        n_1\times n_2=(0, 9, -9)
    \end{equation*}
    Normalizing we get 
    \begin{equation*}
        n=\left(0, \frac{1}{\sqrt{2}}, -\frac{1}{\sqrt{2}}\right)
    \end{equation*}
\end{proof}


\section{Chapter 2}

\begin{prob}[2]
    Describe some appropriate level surfaces and sections of the graphs of:

    \begin{enumerate}
        \item[(a)] \( f(x, y, z) = 2x^2 + y^2 + z^2 \)
        \item[(b)] \( f(x, y, z) = x^2 \)
        \item[(c)] \( f(x, y, z) = xyz \)
    \end{enumerate}
\end{prob}
\begin{proof}
    Good luck with this one.
\end{proof}

\begin{prob}[3]
    Compute the derivative \( Df(\mathbf{x}) \) of each of the following functions:

    \begin{enumerate}
        \item[(a)] \( f(x, y) = (x^2y, e^{-xy}) \)
        \item[(b)] \( f(x) = (x, x) \)
        \item[(c)] \( f(x, y, z) = e^x + e^y + e^z \)
        \item[(d)] \( f(x, y, z) = (x, y, z) \)
    \end{enumerate}
\end{prob}
\begin{proof}
    \begin{enumerate}
        \item[(a)] The derivative matrix is 
        \begin{equation*}
            Df(x,y)=\begin{bmatrix}
                2xy&x^2\\
                -ye^{-xy}&-xe^{-xy}
            \end{bmatrix}
        \end{equation*}
        \item[(b)] The derivative matrix is 
        \begin{equation*}
            Df(x)=\begin{bmatrix}
                1\\
                1
            \end{bmatrix}
        \end{equation*}
        \item[(c)] The derivative matrix is 
        \begin{equation*}
            \begin{bmatrix}
                e^x&e^y&e^z
            \end{bmatrix}
        \end{equation*}
        \item[(d)] The derivative matrix is
        \begin{equation*}
            \begin{bmatrix}
                1&0&0\\
                0&1&0\\
                0&0&1
            \end{bmatrix}
        \end{equation*}
    \end{enumerate}
\end{proof}

\begin{prob}[5]
    Let \( f(u, v) = (\cos u, v + \sin u) \) and  
    \( g(x, y, z) = (x^2 + \pi y^2, xz) \). Compute \( D(f \circ g) \) at \((0, 1, 1)\) using the chain rule.
\end{prob}
\begin{proof}
    Using the chain rule, we get 
    \begin{align*}
        D(f\circ g)(0,1,1)&=Df(g(x_0))Dg(x_0)\\
        &=Df(\pi, 0)Dg(0,1,1)\\
        &=\begin{bmatrix}
            0&0\\
            -1&1
        \end{bmatrix}\begin{bmatrix}
            0&2\pi &0\\
            1&0&0
        \end{bmatrix}\\
        &=\begin{bmatrix}
            0&0&0\\
            1&-2\pi&0
        \end{bmatrix}
    \end{align*}
\end{proof}

\begin{prob}[20]
    Let \( f: \mathbb{R}^2 \rightarrow \mathbb{R}^4 \) and \( g: \mathbb{R}^2 \rightarrow \mathbb{R}^2 \) be given by
    \[
    f(x, y) = (x^2 - y^2, 0, \sin(xy), 1)
    \]
    and
    \[
    g(x, y) = (ye^{x^2}, xe^{y^2}).
    \]
    Compute \( D(f \circ g)(1, 2) \).
\end{prob}
\begin{proof}
    We use chain rule again just like the above:
    \begin{align*}
        D(f\circ g)(1,2)&=Df(2e, e^4)Dg(1,2)
    \end{align*}
    and you compute :)
\end{proof}


\begin{prob}[24]
    Find the plane tangent to the surface \( z = x^2 + y^2 \) at the point \( (1, -2, 5) \). Explain the geometric significance, for this surface, of the gradient of \( f(x, y) = x^2 + y^2 \) (see Exercise 23).
\end{prob}
\begin{proof}
    The surface is $x^2+y^2-z=0$, let $g=x^2+y^2-z$, and the gradient is normal to the surface:
    \begin{equation*}
        \nabla g=(2x, 2y, -1)
    \end{equation*}
    Thus the plane tangent to the surface is 
    \begin{equation*}
        (2, -4, -1)\cdot (x-1, y+2, z-5)=0
    \end{equation*}
    simplifying we get 
    \begin{equation*}
        2x-4y-z-5=0
    \end{equation*}
\end{proof}


\begin{prob}[26]
    Find the directional derivative of the given function at the given point and in the direction of the given vector.

    \begin{enumerate}
        \item[(a)] \( f(x, y, z) = e^x \cos(yz), \quad p_0 = (0, 0, 0), \quad \mathbf{v} = (2, 1, -2) \)
        \item[(b)] \( f(x, y, z) = xy + yz + zx, \quad p_0 = (1, 1, 2), \quad \mathbf{v} = (10, -1, 2) \)
    \end{enumerate}
\end{prob}
\begin{proof}
    The directional derivative is given by 
    \begin{equation*}
        D_vf(p_0)=\nabla f(p_0)\cdot\frac{v}{\|v\|}
    \end{equation*}
    I will not compute them here, but it is important that you normalize the direction vector $v$!
\end{proof}


\begin{prob}[30]
    Find the direction in which the function \( w = x^2 + xy \) increases most rapidly at the point \((-1, 1)\). What is the magnitude of \( \nabla w \) at this point? Interpret this magnitude geometrically.
\end{prob}
\begin{proof}
    The gradient is the direction that $w$ increases most rapidly, i.e., 
    \begin{equation*}
        \nabla w=(2x+y, x)
    \end{equation*}
    evaluated at $(-1,1)$, we get 
    \begin{equation*}
        \nabla w=(-1, -1), \quad |\nabla w|=\sqrt{2}
    \end{equation*}
    This magnitude is the maximum rate of change of $w$ at this point.
\end{proof}

\begin{prob}[40]
    Compute an equation for the plane tangent to the graph of
    \[
    f(x, y) = \frac{e^x}{x^2 + y^2}
    \]
    at \( x = 1, y = 2 \).
\end{prob}
\begin{proof}
    The tangent plane to the graph of $f$ is given by 
    \begin{equation*}
        z=f(1,2)+\frac{\partial f}{\partial x}(1,2)(x-1)+\frac{\partial f}{\partial y}(1,2)(y-2)
    \end{equation*}
\end{proof}


\begin{prob}[59]
    \begin{enumerate}
        \item[(a)] In what direction is the directional derivative of  
        \[
        f(x, y) = \frac{x^2 - y^2}{x^2 + y^2}
        \]
        at \((1, 1)\) equal to zero?  
        \item[(b)] How about at an arbitrary point \((x_0, y_0)\) in the first quadrant?  
        \item[(c)] Describe the level curves of \(f\). In particular, discuss them in terms of the result of part (b).
    \end{enumerate}
\end{prob}
\begin{proof}
    \begin{enumerate}
        \item[(a)] We first compute $\nabla f(1,1)=(1,-1)$, thus a directional vector such as $(1,1)$ would make the directional derivative of $f$ zero (by taking the dot product).
        \item[(b)] You would do the same process as in (a) to obtain a general form for each $(x_0, y_0)$.
        \item[(c)] The level curves of $f$ are straight lines through the origin.
    \end{enumerate}
\end{proof}






\section{Chapter 3}
\begin{prob}[3]
    Let \( f(x, y) = x^2 - y^2 - xy + 5 \). Find all critical points of \( f \) and determine whether they are local minima, local maxima, or saddle points.
\end{prob}
\begin{proof}
    To find the critical points, we need to know where $f_x=2x-y$ and $f_y=-2y-x$ both vanish. Well, if $2x-y=0$ and $-2y-x=0$, then $x=-2y$ and $y=2x$, so
    \[x=-2(2x)=-4x,\]
    so $x=0$, so $y=0$. Thus, the only critical point is $(0,0)$.

    It remains to determine what kind of critical point $(0,0)$ is. Well, $f_{xx}=2$, so $(0,0)$ cannot be a local maximum, and $f_{yy}=-2$, so $(0,0)$ cannot be a local minimum.
\end{proof}

\begin{prob}[5]
    Find the second-order Taylor polynomial for  
    \[
    f(x, y) = y^2 e^{-x^2} \text{ at } (1, 1).
    \]
\end{prob}
\begin{proof}
    The second-order Taylor polynomial is
    \[f(1,1)+(x-1)f_x(1,1)+(y-1)f_y(1,1)+\frac12(x-1)^2f_{xx}(1,1)+\frac12(y-1)^2f_{yy}(1,1)+(x-1)(y-1)f_{xy}(1,1).\]
    Thus, we see that we have many derivatives to compute.
    \begin{itemize}
        \item Note $f(1,1)=e^{-1}$.
        \item Note $f_x=-2xy^2e^{-x^2}$, so $f_x(1,1)=-2e^{-1}$.
        \item Note $f_y=2ye^{-x^2}$, so $f_y(1,1)=2e^{-1}$.
        \item Note $f_{xx}=\left(4x^2-2\right)y^2e^{-x^2}$, so $f_{xx}(1,1)=2e^{-1}$.
        \item Note $f_{yy}=2e^{-x^2}$, so $f_{yy}(1,1)=2e^{-1}$.
        \item Note $f_{xy}=-4xye^{-x^2}$, so $f_{xy}(1,1)=-4e^{-1}$.
    \end{itemize}
    Thus, the second-order Taylor polynomial is
    \[\frac1e\left(1-2(x-1)+2(y-1)+(x-1)^2+(y-1)^2-4(x-1)(y-1)\right).\]
\end{proof}

\begin{prob}[15]
    \( f(x, y) = x^2 - 2xy + 2y^2 \), subject to \( x^2 + y^2 = 1 \).
\end{prob}
\begin{proof}
    We use Lagrange multipliers. Our constraint function is $g(x,y)=x^2+y^2-1$, which has gradient $\nabla g=(2x,2y)$. Under this constraint, $f(x,y)=1-2xy+y^2$, which has gradient $(-2y,2y-2x)$, so there is $\lambda$ such that
    \[\begin{cases}
        2\lambda x=-2y, \\
        2\lambda y=2y-2x, \\
        x^2+y^2=1.
    \end{cases}\]
    The first equation gives $-\lambda x=y$, and the second equation gives $(1-\lambda)y=x$. Thus, $(1-\lambda)(-\lambda)x=x$, so $((1-\lambda)(-\lambda)-1)x=0$. Thus, either $x=0$ or $(1-\lambda)(-\lambda)=1$.

    Now, plugging into $x^2+y^2=1$ for $y$ gives
    \[\left(\lambda^2+1\right)x^2=1.\]
    We conclude that $x$ is nonzero and equals
    \[x=\pm\sqrt{\frac1{\lambda^2+1}}.\]
    To simplify this, recall that $(1-\lambda)(-\lambda)=1$, so $\lambda^2-\lambda-1=0$, so $\lambda=\frac{1\pm\sqrt5}2$. Thus, $x^2=1/\left(\lambda^2+1\right)=1/(\lambda+2)$. But note $(\lambda+2)(\lambda-3)=\lambda^2-\lambda-6=-5$, so $x^2=(3-\lambda)/5$.

    Recall that we want to maximize the function $x^2-2xy+2y^2$, which with $y=-\lambda x$ is $x^2\left(1+2\lambda+2\lambda^2\right)$. But $\lambda^2=\lambda+1$, so we are trying to maximize
    \[x^2(3+4\lambda)=\frac{(3+4\lambda)(\lambda-3)}5=\frac{(4\lambda^2-9\lambda-9)}5=(\lambda+1).\]
    Thus, the extremal values are $\lambda=\frac{3\pm\sqrt5}2$, which occur at the points $(x,y)=\pm\left(1/\sqrt{\lambda^2+1},-\lambda/\sqrt{\lambda^2+1}\right)$.
\end{proof}

\begin{prob}[17]
    \( f(x, y) = \cos(x^2 - y^2) \), subject to \( x^2 + y^2 = 1 \).
\end{prob}
\begin{proof}
    Under the constraint $x^2+y^2=1$, we have
    \[\cos\left(x^2-y^2\right)=\cos\left(1-2y^2\right).\]
    Because $y^2\in[0,1]$, we see that we are finding the extrema of the function $\cos$ in the interval $[-1,1]$.
    
    Well, the only critical points of $\cos$ are at multiples of $\pi$, so the only critical point in this interval is $0$. Accordingly, we see that the maximum occurs at $\cos0=1$, which occurs when $1-2y^2=0$, or $(x,y)=(\pm1/\sqrt2,\pm1/\sqrt2)$. Lastly, we have to check the endpoints, for which we note that $\cos1=\cos(-1)$, so $\cos1$ is the minimum, which occurs when $1-2y^2=1$, or $(x,y)=(\pm1,0)$.
\end{proof}

\begin{prob}[24]
    Determine all values of \( k \) for which the function \( g(x, y, z) = x^2 + kxy + kxz + ky^2 + kz^2 \) has a local minimum at \( (0, 0, 0) \).
\end{prob}
\begin{proof}
    To be a local minimum, $(0,0,0)$ must be a critical point, which we can compute is the case. To be a local minimum, $g_{xx}=2$ must be nonnegative (which is true), $g_{yy}=2k$ must be nonnegative (which holds if and only if $k\ge0$), and the Hessian should be nonpositive. Well, the Hessian is
    \[\det\begin{bmatrix}
        g_{xx} & g_{xy} & g_{xz} \\
        g_{yx} & g_{yy} & g_{yz} \\
        g_{zx} & g_{zy} & g_{zz}
    \end{bmatrix}=\det\begin{bmatrix}
        2 & k & k \\
        k & 2k & 0 \\
        k & 0 & 2k
    \end{bmatrix}=8k^2-4k^3=4k^2\left(2k-1\right).\]
    For this to be nonpositive, we see that $k\le1/2$.

    Thus, we see that $g$ has a local minimum at $(0,0,0)$ automatically for $k\in(0,1/2)$. It remains to check $k\in\{0,1/2\}$. At $k=0$, the function is $g(x,y,z)=x^2$, which does not have $(0,0,0)$ as a local minimum (e.g., increasing in the $y$ direction does not change the value of $g$). At $k=1/2$,
    \[g(x,y,z)=x^2+\frac12xy+\frac12xz+\frac12y^2+\frac12z^2=\frac18(x+2y)^2+\frac18(x+2z)^2+\frac34x^2.\]
    We can now see that this function has a global minimum at $(0,0,0)$: the only way to get the value of $0$ is for $x+2y=x+2z=x=0$ is for $x=y=z=0$. In total, $g$ has a local minimum at $(0,0,0)$ if and only if $k\in(0,1/2]$.
\end{proof}

\begin{prob}[30]
    Find the maximum and minimum of \( f(x, y) = xy - y + x - 1 \) on the set \( x^2 + y^2 \leq 2 \).
\end{prob}
\begin{proof}
    We first find the critical points in the interior of the disk, and then we use Lagrange multipliers to find the extrema on the boundary.
    \begin{itemize}
        \item Note that $f_x=y+1$ and $f_y=x-1$, so the only critical point of $f$ is $(x,y)=(1,-1)$. We calculate $f(1,-1)=0$. % (x-1)(y+1)
        \item We maximize $f(x,y)=(x-1)(y+1)$ with respect to the constraint $g(x,y)=x^2+y^2-2$. Note $\nabla f=(y+1,x-1)$ and $\nabla g=(2x,2y)$, so we want to solve the system
        \[\begin{cases}
            2\lambda x=y+1, \\
            2\lambda y=x-1, \\
            x^2+y^2=2.
        \end{cases}\]
        Note $x=2\lambda y+1$ from the second equation, so we receive the system
        \[\begin{cases}
            4\lambda^2y+2\lambda=y+1, \\
            (2\lambda y+1)^2+y^2=2.
        \end{cases}\]
        The first equation gives $y=(1-2\lambda)/\left(4\lambda^2-1\right)=-1/(2\lambda+1)$ as long as $\lambda\notin\{\pm1/2\}$. Thus, provided $\lambda\notin\{\pm1/2\}$, we find $x=2\lambda+1=1/(2\lambda+1)=-y$. Along with $x^2+y^2=2$, we see that $(x,y)\in\{(-1,1),(1,-1)\}$, which have $f(1,-1)=0$ and $f(-1,1)=-4$.

        It remains to handle $\lambda\in\{\pm1/2\}$. In this case, the equation $4\lambda^2y+2\lambda=y+1$ reads $2\lambda=1$, so we actually must have $\lambda=1/2$. Thus, our original system is
        \[\begin{cases}
            x=y+1, \\
            y=x-1, \\
            x^2+y^2=2.
        \end{cases}\]
        The first two equations are equivalent. Plugging into the third equation gives $(y+1)^2+y^2=2$, so $2y^2+2y-1=0$, so $y=\frac{-1\pm\sqrt3}2$ and so $x=y+1=\frac{1\pm\sqrt3}2$. Thus,
        \[f(x,y)=(x-1)(y+1)=\frac{-1\pm\sqrt3}2\cdot\frac{1\pm\sqrt3}2=\frac12.\]
    \end{itemize}
    Gathering everything together, we see that the maximum is $1/2$, and the minimum is $-4$.
\end{proof}

\begin{prob}[38]
    Consider the surface \( S \) given by
    \[
    x^2z + x \sin y + ye^{z-1} = 1.
    \]
    \begin{enumerate}
        \item[(a)] Find the equation of the tangent plane to \( S \) at the point \( (1, 0, 1) \).
        \item[(b)] Is it possible to solve the equation defining \( S \) for the variable \( y \) as a function of the variables \( x \) and \( z \) near \( (1, 0, 1) \)? Why?
        \item[(c)] Find \( \frac{\partial y}{\partial x} \) at \( (1, 0, 1) \).
    \end{enumerate}
\end{prob}
\begin{proof}
    Define $f(x,y,z)=x^2z+x\sin y+ye^{z-1}$.
    \begin{enumerate}
        \item[(a)] The tangent plane is defined by $\nabla f(1,0,1)\cdot(x,y,z)=\nabla f(1,0,1)\cdot(1,0,1)$. Accordingly, we calculate
        \[\nabla f=\left(2xz+\sin y,x\cos y+e^{z-1},x^2+ye^{z-1}\right),\]
        so $\nabla f(1,0,1)=(2,2,1)$. Thus, the tangent plane is $2x+2y+z=3$.
        \item[(b)] Yes. By the implicit function theorem (as in Theorem 11), it is enough to check that $f_y(1,0,1)\ne0$. But in (a), we calculated $f_y(1,0,1)=2$.
        \item[(c)] As in (b), we may view $y$ as a function of $x$, where $z$ is constantly $1$. Setting $z=1$ leaves us with the curve
        \[x^2+x\sin y+y=1.\]
        Viewing $y$ as a function of $x$ and taking the derivative with respect to $x$, we get
        \[2x+\sin y+x\sin y\cdot y'+y'=0.\]
        Plugging in $(x,y)=(1,0)$ gives $y'=-2$.
        \qedhere
    \end{enumerate}
\end{proof}




\section{Chapter 4}
\begin{prob}[5]
    Calculate the tangent and acceleration vectors for the helix \( c(t) = (\cos t, \sin t, t) \) at \( t = \pi/4 \).
\end{prob}
\begin{proof}
    The tangent vector is $c'(t)=(-\sin t,\cos t,1)$, which is $(-\sqrt2/2,\sqrt2/2,1)$ at $t=\pi/4$. The acceleration vector is $(-\cos t,-\sin t,0)$, which is $(-\sqrt2/2,-\sqrt2/2,0)$ at $t=\pi/4$.
\end{proof}

\begin{prob}[8]
    \begin{enumerate}
        \item[(a)]  Let \( c(t) \) be a path with \( \|c(t)\| = \text{constant} \); that is, the curve lies on a sphere. Show that \( c'(t) \) is orthogonal to \( c(t) \).
        \item[(b)] Let \( c \) be a path whose speed is never zero. Show that \( c \) has constant speed if and only if the acceleration vector \( c'' \) is always perpendicular to the velocity vector \( c' \).
    \end{enumerate}
\end{prob}
\begin{proof}
    Write $c(t)=(x(t),y(t),z(t))$ throughout.
    \begin{enumerate}
        \item[(a)] We are given that $\lVert c(t)\rVert^2=x(t)^2+y(t)^2+z(t)^2$ is constant, so its derivative vanishes. However, the derivative is
        \[2c(t)\cdot c'(t)=2x(t)x'(t)+2y(t)y'(t)+2z(t)z'(t).\]
        It follows that $c'$ is orthogonal to $c$ at all times $t$.
        \item[(b)] Consider the ``velocity path'' $c'$. By the argument of (a), we know that
        \[\frac d{dt}\lVert c'(t)\rVert^2=2c'(t)\cdot c''(t).\]
        Thus, $c'$ has constant magnitude if and only if its derivative vanishes, which is then equivalent to $c'$ and $c''$ being perpendicular.
        \qedhere
    \end{enumerate}
\end{proof}

\begin{prob}[10]
    Let \( \mathbf{F}(x, y, z) = (\sin(xz), e^{xy}, x^2y^3z^5) \).

    \begin{enumerate}
        \item[(a)] Find the divergence of \( \mathbf{F} \).
        \item[(b)] Find the curl of \( \mathbf{F} \).
    \end{enumerate}
\end{prob}
\begin{proof}
    Set $F_x=\sin(xz)$ and $F_y=e^{xy}$ and $F_z=x^2y^3z^5$ for brevity.
    \begin{enumerate}
        \item[(a)] We calculate
        \begin{align*}
            \nabla\cdot\mathbf F &= \nabla\cdot\mathbf F \\
            &= \frac{\partial F_x}{\partial x}+\frac{\partial F_y}{\partial y}+\frac{\partial F_z}{\partial z} \\
            &= z\cos xz+xe^{xy}+5x^2y^3z^4.
        \end{align*}
        \item[(b)] We calculate
        \begin{align*}
            \nabla\times\mathbf F &= \det\begin{bmatrix}
                i & j & k \\
                \partial/\partial_x & \partial/\partial_y & \partial/\partial_z \\
                \sin xz & e^{xy} & x^2y^3z^5
            \end{bmatrix} \\
            &= \left(3x^2y^2z^5,-2xy^3z^5+x\cos xz,ye^{xy}\right).
        \end{align*}
    \end{enumerate}
\end{proof}

\begin{prob}[13]
    Express the arc length of the curve \( x^2 = y^3 = z^5 \) between \( x = 1 \) and \( x = 4 \) as an integral, using a suitable parametrization.
\end{prob}
\begin{proof}
    For convenience, we note that each $x$ in the region $1\le x\le 4$ admits a unique $t$ for which $x=t^{15}$, where $1\le t\le 4^{1/15}$. Then $y=x^{2/3}=t^{10}$ and $z=x^{2/5}=t^6$. Thus, we may parameterize the curve by
    \[c(t)=\left(t^{15},t^{10},t^{6}\right).\]
    Thus, the arc length is
    \begin{align*}
        \int_1^{4^{1/15}}\lVert{c'(t)}\rVert\,dt &= \int_1^{4^{1/15}}\sqrt{\left(15t^{14}\right)^2+\left(10t^9\right)^2+\left(6t^5\right)^2}\,dt \\
        &= \int_1^{4^{1/15}}\sqrt{225t^{28}+100t^{18}+36t^{10}}\,dt.
    \end{align*}
    We remark that different parameterizations will yield different (but equal) integrals. For example, the parameterization $\left(t,t^{2/3},t^{2/5}\right)$ gives the integral
    \[\int_{1}^{4}\sqrt{1+\frac{4}{9}t^{-\frac{2}{3}}+\frac{4}{25}t^{-\frac{6}{5}}}\,dt.\]
\end{proof}

\begin{prob}[34]
    Let a particle of mass \( m \) move along the elliptical helix  
    \[
    c(t) = (4 \cos t, \sin t, t).
    \]  

    \begin{enumerate}
        \item[(a)] Find the equation of the tangent line to the helix at \( t = \pi / 4 \).  
        \item[(b)] Find the force acting on the particle at time \( t = \pi / 4 \).  
        \item[(c)] Write an expression (in terms of an integral) for the arc length of the curve \( c(t) \) between \( t = 0 \) and \( t = \pi / 4 \).
    \end{enumerate}
\end{prob}
\begin{proof}
    Throughout, it will be useful to know that $c'(t)=(-4\sin t,\cos t,1)$.
    \begin{enumerate}
        \item[(a)] The velocity at $t=\pi/4$ is $c'(\pi/4)=(-2\sqrt2,\sqrt2/2,1)$, so the tangent line is given by
        \[\left(2\sqrt2-2\sqrt2\cdot t,\frac{\sqrt2}2+\frac{\sqrt2}2\cdot t,\frac\pi4+t\right).\]
        Alternatively, this is cut out by the equations
        \[\frac{x-2\sqrt2}{-2\sqrt2}=\frac{y-\sqrt2/2}{\sqrt2/2}=z-\frac\pi4.\]

        \item[(b)] The acceleration is $c''(t)=(-4\cos t,-\sin t,0)$, so the force is $(-4m\cos t,-m\sin t,0)$.

        \item[(c)] The arc length is
        \begin{align*}
            \int_0^{\pi/4}\lVert c'(t)\rVert \,dt &= \int_0^{\pi/4}\sqrt{\left(-4\sin t\right)^2+(\cos t)^2+1} \,dt \\
            &= \int_0^{\pi/4}\sqrt{16\sin^2 t+\cos^2 t+1} \,dt.
        \end{align*}
    \end{enumerate}
\end{proof}

\begin{prob}[36]
    \begin{enumerate}
        \item[(a)] Write in parametric form the curve that is the intersection of the surfaces \( x^2 + y^2 + z^2 = 3 \) and \( y = 1 \).
        \item[(b)]  Find the equation of the line tangent to this curve at \((1, 1, 1)\).
        \item[(c)] Write an integral expression for the arc length of this curve. What is the value of this integral?
    \end{enumerate}
\end{prob}
\begin{proof}
    The given curve is the intersection of the surfaces $x^2+y^2+z^2=3$ and $y=1$. Plugging in $y$ means that this curve is alternatively given by the intersection of the cylinder $x^2+z^2=2$ and the plane $y=1$, which is a circle.
    \begin{enumerate}
        \item[(a)] The circle given as the intersection of $x^2+z^2=2$ and $y=1$ can be parameterized by
        \[c(t)=\left(\sqrt2\cos t,\sqrt2\sin t,1\right).\]
        \item[(b)] Using the parameterization of (a), the point $(1,1,1)$ occurs at $t=\pi/4$. Here, the velocity
        \[c'(t)=\left(-\sqrt2\sin t,\sqrt2\cos t,0\right)\]
        equals $(-1,1,0)$, so the tangent line is given by
        \[(1-t,1+t,1).\]
        Alternatively, the tangent line is given by the equations
        \[\begin{cases}1-x=y-1,\\ z=1.\end{cases}\]
        \item[(c)] Using the velocity computed in (b), the arc length is
        \begin{align*}
            \int_0^{2\pi}\lVert c'(t)\rVert\,dt &= \int_0^{2\pi}\sqrt{\left(-\sqrt2\sin t\right)^2+\left(\sqrt2\cos t\right)^2+0}\,dt \\
            &= \int_0^{2\pi}\sqrt{2\sin^2t+2\cos^2t}\,dt.
        \end{align*}
        This integral evaluates to $\int_0^{2\pi}\sqrt2\,dt=2\sqrt2\pi$.
        \qedhere
    \end{enumerate}
\end{proof}





\section{Chapter 5}
\begin{prob}[3]
    Evaluate the integral
    \[
    \int_{0}^{1} \int_{\sqrt{x}}^{1} (x+y)^{2} \, dy \, dx.
    \]
\end{prob}
\begin{proof}
    The domain of integration is the region in the first quadrant above $y=\sqrt x$ but below $y=1$. Thus, we may change the order of integration to instead compute
    \[\int_0^1\int_{0}^{y^2}(x+y)^2\,dx\,dy.\]
    Expanding $(x+y)^2=x^2+2xy+y^2$ and valuating the inner integral gives
    \[\int_{0}^{1}\left(\frac{y^{6}}{3}+y^{5}+y^{4}\right)dy.\]
    This is $\frac1{21}+\frac16+\frac15=\frac{29}{70}$.
\end{proof}

\begin{prob}[9]
    Evaluate the integral
    \[
    \int_{0}^{1} \int_{0}^{x} \int_{0}^{y} (y+z) \, dz \, dy \, dx.
    \]
\end{prob}
\begin{proof}
    Evaluating the integral with respect to $z$ gives
    \[\int_0^1\int_0^x\frac32y^2\,dy\,dx.\]
    Evaluating the integral with respect to $y$ gives
    \[\int_0^1\frac12x^3\,dx.\]
    This is $\frac18$.
\end{proof}

\begin{prob}[11]
    Evaluate
    \[
    \int_{0}^{1} \int_{0}^{(\arcsin y)/y} y \cos(xy) \, dx \, dy.
    \]
\end{prob}
\begin{proof}
    The integral of $\cos x$ with respect to $x$ is $\sin x+C$, so the inner integral is
    \[\int_{0}^{1} \sin(xy)\bigg|_{0}^{(\arcsin y)/y} \, dy=\int_0^1\sin(\arcsin y)\,dy.\]
    Note $\sin\arcsin y=y$ in the region $y\in[0,1]$, so the integral is $\frac12$.
\end{proof}

\begin{prob}[16]
    Find
    \[
    \iint_{D} y[1-\cos(\pi x/4)] \, dx \, dy,
    \]
    where \( D \) is the region in Figure 5.R.1.
\end{prob}
\begin{proof}
    The region $D$ is in the first quadrant bounded below by $y=\sqrt x$ and above by $y=2$. It will turn out to be easier to integrate against $x$ first (because $\cos$ is the most complicated function present). Thus, we write the integral as
    \[\int_0^2\int_0^{y^2}y\left(1-\cos\frac{\pi x}4\right)\,dx\,dy=\int_{0}^{2}y\left(y^{2}-\frac{4}{\pi}\sin\frac{\pi y^{2}}{4}\right)dy.\]
    This is
    \[\frac{y^4}4\bigg|_0^2+\frac{8}{\pi^2}\cos\frac{\pi y^2}4\bigg|_0^2,\]
    which is $4-\frac{16}{\pi^2}$.
\end{proof}

\begin{prob}[21]
    Evaluate the integral
    \[
    \int_{0}^{1} \int_{0}^{x^2} \left(x^2 + xy - y^2\right) \, dy \, dx.
    \]
    Sketch and identify the type of the region (corresponding to the way the integral is written).
\end{prob}
\begin{proof}
    We will not provide a sketch, but we will say that the region is in the first quadrant bounded above by $y=x^2$ and to the right by $x=1$. Evaluating the inner integral gives
    \[\int_0^1\left(x^2y+\frac12xy^2-\frac13y^3\bigg|_0^{x^2}\right)\,dx=\int_0^1\left(x^4+\frac12x^5-\frac13x^6\right)\,dx.\]
    This is $\frac15+\frac1{12}-\frac1{21}=\frac15+\frac1{28}=\frac{33}{140}$.
\end{proof}

\begin{prob}[23]
    Evaluate the integral
    \[
    \int_{0}^{1} \int_{x^2}^{x} (x + y)^2 \, dy \, dx.
    \]
    Sketch and identify the type of the region (corresponding to the way the integral is written).
\end{prob}
\begin{proof}
    We will again not provide a sketch, but we will say that the region is in the first quadrant bounded above by $y=x$ and bounded below by $y=x^2$; this region lives in the unit square $[0,1]\times[0,1]$. Now, expanding $(x+y)^2=x^2+2xy+y^2$, we see that evaluating the inner integral gives
    \[\int_0^1\left(x^2y+xy^2+\frac13y^3\bigg|_{x^2}^x\right)\,dx=\int_0^1\left(x^3+x^3+\frac13x^3-x^4-x^5-\frac13x^6\right)\,dx.\]
    This is $\frac7{12}-\frac15-\frac16+\frac1{21}=\frac{23}{60}-\frac{1}{6}-\frac{1}{21}=\frac{13}{60}-\frac{1}{21}=\frac{71}{420}$.
\end{proof}

\begin{prob}[33]
    Prove:
    \[
    \int_{0}^{x} \left[ \int_{0}^{t} F(u) \, du \right] dt = \int_{0}^{x} (x - u) F(u) \, du.
    \]
\end{prob}
\begin{proof}
    We change the order of integration. The region of integration is bounded by $u,t\ge0$ and $u\le t\le x$. Thus, we could integrate with respect to $t$ first, yielding
    \[\int_0^x\int_u^xF(u)\,dt\,du.\]
    Evaluating the inner integral gives $(x-u)F(u)$, so the result follows.
\end{proof}


\section{Chapter 6}



\begin{prob}[1]
    \begin{enumerate}
        \item[(a)]  Find a linear transformation taking the square \( S = [0, 1] \times [0, 1] \) to the parallelogram \( P \) with vertices \((0, 0), (2, 0), (1, 2), (3, 2)\).  
        \item[(b)]  Write down a change of variables formula appropriate to the transformation you found in part (a).
    \end{enumerate}
\end{prob}
\begin{proof}
    \begin{enumerate}
        \item[(a)]  The matrix is given by 
        \begin{equation*}
            T=\begin{pmatrix}
                2&1\\
                0&2
            \end{pmatrix}
        \end{equation*}
        \item[(b)] Writing the linear transformation as $T(u,v)=(2u+v, 2v)\coloneq(x,y)$, we fine 
        \begin{equation*}
            \det(J)=\det\left|\begin{matrix}
                \frac{\partial x}{\partial u}&\frac{\partial x}{\partial v}\\
                \frac{\partial y}{\partial u}&\frac{\partial y}{\partial v}
            \end{matrix}\right|=4
        \end{equation*}
        Thus the change of variable formula is 
        \begin{equation*}
            dxdy=4dudv
        \end{equation*}
    \end{enumerate}
\end{proof}

\begin{prob}[4]
    In parts (a) to (d), make the indicated change of variables. (Do not evaluate.)

    \begin{enumerate}
        \item[(a)] \(\displaystyle \int_{0}^{1} \int_{-1}^{1} \int_{-\sqrt{1-y^2}}^{\sqrt{1-y^2}} (x^2 + y^2)^{1/2} \, dx \, dy \, dz\), cylindrical coordinates
        \item[(b)] \(\displaystyle \int_{-1}^{1} \int_{-\sqrt{1-y^2}}^{\sqrt{1-y^2}} \int_{-\sqrt{4-x^2-y^2}}^{\sqrt{4-x^2-y^2}} xyz \, dz \, dy \, dx\), cylindrical coordinates
        \item[(c)] \(\displaystyle \int_{-\sqrt{2}}^{\sqrt{2}} \int_{-\sqrt{2-y^2}}^{\sqrt{2-y^2}} \int_{\sqrt{x^2+y^2}}^{\sqrt{4-x^2-y^2}} z^2 \, dz \, dx \, dy\), spherical coordinates
        \item[(d)] \(\displaystyle \int_{0}^{1} \int_{0}^{\pi/4} \int_{0}^{2\pi} \rho^3 \sin 2\phi \, d\theta \, d\phi \, d\rho\), rectangular coordinates
    \end{enumerate}
\end{prob}
\begin{proof}
    \begin{enumerate}
        \item[(a)] 
        \begin{equation*}
            \int_0^1\int_0^{2\pi}\int_{0}^1r^2drd\theta dz
        \end{equation*}
        \item[(b)]
        \begin{equation*}
            \int_0^{2\pi}\int_0^1\int_{\sqrt{4-r^2}}^{\sqrt{4-r^2}}r^3\sin\theta\cos\theta zdzdrd\theta
        \end{equation*}
        \item[(c)] Spherical coordinates has $(x,y,z)=(\rho\sin\varphi\cos\theta,\rho\sin\varphi\sin\theta,\rho\cos\varphi)$, so
        \[dx\,dy\,dz=\rho^2\sin\varphi\,d\rho\,d\theta\,d\varphi.\]
        Now, the region lives above the circle $x^2+y^2\le2$ as contained in a large ellipse $x^2+y^2+z^2\le4$ but outside $z^2=x^2+y^2$. Thus, the integral is
        \[\int_0^{2\pi}\int_0^{\pi/4}\int_0^4\rho^4\sin\varphi\cos^2\varphi\,d\rho\,d\varphi\,d\theta.\]
        \item[(d)] As in (c), we note that $z\,dx\,dy\,dz=\rho^3\sin\varphi\cos\varphi\,d\rho\,d\varphi\,d\theta$. The region here is analogous to the region in (c), so we receive the integral
        \[\int_{-1}^{1} \int_{-\sqrt{1-y^2}}^{\sqrt{1-y^2}} \int_{\sqrt{x^2+y^2}}^{\sqrt{1-x^2-y^2}} 2z \, dz \, dx \, dy.\]
    \end{enumerate}
\end{proof}

\begin{prob}[8]
    Let \( C_1 \) and \( C_2 \) be two cylinders of infinite extent, of diameter 2, and with axes on the \( x \) and \( y \) axes, respectively. Find the volume of their intersection, \( C_1 \cap C_2 \).
\end{prob}
\begin{proof}
    The two cylinders are given by the inequalities $x^2+z^2\le1$ and $y^2+z^2\le1$. We will integrate our volume against $z\in[-1,+1]$. Now, for each $z$, the plane at this value of $z$ intersects $C_1\cap C_2$ by the inequalities
    \[\begin{cases}
        \left|x\right|\le\sqrt{1-z^2}, \\
        \left|y\right|\le\sqrt{1-z^2}.
    \end{cases}\]
    Thus, our volume is
    \[\int_{-1}^1\int_{-\sqrt{1-z^2}}^{\sqrt{1-z^2}}\int_{-\sqrt{1-z^2}}^{\sqrt{1-z^2}}\,dx\,dy\,dz.\]
    Evaluating the inner integrals reveals this is
    \[4\int_{-1}^1\left(1-z^2\right)\,dz=8\int_0^1\left(1-z^2\right)\,dz.\]
    This evaluates to $8\left(1-\frac13\right)=\frac{16}3$.
\end{proof}

\begin{prob}[22]
    Evaluate
    \[
    \iint_{B} e^{-x^2-y^2} \, dx \, dy,
    \]
    where \( B \) consists of those \( (x, y) \) satisfying \( x^2 + y^2 \leq 1 \) and \( y \leq 0 \).
\end{prob}
\begin{proof}
    Using polar cooridnates, we get 
    \begin{align*}
        \iint_{B} e^{-x^2-y^2} \, dx \, dy&=\int_{\pi}^{2\pi}\int_0^1e^{-r^2}rdrd\theta\\
        &=\frac{\pi}{2}\left(1-\frac{1}{e}\right)
    \end{align*}
\end{proof}

\begin{prob}[24]
    Evaluate
    \[
    \iiint_{D} (x^2 + y^2 + z^2) xyz \, dx \, dy \, dz
    \]
    over each of the following regions.

    \begin{enumerate}
        \item[(a)] The sphere \( D = \{(x, y, z) \mid x^2 + y^2 + z^2 \leq R^2\} \)
        \item[(b)] The hemisphere \( D = \{(x, y, z) \mid x^2 + y^2 + z^2 \leq R^2 \text{ and } z \geq 0\} \)
        \item[(c)] The octant \( D = \{(x, y, z) \mid x \geq 0, y \geq 0, z \geq 0, \text{ and } x^2 + y^2 + z^2 \leq R^2\} \)
    \end{enumerate}
\end{prob}
\begin{proof}
    We will convert to spherical coordinates. Note
    \begin{align*}
        \left(x^2+y^2+z^2\right)xyz\,dx\,dy\,dz &= \rho^2\cdot\rho^3\sin^2\varphi\cos\theta\sin\theta\cos\varphi\cdot\rho^2\sin\varphi\,d\rho\,d\theta\,d\varphi \\
        &= \rho^7\cdot\sin^3\varphi\cos\varphi\cdot\sin\theta\cos\theta\,d\rho\,d\theta\,d\varphi.
    \end{align*}
    \begin{enumerate}
        \item[(a)] The integral is
        \begin{align*}
            & \int_0^R\int_0^{2\pi}\int_0^\pi\rho^7\cdot\sin^3\varphi\cos\varphi\cdot\sin\theta\cos\theta\,d\varphi\,d\theta\,d\rho \\
            ={}& \int_0^R\rho^7\,d\rho\int_0^{2\pi}\sin\theta\cos\theta\,d\theta\int_0^\pi\sin^3\varphi\cos\varphi\,d\varphi \\
            ={}& \cdot\frac{R^8}8\cdot\frac12\sin^2\theta\bigg|_0^{2\pi}\cdot\frac14\sin^4\varphi\bigg|_0^\pi,
        \end{align*}
        which vanishes because the $\theta$ factor vanishes.
        \item[(b)] Repeating the manipulations in (a), the integral is
        \[\cdot\frac{R^8}8\cdot\frac12\sin^2\theta\bigg|_0^{2\pi}\cdot\frac14\sin^4\varphi\bigg|_0^{\pi/2}.\]
        This again vanishes because the $\theta$ factor vanishes.
        \item[(c)] Repeating the manipulations in (a), the integral is
        \[\cdot\frac{R^8}8\cdot\frac12\sin^2\theta\bigg|_0^{\pi/2}\cdot\frac14\sin^4\varphi\bigg|_0^{\pi/2}.\]
        This evaluates to $\frac18\cdot R^3\cdot\frac12\cdot\frac14=\frac1{64}R^8$.
        \qedhere
    \end{enumerate}
\end{proof}





\section{Chapter 7}

\begin{prob}[3]
    Compute each of the following line integrals:
    \begin{enumerate}
        \item[(a)] 
        \[
        \int_C (\sin \pi x) \, dy - (\cos \pi y) \, dz,
        \]
        where \( C \) is the triangle whose vertices are \((1, 0, 0)\), \((0, 1, 0)\), and \((0, 0, 1)\), in that order.
    
        \item[(b)]
        \[
        \int_C (\sin z) \, dx + (\cos z) \, dy - (xy)^{1/3} \, dz,
        \]
        where \( C \) is the path \( c(\theta) = (\cos^3 \theta, \sin^3 \theta, \theta) \), \( 0 \leq \theta \leq \frac{7\pi}{2} \).
    \end{enumerate}
\end{prob}
\begin{proof}
    We do the calculations separately.
    \begin{enumerate}
        \item[(a)] The vector field is $\mathbf F(x,y,z)=(0,\sin\pi x,-\cos\pi y)$. We must calculate the line integrals over each of the three segments separately.
        \begin{itemize}
            \item For the segment from $(1,0,0)$ to $(0,1,0)$, we use the parameterization $c(t)=(1-t,t,0)$ for $t\in[0,1]$, so the integral is
            \begin{align*}
                \int_0^1\mathbf F(c(t))\cdot c'(t)\,dt &= \int_0^1(0,\sin\pi (1-t),-\cos\pi t)\cdot (-1,1,0)\,dt \\
                &= \int_0^1\sin\pi (1-t)\,dt \\
                &= \frac1\pi\cos\pi(1-t)\bigg|_0^1 \\
                &= \frac2\pi.
            \end{align*}

            \item For the segment from $(0,1,0)$ to $(0,0,1)$, we use the parameterization $c(t)=(0,1-t,t)$ for $t\in[0,1]$, so the integral is
            \begin{align*}
                \int_0^1\mathbf F(c(t))\cdot c'(t)\,dt &= \int_0^1(0,0,-\cos\pi (1-t))\cdot (0,-1,1)\,dt \\
                &= \int_0^1-\cos\pi (1-t)\,dt \\
                &= \frac1\pi\sin\pi(1-t)\bigg|_0^1 \\
                &= 0.
            \end{align*}

            \item For the segment from $(0,0,1)$ to $(1,0,0)$, we use the parameterization $c(t)=(t,0,1-t)$ for $t\in[0,1]$, so the integral is
            \begin{align*}
                \int_0^1\mathbf F(c(t))\cdot c'(t)\,dt &= \int_0^1(0,\sin\pi t,-1)\cdot (1,0,-1)\,dt \\
                &= \int_0^11\,dt \\
                &= 1.
            \end{align*}
        \end{itemize}
        In total, the three pieces sum to $\frac2\pi+1$.

        \item[(b)] The vector field is $\mathbf F(x,y,z)=(\sin z,\cos z,-\sqrt[3]{xy})$. Thus, the integral is
        \begin{align*}
            \int_0^{7\pi/2}\mathbf F(c(t))\cdot c'(t)\,dt &= \int_0^{7\pi/2}(\sin\theta,\cos\theta,-\cos\theta\sin\theta)\cdot\left(-3\cos^2\theta\sin\theta,3\sin^2\theta\cos\theta,1\right)\,d\theta \\
            &= \int_0^{7\pi/2}\left(-\sin\theta\cos\theta\right)\,d\theta \\
            &= \frac12\cos^2\theta\bigg|_0^{7\pi/2} \\
            &= \frac12.
        \end{align*}
    \end{enumerate}
\end{proof}

\begin{prob}[5]
    Find the work done by the force  
    \[
    \mathbf{F}(x, y) = (x^2 - y^2)\,\mathbf{i} + 2xy\,\mathbf{j}
    \]
    in moving a particle counterclockwise around the square with corners \((0, 0)\), \((a, 0)\), \((a, a)\), \((0, a)\), \(a > 0\).
\end{prob}
\begin{proof}
    By Green's theorem, we have 
    \begin{align*}
        \int_CF\cdot ds&=\iint_D\frac{\partial Q}{\partial x}-\frac{\partial P}{\partial y}dA\\
        &=\int_0^a\int_0^a4ydxdy\\
        &=2a^3
    \end{align*}
\end{proof}

\begin{prob}[7]
Find a parametrization for each of the following surfaces:

\begin{enumerate}
    \item[(a)] \( x^2 + y^2 + z^2 - 4x - 6y = 12 \)
    \item[(b)] \( 2x^2 + y^2 + z^2 - 8x = 1 \)
    \item[(c)] \( 4x^2 + 9y^2 - 2z^2 = 8 \)
\end{enumerate}
\end{prob}
\begin{proof}
    \begin{enumerate}
        \item[(a)] We would first complete some squares 
        \begin{equation*}
            (x^2-4x+4)+(y^2-6y+9)+z^2=25
        \end{equation*}
        Thus the parametrization can be given by 
        \begin{equation*}
            \Phi(\phi,\theta)=(2+5\sin\phi\cos\theta, 3+5\sin\phi\sin\theta, 5\cos\phi), 0\leq\theta\leq 2\pi, 0\leq\phi\leq\pi
        \end{equation*}
        \item[(b)] You would do the same as in (a).
        \item[(c)] This one is kinda complicated. 
    \end{enumerate}
\end{proof}

\begin{prob}[12]
Find \(\displaystyle \iint_S f \, dS\) in each of the following cases:

\begin{enumerate}
    \item[(a)] \( f(x, y, z) = x \); \( S \) is the part of the plane  
          \( x + y + z = 1 \) in the positive octant defined by  
          \( x \geq 0, \; y \geq 0, \; z \geq 0 \)
    \item[(b)] \( f(x, y, z) = x^2 \); \( S \) is the part of the plane \( x = z \) inside the cylinder \( x^2 + y^2 = 1 \)
    \item[(c)] \( f(x, y, z) = x \); \( S \) is the part of the cylinder \( x^2 + y^2 = 2x \) with \( 0 \leq z \leq \sqrt{x^2 + y^2} \)
\end{enumerate}
\end{prob}
\begin{proof}
    \begin{enumerate}
        \item[(a)] We parametrize the surface by  
        \begin{equation*}
            \Phi(x,y)=(x,y, 1-x-y), 0\leq x\leq 1, 0\leq y\leq 1-x
        \end{equation*}
        Then one would compute 
        \begin{equation*}
            \iint_SfdS=\int_0^1\int_0^{1-x}\|T_x\times T_y\|dydx
        \end{equation*}
        \item[(b)] One can parametrize the surface as 
        \begin{equation*}
            \Phi(x,y)=(x,y,x), x^2+y^2\leq 1
        \end{equation*}
        and use the same formula.
        \item[(c)] One can parametrize the surface as follows 
        \begin{equation*}
            \Phi(\theta,z)=(1+\cos\theta, \sin\theta, z), 0\leq\theta\leq 2\pi, 0\leq z\leq \sqrt{2(1+\cos\theta)}
        \end{equation*}
        and use the same formula.
    \end{enumerate}
\end{proof}



\begin{prob}[16]
    A paraboloid of revolution \( S \) is parametrized by 
    \[
    \Phi(u, v) = (u \cos v, u \sin v, u^2), \quad 0 \leq u \leq 2, \; 0 \leq v \leq 2\pi.
    \]
    
    \begin{enumerate}
        \item[(a)] Find an equation in \( x, y, \) and \( z \) describing the surface.
        \item[(b)] What are the geometric meanings of the parameters \( u \) and \( v \)?
        \item[(c)] Find a unit vector orthogonal to the surface at \( \Phi(u, v) \).
        \item[(d)] Find the equation for the tangent plane at \( \Phi(u_0, v_0) = (1, 1, 2) \) and express your answer in the following two ways:
        
        \begin{enumerate}
            \item[(i)] parametrized by \( u \) and \( v \); and
            \item[(ii)] in terms of \( x, y, \) and \( z \).
        \end{enumerate}
        
        \item[(e)] Find the area of \( S \).
    \end{enumerate}
\end{prob}
\begin{proof}
    \begin{enumerate}
        \item[(a)] $z=x^2+y^2$.
        \item[(b)] $u$ is the radius of the paraboloid and also related to the height, and $v$ is the angle.
        \item[(c)] You would compute 
        \begin{equation*}
            n=\frac{\partial\Phi}{\partial u}\times\frac{\partial\Phi}{\partial v}
        \end{equation*}
        \item[(d)] You would use the formula for the tangent plane using the normal vector found in (c).
        \item[(e)] By the area formula, it is given by 
        \begin{equation*}
            \text{Area}=\int_0^2\int_0^{2\pi}\|n\| dvdu
        \end{equation*}
    \end{enumerate}
\end{proof}

\begin{prob}[26]
    Calculate \( \iint_S \mathbf{F} \cdot d\mathbf{S} \), where \( \mathbf{F}(x, y, z) = (x, y, -y) \) and \( S \) is the cylindrical surface defined by \( x^2 + y^2 = 1 \), \( 0 \leq z \leq 1 \), with normal pointing out of the cylinder.
\end{prob}
\begin{proof}
    We parametrize the surface as follows:
    \begin{equation*}
        \Phi(\theta,z)=(\cos\theta, \sin\theta, z), 0\leq\theta\leq 2\pi, 0\leq z\leq 1
    \end{equation*}
    Then we compute $T_\theta\times T_z$,
    \begin{equation*}
        T_\theta\times T_z=(\cos\theta, \sin\theta, 0)
    \end{equation*}
    By the formula for surface integrals of vector fields, we get 
    \begin{align*}
        \iint_SF\cdot dS&=\int_0^{2\pi}\int_0^1F(\Phi(r,\theta))\cdot T_\theta\times T_z dzd\theta\\
        &=\int_0^{2\pi}\int_0^1(\cos\theta, \sin\theta, -\sin\theta)\cdot(\cos\theta, \sin\theta, 0)dzd\theta\\
        &=2\pi
    \end{align*}
\end{proof}

\begin{prob}[27]
    Let \( S \) be the portion of the cylinder \( x^2 + y^2 = 4 \) between the planes \( z = 0 \) and \( z = x + 3 \). Compute the following:
    
    \begin{enumerate}
        \item[(a)] \( \iint_S x^2 \, dS \)
        \item[(b)] \( \iint_S y^2 \, dS \)
        \item[(c)] \( \iint_S z^2 \, dS \)
    \end{enumerate}
\end{prob}
\begin{proof}
    We parametrize the portion of the cylinder as follows:
    \begin{equation*}
        (2\cos\theta, 2\sin\theta, z), \quad 0\leq\theta\leq 2\pi, 0\leq z\leq 3+2\cos\theta
    \end{equation*}
    Thus computing $T_\theta, T_z$ and taking the cross product $T_\theta\times T_z$, we have 
    \begin{equation*}
        \|T_\theta\times T_z\|=2
    \end{equation*}
    Thus by the formula of surface integral of scalar-valued functions, we get 
    \begin{enumerate}
        \item[(a)] 
        \begin{align*}
            \iint_Sx^2dS&=\iint_D4\cos^2\theta\|T_\theta\times T_z\| dA\\
            &=\int_0^{2\pi}\int_{0}^{3+2\cos\theta}8\cos^2\theta dzd\theta\\
            &=24\pi
        \end{align*}
        One would do the same for (b) and (c).
    \end{enumerate}
\end{proof}



\section{Chapter 8}

\begin{prob}[3]
    Let \( \mathbf{F} = x^2 y \, \mathbf{i} + z^8 \, \mathbf{j} - 2xyz \, \mathbf{k} \). Evaluate the integral of \( \mathbf{F} \) over the surface of the unit cube.
\end{prob}
\begin{proof}
    By Gauss' divergence theorem, we know
    \begin{align*}
        \int_SF\cdot dS=\int_{V}\nabla\cdot FdV
    \end{align*}
    where 
    \begin{equation*}
        \nabla\cdot F=2xy+0-2xy=0
    \end{equation*}
    hence 
    \begin{equation*}
        \int_SF\cdot dS=\iiint_{V}\nabla\cdot FdV=0
    \end{equation*}
\end{proof}

\begin{prob}[4]
    Verify Green's theorem for the line integral
    \[
    \int_C x^2 y \, dx + y \, dy,
    \]
    when \( C \) is the boundary of the region between the curves \( y = x \) and \( y = x^3 \), \( 0 \leq x \leq 1 \).
\end{prob}
\begin{proof}
    By Green's theorem, we have 
    \begin{align*}
        \int_Cx^2ydx+ydy&=\int_D\frac{\partial Q}{\partial x}-\frac{\partial P}{\partial y}dA\\
        &=\int_0^1\int_{x^3}^x-x^2dydx\\
        &=-\frac{1}{12}
    \end{align*}
\end{proof}

\begin{prob}[7]
    \begin{enumerate}
        \item[(a)] Show that  
        \[
        \mathbf{F} = 6xy(\cos z) \, \mathbf{i} + 3x^2(\cos z) \, \mathbf{j} - 3x^2 y(\sin z) \, \mathbf{k}
        \]
        is conservative (see Section 8.3). 
        \item[(b)] Find \( f \) such that \( \mathbf{F} = \nabla f \).  
        \item[(c)] Evaluate the integral of \( \mathbf{F} \) along the curve \( x = \cos^3 \theta, \; y = \sin^3 \theta, \; z = 0, \; 0 \leq \theta \leq \pi/2 \).
    \end{enumerate}
\end{prob}
\begin{proof}
    \begin{enumerate}
        \item[(a)] The idea would be to show $\nabla\times F=0$.
        \item[(b)] First integrating $F_1$ with respect to $x$ to get
        \begin{equation*}
            f=3x^2y\cos z
        \end{equation*}
        then verify that $\partial f/\partial y=F_2, \partial f/\partial z=F_3$.
        \item[(c)] Since $F$ is conservative, the integral of $F$ over a curve only depends on the endpoints, which are $(1,0,0), (0,1,0)$.
        \begin{equation*}
            \int_cF\cdot ds=f(0,1,0)-f(1,0,0)=0
        \end{equation*}
    \end{enumerate}
\end{proof}


\begin{prob}[12]
    Show that the fields \( \mathbf{F} \) in (a) and (b) are conservative and find a function \( f \) such that \( \mathbf{F} = \nabla f \).  

    \begin{enumerate}
        \item[(a)] \( \mathbf{F} = (y^2 e^x) \, \mathbf{i} + (2y e^x) \, \mathbf{j} \)
        \item[(b)] \( \mathbf{F} = (\sin y) \, \mathbf{i} + (x \cos y) \, \mathbf{j} + (e^x) \, \mathbf{k} \)
    \end{enumerate}
\end{prob}
\begin{proof}
    \begin{enumerate}
        \item[(a)] We see that 
        \begin{equation*}
            \frac{\partial Q}{\partial x}=\frac{\partial P}{\partial y}=2ye^x
        \end{equation*}
        thus $F$ is conservative. We find 
        \begin{equation*}
            f(x,y)=y^2e^x
        \end{equation*}
        \item[(b)] This is a typo, $F$ is not conservative, and you can show $\nabla\times F\neq 0$.
    \end{enumerate}
\end{proof}

\begin{prob}[13]
    \begin{enumerate}
        \item[(a)] Let \( f(x, y, z) = 3xy e^z \). Compute \( \nabla f \).  
        \item[(b)] Let \( \mathbf{c}(t) = (3 \cos^3 t, \sin^2 t, e^t), \; 0 \leq t \leq \pi \). Evaluate
        \[
        \int_C \nabla f \cdot d\mathbf{s}.
        \]
        \item[(c)] Verify directly Stokes' theorem for gradient vector fields \( \mathbf{F} = \nabla f \).
    \end{enumerate}
\end{prob}
\begin{proof}
    \begin{enumerate}
        \item[(a)] \begin{equation*}
            \nabla f(x,y,z)=\left(3ye^z, 3xe^z, 3xye^z\right)
        \end{equation*} 
        \item[(b)] \begin{equation*}
            \int_C\nabla f\cdot ds=f(-3, 0, e^\pi)-f(3, 0, 1)=0
        \end{equation*}
        \item[(c)] Stokes's theorem states that 
        \begin{equation*}
            \int_CF\cdot ds=\int_S\left(\nabla\times F\right)\cdot ndS
        \end{equation*}
        where $S$ is the surface for which $C$ is the boundary of. Thus we get 
        \begin{equation*}
            \int_CF\cdot ds=\int_S\left(\nabla\times F\right)\cdot ndS=\int_S\left(\nabla\times\nabla f\right)\cdot ndS=0
        \end{equation*}
    \end{enumerate}
\end{proof}

\begin{prob}[20]
    Which of the following are conservative fields on \( \mathbb{R}^3 \)?  
    For those that are, find a function \( f \) such that \( \mathbf{F} = \nabla f \).  

    \begin{enumerate}
        \item[(a)] \( \mathbf{F}(x, y, z) = 3x^2 y \, \mathbf{i} + x^3 \, \mathbf{j} + 5 \, \mathbf{k} \)
        \item[(b)] \( \mathbf{F}(x, y, z) = (x + z) \, \mathbf{i} - (y + z) \, \mathbf{j} + (x - y) \, \mathbf{k} \)
        \item[(c)] \( \mathbf{F}(x, y, z) = 2xy^3 \, \mathbf{i} + x^2 z^3 \, \mathbf{j} + 3x^2 y z^2 \, \mathbf{k} \)
    \end{enumerate}
\end{prob}
\begin{proof}
    \begin{enumerate}
        \item[(a)] It is conservative, one choice of the potential function is
        \begin{equation*}
            f=x^3y+5z
        \end{equation*}
        \item[(b)] It is conservative, one choice of the potential function is 
        \begin{equation*}
            f=\frac{1}{2}x^2-\frac{1}{2}y^2+xz-yz
        \end{equation*}
        \item[(c)] It is not conservative.
    \end{enumerate}
\end{proof}



\begin{prob}[21]
    Consider the following two vector fields in \( \mathbb{R}^3 \):  

    \begin{enumerate}
        \item[(i)] \( \mathbf{F}(x, y, z) = y^2 \, \mathbf{i} - z^2 \, \mathbf{j} + x^2 \, \mathbf{k} \)
        \item[(ii)] \( \mathbf{G}(x, y, z) = (x^3 - 3xy^2) \, \mathbf{i} + (y^3 - 3x^2 y) \, \mathbf{j} + z \, \mathbf{k} \)
    \end{enumerate}

    \begin{enumerate}
        \item[(a)] Which of these fields (if any) are conservative on \( \mathbb{R}^3 \)? (That is, which are gradient fields?) Give reasons for your answer.  

        \item[(b)] Find a potential for the fields that are conservative.  

        \item[(c)] Let \( \alpha \) be the path that goes from \( (0, 0, 0) \) to \( (1, 1, 1) \) by following edges of the cube \( 0 \leq x \leq 1 \),  
        \( 0 \leq y \leq 1 \), \( 0 \leq z \leq 1 \) from \( (0, 0, 0) \) to \( (0, 0, 1) \) to \( (0, 1, 1) \) to \( (1, 1, 1) \).  
        Let \( \beta \) be the path from \( (0, 0, 0) \) to \( (1, 1, 1) \) directly along the diagonal of the cube.  
        Find the values of the line integrals  
        \[
        \int_{\alpha} \mathbf{F} \cdot d\mathbf{s}, \quad 
        \int_{\alpha} \mathbf{G} \cdot d\mathbf{s}, \quad 
        \int_{\beta} \mathbf{F} \cdot d\mathbf{s}, \quad 
        \int_{\beta} \mathbf{G} \cdot d\mathbf{s}.
        \]
    \end{enumerate}
\end{prob}
\begin{proof}
    \begin{enumerate}
        \item[(a)] $F$ is not conservative, $G$ is conservative. One can verify this by checking
        \begin{equation*}
            \nabla\times F\neq 0, \quad \nabla\times G=0
        \end{equation*}
        \item[(b)] Potential for $G$ can be given by 
        \begin{equation*}
            g(x,y,z)=\frac{1}{4}x^4+\frac{1}{4}y^4-\frac{3}{2}x^2y^2+\frac{1}{2}z^2
        \end{equation*}
        \item[(c)] Because $G$ is conservative, we know the line integral only depends on the endpoints: 
        \begin{equation*}
            \int_\alpha G\cdot ds=\int_\beta G\cdot ds=g(1,1,1)-g(0,0,0)=-\frac{1}{2}
        \end{equation*}
        For $F$, one would have to parametrize each line segment and evaluate, for example, we parametrize $\beta$ by 
        \begin{equation*}
            \beta(t)=(t,t,t), 0\leq t\leq 1
        \end{equation*}
        and 
        \begin{equation*}
            \int_\beta F\cdot ds=\int_0^1(t^2, -t^2, t^2)\cdot \beta'(t)=\int_0^1(t^2, -t^2, t^2)\cdot(1,1,1)dt=\frac{1}{3}
        \end{equation*}
        Similarly, you do this for every segment of $\alpha$ and get 
        \begin{equation*}
            \int_\alpha F\cdot ds=0
        \end{equation*}
    \end{enumerate}
\end{proof}





\end{document}