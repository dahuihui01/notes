\documentclass[openany]{book}

\usepackage[margin=1in]{geometry}
\usepackage{amsmath,amsfonts,amsthm, amssymb}
\usepackage{yhmath}
\usepackage{mathrsfs}
\usepackage{mathtools}
\usepackage{xcolor}
\usepackage{graphicx}
\usepackage{comment}
\usepackage{tikz-cd}
\usepackage{quiver}
\usepackage{hyperref,cleveref}
\renewcommand{\familydefault}{ppl}
\newcommand{\tr}{\text{tr}}
\newcommand{\R}{\mathbb{R}}
\newcommand{\E}{\mathbb{E}}
\newcommand{\Z}{\mathbb{Z}}
\newcommand{\C}{\mathbb{C}}
\newcommand{\F}{\mathbb{F}}
\newcommand{\la}{\langle}
\newcommand{\ra}{\rangle}
\newcommand{\colim}{\text{colim}}
\DeclareMathOperator{\im}{im}
\let\oldemptyset\emptyset
\let\emptyset\varnothing
\newcommand{\tor}{\text{Tor}}
\newcommand{\id}{\text{id}}
\newcommand{\ext}{\text{Ext}}
\newcommand{\ptop}{\text{PTop}}
\newcommand{\pt}{\text{pt}}
\newcommand{\ach}{\text{Ach}}
\newcommand{\Q}{\mathbb{Q}}
\newcommand{\gal}{\text{Gal}}


\input{hui_r.tex}

\title{Calc III Midterm Essay Review
\\ 
\vspace{0.4cm}
\large Fall 2025}




\date{\today}
\author{Hui Sun}


\begin{document}

\maketitle

\tableofcontents
\newpage

\chapter{Definition review}


\begin{enumerate}
    \item definition review 
    \item proposition review
    \item practice problems
\end{enumerate}



1.1, 1.2, 1.3, 
2.1, 2.2, 2.3, 2.4, 2.5, 2.6
3.1, 3.2


\begin{defn}[standard basis in $\R^3$]
    The vectors 
    \begin{equation*}
        i=(1,0,0), j=(0,1,0), k=(0,0,1)
    \end{equation*}
    are called the standard basis vectors of $\R^3$, and for any vector $a=(a_1,a_2,a_3)\in\R^3$, we can write 
    \begin{equation*}
        a=a_1i+a_2j+a_3k
    \end{equation*}
\end{defn}


\begin{defn}[Equation of a line]\label{line}
    A line $l$ in $\R^3$ through the tip of $a=(a_1,a_2,a_3)$ pointing in the direction of a vector $v=(v_1,v_2,v_3)$ is given by 
    \begin{equation*}
        l(t)=a+tv
    \end{equation*}
    where $t\in\R$. Alternatively, a line passing through two points $P=(x_1,y_1,z_1), Q=(x_2,y_2,z_2)$ is given by 
    \begin{equation*}
        l(t)=(x(t), y(t),(z))
    \end{equation*}
    where 
    \begin{equation*}
        \begin{cases}
            x(t)=x_1+(x_2-x_1)t\\
            y(t)=y_1+(y_2-y_1)t\\
            z(t)=z_1+(z_2-z_1)t
        \end{cases}
    \end{equation*}
\end{defn}




\chapter{Practice Problems}


\begin{prob}
    Find the equation of the line passing through $(1,0,2)$ in the direction $(2,-1,3)$.
\end{prob}
\begin{proof}
    By definition \ref{line} The line is given by 
    \begin{equation*}
        l(t)=(1+2t, -t, 2+3t)
    \end{equation*}
\end{proof}

\begin{prob}
    In which direction does the line 
    \begin{equation*}
        l(t)=(3-2t, 2+5t, 1+t)
    \end{equation*}
    point?
\end{prob}
\begin{proof}
    In the direction of the vector $(-2, 5, 1)$.
\end{proof}










\begin{prob}
    Compute the following limits if they exist; if the limits don't exist, please explain why.
    \begin{enumerate}
        \item \begin{equation*}
            \lim_{(x,y)\to(1,1)}\frac{x^2+y^2-2xy}{x-y}
        \end{equation*}
    \end{enumerate}
\end{prob}

\begin{prob}
    Find the tangent plane of $f(x,y)=\ln(x+y)-2x$ at $(1,2)$.
\end{prob}



\begin{prob}
    
\end{prob}



\chapter{Answer Key}



\renewcommand\thesection{\arabic{section}}

\noindent























\end{document}