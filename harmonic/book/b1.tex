\chapter{Fourier Series and Integrals}

We will go through the book's notes in this document.
Chapter 1 is organized as follows:
\begin{enumerate}
    \item Fourier coefficients and Series
    \item Criteria for pointwise convergence
    \item Convergence in norm
    \item summability methods
    \item The fourier transform of $L^1$ functions
    \item Schwartz class and tempered distributions
    \item Fourier transform on $L^p$, for $1<p\leq 2$
    \item Convergence and summability of Fourier Integrals
    \item Further results
\end{enumerate}

\textbf{Some Notations}
The Lebesgue measure in $\R^n$ will be denoted using $dx$, and on the unit sphere $S^{n-1}$ will be $d\sigma$.

Let $a=(a_1, ..., a_n)\in\mathbb{N}^n$ be a multiindex, and $f:\R^n\to\C$, then
\begin{equation*}
    D^af=\frac{\partial^{|a|}f}{\partial_{x_1}^{a_1}...\partial_{x_n}^{a_n}}
\end{equation*}
where $|a|=a_1+...+a_n$.


\begin{theorem}[Minkowski's integral inequality.]
    Given $(X,\mu), (Y,\nu)$ as $\sigma$-finite measure spaces, we have the following inequality
    \begin{equation*}
        \left(\int_X\left|\int_Yf(x,y)d\nu(y) \right|^pd\mu(x) \right)^{1/p}\leq\int_Y \left(\int_X|f(x,y)^pd\mu(x) \right)^{1/p}d\nu(y)
    \end{equation*}
\end{theorem}

Taking $\nu$ to be the counting measure over a two point set $S={1,2}$ gives the usual Minkowski inequality
\begin{equation*}
    \|f_1+f_2\|_{L^p}\leq\|f_!\|_{L^p}+\|f_2\|_{L^p}
\end{equation*}

We will use $\mathcal{D}$ to denote the space of test functions, i.e. $C_c^\infty$, and $\mathcal{S}$ to denote the space of Schwartz functions. Recall the dual of $\mathcal{D}$, denoted as $\mathcal{D}'$ is the space of distrubitons, and $\mathcal{S}'$ is the space of temperate distributions.

\begin{definition}[Convolution of distribution]
    Let $T\in\mathcal{D}'$, and $f\in\mathcal{D}$, then we define
    \begin{equation*}
        T\ast f(x)=\langle T, \tau_x\tilde{f}\rangle
    \end{equation*}
    where $\tilde{f}(y)=f(-y)$, and $\tau_xf(y)=f(x+y)$. Hence it can be read as $T\ast f(x)=\langle T, f(x-y)\rangle$
\end{definition}
\begin{comment}
The theory of distribution will come in when we need meaningful differentiation but derivatives don't exist in the classical sense.
\end{comment}
\begin{comment}
    Tradionally, functions are thought of as sending points to points, but distributions are linear functions that act on a test function and send it to a point. In other words, distributions send functions to points using $\phi\mapsto\int u\phi dx$.
\end{comment}

\subsection{1.1 Review of definitions}
We now do some math.
If $f$ is a trigonometric series, of the form
\begin{equation*}
    f(x)=\sum_{n=-\infty}^\infty c_ne^{2\pi inx}
\end{equation*}
Then we find $c_n$ for any fixed $n$ by multiplying $f(x)$ by $e^{-2\pi inx}$, and integrate. Namely, we have
\begin{equation*}
    \int_0^1\sum_nc_ne^{2\pi inx}\cdot e^{-2\pi inx}=\int_0^1 c_n=c_n
\end{equation*}

We denote the additive group of $\R/\mathbb{Z}$ by $\T$, which gives $[0,1)$, and naturally identifies with $S^1$. Hence, saying a function $f$ defined on $\T$ is the same as saying $f$ is defined on $\R$ with period 1.

\begin{definition}[Fourier coefficients]
    Fix $f\in L^1(\T)$, we associate the sequence $\{\hat{f}(n)\}$ of $f$ defined by
    \begin{equation*}
        \hat{f}(k)=\int_0^1f(x)e^{-2\pi inx}dx
    \end{equation*}
    And its Fourier series defined as
    \begin{equation*}
        \sum_{n=-\infty}^\infty \hat{f}(n)e^{2\pi inx}
    \end{equation*}
\end{definition}


\subsection{1.2}

