\chapter{Fourier Series and Integrals}

We will go through the book's notes in this document.
Chapter 1 is organized as follows:
\begin{enumerate}
    \item Fourier coefficients and Series
    \item Criteria for pointwise convergence
    \item Convergence in norm
    \item summability methods
    \item The fourier transform of $L^1$ functions
    \item Schwartz class and tempered distributions
    \item Fourier transform on $L^p$, for $1<p\leq 2$
    \item Convergence and summability of Fourier Integrals
    \item Further results
\end{enumerate}

The Lebesgue measure in $\R^n$ will be denoted using $dx$, and on the unit sphere $S^{n-1}$ will be $d\sigma$.

Let $a=(a_1, ..., a_n)\in\mathbb{N}^n$ be a multiindex, and $f:\R^n\to\C$, then
\begin{equation*}
    D^af=\frac{\partial^{|a|}f}{\partial_{x_1}^{a_1}...\partial_{x_n}^{a_n}}
\end{equation*}
where $|a|=a_1+...+a_n$.


\begin{theorem}[Minkowski's integral inequality.]
    Given $(X,\mu), (Y,\nu)$ as $\sigma$-finite measure spaces, we have the following inequality
    \begin{equation*}
        \left(\int_X\left|\int_Yf(x,y)d\nu(y) \right|^pd\mu(x) \right)^{1/p}\leq\int_Y \left(\int_X|f(x,y)^pd\mu(x) \right)^{1/p}d\nu(y)
    \end{equation*}
\end{theorem}

Taking $\nu$ to be the counting measure over a two point set $S={1,2}$ gives the usual Minkowski inequality
\begin{equation*}
    \|f_1+f_2\|_{L^p}\leq\|f_!\|_{L^p}+\|f_2\|_{L^p}
\end{equation*}

We will use $\mathcal{D}$ to denote the space of test functions, i.e. $C_c^\infty$, and $\mathcal{S}$ to denote the space of Schwartz functions. 