\chapter{Singular Integrals}
\begin{definition}[Zero Average Functions on $S^{n-1}$]
    Let $\sigma$ denote the angular measure on $S^{n-1}$, then we say a function defined on $S^{n-1}$ has zero average if $y'\in S^{n-1}$
    \begin{equation*}
        \int_{S^{n-1}}\Omega(y')d\sigma(y')=0
    \end{equation*}
\end{definition}
And with this definition, we define singular integrals as the following operator.
\begin{proposition}
    For $\Omega:S^{n-1}\to\C$, if it has zero average, then
    \begin{equation*}
        \int_{\epsilon\leq|x|<1}\frac{\Omega(y')}{|y|^n}dy=0
    \end{equation*}
\end{proposition}
\begin{proof}
    We will demonstrate this using polar coordinates.
    \begin{equation*}
        \int_{\epsilon\leq|x|<1}\frac{\Omega(y')}{|y|^n}dy=\int_\epsilon^1 \frac{1}{r^n}\left(\int_{S^{n-1}}\Omega(y') d\sigma(y')\right)r^{n-1}dr=\int_{S^{n-1}}\Omega(y')d\sigma(y')\log(1/\epsilon)=0
    \end{equation*}
\end{proof}
\qed

\begin{definition}[Singular integrals]
    We call $T$ a singular integral is $T$ is an operator of the form: for $f\in\mathcal{S}$
    \begin{equation*}
        Tf(x)=\lim_{\epsilon\to 0}\int_{|y|>\epsilon}\frac{\Omega(y')}{|y|^n}f(x-y)dy
    \end{equation*}
    where $\Omega:S^{n-1}\to\C$, and $y'=\frac{y}{|y|}\in S^{n-1}$ and $\Omega\in L^1$, and with zero average, i.e.
    \begin{equation*}
            \int_{S^{n-1}}\Omega(y')d\sigma(y')=0
    \end{equation*}
\end{definition}
We would of course, want to make sense of the integral in the definition, i.e. does it converge for $f\in\mathcal{S}$.
\begin{proposition}
    $p.v. \frac{\Omega(x')}{|x|^n}$ is a tempered distribution.
\end{proposition}
\begin{proof}
    \begin{equation*}
        p.v.\frac{\Omega(x')}{|x|^n}(\phi)=\lim_{\epsilon\to 0}\int_{|x|>\epsilon}\frac{\Omega(x')}{|x|^n}\phi(x)dx=\int_{|x|<1}\frac{\Omega(x')}{|x|^n}(\phi(x)-\phi(0))+\int_{|x|>1}\frac{\Omega(x')}{|x|^n}\phi(x)dx
    \end{equation*}
    By $\phi\in\mathcal{S}$, we have that both 
    \begin{equation*}
        \frac{\phi(x)-\phi(0)}{|x|^n}\in\mathcal{S}, \frac{\phi(x)}{|x|^n}\in\mathcal{S}
    \end{equation*}
    Hence both integrals converge.
\end{proof}
\qed

\begin{corollary}[Necessary condition for singular integrals]
    If $p.v.\frac{\Omega(x')}{|x|^n}$ is a tempered distribution, then $\Omega$ has zero average.
\end{corollary}

\section{Fourier transform of the kernel}
\begin{definition}[Homogeneous functions]
    If $f$ is such that for any $x\in\R^n$ and any $\lambda>0$, we have
    \begin{equation*}
        f(\lambda x)=\lambda^af(x)
    \end{equation*}
    Then we say $f$ is of Homogeneous of degree $a$.
\end{definition}
Similarly, if we scale a test function such that
\begin{equation*}
    \phi_\lambda(x)=\lambda^{-n}\phi(\lambda^{-1}x)
\end{equation*}
And we integrate against a homogeneous of degree $a$ function $f(x)$, we have
\begin{equation*}
    \int\phi_\lambda(x)f(x)=\int \phi(x)f(\lambda x)dx=\lambda^a\int\phi(x)f(x)dx
\end{equation*}
\begin{definition}[Homogeneous distribution]
    A distribution $T$ is homogeneous of degree $a$, if for every test function $\phi$, we have
    \begin{equation*}
        T(\phi_\lambda)=\lambda^aT(\phi)
    \end{equation*}
\end{definition}
\begin{exercise}
$p.v.\frac{\Omega(x')}{|x|^n}$ is homogeneous of degree $-n$.
\begin{equation*}
    p.v.\frac{\Omega(x')}{|x|^n}(\phi_\lambda)=\lim_{\epsilon\to 0}\int_{|x|>\epsilon}\frac{\Omega(x')}{|x|^n}\phi_\lambda(x)dx=\lambda^{-n}p.v.\frac{\Omega(x')}{|x|^n}(\phi)
\end{equation*}
\end{exercise}

\begin{proposition}
    If $T$ is a tempered distribtuion, homogeneous of degree $a$, then its Fourier transform has degree $-n-a$.
\end{proposition}
\begin{proof}
    Let $\phi\in\mathcal{S}$, we have
    \begin{equation*}
        \hat{T}(\phi_\lambda)=T(\hat{\phi_\lambda})=T(\hat{\phi(\lambda\xi)})=\lambda^{-n}T(\lambda^n\phi(\lambda\xi))=\lambda^{-n}T(\phi_{\lambda^{-1}})=\lambda^{-n-a}T(\hat{\phi})=\lambda^{-n-a}\hat{T}(\phi)
    \end{equation*}
\end{proof}
\qed

\begin{theorem}
    The Fourier transform of our kernel $p.v.\frac{\Omega(x')}{|x|^n}$ is of homogeneous of degree 0, and
    \begin{equation*}
        \left(p.v.\frac{\Omega(x')}{|x|^n}\right)^{\widehat{\phantom{.}}}(\xi)=\int_{S^{n-1}}\Omega(u)\left[\log\left(\frac{1}{|u\cdot\xi'|}\right)-i\frac{\pi}{2}sgn(u\cdot\xi')\right]d\sigma(u)
    \end{equation*}
\end{theorem}



\section{Singular integrals with variable kernels}
Consider the following polynomial operator:
\begin{equation*}
    P(x,D)=\sum_{|a|=m}b_\alpha(x) D^\alpha
\end{equation*}
Note now $b_\alpha(x)$ is a variable function with $x$.

When $f\in\mathcal{S}$, we have
\begin{equation*}
    p(x,D)f(x)=\int_{\R^n}P(x,2\pi i\xi)\hat{f}(\xi)e^{2\pi ix\xi}d\xi
\end{equation*}
And let $\Gamma=(-\Delta)^{1/2}$, then we would like to find $T$ such that
\begin{equation*}
    P(x,D)f=T(\Lambda^mf)
\end{equation*}
\begin{align*}
    Tf(x)&=\int_{\R^n}\sigma(x,\xi)\hat{f}(\xi)e^{2\pi ix\cdot\xi}d\xi
\end{align*}
where
\begin{equation*}
    \sigma(x,\xi)=\frac{P(x,i\xi)}{|\xi|^m}
\end{equation*}
where it is homogeneous of degree 0, then 
\begin{equation*}
    Tf(x)=\int\sigma(x,\xi)e^{2\pi ix\cdot\xi}\int f(y)e^{-2\pi iy\cdot\xi}dyd\xi=\int_{\R^n}K(x,x-y)f(y)dy
\end{equation*}
where
\begin{equation*}
    K(x,z)=\int_{\R^n}\sigma(x,\xi)e^{2\pi iz\cdot\xi}d\xi
\end{equation*}
\begin{note}
    It looks like the inverse Fourier transform of a homogeneous function of degree 0, so we could apply the previous theories. (We won't be doing this rigorously).
\end{note}


\section{Lecture 10/27}
Today we will finish chapter 4. We will begin with a comment.
\begin{theorem}
    We have $T_m\in\mathcal{A}$ if and only if $m$ is nowhere-vanishing on $S^{n-1}$
\end{theorem}

We now continue our discussion of section 6, variable kernel.
\begin{equation*}
    K(x,z)=\int_{\R^n}\sigma(x,\xi)e^{2\pi iz\xi}d\xi
\end{equation*}
We have that $\sigma$ is homogeneous of  degree 0 in $\xi$, and 
\begin{equation*}
    \sigma(x,\cdot)\in C^\infty(S^{n-1})
\end{equation*}

By theorem 4.13, for every $x$, there exists $a(x)$ and $\Omega(x, \cdot)$ on $C^\infty(S^{n-1})$ with zero average such that
\begin{equation*}
    K(x,z)=a(x)\delta(z)+p.v.\frac{\Omega(x,z')}{|z|^n}
\end{equation*}
This motivates us to study the behavior:
\begin{equation*}
    Tf=\lim_{\epsilon\to 0}\int_{|y|>\epsilon}\frac{\Omega(x', y')}{|y|^n}f(x-y)dy
\end{equation*}
\begin{remark}
    Chapter 5 deals with the case where $\Omega$ is smooth.
\end{remark}

\begin{note}
    Convergence issues, leads us to study Hilbert transform (being bounded), and Hilbert transform naturally generalizes to higher dimensions of singular integrals, and now variable coefficients are natural in PDE's.
\end{note}
\begin{definition}[Psedodifferential operator]
    \begin{equation*}
        T_\sigma f(x)=\int\sigma(x,\xi)\hat{f}(\xi)e^{2\pi ix\cdot\xi}d\xi
    \end{equation*}
    is called a pseudo differential operator, where $\sigma$ is called a symbol. Special case: differential operator of variable coefficients.
\end{definition}
\begin{theorem}
    Let $\Omega(x,y)$ be a function that is homogeneous of degree 0 in $y$ such that
    \begin{enumerate}
        \item $\Omega(x,-y)=-\Omega(x,y)$
        \item $\Omega^*(u)=\sup_x|\Omega(x,u)|\in L^1(S^{n-1})$
    \end{enumerate}
    Then $T$ defined by 
    \begin{equation*}
         Tf=\lim_{\epsilon\to 0}\int_{|y|>\epsilon}\frac{\Omega(x', y')}{|y|^n}f(x-y)dy
    \end{equation*}
    is strong (p,p), $p<\infty$.
\end{theorem}
\begin{proof}
    Let $f\in\mathcal{S}$, then we have
    \begin{equation*}
        Tf(x)=\frac{\pi}{2}\int_{S^{n-1}}\Omega(x,u)H_uf(x)d\sigma(u)
    \end{equation*}
    Hence we have
    \begin{equation*}
        |Tf(x)|\leq\frac{\pi}{2}\int_{S^{n-1}}|\Omega^*(u)|H_uf(x)d\sigma(u)
    \end{equation*}
    And the conclusion follows from Minkowski's inequality.
    And we know that (from proposition 4.6), we have the operator
    \begin{equation*}
        T_\omega f(x)=\int_{S^{n-1}}T_uf(x)d\sigma(u)
    \end{equation*}
    is bounded on $L^p$. Hence we are done.
\end{proof}
\qed

\begin{note}
    If our second condition (2) is replaced with 
    \begin{equation*}
        \sup_x\left(\int_{S^{n-1}}|\Omega(x,u)|^pd\sigma(u)\right)^{\frac{1}{p}}=B_p<\infty
    \end{equation*}
    for $1<q<\infty$, then we have that $T$ is bounded on $L^p(\R^n)$, and $q'<p<\infty$.
\end{note}
\begin{proof}
    \begin{equation*}
        |Tf(x)|\leq\frac{\pi}{2}B_q\left(\int_{S^{n-1}}|H_uf(x)|^{q'}d\sigma(u)\right)^\frac{1}{q'}
    \end{equation*}
    Interchanging the order, we have that
    \begin{equation*}
        \int |Tf(x)|^{q'}dx\lesssim B_q^{q'}\int_{S^{n-1}}\int_{\R^n}|H_uf(x)|^{q'}dxd\sigma(u)\lesssim B_q^{q'}\|f\|_{L^{q'}}^{q'}
    \end{equation*}
    Since it holds for $q$, it also olds for $p'\in (1,q]$.
\end{proof}
\qed

\chapter{Singular integrals (II)}

\section{Calderon-Zygmund theorem}
\begin{theorem}[Calderon-Zygmund]
    Let $K\in\mathcal{S}'$ agreeing with a locally integral function on $\R^n\setminus\{0\}$ such that 
    \begin{enumerate}
        \item $|\hat{K}(\xi)\leq A$
        \item \begin{equation*}
            \int_{|x|>2|y|}|K(x-y)-K(x)|dx\leq B
        \end{equation*}
    \end{enumerate}
    Then we have 
    \begin{equation*}
        Tf=K\ast f
    \end{equation*}
    is weak $(1,1)$ and sstrong $(p,p)$, for $1<p<\infty$ with implied constants depend only on $n, p, A, B$.
\end{theorem}
\begin{proposition}
    (5.2) holds if we have
    \begin{equation*}
        |\nabla K(x)|\leq\frac{C}{|x|^{n+1}}
    \end{equation*}
    for all $x\neq 0$.
\end{proposition}
