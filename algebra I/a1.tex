\chapter{Groups}
We will talk about some facts about groups.
\begin{prop}
    Here are some basic properties of groups.
    \begin{enumerate}
        \item If $G$ is a group, and $e'\in G$ is an identity element, then $e'=e$.
        \item Moreover, if $g\in G$ has inverses $h_1, h_2$, then $h_1, h_2$.
        \item Let $g\in G$, and $gh=gf$, then $h=f$.
    \end{enumerate}
\end{prop}
\begin{defn}[abelian]
    A group $G$ is commutative or abelian if for all $a,b\in G$, we have $ab=ba$.
\end{defn}
Here are some examples:
\begin{enumerate}
    \item Cyclic groups are commutative. A cyclic group $G=\la g: g^n=e\ra$, i.e., it is the group generated subject to this condition and generated by one element. Equivalently, for every $h\in G$, there exists $m$ such that $h=g^m$.
    \item $M_n(\Z)$ under addition is commutative, under multiplication is not commutative.  
    \item $GL_n(\Z)$ is a group under multiplication since it's determinant is a unit. \[\begin{tikzcd}
        {Gl_n(\mathbb{Z})} & {M_n(\mathbb{Z})} \\
        {\mathbb{Z}^*} & {\mathbb{Z}}
        \arrow[from=1-1, to=1-2]
        \arrow["\det", from=1-2, to=2-2]
        \arrow[from=2-1, to=1-1]
        \arrow[from=2-2, to=2-1]
    \end{tikzcd}\]
    \item $GL_1(\Z)=Z^*$ is abelian.
    \item $GL_2(\Z)$ is not abelian, and so is not higher $n$
    \item The dihedral group $D_n=\la r,s: r^n=e, s^2=e, rs=sr^{-1}\ra$. Since $r^{-1}\neq r$ for $n>2$, we have $rs\neq sr$. Hence $D_n$ is not abelian for $n>2$.
    \item Alternatively, we can describe $D_n$ explicity, i.e., by $rs=sr^{-1}$, then we can always write $s$ in front of an $r$.
    \begin{equation*}
        D_n=\{e,r,\ldots, r^{-1}, s, sr, \dots, sr^{n-1}\}
    \end{equation*}
    \item $S_n$ is also not abelian for $n>2$. For example, $(123)(12), (23)(123)$. However, disjoint cycles commute.
\end{enumerate}
Remark: orders of elements in groups.
\begin{enumerate}
    \item $c_n=\la f; f^n=e\ra=\{e, f, f^2, \dots, f^{n-1}\}\cong \{0,1,\dots, n-1\}$ under addition modulo $n$. Now given $m\in\{0,1,\dots,n-1\}$, what is $|m|$?
\end{enumerate}
\begin{defn}[order]
    The order of $m$, is the least positive integer $l$, denoted $|m|$ such that $lm=0$. Moreover, if there exists integer $k$ such that $lm=kn$, then $l$ is the least positive integer suc that $\frac{lm}{n}\in\Z$.
\end{defn}
\begin{prop}
    Elements $m$ with $\gcd(m,n)=1$ has order $n$.
    Moreover, 
    \begin{equation*}
        |m|=\frac{n}{\gcd(m,n)}
    \end{equation*}
\end{prop}
\begin{prop}
    If $\gcd(m,n)=1$, then $m\in(\Z/n\Z)^*$.
\end{prop}
\begin{proof}
    $m\in(\Z/n\Z)^*$ if there exists $l$ such that $lm=1\mod n$, which implies that $lm=1+kn$, i.e. $lm-kn=1$, this implies that $m,n$ are relatively prime. Moreover, this is if and only if $|m|=n$ in the additive group.
\end{proof}
\begin{example}
    $\Z/12\Z: \{0,1,2,\dots, 11\}$, and $(\Z/12\Z)^*=\{1,5,7,11\}$, for the multiplicative group, $|5|=2, |7|=2, |11|=2$. This implies 
    \begin{equation*}
        (\Z/12\Z)^*\cong C_2\times C_2
    \end{equation*}
    where $C_2\times C_2=\{(a,b):a, b\in \pm 1\}$.
\end{example}
\begin{defn}[group homomorphism]
    A group homomorphism $\varphi:G\to H$ is a function $\varphi$ such that 
    \begin{equation*}
        \varphi(g_1g_2)=\varphi(g_1)\varphi(g_2), \varphi(e_G)=e_H, \varphi(g^{-1})=\varphi(g)^{-1}
    \end{equation*}
\end{defn}
\begin{defn}[isomorphism]
    An isomorphism $\varphi$ is a bijective isomorphism. In other words, in there exists $\psi: H\to G$ such that 
    \begin{equation*}
        \varphi\circ\psi=id_H, \psi\circ\varphi=id_G
    \end{equation*}
    In fact,requiring $\varphi$ as a bijection we can show $\psi$ is indeed a homomorphism.
\end{defn}
