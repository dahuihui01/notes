\chapter{Lecture 1}

here we go.

\section*{Course overview}
We will be discussing the nonlinear Schrodinger quations, which is a subcategory of nonlinear pde's, nonlinear dispersive equations, and infinite speed of propogation, 

Let's start with the linear Schrondinger equation.

\begin{equation*}
    i\partial_tu+\Delta u=0 \text{ in } \R\times\R^n, u(t=0)=u_0
\end{equation*}
The fundamental solution to a Schrondinger equation, is the $K(t,x)$ is such that $u_0=\delta_0$.

Instead, one could look at other intial data, for exmaple, $\widehat{u}_0=\delta_{\xi_0}$, or $u_0=e^{ix\xi_0}$.

\begin{remark}
    If you localize the initial data in the physical space, then the fourier transform is constant and therefore cannot be localized in the Fourier space. The reverse is also true if you try to localize in the Foureir space.
\end{remark}

If we have
\begin{equation*}
    u_0=e^{-\frac{(x-x_0)^2}{2}}e^{i(x-x_0)\cdot\xi_0}
\end{equation*}
For this type of initial data, we call it the coherent state, localized at $(x_0, \xi_0)$

In non-coherent state, the solution spreads out immediately; in the coherent state, the solution remains nicely behaved and coherent for a period of time, then it spreads out evenetually.

\begin{remark}
    This is the idea of group velocity, waves with frequence $\xi_0$ move with velocity $2\xi_0$. This $2\xi_0$ is called the group velocity. 
\end{remark}

Dispersive equation: waves with different frequencies travel in different directions.


\section{Nonlinear}
We will start with the nonlinear boys now.
\begin{equation*}
    i\partial_tu+\Delta u=\lambda u\cdot |u|^p
\end{equation*}
We will ask the following standard pde questions.
\begin{enumerate}
    \item existence
    \item uniqueness
    \item continuous dependent
    \item global in time behavior, i.e. linear vs nonlinear effects
\end{enumerate}
\begin{remark}
    If one just observes the RHS, then there is linear and nonlinear contributions, and one probalby would expect that one dominates over the other over time.
\end{remark}

\textbf{linear}: scattering. nonlinear solution looks like the linear solution

\textbf{nonlinear}: solitons (solutions that remain concentrated for a very long time, such as a bump function), blow-ups.

We will comment on the dispersive aspect of the Schrondinger equation before the nonlinear aspect.


\section{Dispersion}
Here are ways to measure dispersion. Given nicely behaved initial data, $u(t=0)=u_0\in H^s$.
\begin{enumerate}
    \item  dispersive estimates $\|u\|_{L^\infty}\leq t^{O(1)}\|u_0\|_{L^1}$
    \item Strichartz estimates $\|u\|_{L_t^pL_x^q}\leq \|u_0\|_{L^2}$
    \item Lateral Strichartz esiamtes, exchanging the role of $t, x$.
    \item Improved function spaces (Bourgain spaces, $U^p$, $V^p$)
    \item Local energy decay. If you have a dispersive solution, instead of measuring the solution everywhere, say, you measure it in the vertical cylinder.
\end{enumerate}

\textbf{Back to NLS.}
\begin{equation*}
    i\partial_t u+\Delta u=\lambda u|u|^p
\end{equation*}

We will talk about the following:
\begin{enumerate}
    \item local well-posedness
    \item global well-posedness for small initial data
    \item large initial data problem 
    \item energy critical problem $\int|\nabla u|^2$ and the mass critical problem $\int|u|^2$
\end{enumerate}
\begin{remark}
    The exponenet $p$ that we put on the RHS plays an important role int he above questions.
\end{remark}
Some topics in the foreseeable future: Littlewood-Paley theory, Bessel's problem, etc

\textbf{References}: Tao's on nonlinear and dispersive pde.

Now we will talk about Schrondinger maps
\begin{equation*}
    u: \R\times \R^n\to (M, g)
\end{equation*}
Sasy we have $u_t=i\Delta u$, then $u_t\in TM$, where $T$ stands for tangent, as we have rotated $\Delta u$ 90 degrees hence should live in the tangent of the manifold.
\begin{equation*}
    u_t=P\Delta u, P \text{ projection on } TM
\end{equation*}
The RHS $P\Delta u$ is called the heat flow. Let $M$ be a kahlan manifold.

Spherical case, $(M,g)=\mathbb{S}^2$. One can identify $\mathbb{S}^2$ as the comlex plane and compactified.
Hence if we would like an object that is perpendicular to both $u$ and $\Delta u$, and rotate by 90 degrees, then we look at the following equation
\begin{equation*}
    u_t=u\times \Delta u, u(t=0)=u_0
\end{equation*}


Then we come to the next section of the class, Quasilinear Schrondinger equations.
\begin{equation*}
    iu_t+g^{jk}(u)\partial_j\partial_k(u)=N(u, \nabla u), u(t=0)=u_0
\end{equation*}
Suppose $g^{jk}$ is a positive definite matrix, and the $N$ stands for nonlinear We will look at the local solvability. 

If we start with a simple guess, $N=\partial_j u$, and this becomes a ill-posed linear problem due to exponential growth (by taking the Fourier transform). Then we can probably repalce $N=(\nabla u)^2, N=(\nabla)^3$.

Another difficulty is how waves propogate, and ``trapping'' refers to when waves are localized eternally and do not propogate (sit in the vertical cylinder for example). This leads to the discussion of local well-posed theory.

For the \textbf{last part of the course}, we wil look at global solutions for quasilinear Schrondinger equations for small initial data, if $n\geq 3$, then somehow you can use the dispersive estimates mentioned above, via Strichartz. In higher dimension, the decay is faster, than the estimate is stronger, and the linear component plays more role. In low dimension, the nonlinear interactiosn are more prominennt. 

In $n=1$, there exists a following conjecture.
\begin{proposition}[Conjecture]
    If one has $1-d$ dispersive problem, that is cubic defocusing, then there exists a global solution for small data $u_0$.
\end{proposition}
In the case of QNLS, there is a proved theorem as above in 2023.



\section{Lecture 2}

We will now interpret the three quantities introduced above, mass, momentum and energy.

\textbf{First interpretation}
The Hamiltonian interpretation,

denote $H(u)=E(u)$, and $w(u,v)$ antisymmetric in $L^2$, and 
\begin{equation*}
    w(u,v)=\int Im(u\overline{v}), J=i
\end{equation*}
And $\partial_tu=JDH=-id\Delta u$, where $D$ is the differential form.

Given two Hamiltonians, $\{H_1, H_2\}=0$, can ask if they commute. This is to ask whether $H_1, H_2$ flows commute.

\begin{theorem}[Noether]
    Each sympletic symmetry of one Hamiltonian flow is generated by a commuting Hamiltonian.
\end{theorem}
\begin{remark}
    Sympletic refers to the solution preserving the sympletic form.
\end{remark}

$E$ generate the linear Schrodinger equation.

If we look at
\begin{equation*}
    \frac{\delta P_j}{\delta u}-i\partial_j u, \partial)t u-i\cdot i\partial_j u=-\partial_j u
\end{equation*}
This gives
\begin{equation*}
    \partial_t u+\partial u=0
\end{equation*}

This generates translations.

If we look at mass $M$, we have
\begin{equation*}
    \frac{\partial M}{\partial u}=2u, \partial_\theta u=i2u
\end{equation*}

This generates phase rotations.

We note that although mass is a conserved quantity, the mass is moving around. Hence, we associate a flux to it, i.e. a mass density. We define the mass density as follows
\begin{equation*}
    m(u)=|u|^2
\end{equation*}
And we take the time derivative
\begin{equation*}
    \partial_t m=\partial_jf_j
\end{equation*}
where $f_j$ is the mass transfer in the $j$-th direction.

\begin{equation*}
    \partial_tm=2Re(\partial_t u\cdot\overline{u})=2 Re(i\Delta u\cdot\overline{u})
\end{equation*}
Then by integration by parts, we have the above equal to
\begin{equation*}
    2Re(i\partial_j(\partial_j u\cdot\overline{u})-i\partial_ju\cdot\partial_j\overline{u})=\partial_j[2Im(\partial_ju\cdot\overline{u})]=-\partial_jp_j
\end{equation*}


We have shown that
\begin{equation*}
    \partial_jm(u)+\partial_jp_j=0
\end{equation*}
We could do similar computations for momentum.
\begin{equation*}
    \partial_tp_j+\partial_ke_{jk}=0
\end{equation*}
Where for the matrix $e_{jk}$, the trace of the matrix is equal to $e$, denoted as the energy density.
The above two equations give rise to the "Energy-momentum tensor is divergence free."?

$\begin{bmatrix}
    m & P \\
    P & e
\end{bmatrix}$
The divergence of the above matrix is equal to 0.

Solutions for the (LS).
We go back to the fundamental solution. For $u+0=\delta_0$. We have
\begin{equation*}
    \widehat{K}=e^{it\xi^2}\widehat{u}_0=e^{it\xi^2}
\end{equation*}
We thus have
\begin{equation*}
    K(t,x)=\frac{1}{(2\pi it)^{n/2}}e^{-ix^2/4t}
\end{equation*}

The dirac is Galilean invariant, and so is the fundamental solution, and $|K(t,x)|$ has to say constant due to Galilean invariance. Hence we have
\begin{equation*}
    |K(t,x)|=c_n\cdot t^{-n/2}
\end{equation*}

And we also have infinite speed of propogation, wave packets travel in all directions with the same speed, and do no discriminate different speeds.

Connection between propogation speed and frequency.
For
\begin{equation*}
    \partial_t\widehat{u}=it\xi^2\cdot u
\end{equation*}

Suppose $u$ only has frequencies close to $\xi_0$. Let's approximate $\xi$, Given
\begin{equation*}
    \xi^2=\xi_0^2+2\xi_0(\xi-\xi_0)
\end{equation*}

Approximate equation
\begin{equation*}
    \partial_t\widehat{u}=i[\xi_0^2+2\xi_0(\xi-\xi_0)]\widehat{u}=(2i\xi_0\cdot\xi-i\xi_0^2)\widehat{u}
\end{equation*}
Coming back to the physical space, we have
\begin{equation*}
    \partial_t{u}=2\xi_0\cdot\partial_xu-i\xi_0^2u
\end{equation*}
The first partial refers to the transport, and the second refers to the phase rotation.

This gives that $u(t,x)=u_0(x,x+2t\xi_0)e^{-it\xi_0^2}$. This gives the conclusion that waves with frequency $\xi_0$ now move with velocity $2\xi_0$.

(We may have a sign error, but imagine we have $\tau+\xi_0^2$), and the velocity $2\xi_0$ is called the roup velocity. If we denote $\tau+\xi^2$ as $\tau+\alpha(\xi)$ , then the group velocity is $\partial_\xi\alpha(\xi)$.

If the velocity of waves depends on frequency, we thus call this dispersive.

If I choose $u_0=e^{ix\xi_0}$, then we get
\begin{equation*}
    u(t)=e^{i(x_2t\xi_0)}\cdot e^{-it\xi_0^2}
\end{equation*}
The first part comes from the transport and the second part comes from the phase shift.

We have $\widehat{u}_0=\delta_{\xi_0}$. 

If we have $u_0=e^{-x^2/2}$, and $\widehat{u}_0=e^{-\xi^2/2}$, and $(x_0, \xi_0)=(0,0)$, and $(\delta x, \delta \xi)=(1,1)$.

Then $\widehat{u}(t,\xi)=e^{-it\xi^2}e^{-\xi^2/2}$. 

This gives
\begin{equation*}
    u(t,x)=\frac{1}{((1/2_it)2\pi)^{n/2}}e^{-x^2/(2-4it)}
\end{equation*}
Note that if $t\lesssim 1$, then it behaves nicely like a Gaussian, and if otherwise, we have waves spread out because the $4it$ term dominates.

Before a time threshold, we have the coherent state, where everything stays like a Gaussian, but becomes dispersive after some time.

Given Galilean invariance and translation invariance, we can move to $\xi_0$ and then to $x_0$. Hence now we have
\begin{equation*}
    u_0=e^{-(x-x_0)^2/2}e^{i(x-x_0)\xi_0^2}
\end{equation*}

These translations do not commute, however, it only varies our solution by a constnat factor, say $e^{i\xi_0^2}$ or something. The form of above $u_0$ is called the coherent state.

And we call the coherent state solutions, moving with velocity $2\xi_0$ as wave packets. And still $\delta x=1, \delta\xi=1$. And the time of coherent is 1, $\delta t=1$.

What if we want to change the scale, i.e. the sclaing symmetry. $(t,x)\mapsto (\lambda^2 t, \lambda x)$.

\begin{equation*}
    \delta t=T, \delta x=\sqrt{T}, \delta\xi=\frac{1}{\sqrt{T}}
\end{equation*}

By this scaling relation, We can adjust the time of the coherent state.

We can also think of LS solutions as superpositions of wave packets.

\section{Lecture 3}
We recall what we did last time.
\begin{equation*}
    (i\partial_t+\Delta)u=0, u(0)=u_0
\end{equation*}
And we have
\begin{equation*}
    u_0(x)=e^{-(x-x_0)^2/2}e^{ix_0(x-x_0)}
\end{equation*}
This refers to the coherent state, and it suffices to study $x_0=\xi_0=0$, and use translation invariance.


And note that we have $(x_0, \xi_0)$ as the center and the scales are $\delta x=1, \delta\xi=1$. 

We note this is not the only scale. We could instead have $\delta x=\sqrt{T}, \delta\xi=\frac{1}{\sqrt{T}}$, and for scale=1, we have coherent $T=1$. And for the other scale, we have the coherent time $T$ as $T$.

Now we ask the following questions:
Question: can we think of arbitrary solutions as superpositions of wave packets?

We want $u_0$= linear combination of coherent states. In other words, we now replace our typical $e^{-x^2/2}$ with any Schwartz function $\varphi$, and we have
\begin{equation*}
    u_0(x)=\varphi(x-x_0)e^{i\xi_0(x-x_0)}
\end{equation*}
Moreover, another justification to use arbitrary schwartz function is that if we are trying to solve
\begin{equation*}
    i\partial_tu-A(D)u=0, a\mapsto a(\xi)
\end{equation*}
Then there is no reason to use $u_0$ as the Gausssian $e^{-x^2/2}$ because the Gaussian does not stay Gaussian as it evolves.

We could use partition of unity. Let $\mathbb{Z}\subset\mathbb{R}^n$, we have
\begin{equation*}
    1=\sum_{j\in\mathbb{Z}^n}\chi_j(x)
\end{equation*}
And $\chi_j(x)=\chi(x-x_j)\in\mathcal{S}$. Now we have each $\chi_j$ contained in a unit cube.

And we have
\begin{equation*}
    u_0=\sum\chi_j(x)u_0=\sum_{k\in\mathbb{Z}^n}\sum_{j\in\mathbb{Z}^n}\chi_k(D)\chi_j(x)\cdot u_0
\end{equation*}
The above is called the spatial unit scale decomposition. However, this decomposition is not smooth compared to our initial form of the data.

Now we ask the question, how about a smooth wave packet decomposition.
\begin{equation*}
    u=\int f(x_0, \xi_0)u_{x_0, \xi_0}(x)dx_0d\xi_0
\end{equation*}
\begin{remark}
    The representation is not unique, and it one wishes to make this unique, they would have to have some sort of restriction on $f(x_0, \xi_0)$
\end{remark}

This is the Bargmann trnsform, or the Segal transform. If one includes the scaling, this is called FBi transform.

\begin{definition}
    Let $T$ be defined as follows: (using Gaussians)
    \begin{equation*}
        f(x_0, \xi_0)=Tu(x_0, \xi_0)=\int e^{-(x-x_0)^2/2}e^{-i\xi_0(x-x_0)}u(x)dx
    \end{equation*}
    Note we cannot hope this to be surjective, hence we have twice the amount of variables of $x_0, \xi_0$. Note we start with $x$, and end with a function in two variables.
\end{definition}

If we differentiate with respect to $\xi_0$, we get a $i(x-x_0)$ term, and if we differentiate with respect to $x_0$, we get a $(x-x_0)$ term, as well as $i \xi_0$ term, hence we have the following operator that kills the phase.
\begin{equation*}
    \left[\partial_{\xi_0}-i(\partial_{x_0}-\xi_0)\right]Tu=0
\end{equation*}

\begin{proposition}
    Such $T$ defined above, as the Bargmann transform, is an $L^2$ isometry.
    \begin{equation*}
        T*\circ T=I
    \end{equation*}
\end{proposition}


We define a slightly different transform:
\begin{equation*}
    \tilde{T}u(z)=\int e^{-1/2(x-z)^2}u(x)dx, z=x_0-i\xi_0
\end{equation*}
And now we have
\begin{equation*}
    \tilde{T}: L^2\to L^2(e^{-\xi_0^2})
\end{equation*}


\begin{definition}
    The FBi transform is simply a recaled version of the Bargmann transform.
    \begin{equation*}
        T_\lambda u(x_0, \xi_0)=\int e^{-(x-x_0)^2/2\lambda}e^{-i\xi_0(x-x_0)}u(x)dx
    \end{equation*}
\end{definition}
Now we have $\Delta x=\sqrt{\lambda}$, and $\xi=1/\sqrt{\lambda}$.

We have
\begin{equation*}
    u_0=\sum_{x_0, \xi_0\in\mathbb{Z}^n}c_{x_0, \xi_0}u_0^{x_0, \xi_0}
\end{equation*}
Then we have
\begin{equation*}
    u=\sum c_{x_0, \xi_0}e^{-itD^2}u_0(x_0, \xi_0)
\end{equation*}
The above is the solution to the linear Schrondinger. This is useful up to time 1, and it is viewed as a superposition of wave packets.

We shall approximate wave packets, we have $u_0$ localized at $(x_0, \xi_0)$. We approximate the solution as follows:
\begin{equation*}
    u(x,t)\sim u_0(x-2t\xi_0)e^{it\xi_0^2}
\end{equation*}
If we start with $e^{i(x-x_0)\xi_0}$, then roughtly it is $e^{i(x-2t\xi_0-x_0)}\xi_0$, where $\xi\sim\xi_0$, and $\tau\sim-\xi_0^2$. 

Then $u$ is a good approximate solution, you can verify this by 
\begin{equation*}
    (i\partial_t+\Delta)u\sim O(1)
\end{equation*}
However, this error adds up and is acceptable if $t<< 1$. Especially if you have nonhomogenous equation, the error adds up as time progresses.

We have
\begin{equation*}
    u(t,x)=\sum_{x_0,\xi_0}c_{x_0, \xi_0}u_0(x-2t\xi_0)e^{it\xi_0^2}
\end{equation*}
And this is a good approxiamte solution up to time 1.

\begin{remark}
    We note that the composition here is almost orthogonal, hence we have different frequencies, just like we had agove when we had $u=\sum c_{x_0,\xi_0}e^{-itD^2}u_0(x_0, \xi_0)$
\end{remark}

In the constant coefficient case, you start with $(x_0, \xi_0)$, at $t=0$, you end up at $(x_0+2\xi_0, \xi_0)$
But in variable coefficients, you have
\begin{equation*}
    (x_0, \xi_0)\to (x_t, \xi_t)
\end{equation*}
We no longer move in the linear fashion.

We now explore the case where we don't assume infinitely many derivatives. 

Let's consider wave packets with less localization. Simply consider the case at $(0,0)$ and use translation invariance and Galilean symmetry.

Take $u_0\in L^2$, what does it mean for this to be localized at (0,0)?

One proposition could be
\begin{equation*}
    x\cdot u_0\in L^2
\end{equation*}
This means as long as you move away from (0,0), $u_0$ decays. And we would also like to have decay in frequence, hence
\begin{equation*}
    \partial_xu_0(x,\xi)\in L^2
\end{equation*}
The above implies decay in $\xi$. Together they imply the solution $u$ is localizated at $(0,0)$ up to time $O(1)$.

If you wish it to localize at $(x_0-2t\xi_0, \xi_0)$, then we can have
\begin{equation*}
    (x-x_0-2t\xi_0)u\in L^2, [D_x-\xi_0]u_0\in L^2
\end{equation*}

By energy estimates, we have if $u$ solves the equation, then $D_xu$ also solves the equation, we have
\begin{equation*}
    \|D_xu\|_{L^2}=\|D_xu(0)\|_{L^2}
\end{equation*}
So it remains to consider $xu$, and how it interacts with the equation. We have
\begin{equation*}
    (i\partial_t+\Delta)x_ju=x_j(i\partial_t+\Delta)u+2\partial_ju
\end{equation*}
And the first term is 0, hence we have
\begin{equation*}
    \frac{d}{dt}\|xu\|_{L^2}^2=Re\int 2\overline{u}\partial_j udx=O(1)
\end{equation*}
Then we have $\overline{u}\in O_{L^2}(1), \partial_j u\in O_{l^2}(1)$, hence the entire term is of $O(1)$ in $L^2$.

Recall the fundamental solution of linear Schrondinger equation.
\begin{equation*}
    K(t,x)=\frac{1}{(4\pi it)^{n/2}}e^{ix^2/4t}, |K|\lesssim t^{-n/2}
\end{equation*}
Hence we have $u(t)=K(t)\ast u_0$. 
And we have the dispersive estimate for linear Schrondinger
\begin{equation*}
    \|u(t)\|_{L^\infty}\leq\|K(t)\|_{L^\infty}\|u_0\|_{L^1}\lesssim t^{-n/2}\|u_0\|_{L^1}
\end{equation*}

Dispersive estimates via stationary phase.

Let's generalize to variable coefficents.
\begin{equation*}
    (i\partial_t+A(D)\Delta)u=0
\end{equation*}
And our solution is of the form
\begin{equation*}
    u(t)=e^{itA(D)}u(0)
\end{equation*}
and
\begin{equation*}
    K(t)=\int_{\R^n}e^{ix\xi}e^{-ita(\xi)}d\xi
\end{equation*}

We have the oscillatory integral
\begin{equation*}
    I_\lambda=\int e^{i\lambda \varphi(\xi)}a(\xi)d\xi
\end{equation*}
Where we assume $a(\xi)$ having compact support.


And the value of the integral depends on the stationary points, defined by $D\varphi=0$.

If there are no stationary points, then we get
\begin{equation*}
    |I_\lambda|\leq \lambda^{-N}
\end{equation*}

Replace $\varphi$ with a quadratic expansion, and ``replace'' with a Gaussian. If we have nondegenerate points, $D\varphi\neq 0$. We have
\begin{equation*}
    |I_\lambda|\leq\lambda^{-n/2}
\end{equation*}
It's like putting $\varphi$ in $n-1$ dimension and operate with separation of variables.

We now examine our solution to the Schrondinger equation, and find the critical points here. We differentiate with respect to $x_j$
\begin{equation*}
    x_j+t\partial_{\xi_j}a(\xi)=0
\end{equation*}
We have
\begin{equation*}
    v\sim x/t=a_\xi
\end{equation*}
where $a_\xi$ is our previous group velocity.

The nondegenerate points are characterized as nonvanishing Hessian, which is $\sim D^2a$.
\begin{definition}
    The nondegenerate points are defined by the Hessian $\sim D^2a$, being a nondegenerate matrix.
\end{definition}
In LS, we have
\begin{equation*}
    a=\xi^2, D^2a=2I
\end{equation*}

Now we introduce a third way of viewing dispersive estimates, which is via \textbf{wave packets}.

We first note that the dispersive estimate
\begin{equation*}
    \|u\|_{L^\infty}\lesssim t^{-n/2}\|u_0\|_{L^1}
\end{equation*}
is scale invariant.  Hence it suffices to focus at $t=1$.

Let's take
\begin{equation*}
    u_0=\delta_0=\sum_{x_0,\xi_0\in\mathbb{Z}^n}c_{x_0, \xi_0}\varphi_{x_0, \xi_0}
\end{equation*}
we have
\begin{equation*}
    u_0=\sum\chi_j(x)u_0=u_0
\end{equation*}
And we want to localize frequency
\begin{equation*}
    \widehat{u_0}=\widehat{\delta_0}=1=\sum_{k\in\mathbb{Z}^n}\chi_k(\xi)
\end{equation*}
We have the $\|\chi_k\|_{L^2}=1$, and this says the Fourier coefficents are of $O(1)$.

Then
\begin{equation*}
    u=\sum_{x_0=0,\xi_0\in\mathbb{Z}^n}c u_{x_0, \xi_0}
\end{equation*}
At time $t=1$, we get one of the wave packet each, hence we have
\begin{equation*}
    |K(t=1)|\leq \sup_{\xi_0}|u_{x_0,\xi_0}|=1
\end{equation*}
\begin{remark}
    This no longe words for $t<<1$, hence there exists overlappings now. But to fix this, we rescale $\delta x=\sqrt{T}$, to get smaller and thinner tubes, nonoverlapping at time $T$.
\end{remark}

Now we ask the question, what is the good way to measure dispersive decay for $u_0\in L^2$.

\textbf{A1}: there is no uniform decay bound of the form
\begin{equation*}
    \|u(t)\|_{Sobolev}\leq C(t)\|u_0\|_{L^2}, \lim_{t\to\infty}C(t)=0
\end{equation*}
This is because $\|u(t)\|_{L^2}=\|u_0\|_{L^2}$, and taking $t$ infinitely large, $C(t)$ tends to 0, and we would get 
\begin{equation*}
    \|u\|_{sobolev}\leq 0, and t\to\infty
\end{equation*}

Next time, we will talk about Strichartaz estimates, not in a uniform way, but in an average sense of $\|u\|_{L_t^pL_x^2}$. 
