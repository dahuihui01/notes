\chapter{Homological Algebra}

% Page 29-34



\begin{prob}[S2012-Q2]
    \phantom{text}
    \begin{itemize}
        \item[(a)] Prove that if \(M\) is an abelian group and \(n\) is a positive integer, the tensor product \(M\otimes_{\mathbb{Z}}\mathbb{Z}/n\mathbb{Z}\) can be naturally identified with \(M/nM\).
        \item[(b)] Compute the tensor product over \(\mathbb{Z}\) of \(\mathbb{Z}/n\mathbb{Z}\) with each of \(\mathbb{Z}/m\mathbb{Z}\), \(\mathbb{Q}\) and \(\mathbb{Q}/\mathbb{Z}\). Also compute the tensor products \(\mathbb{Q}\otimes_{\mathbb{Z}}\mathbb{Q}\), \(\mathbb{Q}\otimes_{\mathbb{Z}}(\mathbb{Q}/\mathbb{Z})\), and \((\mathbb{Q}/\mathbb{Z})\otimes_{\mathbb{Z}}(\mathbb{Q}/\mathbb{Z})\).
        \item[(c)] Let \(\mathbb{Z}^\mathbb{N}\) denote the (abelian) group of sequences \((a_i)_{i\in\mathbb{N}}\) in \(\mathbb{Z}\) under termwise addition, and \(\mathbb{Z}^{(\mathbb{N})}\) the subgroup of sequences for which \(a_i = 0\) for all but finitely many \(i\). Define \(\mathbb{Q}^\mathbb{N}\) and \(\mathbb{Q}^{(\mathbb{N})}\) analogously. Compare \(\mathbb{Z}^{(\mathbb{N})}\otimes_{\mathbb{Z}}\mathbb{Q}\) to \(\mathbb{Q}^{(\mathbb{N})}\), and \(\mathbb{Z}^\mathbb{N}\otimes_{\mathbb{Z}}\mathbb{Q}\) to \(\mathbb{Q}^\mathbb{N}\).
    \end{itemize}
\end{prob}

\begin{prob}[F2006-Q4]
    Let \(R\) be a commutative ring. Let \(M\) be an \(R\)-module.
    \begin{itemize}
        \item[(1)] Write down the definition of \(\mathcal{T}(M)\), the tensor algebra of \(M\).
        \item[(2)] Assume \(R = \mathbb{Z}\) and \(M = \mathbb{Q}/\mathbb{Z}\). Compute \(\mathcal{T}(M)\).
        \item[(3)] If \(M\) is a vector space over a field \(R\), show that \(\mathcal{T}(M)\) contains no zero divisors.
    \end{itemize}
\end{prob}

\begin{prob}[S2009-Q5]
    Consider the \(\mathbb{Z}\)-modules \(M_i = \mathbb{Z}/2^i\mathbb{Z}\) for all positive integers \(i\). Let \(M = \prod_{i=1}^{\infty} M_i\). Let \(S = \mathbb{Z} - \{0\}\).
    \begin{itemize}
        \item[(a)] Show that
        \[\mathbb{Q} \otimes_{\mathbb{Z}} M \cong S^{-1}M.\]
        Here \(S^{-1}M\) is the localization of \(M\).
        \item[(b)] Show that
        \[\mathbb{Q} \otimes_{\mathbb{Z}} \prod_{i=1}^{\infty} M_i \neq \prod_{i=1}^{\infty} (\mathbb{Q} \otimes_{\mathbb{Z}} M_i).\]
    \end{itemize}
\end{prob}

% \begin{prob}[S2013-Q1]
%     Prove that, as a $\mathbb{Z}$-module, $\mathbb{Q}$ is flat but not projective.
% \end{prob}


\begin{prob}[F2008-Q5]
    For each \(n \in \mathbb{Z}\), define the ring homomorphism
    \[\phi_n : \mathbb{Z}[x] \to \mathbb{Z} \text{ by } \phi_n(f) = f(n).\]
    This gives a \(\mathbb{Z}[x]\)-module structure on \(\mathbb{Z}\), i.e,
    \[f \circ a = f(n) \cdot a \text{ for all } f \in \mathbb{Z}[x] \text{ and } a \in \mathbb{Z}.\]
    Now given two integers \(m,n \in \mathbb{Z}\), compute the tensor product \(\mathbb{Z} \otimes_{\mathbb{Z}[x]} \mathbb{Z}\) where the left-hand copy of \(\mathbb{Z}\) uses the module structure from \(\phi_n\) and the right-hand copy of \(\mathbb{Z}\) uses the module structure from \(\phi_m\). (Note: The answer depends on the numbers \(n\) and \(m\).)
\end{prob}


\begin{prob}[F2014-Q2]
    Let \(R = \mathbb{Q}[X]\), \(I\) and \(J\) the principal ideals generated by \(X^2 - 1\) and \(X^3 - 1\) respectively. Let \(M = R/I\) and \(N = R/J\). Express in simplest terms [the isomorphism type of] the \(R\)-modules \(M \otimes_R N\) and \(\text{Hom}_R(M, N)\). \textbf{Explain.}
\end{prob}


\begin{prob}[F2004-Q5]
    Consider the ideal \(I = (2,x)\) in \(R = \mathbb{Z}[x]\).
    \begin{itemize}
        \item[(a)] Construct a non-trivial \(R\)-module homomorphism \(I \otimes_R I \to R/I\), and use that to show that \(2 \otimes x - x \otimes 2\) is a non-zero element in \(I \otimes_R I\).
        \item[(b)] Determine the annihilator of \(2 \otimes x - x \otimes 2\).
    \end{itemize}
\end{prob}


% \begin{prob}[S2018-Q5]
%     Let \(n\) be a positive integer and \(A\) an abelian group. Prove that
%     \[\text{Ext}^1(\mathbb{Z}/n\mathbb{Z}, A) \cong A/nA.\]
% \end{prob}


% \begin{prob}[F2002-Q3]
%     Working over the integers, calculate (and show your work in a readable fashion) \(\text{Tor}(\mathbb{Z}/(p), \mathbb{Z}/(p))\).
% \end{prob}

% \begin{prob}[F2002-Q4]
%     Working over the integers, calculate (and show your work in a readable fashion) \(\text{Ext}(\mathbb{Z}/(p), \mathbb{Z}/(p))\).
% \end{prob}



\begin{prob}[S2018-Q2]
    Let \(R\) be a commutative ring. An \(R\)-module \(M\) is said to be finitely presented if there exists a right-exact sequence
    \[R^m \longrightarrow R^n \longrightarrow M \longrightarrow 0\]
    for some non-negative integers \(m,n\). Prove that any finitely generated projective \(R\)-module \(P\) is finitely presented.
\end{prob}



\begin{prob}[F2013-Q3]
    Let \(R\) be a commutative ring with unity. Given an \(R\)-module \(A\) and an ideal \(I \subset R\), there is a natural \(R\)-module homomorphism \(A \otimes_R I \to A \otimes_R R \cong A\) induced by the inclusion \(I \subset R\). In the following three steps you shall prove the flatness criterion: \textit{A is flat if and only if for every finitely generated ideal \(I \subset R\) the natural map \(A \otimes_R I \to A \otimes_R R\) is injective.}
    \begin{itemize}
        \item[(a)] Prove that if \(A\) is flat and \(I \subset R\) is a finitely generated ideal then \(A \otimes_R I \to A \otimes_R R\) is injective.
        \item[(b)] If \(A \otimes_R I \to A \otimes_R R\) is injective for every finitely generated ideal \(I\), prove that \(A \otimes_R I \to A \otimes_R R\) is injective for every ideal \(I\). Show that if \(K\) is any submodule of a free module \(F\) then the natural map \(A \otimes_R K \to A \otimes_R F \cong A\) induced by the inclusion \(K \subset F\) is injective (\textit{Hint}: the general case reduces to the case when \(F\) has finite rank).
        \item[(c)] Let \(\psi : L \to M\) be an injective homomorphism of \(R\)-modules. Prove that the induced map \(1 \otimes \psi : A \otimes_R L \to A \otimes_R M\) is injective (\textit{Hint}: Write \(M\) as a quotient \(f : F \to M\) of a free module \(F\), giving a short exact sequence \(0 \to K \to F \to M \to 0\) and consider the commutative diagram
        \[\begin{tikzcd}
            0 & K & J & L & 0 \\
            0 & K & F & M & 0
            \arrow[from=1-1, to=1-2]
            \arrow[from=1-2, to=1-3]
            \arrow["{\text{id}}", from=1-2, to=2-2]
            \arrow[from=1-3, to=1-4]
            \arrow[from=1-4, to=1-5]
            \arrow["\varphi", from=1-4, to=2-4]
            \arrow[from=2-1, to=2-2]
            \arrow[from=2-2, to=2-3]
            \arrow["f", from=2-3, to=2-4]
            \arrow[from=2-4, to=2-5]
        \end{tikzcd}\]
        where \(J = f^{-1}(\psi(L))\)).
    \end{itemize}
\end{prob}



% \begin{prob}[F2013-Q4]
%     \phantom{text}
%     \begin{itemize}
%         \item[(a)] Let \(R\) be a P.I.D. Prove that a finitely generated \(R\)-module \(M\) is flat if and only if \(M\) is torsion-free (hence, free by the structure theorem).
%         \item[(b)] Give an example of an integral domain \(R\) and a torsion-free \(R\)-module \(M\) such that \(M\) is not free.
%     \end{itemize}
% \end{prob}

% \begin{prob}[F2000-Q6]
%     Let \(R\) be the ring \(\mathbb{Q}[X]/(X^7-1)\), where \((X^7-1)\) is the ideal generated by \(X^7-1\) in \(\mathbb{Q}[X]\). Give an example of a finitely generated projective \(R\)-module which is not \(R\)-free. (We remind you that an \(R\)-module is called projective if it is a direct summand of a free \(R\)-module.)
% \end{prob}










% \chapter{Commutative Algebra}
% Topics: basic properties, Nakayama's lemma, integrality.

% % Page 35-40


% \begin{prob}[S2017-Q1]
%     Let \(A\) be a commutative ring, and define the \textit{nilradical} \(\sqrt{0}\) to be the set of nilpotent elements in \(A\). Show that \(\sqrt{0}\) is equal to the intersection of all prime ideals in \(A\). Show that if \(A\) is reduced, then \(A\) can be embedded into a product of fields.
% \end{prob}
% This one is complicated manipulation, so we omit.


% \begin{prob}[F2004-Q2]
%     Let \(\mathfrak{N}\) be the set of all nilpotent elements in a ring \(R\). Assume first that \(R\) is commutative.
%     \begin{itemize}
%         \item[(a)] Show that \(\mathfrak{N}\) is an ideal in \(R\), and \(R/\mathfrak{N}\) contains no non-zero nilpotent elements.
%         \item[(b)] Show that \(\mathfrak{N}\) is the intersection of all the prime ideals of \(R\).
%         \item[(c)] Give an example with \(R\) \textbf{non}-commutative where \(\mathfrak{N}\) is not an ideal in \(R\).
%     \end{itemize}
% \end{prob}

% \begin{prob}[S2014-Q4]
%     Let \(L/K\) be a Galois extension of degree \(p\) with \(\text{char}K=p\). Show that \(L=K(\theta)\), where \(\theta\) is a root of \(x^{p}-x-a,a\in K\), and, conversely, any such extension is Galois of degree 1 or \(p\).
% \end{prob}





% \begin{prob}[S2009-Q2]
%     Consider \(\mathbb{Z}[\omega] = \{a + b\omega \mid a, b \in \mathbb{Z}\}\) where \(\omega\) is a non-trivial cube root of 1. Show that \(\mathbb{Z}[\omega]\) is an Euclidean domain.
% \end{prob}


% \begin{prob}[F2006-Q3]
%     Let \(A\) be a principal integral domain and \(K\) be its field of fractions. Assume that \(R\) is a ring such that \(A \subset R \subset K\). Show that \(R\) is also a principal integral domain.
% \end{prob}


% \begin{prob}[F2001-Q2]
%     Let \(S\) denote the ring \(\mathbb{Z}[X]/X^2\mathbb{Z}[X]\), where \(X\) is a variable.
%     \begin{itemize}
%         \item[(a)] Show that \(S\) is a free \(\mathbb{Z}\)-module and find a \(\mathbb{Z}\)-basis for \(S\).
%         \item[(b)] Which elements of \(S\) are units (i.e. invertible with respect to multiplication)?
%         \item[(c)] List all the ideals of \(S\).
%         \item[(d)] Find all the nontrivial ring morphisms defined on \(S\) and taking values in the ring of Gaussian integers \(\mathbb{Z}[i]\).
%     \end{itemize}
% \end{prob}

% \begin{prob}[S2001-Q6]
%     Let \(R\) be the ring \(\mathbb{Z}[X,Y]/(YX^2-Y)\), where \(X\) and \(Y\) are two algebraically independent variables, and \((YX^2-Y)\) is the ideal generated by \(YX^2-Y\) in \(\mathbb{Z}[X,Y]\).
%     \begin{itemize}
%         \item[(a)] Show that the ideal \(I\) generated by \(Y-4\) in \(R\) is not prime.
%         \item[(b)] Provide the complete list of prime ideals in \(R\) containing the ideal \(I\) described in question (a).
%         \item[(c)] Which of the ideals found in (b) are maximal?
%     \end{itemize}
% \end{prob}

% \begin{prob}[F2017-Q3]
%     In this problem all rings are commutative.
%     \begin{itemize}
%         \item[(a)] Let \(R\) be a local ring with maximal ideal \(\mathfrak{m}\), let \(N\) and \(M\) be finitely generated \(R\)-modules, and let \(f\colon N\to M\) be an \(R\)-linear map. Show that \(f\) is surjective if and only if the induced map \(N/\mathfrak{m}N\to M/\mathfrak{m}M\) is.
%         \item[(b)] Recall that a module \(M\) over a ring \(R\) is \textit{projective} if the functor \(\operatorname{Hom}_{R}(M,-)\) is exact. Show that if \(R\) is local and \(M\) is finitely generated projective, then \(M\) is free.
%     \end{itemize}
% \end{prob}

% \begin{prob}[F2010-Q4]
%     Let \(A\) be a commutative Noetherian local ring with maximal ideal \(\mathfrak{m}\). Assume \(\mathfrak{m}^n = \mathfrak{m}^{n+1}\) for some \(n > 0\). Show that \(A\) is Artinian.
% \end{prob}

% \begin{prob}[F2009-Q5]
%     Let \(A, B\) be two Noetherian local rings with maxima ideals \(m_A, m_B\), respectively. Let \(f : A \to B\) be a ring homomorphism such that \(f^{-1}(m_B) = m_A\). Assume that:
%     \begin{itemize}
%         \item[1.] \(A/m_A \to B/m_B\) is an isomorphism.
%         \item[2.] \(m_A \to m_B/m_B^2\) is surjective.
%         \item[3.] \(B\) is a finitely generated \(A\)-module (via \(f\)). Show that \(f\) is surjective.
%     \end{itemize}
% \end{prob}


% \begin{prob}[F2015-Q6]
%     Let \(K\) be a finite algebraic extension of \(\mathbb{Q}\).
%     \begin{itemize}
%         \item[(a)] Give the definition of an integral element of \(K\).
%         \item[(b)] Show that the set of integral elements in \(K\) form a sub-ring of \(K\).
%         \item[(c)] Determine the ring of integers in each of the following two fields No credit for memorized answers: \(\mathbb{Q}(\sqrt{13})\), and \(\mathbb{Q}(\sqrt[3]{2})\).
%     \end{itemize}
% \end{prob}

% \begin{prob}[F2009-Q2]
%     Consider \(\mathbb{Q}[\sqrt{5}] = \{a + b\sqrt{5}|a,b \in \mathbb{Q}\}\). Determine the integral closure of \(\mathbb{Z}\) in \(\mathbb{Q}[\sqrt{5}]\).
% \end{prob}

% \begin{prob}[S2012-Q5]
%     \phantom{text}
%     \begin{itemize}
%         \item[(a)] Give the definition of a Dedekind domain.
%         \item[(b)] Give an example of a Dedekind domain that is not a principal ideal domain. Verify from the definition that it \textit{is} a Dedekind domain, and also that it isn't a principal ideal domain.
%     \end{itemize}
% \end{prob}


% \begin{prob}[S2005-Q5]
%     Let \(A\) be an integral domain and let \(K\) be its field of fractions. Let \(A'\) be the integral closure of \(A\) in \(K\). Let \(P \subset A\) be a prime ideal and let \(S = A - P\). (Note that \(A_P = S^{-1}A\) is contained in \(K\).) Show that \(A_P\) is integrally closed in \(K\) if and only if \((A'A) \otimes_A A_P = 0\).
% \end{prob}


% \begin{prob}[F2013-Q2]
%     Let \(a\) be an integral algebraic number such that its norm is 1 for any imbedding into \(\mathbb{C}\), the field of complex numbers. Prove that \(a\) is a root of unity.
% \end{prob}

% \begin{prob}[F2004-Q4]
%     Let \(\lambda_1, \ldots, \lambda_n\) be roots of unity, with \(n \geq 2\). Assume that \(\frac{1}{n} \sum_{i=1}^n \lambda_i\) is integral over \(\mathbb{Z}\). Show that either \(\sum_{i=1}^n \lambda_i = 0\) or \(\lambda_1 = \lambda_2 = \cdots = \lambda_n\).
% \end{prob}













