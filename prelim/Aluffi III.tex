\chapter{Aluffi III: Rings and Modules}
\begin{defn}
    An integral domain is nonzero commutative ring $R$ such that for all $a,b\in R$
    \begin{equation*}
        ab=0 \Rightarrow a=0 \text{ or } b=0
    \end{equation*}
    Equivalently, left and right multiplication by every nonzero element $u\in R$ is injective.
\end{defn}
In other words, if we define a left zero divisor as $a\in R$ such that for some $b\neq 0$ we have $ab=0$, then the integral domain requires NO nonzero zero divisors (0 is always going to be a zero divisor).
\begin{prop}
    Multiplication cancellation holds in integral domains, i.e., if $ab=ac$, then $b=c$, for $a\neq 0$.
\end{prop}
\begin{proof}
    We will know it holds if multiplication by $a$ is an injective function, this is true if and only if $a$ is not a zero divisor.
\end{proof}
\begin{example}
    $\Z, \R, \mathbb{C}, \mathbb{Q}$ are integral domains, and $\Z/n\Z$ is not, for some $n$. However, $\Z/p\Z$ is a integral domain.
\end{example}
\begin{defn}
    A left unit in $R$ is $u$ such that there exists $v$ and $uv=1$. In other words, a left unit has a right inverse.
\end{defn}
We note that two-sided units have unique inverses, and if we call the two-sided units just units, the units of a ring form a group!

One warning: if $u$ only has a right inverse, and no left inverse, then this $u$ may have many right inverses.
\begin{defn}[field]
   A field is a commutative ring such that every nonzero element is a unit. (And of course, fields are integral domains, integral domains are not always fields, for example, $\Z$).
\end{defn}
\begin{prop}
    In a field, left (and right) multiplication by any $u\neq 0$ is injective and surjective.
\end{prop}
\begin{proof}
    This uses the fact that $u\neq 0$ is a two-sided unit.
\end{proof}

\begin{prop}
    A finite integral domain is a field. (In other words, a finite commutative ring is a field if and only if it is an integral domain).
\end{prop}
\begin{proof}
    It suffices to show that the left and right multiplications by an element in this integral domain is surjective (this would imply this element is a unit). We know the multiplication is injective, and an injective map from a finite set to itself is also surjective.
\end{proof}
\begin{example}
    $\Z/p\Z$ is a field, hence an integral domain. This is because the group of units in $\Z/n\Z$ is those $m$ such that $\gcd(m,n)=1$. 

    If $m$ is a unit in $\Z/n\Z$, then $1\equiv am$ for some $m$, then $m$ is a generator of $\Z/n\Z$, hence $\gcd(m,n)=1$. Proof: $am=1 \mod n$, hence there exists some $b$ such that $am+bn=1$, hence $\gcd(m,n)=1$.
\end{example}
\begin{prop}
    A polynomial ring $R[x]$ is an integral domain if $R$ is an integral domain.
\end{prop}

\begin{example}
    A commutative ring $R$ is a field if and only if the only ideals are $R$ and $\{0\}$.
    \begin{proof}
        If $a\neq 0$, then $aR=R$, then the left multiplication by $a$ is surjective, hence $a$ is a unit.
    \end{proof}
\end{example}

\begin{example}
    Let $I$ be an ideal, $\varphi:R\to S$ be a ring homomorphism, $\varphi(I)$ need not to be an ideal.

    Take $R=\Z, I=2\Z, S=\mathbb{Q}$.
\end{example}
\begin{example}
    $\Z[x]$ is not a PID. We can take the ideal generated by $(2,x)$, it is the polynomials with even constant terms, but cannot be generated by a single element of $\Z[x]$.

    Alternatively, $(x)\subset (2,x)$, where $(x)$ is a prime ideal, but clearly $(x)$ is not maximal. But in a PID, nonzero prime ideal $\iff$ maximal ideal.

    However, $k[X]$ is a PID, for $k$ field.
\end{example}

\begin{example}
    A finite commutative ring $R$, and an ideal $I$ in $R$, $R/I$ is a field if and only if $R/I$ is an integral domain. \textbf{Hence}, for a commutative ring $R$, if $R/I$ is finite, an ideal is maximal if and only if it is prime.
\end{example}
\begin{example}
    A prime ideal in $\Z$ is exactly $(p)$, where $p$ is prime.
\end{example}
\begin{prop}
    Here are several conditions where the prime ideal $I$ is equivalent to maximal ideals.
    \begin{enumerate}
        \item $R/I$ is commutative and finite.
        \item $R$ is a commutative PID.
    \end{enumerate}
\end{prop}

\begin{example}
    $k[x]$ is a PID, just like $\Z$ is a PID.

    You choose a monic polynomial with the smallest degree and use the division algorithm to show that everything in $k[x]$ is divisible by this polynomial. Hence every prime ideal in $k[x]$ is also maximal.
\end{example}

\begin{defn}[dimension of rings]
    The Krull dimension of a commutative ring $R$ is the length of the longest chain of prime ideals in $R$.
\end{defn}
For example, for $k[x]$, where $k$ is field, $k[x]$ is a PID, and all prime ideals are maximal, so the Krull dimension of a PID is 1. Likewise, $\Z$ also has dimension 1. However, take $\Z[x]$, the length of prime ideals $(2,x)$ is 2, hence it has dimension 2. And $k[x_1, \ldots, x_n]$ has dimension n, where $(x_1)\subset (x_1, x_2)\subset\ldots\subset (x_1, \ldots, x_n)$.

\begin{prop}
    If $R/I$ is reduced (no nilpotent elements), then the ideal $I$ is radical. In other words, for $r^n\in I$, for some $n$, we have $r\in I$.
\end{prop}
\begin{proof}
    $(r+I)^n=I$ implies that $r^n\in I$, by $R/I$ being reduced, we know $r\in I$. Hence $R/I$ is reduced means that if $r^n\in I$, then $r\in I$. 
\end{proof}

\begin{example}
    Assume $R/IJ$ is reduced, then $I\cap J\subset IJ$. 

    We show that if $r\in I\cap J$, then $r^2\in IJ$, and by the previous proposition, $r\in IJ$.
\end{example}
