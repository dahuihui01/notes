\chapter{Aluffi III: Rings and Modules}
\begin{defn}
    An integral domain is nonzero commutative ring $R$ such that for all $a,b\in R$
    \begin{equation*}
        ab=0 \Rightarrow a=0 \text{ or } b=0
    \end{equation*}
    Equivalently, left and right multiplication by every nonzero element $u\in R$ is injective.
\end{defn}
In other words, if we define a left zero divisor as $a\in R$ such that for some $b\neq 0$ we have $ab=0$, then the integral domain requires NO nonzero zero divisors (0 is always going to be a zero divisor).
\begin{prop}
    Multiplication cancellation holds in integral domains, i.e., if $ab=ac$, then $b=c$, for $a\neq 0$.
\end{prop}
\begin{proof}
    We will know it holds if multiplication by $a$ is an injective function, this is true if and only if $a$ is not a zero divisor.
\end{proof}
\begin{example}
    $\Z, \R, \mathbb{C}, \mathbb{Q}$ are integral domains, and $\Z/n\Z$ is not, for some $n$. However, $\Z/p\Z$ is a integral domain.
\end{example}
\begin{defn}
    A left unit in $R$ is $u$ such that there exists $v$ and $uv=1$. In other words, a left unit has a right inverse.
\end{defn}
We note that two-sided units have unique inverses, and if we call the two-sided units just units, the units of a ring form a group!

One warning: if $u$ only has a right inverse, and no left inverse, then this $u$ may have many right inverses.
\begin{defn}[field]
   A field is a commutative ring such that every nonzero element is a unit. (And of course, fields are integral domains, integral domains are not always fields, for example, $\Z$).
\end{defn}
\begin{prop}
    In a field, left (and right) multiplication by any $u\neq 0$ is injective and surjective.
\end{prop}
\begin{proof}
    This uses the fact that $u\neq 0$ is a two-sided unit.
\end{proof}

\begin{prop}
    A finite integral domain is a field.
\end{prop}
\begin{proof}
    It suffices to show that the left and right multiplications by an element in this integral domain is surjective (this would imply this element is a unit). We know the multiplication is injective, and an injective map from a finite set to itself is also surjective.
\end{proof}
\begin{example}
    $\Z/p\Z$ is a field, hence an integral domain. This is because the group of units in $\Z/n\Z$ is those $m$ such that $\gcd(m,n)=1$. 

    If $m$ is a unit in $\Z/n\Z$, then $1\equiv am$ for some $m$, then $m$ is a generator of $\Z/n\Z$, hence $\gcd(m,n)=1$. Proof: $am=1 \mod n$, hence there exists some $b$ such that $am+bn=1$, hence $\gcd(m,n)=1$.
\end{example}
\begin{prop}
    A polynomial ring $R[x]$ is an integral domain if $R$ is an integral domain.
\end{prop}







