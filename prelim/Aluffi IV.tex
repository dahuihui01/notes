\chapter{Groups II}

\begin{thm}[$p$-group, fixed point theorem]
    Let $G$ be a $p$-group acting on a fintie set $A$ (a finite group with a power of $p$ elements), let $Z=\{a\in A: ga=a \text{for all} g\}$, then we have 
    \begin{equation*}
        |A|\equiv |Z| \mod p
    \end{equation*}
\end{thm}
For example, one could apply this to the conjugacy action, giving the class formula.
\begin{thm}[Class formula]
    Let $G$ be a finite set, and $Z(G)$ be its center, then we have
    \begin{equation*}
        |G|=|Z(G)|+\sum_{a\in A}|O_a|
    \end{equation*}
    where $A$ is a set that contains one representative for every nontrivial conjugacy class in $G$.
\end{thm}
With the class formula, we can do a lot of things. For example, we know every group of prime order is cyclic, by consider $\la g\ra, g\neq 0$ and applying Lagrange. Hence it's commutative. Now we also know that every group of order $p^2$ is commutative.
\begin{prop}
    Every group of order $p^2$ is commutative.
\end{prop}
\begin{proof}
    $G$ is a $p$-group, hence has no trivial center $Z(G)$, hence $|Z(G)|=p$ or $p^2$. If $|Z(G)|=p$, then we know that $|G/Z(G)|=p$, hence it is cyclic, then $G$ is commutative (can check this). If $|Z(G)|=p^2$, then it is commutative anyways.

    More general statement: let $|G|=pq$, where $p,q$ are primes, then either $G$ is commutative or $G$ has trivial center.
\end{proof}
\begin{example}
    $S_3$ cannot have a normal subgroup $H$ of order 2.

    This is because $S_3$ has trivial center, and any normal subgroup is a union of conjugacy classes, hence $H$ must have one other conjugacy class containing of one element. However, a conjugacy class with one element lives in the center. This is a contradiction.
\end{example}


