\chapter{Groups II}

\begin{thm}[$p$-group, fixed point theorem]
    Let $G$ be a $p$-group acting on a fintie set $A$ (a finite group with a power of $p$ elements), let $Z=\{a\in A: ga=a \text{for all} g\}$, then we have 
    \begin{equation*}
        |A|\equiv |Z| \mod p
    \end{equation*}
\end{thm}
For example, one could apply this to the conjugacy action, giving the class formula.
\begin{thm}[Class formula]
    Let $G$ be a finite set, and $Z(G)$ be its center, then we have
    \begin{equation*}
        |G|=|Z(G)|+\sum_{a\in A}|O_a|
    \end{equation*}
    where $A$ is a set that contains one representative for every nontrivial conjugacy class in $G$.
\end{thm}
With the class formula, we can do a lot of things. For example, we know every group of prime order is cyclic, by consider $\la g\ra, g\neq 0$ and applying Lagrange. Hence it's commutative. Now we also know that every group of order $p^2$ is commutative.
\begin{prop}
    Every group of order $p^2$ is commutative.
\end{prop}
\begin{proof}
    $G$ is a $p$-group, hence has no trivial center $Z(G)$, hence $|Z(G)|=p$ or $p^2$. If $|Z(G)|=p$, then we know that $|G/Z(G)|=p$, hence it is cyclic, then $G$ is commutative (can check this). If $|Z(G)|=p^2$, then it is commutative anyways.

    More general statement: let $|G|=pq$, where $p,q$ are primes, then either $G$ is commutative or $G$ has trivial center.
\end{proof}
\begin{example}
    $S_3$ cannot have a normal subgroup $H$ of order 2.

    This is because $S_3$ has trivial center, and any normal subgroup is a union of conjugacy classes, hence $H$ must have one other conjugacy class containing of one element. However, a conjugacy class with one element lives in the center. This is a contradiction.
\end{example}


\begin{example}
    \begin{enumerate}
        \item If there is only 1 cyclic subgroup of order $p$, then this subgroup must be normal.
    \end{enumerate}
\end{example}

\begin{thm}
    Fix $p$, all $p$-Sylow subgroups are conjugate to each other.
\end{thm}

\begin{example}
    Every simple $p$-groups are cyclic. (Hint: they must have order $p$).
\end{example}


Slogan: size of the conjugacy class $[a]$ is equal to the index of the centralizer $[G: Z(a)]$.

\section{IV.4 Conjugacy classes in $S_n$}
\begin{defn}[type]
    The type of $\sigma\in S_n$ is the length of disjoint cycles(orbits) it makes. For example, if $\sigma$ acts on $\{1, \ldots, 8\}$ as $(12346)(78)$, then the type is $[5,2,1]$.

    Please note that the cycles used in a type are disjoint, since they are orbits.
\end{defn}
\begin{prop}
    Let $\tau\in S_n$, and $(a_1, \ldots, a_n)$ be a cycle, then
    \begin{equation*}
        \tau^{-1}(a_1, \ldots, a_n)\tau=(\tau^{-1}(a_1),\dots, \tau^{-1}(a_n))
    \end{equation*}
\end{prop}
\begin{thm}
    $\sigma_1, \sigma_2$ are conjugates of each other if and only if they have the same type.

    In other words, one type of $\sigma\in S_n$ corresponds to one conjugacy class.
\end{thm}
\begin{proof}
    Disjoint cycles are still disjoint under conjugation (since it's a bijection). And if they have the same type, one can define $\tau(a_i)=a_i'$ to establish conjugation between the two elements in $S_n$.
\end{proof}
\begin{thm}[Conjugacy in $S_n$]
    The number of conjugacy classes in $S_n$ is the number of partitions of $n$.
\end{thm}

\begin{thm}
    There are 5 conjugacy classes $[\sigma]_{S_n}$ of even permutations in $S_n$, but there are 6 conjugacy classes $[\sigma]_{A_n}$, as the type $[5]$ splits due to prop 4.15.
\end{thm}

\section{IV.5 Short exact sequences}
\begin{defn}[short exact sequence]
    Let $N,H$ be groups, and $\alpha, \beta$ be homomorphisms, then an exact short sequence is the following:
    \[\begin{tikzcd}
        1 & N & G & H & 1
        \arrow[from=1-1, to=1-2]
        \arrow["\alpha", from=1-2, to=1-3]
        \arrow["\beta", from=1-3, to=1-4]
        \arrow[from=1-4, to=1-5]
    \end{tikzcd}\]
    where $\alpha$ is injective and $\beta$ is surjective, and $\ker(\beta)=Im(\alpha)$.

    A short exact sequence is said to split if $H$ can be realized as a subgroup of $G$, and $H\cap N=\{e\}$.
\end{defn}
