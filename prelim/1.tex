\chapter{Set Theory}

We will review some definitions that we might forget.

\begin{defn}[order relation or simple order]
    An ordered relation $C$ is one such that 
    \begin{enumerate}
        \item For every $x,y\in A$, either $xCy$ or $yCx$.
        \item $xCx$ is not true for all $x$.
        \item If $xCy, yCz$, then $xCz$. 
    \end{enumerate}
\end{defn}

\begin{defn}[well-ordered set]
    A set $A$ with an order relation $<$ is called well-ordered if every set has a smallest element.
\end{defn}
If $A$ is a well-ordered set, then any subset of $A$ with restricted order is a well-ordered set. And if $A,B$ are well-ordered sets, then the directionary product of $A\times B$ is well-ordered.

\begin{thm}
    Let $A$ by any set, then there exists a order relation on $A$ such that $A$ is well-ordered.
\end{thm}

\begin{defn}[strict partial order, partial order]
    Given a set $A$, a relation $\prec$ is called a stric partial order if 
    \begin{enumerate}
        \item There is no $x\in A$, such that $x\prec x$.
        \item For $x\prec y, y\prec z$, we have $x\prec z$.
    \end{enumerate}
    A partial order $\leq$ is such that 
    \begin{enumerate}
        \item $x\leq x$ for all $x$
        \item If $x\leq y, y\leq x$, then $x=y$.
        \item If $x\leq y, y\leq z$, then we have $x\leq z$.
    \end{enumerate}
\end{defn}
Note that a strict partial order is a order relation without the comparability.
\begin{prop}
    Let $A$ be a strictly partially ordered set, then there exists a maximal simply ordered subset $B$ of $A$.
\end{prop}
\begin{lem}[Zorn's lemma]
    Let $A$ be a strictly partially ordered set. If every simply ordered subset of $A$ has an upper bound in $A$, then there exists a maximal element in $A$.
\end{lem}



