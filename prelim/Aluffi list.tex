\chapter{Definition and Theorem List}
\subsection{groups}
\begin{prop}
    Cancellation holds in groups, i.e., if $g_1a=g_2a$, then $g_1=g_2$.
\end{prop}

\begin{defn}[order]
    An order of an element $g$ is the smallest positive integer $n$ such that $g^n=e$, denoted $|g|$, and $|g|=\infty$ if it does not have finite order.
\end{defn}

\begin{prop}
    If $g^k=e$ for some $k$, then $|g|$ divides $k$.
\end{prop}
\begin{prop}
    Let $g\in G$ have finite order, then $g^m$ also has finite order, and 
    \begin{equation*}
        |g^m|=\frac{|g|}{\gcd(m,|g|)}
    \end{equation*}
\end{prop}
The next one concerns with the order of commuting elements.
\begin{prop}
    If $gh=hg$, then $|gh|$ divides $lcm(|g|, |h|)$.
\end{prop}
\begin{defn}[symmetric group]
    The symmetric group $S_n$ is the group Aut$\{1, \dots, n\}$, i.e., the group of permutations of $\{1,\dots, n\}$.
\end{defn}
\begin{defn}[Dihedral group]
    The dihedral group $D_{2n}$ is the group of rotations and reflections of a $n$-polygon. There are $n$ rotations and $n$ reflections, hence $2n$ elements.
\end{defn}
\begin{warn}
    Symmetries in $D_{2n}$ corresponds to permutations in $S_n$, and you can view $D_{2n}$ acting faithfully on the set $\{1, \dots, n\}$.
\end{warn}
\begin{defn}[modulo]
    An equivalence relation defined on $\Z$ with modulo:
    \begin{equation*}
        a\equiv b \mod n \iff n\mid (b-a)
    \end{equation*}
\end{defn}
\begin{prop}
    If $a\equiv a'\mod n$ and $b\equiv b'\mod n$, then $a+b=a'+b'\mod n$. In other words, the cosets addition is well-defined.
\end{prop}
\begin{prop}
    The order of $m\in\Z/n\Z$ is 1 if $n\mid m$, and 
    \begin{equation*}
        |m|=\frac{n}{\gcd(m,n)}
    \end{equation*}
\end{prop}
Then we have the next quite important statement.
\begin{thm}
    $m\in\Z/n\Z$ generates $\Z/n\Z$ if and only if $\gcd(m,n)$.
\end{thm}
\begin{warn}
    This means that every element in $\Z/p\Z$ generates it.  
\end{warn}
\begin{defn}[multiplicative group]
    Let $(\Z/n\Z)*:=\{m\in\Z/n\Z: \gcd(m,n)=1\}$, this is a group with the multiplication operation $(a\mod n)\cdot (b\mod n)=ab\mod n$.
\end{defn}
\begin{prop}
    Let $\varphi:G\to H$ be a group homomorphism, then $\varphi(e_G)=e_H$, and $\varphi(g^{-1})=\varphi(g)^{-1}$.
\end{prop}
\begin{prop}
    The trivial group is both initial and final in Grp. (recall initial is there exists only one homomorphism from $\{e\}$ to any group, and final is there exists only one homomorphism from any group to $\{e\}$).
\end{prop}
Now we see how homomorphisms work with order.

\begin{prop}
    Let $\varphi: G\to H$ be a group homomorphism, and let $g\in G$ be an element of finite order, then $|\varphi(g)|$ divides $|g|$.
\end{prop}
\begin{defn}[cyclic]
    A group is cyclic if it is isomorphic to $\Z$ or to $\Z/n\Z$.
\end{defn}
\begin{prop}
    Let $\varphi:G\to H$ be an isomorphism, then for all $g\in G$, $|\varphi(g)|=|g|$. And $G$ is commutative if and only if $H$ is commutative.
\end{prop}
\begin{thm}
    Two \textit{commutative} finite groups are isomorphic if and only if they have the same number of elements of any given order.
\end{thm}
\begin{example}
    $S_3$ and $\Z/6\Z$ are not isomorphic, since one is abelian, one isn't.
\end{example}
\begin{defn}[group of homomorphisms]
    Let $H$ be commutative, then $\hom(G,H)$ is a commutative group for all group $G$.
\end{defn}


\subsection{Subgroups}
\begin{prop}
    If $\{H_\alpha\}_{\alpha\in A}$ is any family of subgroups a group $G$, then $H=\bigcap_{\alpha}H_\alpha$ is a subgroup of $G$.
\end{prop}
\begin{defn}[finitely generated]
    A group $G$ is finitely generated if there exists a finite subset 
    \begin{equation*}
        \{a_1,\dots, a_n\}\subset G
    \end{equation*}
    such that $G=\la a_1, \dots, a_n\ra$.
\end{defn}
\begin{prop}
    Subgroups of cyclic groups are cyclic.
\end{prop}
\begin{prop}
    Let $G\subset\Z$ be a subgroup, then $G=d\Z$ for some $d\geq 0$. And every subgroup of $\Z$ is isomorphic to $\Z$.
\end{prop}
\begin{prop}
    Let $n>0$ be an integer and let $G\subset\Z/n\Z$ be a subgroup, then $G$ is a cyclic subgroup of $\Z/n\Z$ generated by some $d\in\Z/n\Z$, and $d\mid n$.
\end{prop}
\begin{prop}
    Let $\varphi:G\to H$ be a homomorphism, then the following are equivalent:
    \begin{enumerate}
        \item $\varphi$ is monic
        \item $\ker\varphi=\{e_G\}$.
        \item $\varphi; G\to G'$ is injective as a set function.
    \end{enumerate}
    monic means that for any group $G'$, and two homomorphisms $f, g:G'\to G$, if $\varphi\circ f=\varphi\circ g$, then $f=g$. 
\end{prop}
\subsection{Quotient groups}
\begin{defn}[normal subgroup]
    A subgroup $N\subset G$ is normal if for all $g\in G$, 
    \begin{equation*}
        gNg^{-1}\in N
    \end{equation*}
\end{defn}
\begin{defn}[quotient group]
    Let $H$ be a normal subgroup of a group $G$, then the quotient group $G/H$ is the group obtained by cosets, group operation defined by 
    \begin{equation*}
        (aH)(bH):=(ab)H
    \end{equation*}
\end{defn}
\begin{thm}[first homomorphism theorem]
    Let $H$ be a normal subgroup, then for every group homomorphism $\varphi:G\to G'$ such that $H\subset\ker\varphi$, there exists a unique homomorphism $\tilde{\varphi}:G/H\to G'$ such that
    \begin{equation*}
        \varphi=\tilde{\varphi}\circ\pi
    \end{equation*}
    where $\tilde{\varphi}(aH)=\varphi(a)$.
\end{thm}
\begin{cor}
    Let $\varphi:G\to G'$ be a surjective homomorphism, then
    \begin{equation*}
        G/\ker\varphi\cong G'
    \end{equation*}
\end{cor}

\begin{prop}
    Let $H$ be a normal subgroup, and $K$ is any subgroup containing $H$, then there is a bijection between $K$ and a subgroup of $G/H$, defined by $u(K)=K/H$. In other words, a subgroup of $K$ containing $H$ is a collection of cosets in $G/H$.
\end{prop}
\begin{thm}[fraction holds]
    Let $H,N$ to be normal subgroups of $G$, then 
    \begin{equation*}
        \frac{G/H}{N/H}\cong G/N
    \end{equation*}
\end{thm}
\begin{thm}[second homomorphism theorem]
    Let $H,K$ be subgroups and $H$ is normal, then
    \begin{enumerate}
        \item $HK$ is a subgroup of $G$, and $H$ is normal in $HK$.
        \item $H\cap K$ is normal in $K$, and 
        \begin{equation*}
            \frac{HK}{H}\cong\frac{K}{H\cap K}
        \end{equation*}
    \end{enumerate}
\end{thm}
\subsection{Lagrange's theorem}
\begin{defn}[index]
    Let $H$ be a normal subgroup, the index is the number of cosets of $H$ in $G$, i.e., the number of elements in the quotient group $G/H$.
\end{defn}
\begin{prop}
    Let $H$ be a subgroup of a group $G$. then for all $g\in G$, the functions
    \begin{equation*}
        H\to gH, h\mapsto gh
    \end{equation*}
    is a bijection.
\end{prop}
\begin{thm}[Lagrange's theorem]
    If $G$ is a finite group, and $H$ is a subgroup, then $|G|=[G:H]\cdot|H|$. In particular, $|H|$ divides $|G|$.
\end{thm}
\begin{cor}
    The order of any element $|g|$ divides the order of the group.
\end{cor}
\begin{example}
    Let $G$ be any group of order $p$, then the group is cyclic and abelian. In other words, the group is isomorphic to $\Z/p\Z$.
\end{example}
\begin{proof}
    Let $g\in G$ be any element that is not $e$, then $G\cong\la g\ra$, hence is cyclic of order $p$.
\end{proof}
\begin{thm}[Fermat's last theorem]
    Let $p$ be any prime integer, and let $a$ be any integer, then 
    \begin{equation*}
        a^p\equiv a\mod p
    \end{equation*}
\end{thm}


\subsection{group actions}
\begin{defn}[action]
    An action of a group $G$ on a set $A$ is a set-function $\rho: G\times A\to A$ such that $e\cdot a=a$, and for all $g,h\in G$, and $a\in A$, associativity holds
    \begin{equation*}
        gh\cdot a=g\cdot (h\cdot a)
    \end{equation*}
\end{defn}
\begin{example}
    $G$ acts by left-multiplication on the set $G/H$ of left-cosets, sending $aH$ to $gaH$.
\end{example}
\begin{defn}[transitive action]An action of a group $G$ on a set $A$ is transitive if for all $a,b\in A$, there exists $g\in G$, such that 
    \begin{equation*}
        b=g\cdot a
    \end{equation*}
    In other words, every element can be expressed as another element under an action.
\end{defn}
\begin{defn}[orbit]
    The orbit of $a\in A$ under an action of a group $G$ is the set 
    \begin{equation*}
        O(a)=\{g\cdot a: g\in G\}
    \end{equation*}
\end{defn}
\begin{defn}[stabilizer]
    Let $G$ act on a set $A$, and let $a\in A$, the stabilizer is the subroup of $G$ which fix $a$.
    \begin{equation*}
        \text{Stab}(a)=\{g\in G: g\cdot a=a\}
    \end{equation*}
\end{defn}
\begin{prop}
    Orbits form a partition of the set $A$. The action of $G$ acts transitively on any orbit.
\end{prop}
\begin{prop}
    Every transitive left-action of $G$ on a nonempty set $A$ is isomorphic to the left-multiplication of $G$ on $G/H$, $H=$ the stabilizer of any $a\in A$. More precisely, if $G$ acts transitively on $A$, then $|A|$ divides $|G|$. 

    For example, let $G'$ be a group, and $G$ acts on $G'$, then 
    \begin{equation*}
        G/Stab(a)\cong O(a)
    \end{equation*}
    where $a\in G'$, and $O(a)$ is the orbit of $a$. The isomorphism map is defined by $\varphi(gStab(a))=g\cdot a$.
\end{prop}
\begin{thm}
    If $O(a)$ is an orbit of a finite group $G$, then $|O(a)|$ divides $|G|$. In other words, 
    \begin{equation*}
        |O(a)|\cdot |Stab(a)|=|G|
    \end{equation*}
\end{thm}
\begin{example}
    There are no transitive actions of $S_3$ on a set of 5 elements. This is because transitive actions induce an isomorphism, and 5 does not divide 6.
\end{example}
\begin{prop}
    Suppose a group $G$ acts on a set $A$, and let $a\in A, g\in G$. Let $b=g\cdot a$, then 
    \begin{equation*}
        Stab(b)=gStab(a)g^{-1}
    \end{equation*}
\end{prop}
Now if $Stab(a)$ is normal, then stabilizers for all $g$ is the same.



\section{Chapter III: Rings and Modules}
In a ring, we have $0\times r=0$.
\begin{defn}[integral domain]
    A nonzero commutative ring $R$ is an integral domain if for all $a,b\in R$, such that $ab=0$, we have either $a=0$ or $b=0$.
\end{defn}
\begin{warn}
    $\Z$ is an integral domain, a UFD, a PID, even an Euclidean domain.
\end{warn}
\begin{defn}[zero divisor]
    $a$ is a zero divisor if there exists nonzero $v$ such that $av=0$.
\end{defn}
\begin{prop}
    In a commutative ring, $a$ is not a zero divisor if and only if multiplication by $a$ is injective from $R\to R$.
\end{prop}
\begin{prop}
    Let $R$ be a commutative ring, then $u$ is a unit if and only if multiplication by $u$ is surjective from $R$ to $R$, if and only if it is injective, i.e. $u$ is not a zero divisor.
\end{prop}
\begin{defn}[field]
    A field is a nonzero commutative ring $R$ in which every nonzero element is a unit.
\end{defn}
\begin{thm}
    Finite integral domains are fields.
\end{thm}
\begin{defn}[polynomial rings]
    Let $R$ be a ring, a polynomial ring $R[x]$ is a collection of polynomials with operations:
    $f(x)=\sum a_ix^i$, and $g(x)=\sum b_ix^i$ as
    \begin{equation*}
        f(x)+g(x)=\sum(a_i+b_i)x^i
    \end{equation*}
    \begin{equation*}
        f(x)\cdot g(x)=\sum_{k}\sum_{i+j=k}a_ib_jx^{i+j}
    \end{equation*}
\end{defn}
\begin{defn}[ring homomorphism]
    Let $\varphi:R\to S$ be a ring homomorphism, then for all $a,b\in R$, 
    \begin{equation*}
        \varphi(a+b)=\varphi(a)+\varphi(b)
    \end{equation*}
    \begin{equation*}
        \varphi(ab)=\varphi(a)\varphi(b)
    \end{equation*}
    and $\varphi(1_R)=1_S$.
\end{defn}
\begin{prop}
    \begin{equation*}
        End(\Z)\cong \Z
    \end{equation*}
    defined by $\varphi(f)=f(1)$.
\end{prop}
\begin{prop}
    Let $R$ be a ring, then the function $r\mapsto\lambda_r$ is an injective ring homomorphism $R\to End(R)$, where $\lambda_r(a)=ra$.
\end{prop}
\subsection{Ideals and quotient rings}
\begin{defn}[ideal]
    Let $R$ be a commutative ring, a subgroup $I$ (under addition) is an ideal of $R$ if for all $r\in R$, and $rI\subset I$.
\end{defn}
Remark: the only ideal of a ring $R$ containing 1 is $R$ itself.
\begin{defn}[kernel of a ring homomorphism]
    Let $\varphi:R\to S$ be a ring homomorphism, then $\ker\varphi=\{r\in R: \varphi(r)=0\}$. Note that $\ker\varphi$ is an ideal of $R$.
\end{defn}
\begin{defn}[quotient ring]
    Let $I$ be an ideal, hence a normal subgroup of $R$ under addition, and $R/I$ is a ring, with the following operations:
    \begin{equation*}
        (a+I)+(b+I)=(a+b)+I
    \end{equation*}
    \begin{equation*}
        (a+I)(b+I)=(ab)+I
    \end{equation*}
\end{defn}
For a ring $R$, let $f:\Z\to R$ be the unique ring homomorphism, defined by $a\mapsto a\cdot 1_R$. Then $\ker f=n\Z$.
\begin{defn}[characteristic]
    The characteristic of a ring $R$ is the smallest $n$ such that $1\cdot n=0$. In other words, $char(R)$ is the order of 1.

    Alternatively, the characteristic of $R$ is the above $n$.
\end{defn}
\begin{thm}[first homomorphism theorem]
    Let $I$ be an ideal of a commutative ring $R$, then for every ring homomorphism $\varphi: R\to S$ such that $I\subset\ker\varphi$, there exists a unique homomorphism $\tilde{\varphi}$ such that 
    \begin{equation*}
        \varphi=\tilde{\varphi}\circ\pi
    \end{equation*}
    where $\tilde{\varphi}$ is defined by $\tilde{\varphi}(r+I)=\varphi(r)$.
\end{thm}
Just like the group homomorphism, then subgroups $J$ of $R$ containing $I$ can be identified with subgroups of $R/I$, defined by $u(J)=J/I$.
\begin{thm}[fraction holds]
    Let $I$ be an ideal of a ring $R$, and $J$ be an ideal containing $I$, then $J/I$ is an ideal of $R/I$, and 
    \begin{equation*}
        \frac{R/I}{J/I}\cong R/J
    \end{equation*}
\end{thm}
\begin{defn}[principal ideal]
    A principal ideal is an ideal of $R$ such that it is generated by one element $aR$, denoted by $(a)$.
\end{defn}
\begin{prop}
    The sum of ideals is an ideal. (the sum can be arbitrary). If $I_1, \dots, I_n$ are ideals, then $\sum_{i}I_j$ is also an ideal. (like how sum of subspaces of a vector space is a subspace).

    In particular, the sum of principal ideals is also an ideal, $I=\sum_i(a_i)$, where elements of this ideal are of the form $r_1a_1+\dots+r_na_n$. 
\end{prop}
\begin{defn}[finitely generated]
    If a ring can be written as 
    \begin{equation*}
        R=\sum_{i=1}^n(a_i)
    \end{equation*}
    then we say this ring is finitely generated.
\end{defn}
\begin{defn}[Noetherian]
    A ring $R$ is Noetherian if every ideal $I$ of $R$ is finitely generated.
\end{defn}
\begin{defn}[PID]
    An integral domain $R$ is a PID if every ideal of $R$ is principal.
\end{defn}
\begin{example}
    $\Z, k[x]$ for field $k$ are PID. $\Z[x]$ is not a PID, since $(2,x)$ is not a principal ideal.
\end{example}




\chapter{Fields and Galois theory}
\begin{defn}
    
\end{defn}