\chapter{Topology}

\section{12, 13, 14, 15, 16}
\begin{defn}[topology]
    A topology on a set $X$ is a collection $\mathcal{T}$ of subsets of $X$ such that 
    \begin{enumerate}
        \item $X,\emptyset\in\mathcal{T}$.
        \item $\bigcup_{\alpha\in A}U_\alpha$ is in $\mathcal{T}$.
        \item For any finite intersection $\bigcap_{i=1}^NU_i\in\mathcal{T}$.
    \end{enumerate}
    Any set belonging to $\mathcal{T}$ is called an open set.
\end{defn}

If $X$ is any set, with $\mathcal{T}$ all subsets of $X$, then $\mathcal{T}$ is called the discrete topology. If $\mathcal{T}$ only contains $X, \emptyset$, then it is the indiscrete topology. One can check that the topology defined by 
\begin{equation*}
    \mathcal{T}=\{ U\subset X: U^c \text{ is countable or is all of $X$} \}
\end{equation*}
is a topology on $X$. Moreover, if $\mathcal{T}, \mathcal{T}'$ are two topologies on $X$, and $\mathcal{T}\subset\mathcal{T}'$, then $\mathcal{T}'$ is called finer than $\mathcal{T}$. Now we recall the basis for topologies.
\begin{defn}[basis]
    A basis is a collection $\mathcal{B}$ of subsets of $X$, such that 
    \begin{enumerate}
        \item For each $x\in X$, there exists a $B\in\mathcal{B}$ such that $x\in B$.
        \item If $x$ belongs to the intersection of two basis elements $B_1, B_2$, then there exists $B_3\in\mathcal{B}$ such that $x\in B_3\in B_1\cap B_2$. 
    \end{enumerate}
    The topology generated by $\mathcal{B}$ is defined such that: $U$ is an open set if for every $x\in U$, there exists $B\in\mathcal{B}$ such that $x\in B\subset U$.
\end{defn}
An equivalent condition for 1 is such that $\bigcup B=X$, and an equivalent condition for 2 is such that $B_1\cap B_2=\bigcup_{\alpha} B_\alpha, \text{ with } B_\alpha\in\mathcal{B}$. And it is easy to show that the toplogy $\mathcal{T}$ generated by $\mathcal{B}$ is indeed a basis. There is an equivalent way to get a topolgy using the basis: $\mathcal{T}$ generated by a basis $\mathcal{B}$ can also be defined by taking all arbitrary unions of the basis elements, and it is easy to show that these definitions are equivalent. Next we remember how to go from a topology to a basis.
\begin{lem}
Let $(X,\mathcal{T})$ be a topological space, and $\mathcal{C}$ is a collection of open sets $X$ such that for each open set $U$, with $x\in U$, there exists $C\in\mathcal{C}$ such that $x\in C\subset U$. Then $\mathcal{C}$ is a basis.
\end{lem}
\begin{proof}
    It is easy to show that $C$ is a basis. Note that we also have to show that the topology $\mathcal{T}'$ generated by $\mathcal{C}$ is the same as $\mathcal{T}$. Let $U\in\mathcal{T}$, then for all $x\in U$, there exists $C$ such that $x\in C\subset U$, which is in $\mathcal{T}'$ by definition. Let's assume $O\in\mathcal{T}'$, then $O=\bigcup_\alpha C_\alpha$, note that each $C_\alpha\in\mathcal{T}$, hence $O$ is an open set in $\mathcal{T}$ as well.
\end{proof}
Now we state a lemma to check whether one toplogy is finer than the other using basis.
\begin{lem}
    Let $\mathcal{B}, \mathcal{B}'$ are bases for topologies $\mathcal{T}, \mathcal{T}'$, then the following are equivalent.
    \begin{enumerate}
        \item $\mathcal{T}'$ is finer than $\mathcal{T}$.
        \item For each $x\in X$, and each $B\in\mathcal{B}$ such that $x\in B$, there exists $B'\in\mathcal{B}'$ such that $x\in B\subset B'$.
    \end{enumerate}
\end{lem}
Now one can see that in $\R^2$, the toplogy generated by open balls and open rectangles are the same. The \textbf{standard} topology on $\R$ is defined by 
\begin{equation*}
    U=\{x: a<x<b, a,b\in\R\}
\end{equation*}
and there are other topologies such as the lower-limit topology for $\R$  defined by $\{x: a\leq x< b\}$, or the $K$-topology
\begin{equation*}
    \{x: a<x<b\}-\left\{\frac{1}{n}: n\in\mathbb{Z}_+\right\}
\end{equation*}
One can show that lower limit topology and the $K$-topology are strictly finer than the standard topology. Now we define a subbasis.
\begin{defn}[subbasis]
    A subbasis $\mathcal{S}$ for a topology on $X$ is a collection of $U_\alpha\in\mathcal{T}$, such that $\bigcup_\alpha U_\alpha=X$. The topology generated by the subbasis is the union of all finite intersections of $U_\alpha$'s.
\end{defn}
Note that to show that the topology generated by a subbasis is indeed a topology, it suffices to show that the set of finite intersections of $B\in\mathcal{S}$ is indeed a basis. There are some important consequences.
\begin{enumerate}
    \item Let $\{\mathcal{T}_\alpha\}$ be an arbitrary collection of topologies, then $\bigcap_\alpha\mathcal{T}_\alpha$ is also a topology, but $\bigcup_\alpha\mathcal{T}_\alpha$ is not necessarily a topology.
    \item et $\{\mathcal{T}_\alpha\}$ be an arbitrary collection of topologies, then there exists a unique smallest topology that contains $\bigcup_\alpha T_\alpha$.
    \item The topology generated by a basis $\mathcal{B}$ is equal to the topology of all topologies containing $\mathcal{B}$
\end{enumerate}

\textbf{insert notes}
\begin{prob}
    Show that the directionary order topology on $\R\times\R$ is the same as the product topology $\R_d\times\R$, where $\R_d$ is the discrete topology. 
\end{prob}
\begin{proof}
    The directionary order topology on $\R\times\R$ is generated by the basis $\mathcal{B}$, 
    \begin{equation*}
        \mathcal{B}=\{a\times (b,c): a,b,c\in\R, b<c\}
    \end{equation*}
    And the basis $\mathcal{B}'$ for the product topology $\R_d\times\R$ is
    \begin{equation*}
        \mathcal{B}'=\{(U,V): U \text{ is open in $\R_d$ }, V\text{ is open in $\R$ }\}
    \end{equation*}
    We note that every basis element in $\mathcal{B}$ is also a basis element in $\mathcal{B}'$. Conversely, note that all basis elements $\mathcal{B}'$ are arbitrary unions of elements in $\mathcal{B}$, hence all the basis elements in $\mathcal{B}'$ belong to the  order topology on $\R\times\R$. Because the order topology on $\R\times\R$, and it is contained in $\R_d\times\R$, because there exists a unique smallest topology containing the basis, we have the two topologies are equal.
\end{proof}




\textbf{insert notes here}
\begin{defn}[$T_1$ space]
    A topological space is said to be $T_1$, if for all $x,y$ distinct, there exists open sets $U, V$ such that $x\in U, y\in V$. 
\end{defn}
\begin{lem}
    A space is $T_1$ if and only if setes of finite points $\{x_1, \ldots, x_n\}$ are closed.
\end{lem}
\begin{proof}
    Assume that sets of finite points are closed, then for any $x, y$ distinct, $\{x\}, \{y\}$ are both closed. Hence $X\setminus\{x\}, X\setminus\{y\}$ are both open, and they do not contain the other points. Hence the space is $T_1$ by definition.
    Conversely, assume that a space is $T_1$, then it suffices to show that $\{x\}$ is closed for arbitrary $x$, i.e., it contains all of its limit points. If $y$ is distinct from $x$, then there exists an open set that doesn't intersect $\{x\}$ by the space being $T_1$, hence $y$ is not a limit point of $\{x\}$.
\end{proof}
The next lemma illustrates why we care about Hausdorff spaces.
\begin{lem}
    A sequence in a Hausdorff space, if converges, converges to a unique limit point.
\end{lem}



\begin{prob}
    Let $A,B$ be subsets of $X$, then $\overline{A}\cup\overline{B}=\overline{A\cup B}$.
\end{prob}
\begin{proof}
    For any closed set containing $A$ and a closed set containing $B$, their union also contains $A\cup B$, hence $\overline{A\cup B}\subset\overline{A}\cup\overline{B}$. The other side is checked by different cases, and definition of a limit point.
\end{proof}
\begin{prob}
    We have $\bigcup_\alpha \overline{A_\alpha}\subset\overline{\bigcup_\alpha A_\alpha}$, and this is a strict inclusion.
\end{prob}
\begin{proof}
    The inclusion can be shown using the same argument. Consider the sets 
    \begin{equation*}
        A_n=\left\{x: \frac{1}{n}<x\leq 1, n\in\mathbb{N}\right\}
    \end{equation*}
    Hence we have sets that look like $\left(\frac{1}{n}, 1\right]\phantom{)}$, and 
    \begin{equation*}
        \bigcup_n\overline{A_n}=(0,1]\phantom{)}, \overline{\bigcup_nA_n}=[0,1]
    \end{equation*}
\end{proof}
\begin{prob}
    Let $A=(0,1), B=(1,2)$, then $\overline{A\cap B}=\emptyset$, and $\overline{A}\cap\overline{B}=1$. So I think $\overline{A\cap B}\subset\overline{A}\cap\overline{B}$. 
\end{prob}
\begin{proof}
    For any closed set that contains $A$, and intersected with 
\end{proof}

\begin{prob}
    $X$ is Hausdorff if and only if the diagnoal $\Delta=\{x\times x\in X\times X\}$ is closed in $X\times X$.
\end{prob}
\begin{proof}
    $X$ is Hausdorff iff for all $x,y$ distinct, you can find $U,V$ open such that $x\in U, y\in V$, such that $U\cap V=\emptyset$. Then for all $(y,z)\not\in\Delta$, $(y,z)\in U\times V$, and $U\times V\cap\Delta=\emptyset$. (Assume there exists $(y,z)\in U\times V\cap \Delta$, then $(y,z)=(y,y)$, and $y\in U, y\in V$, but this is impossible since $U\cap V=\emptyset$.) Hence by definition, $(y,z)$ is not a limit point of $\Delta$, this is if and only if $\Delta$ is closed.
\end{proof}
\begin{prob}
    For the finite complement topology on $\R$, the sequence $\left\{\frac{1}{n}:n\in\mathbb{N}\right\}$ converges to every single point in $\R$.
\end{prob}
\begin{proof}
    
\end{proof}

\begin{prob}
    If $U$ is open, then is it true that $U=Int(\overline{U})$?
\end{prob}
\begin{proof}
    No, consider $U=(0,1)\cup(1,2)$
\end{proof}