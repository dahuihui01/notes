\chapter{Irreducibility and Factorization in ID}


\begin{thm}[Hilbert basis theorem]
    Let $R$ be a Noetherian ring, then $R[x]$ is also a Noetherian ring.

    This implies that if $I$ is an ideal of $R[x]$, then $R[x]/I$ is also Noetherian (the ideals of $R[x]/I$ can be identified with ideals of $R[x]$ containing $I$.)
\end{thm}

\section{Prime and irredubile elements}
Let $R$ be a commutative ring for everything below.
\begin{defn}
    Let $a,b,c\in R$, we say $a\vert b$ if there exists $c$ such that $a=bc$, i.e. $a\in (b)$. Moreover, $a,b$ are associates if $(a)=(b)$, i.e., if $a\vert b, or b\vert a$.
\end{defn}
\begin{prop}
    In an integral domain $R$, $a,b$ are associates if and only if $a=bc$ for some unit $c$.
\end{prop}

\begin{defn}[prime, irreducible elements]
    Let $R$ be an integral domain.
    An element $a\in R$ is prime if and only if $(a)$ is prime. In other words, if $a$ is not a unit and $a\vert bc$, then $a\vert b$, or $a\vert c$.

    An element $a$ is irreducible if and only if $a$ is not a unit and $a=bc$, then either $b$ or $c$ is a unit.
\end{defn}


\begin{prop}
    In a PID, prime ideals are maximal ideals.

    \begin{proof}
        You can proceed directly. Alternatively, let $(a)$ be a prime ideal, then $a$ is irreducible, if $(a)\subset (b)$, then $b\vert a$, hence $a=be$ for some $e$. $a$ is irreducible implies that either $b$ or $e$ is a unit.

        Alternatively, $(a)$ prime, $a$ is irreducible, then $(a)$ is maximal in principal ideals, i.e., in all ideals (in a PID), then $(a)$ is maximal.
    \end{proof}
\end{prop}

\begin{prop}
    \begin{enumerate}
        \item $\Z[x]$ is a UFD, but no PID.
        \item In a PID, the greatest common divisor of $a,b$ can be rewritten as a linear combination of them.
    \end{enumerate}
\end{prop}


\begin{thm}[Simple facts you should know]
    Let $a,b\in R$,
    \begin{enumerate}
        \item if $a=br$, for some $r\in R$, then $(a)\subset(b)$
        \item if $a=be$ for some unit $e$, then $(a)=(b)$.
    \end{enumerate}
\end{thm}


\begin{thm}[Gauss's lemma]
    Let $f,g\in R[x]$, then $cont(fg)=cont(f)cont(g)$.
\end{thm}

\newpage
\section{Exercises}
\begin{exer}[1.12]
    Let $R$ be an integral domain, then $a$ is irreducible if and only if $(a)$ is maximal among the proper principal ideals.

    \begin{proof}
        $a$ is irreducible if and only if $a=bc$ implies either $b$ or $c$ is a unit. Let $(a)=(b)$, if $b$ is a unit, then $(b)=(1)$; if $(c)$ is a unit, then $(a)=(b)$ (by Lem 1.5, where $(a)=(b)$ if and only if $a=bc$ for some unit $c$).
    \end{proof}
\end{exer}
\begin{exer}[1.13]
    Showing that in $\Z$, $a$ is irreducible if and only if $a$ is nonzero and prime.

    \begin{proof}
        In integral domains, nonzero prime elements are always irreducible. Now assume $a$ is irreducible, then $(a)$ is maximal since $\Z$ is a PID, and maximal ideals are always prime, hence $(a)$ is prime. 
    \end{proof}
\end{exer}


\begin{example}
    Show that $x^2+y^2-1$ is not irreducible in $\C[x,y]$.

    We can view $\C[x,y]=\C[x][y]$, then $(x-1)$ is prime in $\C[x]$, and it follows the Eisenstein's criterion that the polynomial is indeed irreducible.
\end{example}




