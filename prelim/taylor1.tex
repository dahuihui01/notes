\chapter{Taylor 1: Complex Calculus}
We will start with the basics.
\begin{defn}[convergence]
    $\{z_n\}\subset\C$ converges to $z$ if and only if $|z_n-z|\to 0$ as $n\to\infty$. ($z_n=x_n+iy_n$ this is iff $x_n\to x, y_n\to y$).
\end{defn}
We note that every complex Cauchy sequence converges because every real Cauchy sequence converges.

\begin{defn}[absolutely convergent]
    We say a series is absolutely convergent $\sum_{k=0}^\infty$ if 
    \begin{equation*}
        \sum_{k=0}^\infty |a_k|<\infty
    \end{equation*}
\end{defn}
\begin{prop}
    If $\sum_{k=0}^{\infty} a_kz^k$ converges at $z_1\neq 0$, then either this series converges absolutely for all $z\in\C$, or there exists $R$ such that it is absolutely convergent for all $|z|<R$, and divergent for $|z|>R$ (R is called the radius of convergence).
\end{prop}
\begin{proof}
    We know $\sum a_kz_1^k$ converges hence $|a_kz_1^k|\leq C$ for all $k$. Then for $|z|\leq r|z_1|$ for some $r<1$, we have $\sum|a_kz^k|\leq C\sum r^k<\infty$. This shows that the series converges absolutely for $|z|<R$ for some $R$.
\end{proof}




\begin{example}
    $f(z)=z^2$ is maps the upper half plane $\mathbb{H}$ surjectively onto $\C\setminus \R^{-}$, where $\R=(-\infty, 0])$.
\end{example}


\begin{thm}[Cauchy's integral theorem]
    Let $f\in C^1(\overline{\Omega})$ be holomorphic on $\Omega$, then 
    \begin{equation*}
        \int_{\partial\Omega}f(z)dz=0
    \end{equation*}
\end{thm}
\begin{proof}
    This follows from Green's theorem.
    \begin{equation*}
        \int_{\partial\Omega}fdz=\int_{\partial\Omega} fdx+ifdy=\int\int i\frac{\partial f}{\partial x}-\frac{\partial f}{\partial y}dxdy=0
    \end{equation*}
    the RHS is 0 by Cauchy-Riemann equations for $f$ being holomorhpic.
\end{proof}
\begin{thm}[Cauchy's integral formula]
    Let $f\in C^1(\overline{\Omega})$, then for $z\in\Omega$, then 
    \begin{equation*}
        f(z)=\frac{1}{2\pi i}\int_{\partial\Omega}\frac{f(\zeta)}{(\zeta-z)}d\zeta
    \end{equation*}
\end{thm}
\begin{proof}
    This follows from shrinking $\int_{\partial\Omega}$ to $\int_{\partial D_r}$ with $D_r$ centered at $z$, then parametrizing.
\end{proof}
\begin{prop}[Mean value property]
    Let $z_0\in\Omega$, then
    \begin{equation*}
        f(z_0)=\frac{1}{2\pi}\int_0^{2\pi}f(z_0+re^{i\theta})d\theta
    \end{equation*}
    whenever the closure of $D_r\subset\Omega$.
\end{prop}
Note that one can keep differentiate the Cauchy integral formula, we get 
\begin{equation*}
    f'(z)=\frac{1}{2\pi i}\int_{\partial\Omega}\frac{f(\zeta)}{(\zeta-z)^2}d\zeta
\end{equation*}
Moreover, we have
\begin{equation*}
    f^{(n)}(z)=\frac{n!}{2\pi i}\int_{\partial\Omega}\frac{f(\zeta)}{(\zeta-z)^{n+1}}d\zeta
\end{equation*}
Now we reach a strong conclusion where every holomorphic $f$ can be written as a power series about any point $z\in\Omega$.
\begin{prop}
    If $f\in C^1(\overline{\Omega})$ is holomorhpic, then $z\in D_r(z_0)\subset\Omega$, then $f(z)$ has a convergent power series 
    \begin{equation*}
        f(z)=\sum_{n=0}^\infty a_n(z-z_0)^n=\sum_{n=0}^\infty \frac{f^{(n)}(z_0)}{n!}(z-z_0)^n
    \end{equation*}
\end{prop}


