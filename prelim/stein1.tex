\chapter{Stein 1: preliminaries to complex analysis}

We start with simple formulas for complex numbers.
\begin{equation*}
    Re(z)=\frac{z+\zbar}{2}, Im(z)=\frac{z-\zbar}{2i}, |z|^2=z\zbar, |z^n|=|z|^n
\end{equation*}
Moreover,
\begin{equation*}
    e^{i\theta}=\cos(\theta)+i\sin(\theta)
\end{equation*}
We also know that $\C$ is complete and a set $\Omega\subset\C$ is compact if and only if it it closed and bounded. In $\C$, a set is compact if and only if for any open covering of $\Omega$, there exists a finite subcover.
\begin{defn}[connected]
    A set $\Omega$ is connected if there doesn't exist open sets $U,V$ such that $\Omega=U\bigcup V$.( A connected open set in $\C$ is called a region.)

    Equivalently, $\Omega$ is connected if for any $p,q\in\Omega$, there exists a path $\gamma$ connecting $p,q$ such that $\gamma\subset\Omega$.
\end{defn}
Next we discuss holomorphic functions.
\begin{prop}[Cauchy-Riemann Equations]
    Let $f$ be holomorphic in $\Omega$, then for $z\in\Omega$, we have 
    \begin{equation*}
        \frac{\partial f}{\partial x}+i\frac{\partial f}{\partial y}=0
    \end{equation*}
    Equivalently, we have 
    \begin{equation*}
        \frac{\partial u}{\partial x}=\frac{\partial v}{\partial y}, \frac{\partial u}{\partial y}=-\frac{\partial v}{\partial x}
    \end{equation*}
\end{prop}
If we further define differentiate with respect to $z$ in terms of $x,y$, we have 
\begin{defn}[derivative wrt $z$]
    We define the operators as follows:
    \begin{equation*}
            \frac{\partial f}{\partial z}=\frac{1}{2}\left(\frac{\partial f}{\partial x}+\frac{1}{i}\frac{\partial f}{\partial y}\right)
    \end{equation*}

    \begin{equation*}
        \frac{\partial f}{\partial\zbar}=\frac{1}{2}\left(\frac{\partial f}{\partial x}-\frac{1}{i}\frac{\partial f}{\partial y}\right)
    \end{equation*}
\end{defn}
Then we immediately have the following:
\begin{prop}
    Let $f$ be holomorphic on $\Omega$, then we have for $z\in\Omega$,
    \begin{equation*}
        \frac{\partial f}{\partial\zbar}=0, \frac{\partial f}{\partial z}(z)=2\frac{\partial u}{\partial z}(z)
    \end{equation*}
\end{prop}
So it's clear that holomorphic implies Cauchy-Riemann equations, but it doesn't go exactly conversely. In fact, if $f$ satisfies the CR equations and if $u,v$ are continuously differentiable, then $f$ is holomorphic.

\begin{defn}[radius of convergence]
    For a power series $\sum_{n=0}^\infty a_nz^n$, the radius of convergence $R$ is as follows:
    \begin{equation*}
        1/R=\limsup_{n\to\infty}|a_n|^{1/n}
    \end{equation*}
\end{defn}
We note that $R=\infty$ for $e^z$, and $R=1$ for geometric functions.
\begin{prop}
    Let $f(z)=\sum_{n=0}^\infty a_nz^n$, then $f'(z)=\sum_{n=1}^\infty na_nz^{n-1}$, $f'$ has the same radius of convergence as $f$.
\end{prop}

\chapter{Stein 2: Cauchy's integral theorem and applications}
We state Goursat's theorem.
\begin{thm}[Goursat]
    If $\Omega$ is a region in $\C$, and $T\subset\Omega$ a triangle whose interior is also contained in $\Omega$, if $f$ is holomorphic, then
    \begin{equation*}
        \int_Tf(z)dz=0
    \end{equation*}
\end{thm}
This implies for any rectangle $R$, if $f$ is holomorphic, then $\int_Rf(z)dz=0$. Before we state the Cauchy's integral theorem, we state the following.
\begin{thm}[primitive exists]
    Let $f$ be holomorphic on an open disk $D$, then $f$ has a primitive on this disk.
\end{thm}
\begin{proof}
    You define the primitive as $F(z)=\int_\gamma f(w)dw$, where $\gamma$ is the line connecting the origin and $z$.
\end{proof}
This give Cauchy's theorem.
\begin{thm}[Cauchy's theorem]
    If $f$ is holomorphic in a disk, then for any $\gamma\subset D$, we have 
    \begin{equation*}
        \int_\gamma f(z)dz=0
    \end{equation*}
\end{thm}



\chapter{Stein III: Meromorphic functions}
\begin{defn}[meromorphic function]
    $f$ is meromorphic if there exists a set of points $\{z_1, z_2, \ldots\}$ that has no limit point in $\Omega$ such that $f$ is holomorphic in the punctured disk and has poles at these points $\{z_1, z_2, \ldots\}$.
\end{defn}




