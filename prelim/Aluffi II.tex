\chapter{Aluffi II}

\begin{prop}
    Let $|g|$ be the order of an element $g\in G$, then $|g|\leq |G|$.
\end{prop}
\begin{proof}
    It's trivial if $|G|=\infty$. For $|G|<\infty$, consider the following $|G|+1$ terms,
    \begin{equation*}
        g^0, g, g^2, \ldots, g^{|G|}
    \end{equation*}
    Note all can be distinct, hence there exists $i<j$ such that $g^{i}=g^j$, i.e., $g^{j-i}=e$. This implies that $|g|\leq |G|$.
\end{proof}
\begin{example}
    There exists $g, h$ with finite orders each, and $gh$ has infinite order. For example, the group generated by relations:
    \begin{equation*}
        \langle r,s: r^2=s^2=e\rangle
    \end{equation*}
    The term $rs$ has infinite order. Note there exists geometric examples with matrix groups.
\end{example}
\begin{prop} The following is a list of statements/propositions that you should know.
\begin{enumerate}
    \item If $g^N=e$, then $|g|$ divides $N$.
    \item If $g^m=N$, then $|g|=\frac{lcm(|g|, m)}{m}$
    \item If $gh=hg$, then $|gh|$ divides $lcm(|g|, |h|)$
\end{enumerate}
\end{prop}


\begin{defn}[Dihedrayl group]
    The group $D_{2n}$ is the group of rotations and reflections of $n$-polygons. (There are $2n$ elements in this group).
\end{defn}



