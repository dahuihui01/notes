\chapter{Aluffi II}

\begin{prop}
    Let $|g|$ be the order of an element $g\in G$, then $|g|\leq |G|$.
\end{prop}
\begin{proof}
    It's trivial if $|G|=\infty$. For $|G|<\infty$, consider the following $|G|+1$ terms,
    \begin{equation*}
        g^0, g, g^2, \ldots, g^{|G|}
    \end{equation*}
    Note all can be distinct, hence there exists $i<j$ such that $g^{i}=g^j$, i.e., $g^{j-i}=e$. This implies that $|g|\leq |G|$.
\end{proof}
\begin{example}
    There exists $g, h$ with finite orders each, and $gh$ has infinite order. For example, the group generated by relations:
    \begin{equation*}
        \langle r,s: r^2=s^2=e\rangle
    \end{equation*}
    The term $rs$ has infinite order. Note there exists geometric examples with matrix groups.
\end{example}
\begin{prop} The following is a list of statements/propositions that you should know.
\begin{enumerate}
    \item If $g^N=e$, then $|g|$ divides $N$.
    \item If $g^m=N$, then $|g|=\frac{lcm(|g|, m)}{m}$
    \item If $gh=hg$, then $|gh|$ divides $lcm(|g|, |h|)$
    \item We have $|g^m|=\frac{|g|}{\gcd(m, |g|)}$
\end{enumerate}
\end{prop}

\begin{defn}[Dihedrayl group]
    The group $D_{2n}$ 
    is the group of rotations and reflections of $n$-polygons. (There are $2n$ elements in this group).
\end{defn}

We note that $D_6$ and $S_3$ are isomorphic. They are both generated by $x,y$ such that $x^2=e, y^3=e$.

We first note that $\Z/n\Z$ is a cyclic group of order $n$,and $[1]$ is a generator.
\begin{prop}
    $[m]$ is a generator of $\Z/n\Z$ if and only if $\gcd(m,n)=1$.
\end{prop}
\begin{proof}
    $[m]$ is a generator if and only if $[m]$ has order $n$, which by the above proposition 4, we have $[m]=m[1]$, hence $\gcd(m,n)=1$.
\end{proof}
\begin{prop}
    ${(\Z/n\Z)}^*=\{[m]: \gcd(m,n)=1\}$ is a group under the multiplication defined as 
    \begin{equation*}
        [m]\cdot[n]=[mn]
    \end{equation*}
\end{prop}
\begin{proof}
    We will only check the existence of inverse. For each $[m]$, we know $\gcd(m,n)=1$, hence $[m]$ is a generator of the additive group $\Z/n\Z$, hence we know there exists some integer $q$ such that $q[m]=[1]$, this implies that $[q][m]=[1]$, hence inverse.
\end{proof}

\begin{example}
    $G\cong H\times G$ does not mean that $H$ is the trivial group. For example, consider $G$ as the infinite product $G=\Z\times\Z\times\Z\ldots$, and take $H=\Z$.

    Note that if we take $G=\Z$, then $H$ is the trivial group. Proof: consider where $\varphi(1)$ gets sent to. No matter where it is sent to, there are elements not mapped by $\varphi$.
\end{example}

\begin{example}
    Let $\varphi: S_3\to \Z/2\Z$ be a homomorphism, then $\varphi$ sends elements of order 3 to $0$. One can define the commutator subgroup. Let $G$ be a group, then the commutator subgroup $[A,G]$ is generated by
    \begin{equation*}
        \langle ghg^{-1}h^{1-}, g,h\in G\rangle
    \end{equation*}
    then 
    \begin{prop}
        Let $\varphi:G\to H$ be a homomorphism, then $\varphi([A,G])\subset \text{ker}(\varphi)$.
    \end{prop}
    Moreover, one can show that any homomorphism $\varphi: S_3\to\Z/2\Z$ should send all elements of order 2 to $0$, or all to $1$. One can either use the determinant map $\det: S_3\to\{\pm 1\}$, or consider that 
    \begin{equation*}
        \varphi((12)(23)(12))=\varphi(13)=\varphi(12)\varphi(23)\varphi(12)
    \end{equation*}
    then $\varphi(13)=\varphi(23)$. 
\end{example}

\begin{example}
    Any homomorphism $\varphi: \Q\to\Q$ is linear, i.e., $\varphi(p)=qp$ for some $q\in\Q$. We note that $\varphi(p)=p\varphi(1)$, and $\varphi\left(\frac{1}{p}\right)=\frac{1}{q}\varphi(1)$. Hence we have $\varphi\left(\frac{p}{q}\right)=\frac{p}{q}\varphi(1)$. By the same argument, any homomorphism from $\Z\to\Z$ is also linear.
\end{example}

We note that isomorphisms preserve the following things:
\begin{prop}
    If $G,H$ are isomorphic, then 
    \begin{enumerate}
        \item If $G$ is abelian, then $H$ is abelian.
        \item The order of $g$=the order of $\varphi(g)$
    \end{enumerate}
    If it's just a homomorphism, then the order of $\varphi(g)$ divides the order of $g$. For example, there is no nontrivial homomorphism from $\Z/4\Z$ to $\Z/7\Z$, because $\varphi(g)$'s order would have to divide 4 and 7.
\end{prop}
\begin{example}
    $(\R, \cdot)$ and $(\C, \cdot)$ are not isomorphic. There are no finite order elements in $\R$, but $i$ has order 4 in $\C$.
\end{example}

\begin{example}
    $Hom(\Z/n\Z, \Z/n\Z)$ is a ring, with the gruop being under addition of maps. $Aut(\Z/n\Z)$ is a group under composition of maps.
\end{example}

\begin{prop}
    Let $H$ be a subset of a group $G$, then $H$ is a subgroup if for all $a,b\in H$, $ab^{-1}\in H$.
\end{prop}
Every homomorphism $\varphi:G\to G'$ determines two subgroups naturally, $\ker(G)\subset G, Im(G)\subset G'$. This is because if $H'$ is a subgroup of $G'$, then $\varphi^{-1}(H')$ is a subgroup of $G$. In fact, the image of any subgroup of $G$ is also a subgroup of $G'$.
\begin{defn}[cyclic group]
    A group is cyclic if it is either $\cong\Z$ or $\cong \Z/n\Z$.
\end{defn}
Let's now classify all the subgroups of cyclic groups.
\begin{prop}
    If $H\subset\Z$ is a subgroup, then $H=d\Z$ for some $d\geq 0$. (Proof: let $d$ be the smallest positive integer in $H$). 
    
    If $G\subset \Z/n\Z$ is a subgroup, then $G=\langle [d]_n\rangle$ for some $d$ that divides $n$. Moreover, this exists a bijection between the subgroups of $\Z/n\Z$ with the divisors of $n$.
\end{prop}
\begin{idea}
    This means that all subgroups of cyclic groups are cyclic.
\end{idea}
\begin{example}
    Show that $\Z/12\Z$ has 6 subgroups. Proof: 12 has 6 divisors: 1,2,3,4,6,12.
\end{example}
\begin{prop}
    Let $\varphi:G\to H$ be a surjective homomorphism, then if $G$ is cyclic, $H$ is also cyclic.

    This gives the result that the projection $\pi_n$ allows us to go from a cyclic subgroup of $\Z$ to a cyclic subgroup of $\Z/n\Z$.
\end{prop}
To understand $\varphi:G\to G'$ as a monic morphism means $\varphi$ is injective.
If we consider
\begin{equation*}
    \begin{tikzcd}
        {\ker(\varphi)} & G & {G'}
        \arrow["i", shift left, from=1-1, to=1-2]
        \arrow["e"', shift right, from=1-1, to=1-2]
        \arrow["\varphi", from=1-2, to=1-3]
    \end{tikzcd}
\end{equation*}
Then $\varphi\circ i=\varphi\circ e$, where $e$ is the trivial map and $i$ is the inclusion map, this means that $\ker(\varphi)=\{e\}$.
\begin{prop}
    Let $S\subset G$ be a subset, $S$ generates $G$ if and only if $\pi: F(S)\to G$ is surjective.
\end{prop}
\begin{proof}
    Assume $\pi$ is surjective, then for any $g\in G$, we have $\pi(w)=g$ for some $s\in F(S)$, and $w=s_1^{n_1}\ldots s_n^{n_k}$, hence every $g$ corresponds to that.
\end{proof}
\begin{thm}
    Let $\varphi: G\to G'$ be a homomorphism, then there exists a bijection $\tilde{\varphi}$
    \begin{equation*}
        \tilde{\varphi}: G/\ker(\varphi)\to Im(\varphi)
    \end{equation*}
\end{thm}
\begin{proof}
    Let $\tilde{\varphi}([a]):=\varphi(a)$. Then $Im(\tilde{\varphi})$ is a subgroup of $G'$. We just need to show that $\tilde{\varphi}$ is injective. 
    \begin{equation*}
        \tilde{\varphi}([a])=\tilde{\varphi}([b])\Rightarrow \varphi(a)=\varphi(b)\Rightarrow \varphi(ab^{-1})=e\Rightarrow ab^{-1}\in\ker(\varphi)\Rightarrow [a]=[b]
    \end{equation*}
    Note that the last argument is by $H=\ker(\varphi)$ is normal, hence we want to show $Ha=Hb$, and $ab^{-1}\in H$ means $Hab^{-1}\subset H$, i.e., $Ha\subset Hb$, and vice versa.
\end{proof}
\begin{example}
    The commutator subgroup of a group $G$ is defined roughly to capture the group that is ``not commutative with other elements.'' In other words, they should, in some sense, be complimentary to the center. The commutator subgroup $[G,G]$ of $G$ is the subgroup \textbf{generated by}
    \begin{equation*}
        \langle ghg^{-1}h^{-1}, g,h\in G\rangle 
    \end{equation*}
    \begin{prop}
        The $G/[G,G]$ is abelian. In other words, the quotient group by the commutator subgroup is abelian. (This intuitively makes sense because if we view the ``noncommutative elements'' as the same, then the rest should just be abelian).
    \end{prop}
    \begin{proof}
        We would like to show that $g[G,G]h[G,G]=gh[G,G]=hg[G,G]$. In other words,
        \begin{equation*}
            g^{-1}h^{-1}gh\in [G,G]\Rightarrow g^{-1}h^{-1}gh\subset [G,G]\Rightarrow g^{-1}h^{-1}gh\in [G,G]=[G,G]
        \end{equation*}
        This is because that any coset contained in one coset is equal to the entire coset.
    \end{proof}
    \begin{lem}
        we claim that for any subgroup $H$,
        \begin{equation*}
            gH\subset H\Rightarrow gH=H
        \end{equation*}
        For any $h\in H$, we know that $g^{-1}gH\in H$, and because $gh\in H$, we have that $g^{-1}\in H$, hence we have that $h\in gH$ as well.
    \end{lem}
\end{example}
\begin{warn}
    This fact should be memorized, i.e., if any coset $gH$ is contained in an other coset $g'H$, then they are the same.
\end{warn}

\begin{example}
    Let $H$ be a normal subgroup of $G$, and $K$ be a subgroup of $G$, then $HK=\{hk: h\in H, k\in K\}$ is a normal subgroup.
    
    To show that it is a subgroup, we note that 
    \begin{equation*}
        HK=\pi^{-1}(\pi(K))
    \end{equation*}
    where $\pi: G\to G/H$. In other words, $\pi^{-1}(gH)=gH$.
    \begin{lem}
        For $H$ normal in $G$, the usual projeciton $\pi: G\to G/H$, we have $\pi^{-1}(gH)=gH$.
    \end{lem}
\end{example}
\begin{example}
    Cosets are disjoint. Let $g\in g_1H\cap g_2H$, then $g_1h_1=g_2h_2$, then $g_1=g_2h_2h_1^{-1}$, hence $g_1H\subset g_2H$.
\end{example}
\begin{example}[8.13]
    Let $|G|$ be odd, show that every element is a square.
    
    Method 1: it'd be nice if we can show that $f(g)=g^2$ is an injective homomrophism from $G\to G$, but this is not the case. It need not to be a homomorphism. In fact, it is a homomorphism only when $G$ is abelian. Now consider $\la g\ra$, it is an abelian group, so we can restrict $f$ to all the $\la g\ra$, then we can show that it is injective, hence surjective. 

    Method 2: observe that $g^{|G|}=1, g^{|G|+1}=g$, then $(g^{|G|+1})^2=g$.
\end{example}
\begin{example}[8.25]
    An interesting homomorphism from $G/H\to Aut(G)$, when $H$ is abelian and normal.

    We know that if $H$ is normal, then $f: g\mapsto \gamma_g$ satisfies $\gamma_g(H)\subset H$, and hence an homomorphism would be $g\mapsto\gamma_g\vert_H$. Now if $H$ is abelian, we see that $H\subset\ker(f)$, i.e. there is the homomorphism from $G/H\to Aut(H)$ is well-defined.
    \begin{lem}
        Let $f: G\to G'$, then the following are equivalent:
        \begin{enumerate}
            \item $\overline{f}: G/H\to G'$ is well-defined, i.e., $gH\mapsto f(g)$.
            \item $H\subset\ker(f)$.
        \end{enumerate}
    \end{lem}
\end{example}

\begin{defn}[action on a set]
    An action $\rho$ of $G$ on a set $A$ is a function $\rho: G\times A\to A$ such that 
    \begin{equation*}
        \rho(e,a)=a, \forall a, \rho(gh,a)=\rho(g, \rho(h,a)), \forall g,h, a
    \end{equation*}
    We call an action is transitive if for all $a,b\in A$, there exists $g$ such that
    \begin{equation*}
        b=\rho(g,a)
    \end{equation*}
\end{defn}
We know that left multiplications are faithful and also transitive.
\begin{thm}[Cayley's theorem]
    Every group acts faithfully on some set. 

    Proof: Every group acts faithfully on itself by left multiplication.
\end{thm}
\begin{defn}[orbit, stabilizer]
    The orbit of an element $a\in A$ under the action of $G$ is the set
    \begin{equation*}
        O(a)=\{\rho(g,a): g\in G\}
    \end{equation*}
    The stabilizer subgroup of $G$ of $a\in A$ is
    \begin{equation*}
        Stab(a)=\{g\in G: \rho(g,a)=a\}
    \end{equation*}
\end{defn}
\begin{prop}
    Orbits partition the set $A$.
\end{prop}
\begin{proof}
    We can show this directly: if $\rho(g_1,a)=\rho(g_2, b)$, then $\rho(g',a)\in O(b)$ for ant $g'$.

    Alternatively, we can show that if $c=\rho(g,a)\in O(a)$, then $O(c)\subset O(a)$; then we can show if $c\in O(a)$, then $a\in O(c)$, so $O(a)\subset O(c)$, hence $O(a)=O(c)$, hence any overlap implies equality.
\end{proof}
From this we see that orbits partition $A$, and the induced action on each orbit is transitive. Hence it suffices to consider transitive actions if we want to understand any group actions on a set.

\begin{prop}
    If $G$ acts transitively on a set $A$, then $|A|$ divides $|G|$.
\end{prop}
\begin{proof}
    One can define a bijection between $\varphi: G/H\to A$, where $H=Stab(a)$ for any $a\in A$, by $\varphi: gH\mapsto g\cdot a$. Then by Lagrange's, we know $|G|=|A|\cdot|H|$.
\end{proof}
\begin{example}[9.11]
    Let $G$ be a finite group, and $p$ be the smallest prime such that it divides $|G|$. Let $H$ be a subgroup of $G$ of index $p$, then $H$ is normal.

    The proof relies on showing $\ker(\sigma)=H$, where $\sigma: G\to S_p$ corresponds to an action of $G$ on $G/H$. 
\end{example}


