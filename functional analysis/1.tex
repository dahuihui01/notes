\section{Lecture 1}
Today is a very hot and sunny day, here we go.

\subsection{Course Overview and Logistics}
Some administrative things.
OH are Monday, Fridays 1:45 to 2:45, Wednesdays 12:45-1:45 in Evans 811.

\textbf{Textbook}: an introduction to functional analysis by Conway.
We will be talking about operators on Hilbert spaces, and more generally, Banach spaces, and Frechet spaces (defined by a countable numer of seminomrs).

\begin{remark}
    Let $\mathcal{H}$ be a Hilbert space, then the dual space $\mathcal{H}^*$ is itself. $\mathcal{H}=\mathcal{H}^*$.    
    Hilbert spaces are the best spaces to work with. They are self-dual, and identified with themslves.
\end{remark}
Then in the next section, we will look at groups, motivated by their actions on Banach spaces, connected with Fourier transforms. 


\subsection{Motivation}
Let $X$ be a compact Hausdorff space. Let $C(X)=\{f:X\to\R, f \text{ continuous}\} $ be the algebra of continuous functons on $X$ mapping in to $\R$ or $\mathbb{C}$.
Define the norm as the sup norm $\|\cdot\|_{L^\infty}$.

We will develop the spectral theorem of operators on the Hilbert space, i..e self-adjoint operators can be diagonalized.

If $T$ is a self-adjoint operator on a Hilbert space, then we take the product of $T$ (polynomials of $T$), let $C^*(T, I_\mathcal{H})$ be the sub-algebra of operators generated by $T$ and $I$ the identity operator, then take the closure, i.e. making it closed in the operator norm.

\begin{remark}
    The $*$ is to remind us, $T$ is self-adjoint and when you take the adjoint and generate with it, it gets back into the same space.
\end{remark}

\begin{proposition}
    We have the next two algebra isomorphic to each other.
    \begin{equation}
        C^*(T, I_\mathcal{H})\cong C(X)
    \end{equation}
\end{proposition}
We can generalize this even further.
This is what we are aiminig for. If $T_1,..., T_n$ is a collection of self-adjoint operators on $\mathcal{H}$, that all commute with each other,
then we also have
\begin{equation}
    C^*(T_1,..., T_n, I_\mathcal{H})\cong C(X)
\end{equation}

\subsection{Groups}

Let $G$ be a group, $B$ be a Banach space, for example, groups of automorphisms. Let
\begin{equation*} 
    Aut(B)=\{T:  T \text{is isometric, onto, invertible } on B\}
\end{equation*}    
Isomorphisms mapping into itself is an automorphism.

\begin{definition}
    Suppose that $\alpha$ is a group homomorphisms, and $\alpha: G\to Aut(B)$, is called a representation on $B$ or an action on $B$.
\end{definition}

Then we can consider the subalgebra of $\mathcal{L}(B)$ the bounded linear operators on $B$, generated by
\begin{equation*}
    \{\alpha_x:x\in G\}
\end{equation*}
\begin{remark}
    The identity on $G$ should be mapped into the identity operator on $B$, hence no need to include it.
\end{remark}

Elements of the form $\Sigma_{z_x}\alpha_x, z_x\in\mathbb{C}$, (where $\Sigma$ is a finite sum.)

Let's introduce,
$f\in C_c(G)$ are functions with compact support and in discrete groups, imply they are of finite support.

\begin{equation*}
    \sum_{x\in G}f(x)\alpha_x=\alpha_f
\end{equation*}
note for except finitely many $x$, $f(x)=0$.

Let $f,g\in C_c(G)$, then for 
\begin{equation*}
    \alpha_f\alpha_g=(\sum f(x)\alpha_x)(\sum g(y)\alpha_y)=\sum_{x,y}f(x)g(y)\alpha_x\alpha_y=\sum_{x,y}f(x)g(y)\alpha_{xy}
\end{equation*}
The last inequality follows from $\alpha$ being a group homomorphism. And the sums are finite hence are able to exchange the orders.
We further have,
\begin{equation*}
    \alpha_f\alpha_g=\sum_x\sum_yf(x)g(x^{-1}y)\alpha_y=\sum (f\ast g)(y)\alpha_y
\end{equation*}
where we define $f\ast g(y)=\sum f(x)g(x^{-1}y)$ as the convolution operator.


We get
\begin{equation*}
    \alpha_f\alpha_g=\alpha_{f\ast g}
\end{equation*}
This is how we define convolution on $C_C(G)$
Notice we have, by $\|\alpha_x\|=1$,
\begin{equation*}
    \|\alpha_f\|=\|\sum f(x)\alpha_x\|\leq\sum|f(x)|\|\alpha_x\|=\sum|f(x)|=l^1(f)=\|f\|_{l^1}
\end{equation*}
It is therefore, easy to check

\begin{equation*}
    \|f\ast g\|_{l^1}\leq\|f\|_{l^1}\|g\|_{l^1}
\end{equation*}
We get $l^1(G)$ is an algebra with ??

For $G$ commutative, it is easily connected with the Fourier transform.

Consider $l^2(G)$ with the counting measure on the group. For $x\in G$, let $\xi\in l^2(G)$ define $\alpha_x\xi(y)=\xi(x^{-1}y), \alpha_x$ being unitary.
$l^1(G)$ acts on operators in $l^2(G)$ via $\alpha$.


If $G$ is commutative, then we have
\begin{equation*}
    \overline{\alpha_{l^1(G)}}\cong C(X)
\end{equation*}
where $X$ is some compact space.
Note that $C_c(G)$ operators on $l^2(G)$, and $\|\alpha_f\|\leq\|f\|_{l^1}$.

