\chapter{Some things to remember}


Convergence of Fourier series and Fourier transforms.
We define the partial sum operator for Fourier transforms as follows:
\begin{equation*}
    (S_Rf)^{\widehat{\phantom{.}}}=\chi_{B_R}\widehat{f}
\end{equation*}
We first talk about the $L^p$ convergence of the Fourier transform. For $n=1$, Riesz showed that $\|S_Rf-f\|_p=0$ as $R\to\infty$. For $n>1$, C. Fefferman then showed that $S_R$ is not bounded unless $p=2$.
\begin{thm}
    $S_Rf$ converges to $f$ in $L^p$ if and only if $S_R$ is bounded.
\end{thm}
\begin{proof}
    $(\Rightarrow)$ By Uniform Boundedness Principle, either $S_R$ is bounded or there exists $f\in L^p$ such that $\sup_R\|S_Rf\|_p=\infty$. This contradicts that $\|S_Rf-f\|_p\to 0$ as $R\to\infty$.

    $(\Leftarrow)$ One can find smooth, compactly supported $g$ such that $S_Rg=g$, and $\|f-g\|_p<\epsilon$. And the result follows.
\end{proof}

For pointwise convergence, it is the theorem by Carleson-Hunt. The Fourier series of a $L^p$ function for $1<p<\infty$ converges pointwise. However, the pointwise convergence of Fourier transforms is such that it converges pointwise for $1< p\leq 2$.

