\section*{Lecture 2}
Now we begin with some building blocks.

Suppose $\Omega=[0,N], \theta_j=[j-1. j], \Omega=\bigsqcup_{j=1}^N\theta_j$. And we ask the question, if we have $supp(\widehat{f})\subset[0,1]$, could $|f|$ look like several narrow peaks and almost 0 elsewhere?

We recall how we decouple the function $f$: for $supp(\widehat{f})\subset\Omega$, define $f_{\theta_j}=\int_{[j-1, j]}\widehat{f}(\omega)e^{i\omega x}d\omega$, then $f=\sum_jf_{\theta_j}$.

Now we remind ourselves of the height of $f$.
\begin{proposition}
    Let $f\in\mathcal{S}$ be such that $supp(\widehat{f})\subset[0,1]$, and we have
    \begin{equation*}
        ||f||_{L^\infty}\lesssim||f||_{L^1}
    \end{equation*}
\end{proposition}
\begin{proof}
    We define a cutoff function $\eta\in\mathcal{S}$ such that $\eta=1$ on $[0,1]$, then $\widehat{f}=\eta\widehat{f}$, then $f=f\ast\check{\eta}$, also a Schwartz function.
\begin{align*}
    ||f||_{L^\infty}&=||f\ast\check{\eta}||_{L^\infty}\\
    &\leq||f||_{L^1}||\check{\eta}||_{L^\infty}\\
    &\lesssim||f||_{L^1}
\end{align*}
\end{proof}
Hence the answer is no, because if we have narrow peaks with controlled heights, $||f||_{L^1}$ would be small, which would violate $||f||_{L^\infty}\lesssim||f||_{L^1}$.

Now we ask the following question, can we have flat parts of $|f|$ where $||f||_{L^1}$ is dominated by the flat parts, but still has narrow peaks? To address that, we introduce an important lemma which allows us to control the height of $f$ in one interval using its $L^1$ norm in an even larger interval.

\begin{proposition}[Locally Constant Lemma]
    If $supp\widehat{f_1}\subset[0,1]$, and $I$ is the unit interval $[0,1]$, then we have
    \begin{equation*}
        ||f||_{L^\infty(I)}\lesssim ||f||_{L^1(\omega_I}
    \end{equation*}
    Where the weighted $L^1$ norm is defined to be $||f||_{L^1(\omega_I)}=\int_{\R}|f_1|\omega_I$
\end{proposition}



Why is this not composing?
