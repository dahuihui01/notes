\chapter{Finite Fields, Field Extensions}
Page 47-51

\begin{prob}[F2016-Q3]
    Let \(F\) be a finite field of order \(2^{n}\). Here \(n>0\). Determine all values of \(n\) such that the polynomial \(x^{2}-x+1\) is irreducible in \(F[x]\).
\end{prob}


\begin{prob}[F2015-Q5]
    Let \(L\) be a finite field. Let \(a\) and \(b\) be elements of \(L^\times\) (the multiplicative group of \(L\)) and \(c \in L\). Show that there exist \(x\) and \(y\) in \(L\) such that \(ax^2 + by^2 = c\).
\end{prob}

\begin{prob}[F2013-Q6]
    Let \(p\) be a prime and let \(F\) be a field of characteristic \(p\).
    \begin{itemize}
        \item[(a)] Prove that the map \(\varphi : F \to F, \varphi(a) = a^p\) is a field homomorphism.
        \item[(b)] \(F\) is said to be \textit{perfect} if the above homomorphism \(\varphi\) is an automorphism. Prove that every finite field is perfect.
        \item[(c)] If \(x\) is an indeterminate and \(F\) is any field of characteristic \(p\), prove that the field \(F(x)\) is not perfect.
    \end{itemize}
\end{prob}

\begin{prob}[F2017-Q5]
    Let \(K/k\) be an extension of finite fields with \(\#k=q\), let \(\Phi\colon x\mapsto x^{q}\) denote the \(q\)th power Frobenius map on \(K\), and let \(G:=\text{Gal}(K/k)\).
    \begin{itemize}
        \item[(a)] Compute the minimal polynomial of \(\Phi\) as a \(k\)-linear endomorphism of \(K\).
        \item[(b)] Use (a) to prove the \textit{normal basis theorem} in the case of the extension \(K/k\): there exists \(x\in K\) such that the set \(\{\sigma x\mid\sigma\in G\}\) is a \(k\)-basis for \(K\).
    \end{itemize}
\end{prob}

\begin{prob}[F2010-Q5]
    Let \(\mathbb{F}_q\) be a finite field with \(q = p^n\) elements. Here \(p\) is a prime number. Let \(\varphi : \mathbb{F}_q \rightarrow \mathbb{F}_q\) be given by \(\varphi(x) = x^p\).
    \begin{itemize}
        \item[(a)] Show that \(\varphi\) is a linear transformation on \(\mathbb{F}_q\) (as vector space over \(\mathbb{F}_p\)), then determine its minimal polynomial.
        \item[(b)] Supposed that \(\varphi\) is diagonalizable over \(\mathbb{F}_p\). Show that \(n\) divides \(p - 1\).
    \end{itemize}
\end{prob}

\begin{prob}[S2011-Q2]
    Let \(p\) be a prime, \(F\) a finite field with \(p\) elements and \(K\) a finite extension of \(F\). Denote by \(F^\times\) and \(K^\times\) the multiplicative groups of nonzero elements of fields \(F\) and \(K\), respectively. Prove that the norm homomorphism \(N:K^\times\to F^\times\) is surjective.
\end{prob}


\begin{prob}[F2008-Q3]
    Let \(k\) be a finite field and \(K\) be a finite extension of \(k\). Let \(\mathfrak{Tr} = \text{Tr}_k^K\) be the trace function from \(K\) to \(k\). Determine the image of \(\mathfrak{Tr}\) and prove your answer.
\end{prob}


\begin{prob}[S2014-Q3]
    Let \(L/K\) be a Galois extension of degree \(p\) with \(\text{char}K=p\). Show that \(L=K(\theta)\), where \(\theta\) is a root of \(x^{p}-x-a,a\in K\), and, conversely, any such extension is Galois of degree 1 or \(p\).
\end{prob}


\begin{prob}[S2015-Q1]
    Let \(K\) be a field of characteristic \(p>0\). Prove that a polynomial \(f(x)=x^{p}-x-a\in K[x]\) either irreducible, or is a product of linear factors. Find this factorization if \(f\) has a root \(x_{0}\in K\).
\end{prob}


\begin{prob}[S2002-Q5]
    Let \(\zeta = e^{\frac{2\pi i}{5}}\) and \(K = \mathbb{Q}(\zeta)\) the field generated by \(\zeta\) over the field of rational numbers. Prove that \(K\) contains \(\sqrt{5}\).
\end{prob}

\begin{prob}[S2008-Q2]
    Let \(\xi\) be a primitive 9-th root of unity. Find the minimal polynomial of \(\xi + \xi^{-1}\) over \(\mathbb{Q}\).
\end{prob}


\begin{prob}[F2007-Q1]
    Let \(G\) be a cyclic group of order 12. Construct a Galois extension \(K\) over \(\mathbb{Q}\) so that the Galois group is isomorphic to \(G\).
\end{prob}


\begin{prob}[F2011-Q3]
    Let \(G\) be a cyclic group of order 100. Let \(K=\mathbb{Q}\), the field of rational numbers, or \(K=F_p\), the finite field with \(p\) elements, \(p\) being a prime number. For each such \(K\), construct a Galois extension \(L/K\) whose Galois group \(\text{Gal}(L/K)\) is isomorphic to \(G\). Explain your construction in detail.
\end{prob}



\begin{prob}[S2003-Q2]
    Let \(K\) be a field. A polynomial \(f(x) \in K[x]\) is called separable if, in any field extension, it has distinct roots. Prove that:
    \begin{itemize}
        \item[(a)] if \(K\) has characteristic 0, then each irreducible polynomial in \(K[x]\) is separable; and
        \item[(b)] if \(K\) has characteristic \(p \neq 0\), then an irreducible polynomial \(f(x) \in K[x]\) is separable if and only if has no form \(g(x^p)\) where \(g(x) \in K[x]\).
    \end{itemize}
    Give an example of an inseparable irreducible polynomial.
\end{prob}






\begin{prob}[S2001-Q4]
    Let \(p\) be a prime number, \(\mathbb{F}_p\) the prime field of \(p\) elements, \(X\) and \(Y\) algebraically independent variables over \(\mathbb{F}_p\), \(K=\mathbb{F}_p(X,Y)\), and \(F=\mathbb{F}_p(X^p-X,Y^p-X)\).
    \begin{itemize}
        \item[(a)] Show that \([K:F]=p^2\) and the separability and inseparability degrees of \(K/F\) are both equal to \(p\).
        \item[(b)] Show that there exists a field \(E\), such that \(F \subseteq E \subseteq K\), which is a purely inseparable extension of \(F\) of degree \(p\).
    \end{itemize}
\end{prob}



\begin{prob}[F2003-Q2]
    Let \(k\) be a field of characteristic \(p\) and let \(t, u\) be algebraically independent over \(k\). Prove the following:
    \begin{itemize}
        \item[(a)] \(k(t, u)\) has degree \(p^2\) over \(k(t^p, u^p)\).
        \item[(b)] There exist infinitely many fields between \(k(t, u)\) and \(k(t^p, u^p)\).
    \end{itemize}
\end{prob}




\chapter{Field Theory Random}
Page 62



\begin{prob}[S2016-Q2]
    Let \(F\subset K\) be an algebraic extension of fields. Let \(F\subset R\subset K\) where \(R\) is a \(F\)-subspace of \(K\) with the property such that \(\forall a\in R\), \(a^{k}\in R\) for all \(k\geq 2\).
    \begin{itemize}
        \item[(1)] Assume that \(\text{char}(F)\neq 2\). Show that \(R\) is a subfield of \(K\).
        \item[(2)] Give an example such that \(R\) may not be a field if \(\text{char}(F)=2\).
    \end{itemize}
\end{prob}

\begin{prob}[S2013-Q4]
    Prove that the group of automorphisms \(\text{Aut}_\mathbb{Q}(\mathbb{R})\) of the field \(\mathbb{R}\) that fix \(\mathbb{Q}\) pointwise is trivial (\textit{Hint}: Prove that every such automorphism is continuous).
\end{prob}

\begin{prob}[F2018-Q5]
    Let \(t\) be transcendental over \(\mathbb{Q}\). Set \(K=\mathbb{Q}(t)\), \(K_{1}=\mathbb{Q}(t^{2})\) and \(K_{2}=\mathbb{Q}(t^{2}-t)\).
    
    Show that \(K\) is algebraic over \(K_{1}\) and over \(K_{2}\) and that \(K\) is not algebraic over \(K_{1}\cap K_{2}\).
\end{prob}

\begin{prob}[S2006-Q4]
    Let \(k\) be a field. Let \(p\) be a prime number. Let \(a \in k\). Show that the polynomial \(x^p - a\) either has a root in \(k\) or is irreducible in \(k[x]\).
\end{prob}







\chapter{Random Problems}
Page 63-64

\begin{prob}[F2005-Q1]
    Let \(K\) be a finite field with \(q\) elements. Let \(n > 0\) be a positive integer. Compute the sum
    \[\sum_{x \in K} x^n.\]
\end{prob}


\begin{prob}[F2005-Q2]
    Let \(K\) be the splitting field (in \(\mathbb{C}\)) of the polynomial \(x^4 - 3x^2 + 5\) over \(\mathbb{Q}\).
    \begin{itemize}
        \item[(1)] Determine \(\text{Gal}(K/\mathbb{Q})\).
        \item[(2)] Find all intermediate fields \(\mathbb{Q} \subset E \subset K\) such that \(E\) is Galois over \(\mathbb{Q}\).
    \end{itemize}
\end{prob}

\begin{prob}[F2005-Q3]
    Let \(k \subset E\) be an algebraic extension of fields of characteristic zero. Assume that every non-constant polynomial \(f(x) \in k[x]\) has a root in \(E\). Show that \(E\) is algebraically closed.
\end{prob}

\begin{prob}[F2005-Q4]
    Let \(R\) be a commutative ring. Let \(I\) be a finitely generated ideal. Assume that \(I^2 = I\). Show that \(I\) is a direct summand of \(R\).
\end{prob}

\begin{prob}[S2005-Q2]
    Let \(\mathbb{F}_p\) be the field with \(p\) elements, where \(p\) is a prime number. Let \(f_{n,p}(x) = x^{p^n} - x + 1\), and suppose that \(f_{n,p}(x)\) is irreducible in \(\mathbb{F}_p[x]\). Let \(\alpha\) be a root of \(f_{n,p}(x)\).
    \begin{itemize}
        \item[(a)] Show that \(\mathbb{F}_{p^n} \subset \mathbb{F}_p(\alpha)\) and \([\mathbb{F}_p(\alpha) : \mathbb{F}_{p^n}] = p\).
        \item[(b)] Determine all pairs \((n, p)\) for which \(f_{n,p}(x)\) is irreducible.
    \end{itemize}
\end{prob}


\begin{prob}[S2005-Q3]
    Let \(\xi\) be a primitive \(p^n\)-th root of unity. Here \(p\) is prime and \(n > 0\). Let \(f(x)\) be the minimal polynomial of \(\xi\) over \(\mathbb{Q}\), and let \(m\) be its degree.
    \begin{itemize}
        \item[(a)] Determine \(f(x)\).
        \item[(b)] Let \(\alpha_1, \ldots, \alpha_m\) be all the roots of \(f(x)\). Define the discriminant of \(\xi\) as:
        \[D(\xi) = [\det(\alpha_i^{j-1})_{ij}]^2, \quad i, j = 1, \ldots m.\]
        Show that
        \[D(\xi) = (-1)^{\frac{m(m-1)}{2}} N_{\mathbb{Q}}^{\mathbb{Q}(\xi)}(f'(\xi)).\]
        \item[(c)] Take \(n = 2\). Compute \(D(\xi)\) in this case.
    \end{itemize}
\end{prob}


\begin{prob}[S2004-Q2]
    Let \(K \subset \mathbb{C}\) be the splitting field of \(f(x) = x^6 + 3\) over \(\mathbb{Q}\). Let \(\alpha\) be a root of \(f(x)\) in \(K\).
    \begin{itemize}
        \item[(a)] Show that \(K = \mathbb{Q}(\alpha)\).
        \item[(b)] Determine the Galois group \(\text{Gal}(K/\mathbb{Q})\).
    \end{itemize}
\end{prob}

\begin{prob}[S2004-Q4]
    Let \(k\) be a field. Consider the integral domain \(R = k[x,y]/(x^2 - y^2 + y^3)\).
    \begin{itemize}
        \item[(a)] Show that \(R\) is not a unique factorization domain.
        \item[(b)] Let \(F\) be the field of fractions of \(R\). Find \(t \in F\) such that \(F = k(t)\).
        \item[(c)] Determine the integral closure of \(R\) in \(F\).
    \end{itemize}
\end{prob}

\begin{prob}[F2000-Q2]
    \phantom{text}
    \begin{itemize}
        \item[(a)] Let \(p\) be a prime number. Show that \(f(X)=X^p-pX-1\) is irreducible in \(\mathbb{Q}[X]\). (Hint: use Eisenstein's criterion of irreducibility for the image of \(f(X)\) via a ring automorphism of \(\mathbb{Q}[X]\).)
        \item[(b)] Let \(R\) be the ring \(\mathbb{Z}[X]/(X^4-3X^2-X)\), where \((X^4-3X^2-X)\) is the ideal generated by \(X^4-3X^2-X\) in \(\mathbb{Z}[X]\). Find all the prime ideals of \(R\) containing \(\hat{3}\) (the image of \(3 \in \mathbb{Z}[X]\) via the canonical surjection \(\mathbb{Z}[X] \to R\).)
    \end{itemize}
\end{prob}

\begin{prob}[F2000-Q3]
    Let \(K/k\) be a finite, separable field extension of degree n. Let
    \[\rho, \rho' : K \to M_n(k)\]
    be two morphisms of \(k\)-algebras, where \(M_n(k)\) is the ring of \(n \times n\) matrices with entries in \(k\). Show that there exists an invertible matrix \(A\) in \(M_n(k)\) such that
    \[\rho'(x)=A \cdot \rho(x) \cdot A^{-1}, \text{ for all } x \in K.\]
\end{prob}

\begin{prob}[F2019-Q1]
    Let \(\mathbb{F}_{q}\) be a field with \(q\neq 9\) elements and \(a\) be a generator of the cyclic group \(\mathbb{F}^{*}_{q}\). Show that \(\mathrm{SL}_{2}(\mathbb{F}_{q})\) is generated by
    \[\left(\begin{array}{cc}1&1\\0&1\end{array}\right),\ \left(\begin{array}{cc}1&0\\a&1\end{array}\right).\]
\end{prob}