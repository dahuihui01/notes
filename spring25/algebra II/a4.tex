\chapter{Ring Theory Random}
Page 41

\begin{prob}[S2010-Q2]
    Let \(R\) be a ring such that \(r^3 = r\) for all \(r \in R\). Show that \(R\) is commutative. (Hint: First show that \(r^2\) is central for all \(r \in R\).)
\end{prob}

\begin{prob}[S2006-Q2]
    Let \(R\) be a ring with identity 1. Let \(x, y \in R\) such that \(xy = 1\).
    \begin{itemize}
        \item[(1)] Assume \(R\) has no zero-divisor. Show that \(yx = 1\).
        \item[(2)] Assume \(R\) is finite. Show that \(yx = 1\).
    \end{itemize}
\end{prob}


























\chapter{Tensor Products over Fields}
Page 42-43


\begin{prob}[S2018-Q3]
    Let \(K/k\) be a finite separable field extension, and let \(L/k\) be any field extension. Show that \(K\otimes_{k}L\) is a product of fields.
\end{prob}

\begin{prob}[F2019-Q3]
    Let \(F,L\) be extensions of a field \(K\). Suppose that \(F/K\) is finite. Show that there exists an extension \(E/K\) such that there are monomorphisms of \(F\) into \(E\) and of \(L\) into \(E\) which are identical on \(K\).
\end{prob}


\begin{prob}[F2009-Q4]
    Let \(E\) and \(F\) be finite field extensions of a field \(k\) such that \(E \cap F = k\), and that \(E\) and \(F\) are both contained in a larger field \(L\). Assume that \(E\) is Galois over \(k\). Show that \(E \otimes_k F \cong EF\).
\end{prob}



\begin{prob}[S2008-Q5]
    Let \(k\) be a field of characteristic zero. Assume that \(E\) and \(F\) are algebraic extensions of \(k\) and both contained in a larger field \(L\). Show that the \(k\)-algebra \(E \otimes_k F\) has no nonzero nilpotent elements.
\end{prob}


\begin{prob}[S2004-Q5]
    Show that there is a \(\mathbb{C}\)-algebra isomorphism between \(\mathbb{C} \otimes_{\mathbb{R}} \mathbb{C}\) and \(\mathbb{C} \times \mathbb{C}\).
\end{prob}


\begin{prob}[F2005-Q5]
    Let \(\mathbb{C}\) and \(\mathbb{R}\) be complex and real number fields. Let \(\mathbb{C}(x)\) and \(\mathbb{C}(y)\) be function fields of one variable. Consider \(\mathbb{C}(x) \otimes_{\mathbb{R}} \mathbb{C}(y)\) and \(\mathbb{C}(x) \otimes_{\mathbb{C}} \mathbb{C}(y)\).
    \begin{itemize}
        \item[(1)] Determine if they are integral domains.
        \item[(2)] Determine if they are fields.
    \end{itemize}
\end{prob}

\begin{prob}[F2003-Q4]
    Verify the isomorphism of algebras over a field \(K\):
    \[\mathbb{M}_n(K) \otimes_K \mathbb{M}_{m}(K) \simeq \mathbb{M}_{mn}(K).\]
    [Note: \(\mathbb{M}_n(K)\) denotes the algebra of \(n \times n\) matrices over \(K\).]
\end{prob}

















\chapter{Irreducibility of Polynomials}
Page 44-45

\begin{prob}[F2008-Q2]
    Show that the polynomial \(x^5 - 5x^4 - 6x - 2\) is irreducible in \(\mathbb{Q}[x]\).
\end{prob}

\begin{prob}[F2003-Q3]
    Obtain a factorization into irreducible factors in \(\mathbb{Z}[x]\) of the polynomial \(x^{10} - 1\).
\end{prob}


\begin{prob}[S2004-Q3]
    Let \(k\) be a field with characteristic 0. Let \(m \geq 2\) be an integer. Show that \(f(x,y) = x^m + y^m + 1\) is irreducible in \(k[x,y]\).
\end{prob}

\begin{prob}[S2017-Q2, S2007-Q3]
    Write down the minimal polynomial for \(\sqrt{2}+\sqrt{3}\) over \(\mathbb{Q}\) and prove that it is reducible over \(\mathbb{F}_{p}\) for every prime number \(p\).
\end{prob}

\begin{prob}[S2015-Q4]
    Prove that the polynomial \(x^{4}+1\) is not irreducible over any field of positive characteristic.
\end{prob}



\begin{prob}[F2010-Q2]
    \phantom{text}
    \begin{itemize}
        \item[(a)] Find the complete factorization of the polynomial \(f(x) = x^6 - 17x^4 + 80x^2 - 100\) in \(\mathbb{Z}[x]\).
        \item[(b)] For which prime numbers \(p\) does \(f(x)\) have a root in \(\mathbb{Z}/p\mathbb{Z}\) (i.e, \(f(x)\) has a root modulo \(p\))? Explain your answer.
    \end{itemize}
\end{prob}
















