\chapter{Ring Theory Random}
% Page 41


\begin{prop}
    Let $I\subset R$ be an ideal, then the following are equivalent:
    \begin{enumerate}
        \item $I$ is a prime ideal.
        \item There exists a field $K$ and $\varphi:R\to K$ such that $I=\ker(\varphi)$.
    \end{enumerate}
\end{prop}
\begin{proof}
    (1)$\Rightarrow$(2). Let $K$ be the field of fractions of $R/I$, which is an integral domain. (2)$\Rightarrow$ (1) is obvious given $K$ is a field.
\end{proof}

\begin{prob}[S2010-Q2]
    Let \(R\) be a ring such that \(r^3 = r\) for all \(r \in R\). Show that \(R\) is commutative. (Hint: First show that \(r^2\) is central for all \(r \in R\).)
\end{prob}
\begin{proof}
    This question is not so constructive and is purely computational (as far as I am aware) so I will skip it here.
\end{proof}

\begin{prob}[S2006-Q2]
    Let \(R\) be a ring with identity 1. Let \(x, y \in R\) such that \(xy = 1\).
    \begin{itemize}
        \item[(1)] Assume \(R\) has no zero-divisor. Show that \(yx = 1\).
        \item[(2)] Assume \(R\) is finite. Show that \(yx = 1\).
    \end{itemize}
\end{prob}
\begin{proof}
    \begin{itemize}
        \item[(1)] We know $x,y\neq 0$, therefore consider 
        \begin{equation*}
            x(yx-1)=0
        \end{equation*}
        Since $R$ has no zero-divisor, we must have $yx-1=0$, as desired.
        \item[(2)] We note the right multiplication map $m_x:R\to R$ by $x$ is injective: suppose $r_1,r_2\in R$ and 
        \begin{equation*}
            r_1x=xr_2x
        \end{equation*}
        multiplying both sides by $y$ we see $r_1=r_2$. Since $R$ is finite, this map is also surjective, i.e., there exists $s\in R$ such that 
        \begin{equation*}
            sx=1
        \end{equation*} 
        Now we see 
        \begin{equation*}
            yx-1=sx(yx-1)=sx-sx=0
        \end{equation*}
        as desired.
    \end{itemize}
\end{proof}


























\chapter{Tensor Products over Fields}
Page 42-43

\begin{prop}
    If $L/k$ is finite separate extension, then there exists $\alpha\in L$ such that 
    \begin{equation*}
        L=k(\alpha)
    \end{equation*}
\end{prop}
\begin{example}
    Write $\Q(\sqrt{2})\otimes_\Q\Q(\sqrt{3})$ as a product of fields: 
    \begin{equation*}
        \Q(\sqrt{2})\otimes_\Q\frac{\Q[x]}{(x^2-3)}\cong\frac{\Q(\sqrt{2}[x])}{(x^3-2)}
    \end{equation*}
    and $(x^3-2)$ does not have a root in $\Q(\sqrt{2})$, thus 
    \begin{equation*}
        \Q(\sqrt{2})\otimes_\Q\Q(\sqrt{3})\cong\Q(\sqrt{2})\sqrt{3}
    \end{equation*}
\end{example}

\begin{example}
    Similarly, write the following as a product of fields
    \begin{align*}
        \Q(\sqrt[4]{2})\otimes_\Q\Q(\sqrt[4]{2})&=\Q(\sqrt[4]{2})\otimes_Q\frac{\Q[x]}{(x^4-2)}\\
        &=\frac{\Q(\sqrt[4]{2})[x]}{(x^4-2)}\\
        &=\frac{\Q(\sqrt[4]{2})[x]}{(x-\sqrt[4]{2})(x+\sqrt[4]{2})(x^2+\sqrt{2})}\\
        &=\frac{\Q(\sqrt[4]{2})[x]}{(x-\sqrt[4]{2})}\times\frac{\Q(\sqrt[4]{2})[x]}{(x+\sqrt[4]{2})}\times \frac{\Q(\sqrt[4]{2})[x]}{(x^2+\sqrt{2})}
    \end{align*}
    By the Chinese Remainder theorem 
    \begin{lem}[CRT]
        Let $R$ be a PID, and $I+J=(1)$, then 
        \begin{equation*}
            \frac{R}{IJ}=\frac{R}{I}\times\frac{R}{J}
        \end{equation*}
    \end{lem}
    We have 
    \begin{equation*}
        \Q(\sqrt[4]{2})\otimes_\Q\Q(\sqrt[4]{2})\cong \Q(\sqrt[4]{2})\times \Q(\sqrt[4]{2})\times \Q(\sqrt[4]{2})(i)
    \end{equation*}
\end{example}


\begin{example}
    The field extension generated $(x^p-t)$ of field $\F_p(t)$ is not separable, i.e., 
    \begin{equation*}
        \frac{\F_p(t)[x]}{(x^p-t)}
    \end{equation*}
    is not separable. Consider the element $x$, then the minimal polynomial $m(s)=s^p-t$ can be written as 
    \begin{equation*}
        s^p-t=s^p-x^p=(s-x)^p
    \end{equation*}
\end{example}







\begin{prob}[S2017-Q3]
    Let \(K/k\) be a finite separable field extension, and let \(L/k\) be any field extension. Show that \(K\otimes_{k}L\) is a product of fields.
\end{prob}


\begin{prob}[F2019-Q3]
    Let \(F,L\) be extensions of a field \(K\). Suppose that \(F/K\) is finite. Show that there exists an extension \(E/K\) such that there are monomorphisms of \(F\) into \(E\) and of \(L\) into \(E\) which are identical on \(K\).
\end{prob}


\begin{prob}[F2009-Q4]
    Let \(E\) and \(F\) be finite field extensions of a field \(k\) such that \(E \cap F = k\), and that \(E\) and \(F\) are both contained in a larger field \(L\). Assume that \(E\) is Galois over \(k\). Show that \(E \otimes_k F \cong EF\).
\end{prob}



\begin{prob}[S2008-Q5]
    Let \(k\) be a field of characteristic zero. Assume that \(E\) and \(F\) are algebraic extensions of \(k\) and both contained in a larger field \(L\). Show that the \(k\)-algebra \(E \otimes_k F\) has no nonzero nilpotent elements.
\end{prob}


\begin{prob}[S2004-Q5]
    Show that there is a \(\mathbb{C}\)-algebra isomorphism between \(\mathbb{C} \otimes_{\mathbb{R}} \mathbb{C}\) and \(\mathbb{C} \times \mathbb{C}\).
\end{prob}


\begin{prob}[F2005-Q5]
    Let \(\mathbb{C}\) and \(\mathbb{R}\) be complex and real number fields. Let \(\mathbb{C}(x)\) and \(\mathbb{C}(y)\) be function fields of one variable. Consider \(\mathbb{C}(x) \otimes_{\mathbb{R}} \mathbb{C}(y)\) and \(\mathbb{C}(x) \otimes_{\mathbb{C}} \mathbb{C}(y)\).
    \begin{itemize}
        \item[(1)] Determine if they are integral domains.
        \item[(2)] Determine if they are fields.
    \end{itemize}
\end{prob}

\begin{prob}[F2003-Q4]
    Verify the isomorphism of algebras over a field \(K\):
    \[\mathbb{M}_n(K) \otimes_K \mathbb{M}_{m}(K) \simeq \mathbb{M}_{mn}(K).\]
    [Note: \(\mathbb{M}_n(K)\) denotes the algebra of \(n \times n\) matrices over \(K\).]
\end{prob}

















\chapter{Irreducibility of Polynomials}
Page 44-45


\begin{prop}
    Fix any prime $p$, the polynomial 
    \begin{equation*}
        f(x)=x^{p-1}+\dots+x+1
    \end{equation*}
    is irreducible over $\mathbb{Q}$. Similarly 
    \begin{equation*}
        g(x)=x^{p-1}-x^{p-2}+\dots-x+1
    \end{equation*}
    is irreducible over $\mathbb{Q}$.
\end{prop}
\begin{proof}
    This is an application of Eisenstein. Write 
    \begin{equation*}
        f(x)=\frac{x^p-1}{x-1}
    \end{equation*}
    and replace $x$ with $x+1$ we get 
    \begin{align*}
        f(x)&=\frac{(x+1)^{p}-1}{x}\\
        &=\frac{\sum_{k=1}^n\binom{p}{k}x^k}{x}\\
        &=\sum_{k=1}^n\binom{p}{k}x^{k-1}
    \end{align*}
    We apply Eisenstein with prime $p$ to see $f$ is irreducible.
\end{proof}

\begin{prop}
    For any prime $p$, either $\sqrt{2}\in\F_p$ or $\sqrt{3}\in\F_p$ or $\sqrt{6}\in\F_p$.
\end{prop}
\begin{proof}
    We know there exists a legendre symbol (a character) $\chi: \F_p^\times\to \{\pm 1\}$ such that for $g\in\F_p$,
    \begin{equation*}
        \chi(g)=\begin{cases}
            1, \text{ if $g$ is a square}\\
            -1, \text{ if $g$ is not a square}
        \end{cases}
    \end{equation*}
    Suppose that $\sqrt{2}$ and $\sqrt{3}$ are not in $\F_p$, then 
    \begin{equation*}
        \chi(2)=\chi(3)=-1
    \end{equation*}
    i.e., $2,3$ are not squares. However,
    \begin{equation*}
        \chi(2\cdot 3)=\chi(6)=1
    \end{equation*} 
    This implies that $6$ is a square and $\sqrt{6}\in\F_p$, as desired.
\end{proof}

\begin{cor}
    The following polynomial 
    \begin{equation*}
        f(x)=(x^2-1)(x^3-1)(x^6-1)
    \end{equation*}
    has a linear factor.
\end{cor}



\begin{prob}[S2018-Q3]
    Let \(R\) be the ring \(\mathbb{Z}[\zeta_p]\), where \(p\) is a prime number and \(\zeta_p\) denotes a primitive \(p\)th root of unity in \(\mathbb{C}\). Prove that if an integer \(n \in \mathbb{Z}\) is divisible by \(1 - \zeta_p\) in \(R\), then \(p\) divides \(n\).
\end{prob}


\begin{prob}[F2008-Q2]
    Show that the polynomial \(x^5 - 5x^4 - 6x - 2\) is irreducible in \(\mathbb{Q}[x]\).
\end{prob}

\begin{prob}[F2003-Q3]
    Obtain a factorization into irreducible factors in \(\mathbb{Z}[x]\) of the polynomial \(x^{10} - 1\).
\end{prob}


\begin{prob}[S2004-Q3]
    Let \(k\) be a field with characteristic 0. Let \(m \geq 2\) be an integer. Show that \(f(x,y) = x^m + y^m + 1\) is irreducible in \(k[x,y]\).
\end{prob}

\begin{prob}[S2017-Q2, S2007-Q3]
    Write down the minimal polynomial for \(\sqrt{2}+\sqrt{3}\) over \(\mathbb{Q}\) and prove that it is reducible over \(\mathbb{F}_{p}\) for every prime number \(p\).
\end{prob}
\begin{proof}
    The minimal polynomial of $\sqrt{2}+\sqrt{3}$ is 
    \begin{equation*}
        f(x)=x^4-10x^2+1=0
    \end{equation*}
    By the corollary, we know in any $\F_p$ for any prime $p$, either $\sqrt{2}, \sqrt{3}, \sqrt{6}$ is in $\F_p$. We claim that if $\sqrt{2}\in\F_p$, then $f$ is factors over $\mathbb{Q}(\sqrt{2})$. Suppose that $f$ does not factor over $\Q(\sqrt{2})$, i.e., $f$ is irreducible over $\Q(\sqrt{2})$, then the degree of extension 
    \begin{equation*}
        [\Q(\sqrt{2}+\sqrt{3}):\Q]=[\Q(\sqrt{2}+\sqrt{3}):\Q(\sqrt{2})][\Q(\sqrt{2}):\Q]=8
    \end{equation*}
    which is a contradiction. Hence $f$ factors over $\Q(\sqrt{2})$. Similar arguments work if $\sqrt{3}$ or $\sqrt{6}$ are in $\F_p$.
\end{proof}

\begin{prob}[S2015-Q4]
    Prove that the polynomial \(x^{4}+1\) is not irreducible over any field of positive characteristic.
\end{prob}
\begin{proof}
    The idea is the same as above, and it suffices to note that the field extension generated by $x^4+1$ is $\Q(\sqrt{2}, i)$. Using the Legendre symbol, the proof is similar to the above.
\end{proof}



\begin{prob}[F2010-Q2]
    \phantom{text}
    \begin{itemize}
        \item[(a)] Find the complete factorization of the polynomial \(f(x) = x^6 - 17x^4 + 80x^2 - 100\) in \(\mathbb{Z}[x]\).
        \item[(b)] For which prime numbers \(p\) does \(f(x)\) have a root in \(\mathbb{Z}/p\mathbb{Z}\) (i.e, \(f(x)\) has a root modulo \(p\))? Explain your answer.
    \end{itemize}
\end{prob}
\begin{proof}
    \begin{itemize}
        \item[(a)] Letting $y=x^2$, we need to factorize 
        \begin{equation*}
            f(y)=y^3-17y+80y-100
        \end{equation*}
        Now $f$ is cubic, we need to find the roots of $f$: $5$ is a root, 
        \begin{equation*}
            f(y)=(y-5)(y-2)(y-10)
        \end{equation*}
        i.e., 
        \begin{equation*}
            f(x)=(x^2-2)(x^2-5)(x^2-10)
        \end{equation*}
        which consists of only irreducible factors over $\Z$.
        \item[(b)] $f$ has a root in $\F_p$ for all prime $p$, by the above corollary.
    \end{itemize}
\end{proof}

\section{Quick finite field review}
If $p$ is prime, then $\F_p$ is a field of $p$ elements, isomorphic to $\Z/p\Z$.
\begin{prop}[Fact]
    For every prime power $p^n$, there is exactly one finite field of $p^n$ elements, namely $\F_{p^n}$, up to isomorphisms.
\end{prop}
\begin{thm}[Galois theory of finite fields]
    We have 
    \begin{itemize}
        \item[(1)] $\F_{p^n}/\F$ is a Galois extension, and 
        \begin{equation*}
            \text{Gal}(\F_{p^n}/\F) \text{ is cyclic}
        \end{equation*}
        where the generator is the Forbenius automorphism $\sigma: \F_{p^n}\to\F_{p^n}$ where 
        \begin{equation*}
            \sigma: x\mapsto x^p
        \end{equation*}
        \item[(2)] We also have 
        \begin{equation*}
            \F_{p^n}=\left\{\alpha\in\overline{\F}_p: \alpha^{p^n}-\alpha=0\right\}
        \end{equation*}
        This statement implies that $
        \F_{p^n}$ is the splitting field of $x^{p^n}-1$.
    \end{itemize}
    

\end{thm}
\begin{proof}
    We note that $\F_{p^n}$ is the splitting field of $x^{p^n}-x$ over $\F_p$.  
    \begin{equation*}
        \F_{p^n}=\left\{\alpha\in\overline{\F}_p: \alpha^{p^n}-\alpha=0\right\}
    \end{equation*}
    If $\alpha\in\F_{p^n}$, then we want to show that $\alpha^{p^n}=\alpha$: if $\alpha=0$, then done; if $\alpha\in\F_p^\times$, then using the fact that any finite field is cyclic, we know 
    \begin{equation*}
        \F_{p^n}\cong\Z/(p^n-1)\Z\Rightarrow \alpha^{p^n-1}=1
    \end{equation*}
    and we are done. Now we observe that 
   $ \left\{\alpha\in\overline{\F}_p: \alpha^{p^n}-\alpha=0\right\}$ has $p^n$ elements, and is also a field, thus we are done.

   This fact can be used to show (1) and the above proposition.
\end{proof}


\begin{prop}
    $\F_{p^n}$ embeds into $\F_{p^m}$ iff $n\mid m$.
\end{prop}
\begin{proof}
    If $n\mid m$, then $m=nk$ for some integer $k$. We then notice that 
    \begin{equation*}
        \alpha^{p^n}=\alpha\Rightarrow \alpha^{p^{kn}}=\alpha^{p^m}=\alpha
    \end{equation*}
   Thus $\F_{p^n}$ embeds into $\F_{p^m}$. Conversely, consider the Galois field extensions 
   \begin{equation*}
        \F_p\subset\F_{p^n}\subset\F_{p^m}
   \end{equation*}
   Then by degree of field extensions, we know $n\mid m$.
\end{proof}



\begin{prob}[F2016-Q3]
    If field $|F|=2^n$, find all $n$ such that $x^2-x+1$ is irreducible over $F$.
\end{prob}
\begin{proof}
    We know that $x^2-x+1$ is irreducible over $\F_2$, namely, it has no roots in $\F_2$. Since there is only one field of order $4$, we must have 
    \begin{equation*}
        \F_4\cong\frac{\F_2}{(x^2-x+1)}
    \end{equation*}
    Clearly $x^2-x+1$ is not irreducible over $\F_4$. For any $\F_{2^n}$, we know $(x^2-x+1)$ is irreducible if and only if $\F_4$ does not embed into $\F{2^n}$, i.e., $2\nmid n$. This shows that when $n$ is odd, the polynomial $x^2-x+1$ is irreducible over $\F_{2^n}$.
\end{proof}














