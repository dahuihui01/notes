% \chapter{1}

% We will first talk about algebraic sets and spaces.

% \begin{thm}[1.5 Hilbert's Nullstellnsatz, p380]
%     Let $a\subset k[x]$ be an ideal, where $x=(x_1,\dots,x_n)$, let $f\in k[x]$ be a polynomial such that $f(c)=0$ for every zero $c=(c_1,\dots, c_n)$ of $a\subset k[x]$, then there exists $m\in\mathbb{N}$ such that 
%     \begin{equation*}
%         f^m\in a
%     \end{equation*}
%     Note that $f^m\in a \iff f\in\sqrt{a}\supset a$.
% \end{thm}
% \begin{proof}
%     Suppose $f\neq 0$, and $a$ has a zero. 
% \end{proof}
% The following theorem further characterizes ideals.
% \begin{thm}[1.4]
%     Let $a\subset k[x]=k[x_1,\dots,x_n]$, then either $a=k[x]$ or $a$ has a zero in $k^a$. Now we introduce a new variable $Y$, and consider the ideal $a'$ generated by $a$ and $1-Yf$. By the following theorem, we know that $a'$ must be the entire ring $k[X,Y]$. This implies that there exists polynomials $g_j\in k[X,Y], h_i\in a$ such that 
%     \begin{equation*}
%         1=g_0(1-Yf)+g_1h_1+\dots+g_rh_r
%     \end{equation*}
%     Substitute $Y$ with $f^{-1}$, and multiply both sides by $f^m$, 
%     \begin{equation*}
%         f^m=f^m(g_1h_1+\dots+g_rh_r)\in a
%     \end{equation*}
%     and we are done.
% \end{thm}

% \begin{thm}[1.1 p378]
%     Let $k$ be a field, and $k[x]=k[x_1,\dots, x_n]$ be a finitely generated ring. If $\varphi: k\to L$ is an embedding of an algebraically closed field $L$, then there exists an extension $\bar{\varphi}:k[X]\to L$.
% \end{thm}
% An immediate corollary is as follows:
% \begin{cor}[1.2 p389]
%     Let $k$ be a field and $k[x]$ be finitely generated over $k$. If $k[x]$ is a field, then $k[x]$ is algebraically closed over $k$.
% \end{cor}

% \begin{prop}[3.1 p347]
%     Let $A$ be a subring of $B$, and $B$ is integral over $A$, let $\varphi:A\to L$ where $L$ is a field that is algebraically closed, then $\varphi$ extends to a homomorphism of $B$ to $L$.
% \end{prop}

% Next we proved theorem 1.4 above. I cannot do this.



\chapter{1}
We first talk bout semidirect products. Let $G$ be any group, and $N,H$ be subgroups of $G$.

\begin{defn}
For $\varphi:H\to\text{Aut}(N)$, define $N\rtimes H$ by 
\begin{itemize}
    \item[(1)] $N\rtimes_\varphi H=N\times H$ as a set.
    \item[(b)] Equipped with the group structure 
    \begin{equation*}
        (n_1,h_1)\cdot (n_2,h_2)=(n_1\varphi(h_1)n_2, h_1h_2)
    \end{equation*}
    The structure $(N\rtimes_\varphi H, \cdot)$ forms a group. 
\end{itemize}
\end{defn}

\begin{example}
    If $N$ is a normal subgroup of $G$, and $N\cap H=\{e\}$, and $\varphi:H\to\text{Aut}(N)$ where 
    \begin{equation*}
        \varphi: h\mapsto (n\mapsto hnh^{-1})
    \end{equation*}
    (acting by conjugation), and $G=NH$. Then 
    \begin{equation*}
        N\rtimes_\varphi H\to G
    \end{equation*}
    where 
    \begin{equation*}
        (n,h)\mapsto nh
    \end{equation*}
    is a bijective homomorphism homomorphism. Hence 
    \begin{equation*}
        G\cong N\rtimes_\varphi H
    \end{equation*}
\end{example}

\textcolor{red}{what is happening}

Nxt we present some divisibility results.
\begin{prop}[Lagrange, Orbit-Stabilizer]
    We have the following divisibility results:
    \begin{itemize} 
        \item Let $H$ be a subgroup of $G$, let $[G:H]$ denote the number of cosets of $H$ in $G$, then 
        \begin{equation*}
            |G|=|H|[G:H]
        \end{equation*}
        \item Let $G$ be a finite group acting transitively on a finite set $A$, then for any $a\in A$, we have 
        \begin{equation*}
            |\text{Stab}_G(a)|\cdot|O_G(a)|=|G|
        \end{equation*}
    \end{itemize} 
\end{prop}
The class formula is when $G$ acts on itself by conjugation:
\begin{prop}[class formula]
    Let $G$ act on a finite set $S$, and let $Z(G)$ denote the center of the group $G$, then 
    \begin{equation*}
        |S|=|Z(G)|+\sum_{a\in A}|O_G(a)|
    \end{equation*}
    where $A$ includes exactly one element from each nontrivial orbit.

    If $G$ acts on itself by conjugation, then 
    \begin{equation*}
        |G|=|Z(G)|+\sum_{g}|[g]|=|Z(G)|+\sum_{g}\frac{|G|}{|C_G(g)|}
    \end{equation*}
    where $[g]$ denote the conjugacy class of $g$, and the sum includes exactly one from each nontrivial conjugacy class in $G$.
\end{prop}


\begin{prob}[F19-Q2]
    2. Let \( p, q \) be two prime numbers such that \( p \mid q - 1 \). Prove that  
    \begin{itemize}
    \item[(a)]there exists an integer \( r \neq 1 \mod q \) such that \( r^p \equiv 1 \mod q \);
    \item[(b)] there exists (up to an isomorphism) only one noncommutative group of order \( pq \).
    \end{itemize}
\end{prob}
\begin{proof}
    \begin{itemize}
        \item[(a)] We want to show that there exists an element $r\in(\Z/q\Z)^\times$ such that 
        \begin{equation*}
            r^p\equiv 1\mod q
        \end{equation*}
        We can do this because $(\Z/q\Z)^\times$ has order $(q-1)$ and $p\vert (q-1)$. Therefore by Cauchy's theorem, there exists an element of order $p$ in $(\Z/p\Z)^\times$.
        \item[(b)] Let $n_p,n_q$ denote the number of $p$, $q$-Sylow subgroups. We see that $n_q\vert p$ and $n_q\equiv 1\mod q$, since $p<q$, we must have $n_q=1$. Now $n_p=1$ or $q$ by the same reasoning. Suppose $n_q=1$, let $P,Q$ denote the normal subgroups of order $p,q$, then
        \begin{equation*}
            G\cong P\times Q
        \end{equation*}
        by a standard argument (included in the lemma below). Then $G$ is commutative. Hence $n_p=q$. We therefore have \textcolor{red}{what}
    \end{itemize}
\end{proof}



\begin{lem}
    Let $p,q$ be two primes such that $p<q$, and $N$, $H$ has order $p,q$ respectively, suppose that $N$ is normal in $G$, and $N\cap H=\{e\}$, then 
    \begin{equation*}
        G\cong N\times H
    \end{equation*}
\end{lem}
\begin{proof}
    We consider the map 
    \begin{equation*}
        \psi: N\times H\to G
    \end{equation*}
    such that 
    \begin{equation*}
        (n,h)\mapsto nh
    \end{equation*}
    We want to show that $\psi$ is a homomorphism and $\psi$ is injective (hence bijective by size argument). It is clearly injective: 
    \begin{equation*}
        nh=e\Rightarrow n, h\in N\cap H=\{e\}
    \end{equation*}
    It suffices to show that $\psi$ is a homomorphism. We see that this implies 
    \begin{equation*}
        n_1n_2h_1h_2=n_1h_1n_2h_2
    \end{equation*}
    Therefore it suffices to for any $n\in N, h\in H$, one has
    \begin{equation*}
        nh=hn
    \end{equation*}
    Consider the conjugation action 
    \begin{equation*}
        \varphi: H\to \text{Aut}(N)
    \end{equation*}
    where 
    \begin{equation*}
        h\mapsto \left( n\mapsto hnh^{-1}\right)
    \end{equation*}
    Then we claim that $\varphi$ is trivial. This is because $\ker(\varphi)$ has size either $1$ or $q$. If it has size $q$, then the map is trivial; if it has size $1$, then $H$ embeds in $\text{Aut}(N)$, however, $|H|=q, \text{Aut}(N)=p-1$, and $q\nmid(p-1)$, hence impossible. This shows that the map is trivial, i.e., for $n\in N, h\in H$, 
    \begin{equation*}
        hn=nh
    \end{equation*}
    as desired. 
\end{proof}


\begin{prob}[F2015-Q1]
    Prove every group of order $15$ is cyclic.
\end{prob}
\begin{proof}
    We will show that any group $G$ of order $15$ is isomorphic to 
    \begin{equation*}
        G\cong\frac{\Z}{3\Z}\times\frac{\Z}{5\Z}
    \end{equation*}
    For this, using the above lemma, it suffices to show that there is one normal subgroup of order $3$ and one normal subgroup of order $5$. We repeat the argument above, $n_5\mid 3$ and $n_5\equiv 1\mod 5$, hence $n_5=1$. Moreover, $n_3\mid 5$ and $n_3\equiv 1\mod 3$, hence $n_3=1$ as well. By the lemma above, we know that 
    \begin{equation*}
        G\cong\frac{\Z}{3\Z}\times\frac{\Z}{5\Z}
    \end{equation*}
    hence cyclic as desired.
\end{proof}

\begin{prob}[S2015-Q2]
Let \( p \) and \( q \) be primes with \( p < q \). Let \( G \) be a group of order \( pq \). Prove the following statements:

\begin{itemize}
    \item[(a)] If \( p \) does not divide \( q - 1 \) (i.e., \( p \nmid q - 1 \)), then \( G \) is cyclic.

    \item[(b)] If \( p \) divides \( q - 1 \) (i.e., \( p \mid q - 1 \)), then \( G \) is either cyclic or isomorphic to a non-abelian group on two generators. Give the presentation of this non-abelian group.
\end{itemize}
\end{prob}
\begin{proof}
    This question is exactly the same as  F19-Q2, we will only outline here.
    \begin{itemize}
        \item[(a)] We have $n_q=1$, and $n_p\mid q$, hence $n_p=1$ or $q$, moreover $n_p\equiv 1\mod p$. If $n_p=q$, this implies that $p\mid(q-1)$, hence $n_p=1$. Therefore by the above argument
        \begin{equation*}
            G\cong\frac{\Z}{p\Z}\times\frac{\Z}{q\Z}
        \end{equation*}
        \item[(b)] If $p\mid(q-1)$, then $n_p=1$ or $q$. Hence $G$ is either of the form above or isomorphic to the non-abelian group 
        \begin{equation*}
            G=P\rtimes Q
        \end{equation*}
    \textcolor{red}{not finished, what are the two generators}        
    \end{itemize}
\end{proof}

\begin{prob}[F2007-Q1]
    Prove that no group of order $148$ is simple.
\end{prob}
\begin{proof}
    We note the prime factorization of $148$ is 
    \begin{equation*}
        148=2^2\cdot 37
    \end{equation*}
    We see that $n_{37}\mid 4$ and $n_{37}\equiv 1\mod 37$, therefore $n_{37}=1$. This shows that there exists a normal subgroup of order $37$, i.e., the group is not simple.
\end{proof}


\begin{prob}[F2017-Q1]
    Show that there is no simple group of order $30$.
\end{prob}
\begin{proof}
    This is slightly more complicated, and we will use a counting argument. 
    Same reasoning as the above. The prime factorization of $30$ is as below:
    \begin{equation*}
        30=2\cdot 3\cdot 5
    \end{equation*}
    We see $n_5\mid 6$, and $n_5\equiv 1\mod 5$. Unfortunately, $n_5$ could either be $1$ or $6$. Now $n_3\mid 10$, and $n_3\equiv 1\mod 3$, unfortunately again $n_3$ could be $10$. However, we argue that $n_3=10$ and $n_5=6$ cannot happen at the same time. Suppose this is the case, then there are $20$ elements of order $2$ and $24$ elements of order $5$, but this is too many! Hence either $n_3=1$ or $n_5=1$, as desired.
\end{proof}



\begin{prob}[Richard Borcherds]
    All groups of order less than $60$ are solvable, i.e., there exists a sequence of subgroups of $G$, $G_0, \dots, G_k$ such that $G_i$ is normal in $G_{i+1}$ and $G_{i+1}/G_i$ is abelian, and 
    \begin{equation*}
        1=G_0\subset\dots\subset G_k=G
    \end{equation*}
\end{prob}
\textcolor{red}{...}


\begin{prob}[F2011-Q1]
    \phantom{}
        \begin{itemize}
            \item[(a)] Let \( G \) be a group of order 5046. Show that \( G \) cannot be a simple group. You may not appeal to the classification of finite simple groups.
            
            \item[(b)] Let \( p \) and \( q \) be prime numbers. Show that any group of order \( p^2q \) is solvable.
        \end{itemize}   
\end{prob}
\begin{proof}
    \begin{itemize}
        \item[(a)] The prime factorization of $5049$ is as follows:
        \begin{equation*}
            5049=2\cdot 3\cdot 29^2
        \end{equation*}
        Hence we see $n_{29}=1$, i.e., there is a normal subgroup of order $29$, therefore not simple.
        \item[(b)] We will do discussion by cases.
        \begin{itemize}
            \item[(1)] $p>q$. Then $n_p=1$ or $q$ and $n_p\equiv 1\mod p$, therefore $n_p=1$. Let $P$ be the normal subgroup of $G$ of order $p^2$, we thus have 
            \begin{equation*}
                \{e\}\subset P\subset G
            \end{equation*}
            It is clear that $|G/P|=q$, thus abelian, and $|P|=p^2$ also abelian as well (by the lemma below). This shows that $G$ is solvable.
        \end{itemize}
    \end{itemize}
\end{proof}


\begin{lem}[$p^2$ abelian]
    Fix prime $p$, any group of order $p^2$ is abelian.
\end{lem}
\begin{proof}
    
\end{proof}