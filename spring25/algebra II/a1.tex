% \chapter{1}

% We will first talk about algebraic sets and spaces.

% \begin{thm}[1.5 Hilbert's Nullstellnsatz, p380]
%     Let $a\subset k[x]$ be an ideal, where $x=(x_1,\dots,x_n)$, let $f\in k[x]$ be a polynomial such that $f(c)=0$ for every zero $c=(c_1,\dots, c_n)$ of $a\subset k[x]$, then there exists $m\in\mathbb{N}$ such that 
%     \begin{equation*}
%         f^m\in a
%     \end{equation*}
%     Note that $f^m\in a \iff f\in\sqrt{a}\supset a$.
% \end{thm}
% \begin{proof}
%     Suppose $f\neq 0$, and $a$ has a zero. 
% \end{proof}
% The following theorem further characterizes ideals.
% \begin{thm}[1.4]
%     Let $a\subset k[x]=k[x_1,\dots,x_n]$, then either $a=k[x]$ or $a$ has a zero in $k^a$. Now we introduce a new variable $Y$, and consider the ideal $a'$ generated by $a$ and $1-Yf$. By the following theorem, we know that $a'$ must be the entire ring $k[X,Y]$. This implies that there exists polynomials $g_j\in k[X,Y], h_i\in a$ such that 
%     \begin{equation*}
%         1=g_0(1-Yf)+g_1h_1+\dots+g_rh_r
%     \end{equation*}
%     Substitute $Y$ with $f^{-1}$, and multiply both sides by $f^m$, 
%     \begin{equation*}
%         f^m=f^m(g_1h_1+\dots+g_rh_r)\in a
%     \end{equation*}
%     and we are done.
% \end{thm}

% \begin{thm}[1.1 p378]
%     Let $k$ be a field, and $k[x]=k[x_1,\dots, x_n]$ be a finitely generated ring. If $\varphi: k\to L$ is an embedding of an algebraically closed field $L$, then there exists an extension $\bar{\varphi}:k[X]\to L$.
% \end{thm}
% An immediate corollary is as follows:
% \begin{cor}[1.2 p389]
%     Let $k$ be a field and $k[x]$ be finitely generated over $k$. If $k[x]$ is a field, then $k[x]$ is algebraically closed over $k$.
% \end{cor}

% \begin{prop}[3.1 p347]
%     Let $A$ be a subring of $B$, and $B$ is integral over $A$, let $\varphi:A\to L$ where $L$ is a field that is algebraically closed, then $\varphi$ extends to a homomorphism of $B$ to $L$.
% \end{prop}

% Next we proved theorem 1.4 above. I cannot do this.



\chapter{Group Theory}
\textcolor{red}{S2013-Q2, S2016-Q1, F2018-Q2, F2001-Q1, F2013-Q}

\textcolor{blue}{S2005-Q1, F2009-Q1}


\section{Sylow Theorems}
We first talk bout semidirect products. Let $G$ be any group, and $N,H$ be subgroups of $G$.
\begin{defn}
For $\varphi:H\to\text{Aut}(N)$, define $N\rtimes H$ by 
\begin{itemize}
    \item[(1)] $N\rtimes_\varphi H=N\times H$ as a set.
    \item[(b)] Equipped with the group structure 
    \begin{equation*}
        (n_1,h_1)\cdot (n_2,h_2)=(n_1\varphi(h_1)n_2, h_1h_2)
    \end{equation*}
    The structure $(N\rtimes_\varphi H, \cdot)$ forms a group. 
\end{itemize}
\end{defn}

\begin{example}
    If $N$ is a normal subgroup of $G$, and $N\cap H=\{e\}$, and $\varphi:H\to\text{Aut}(N)$ where 
    \begin{equation*}
        \varphi: h\mapsto (n\mapsto hnh^{-1})
    \end{equation*}
    (acting by conjugation), and $G=NH$. Then 
    \begin{equation*}
        N\rtimes_\varphi H\to G
    \end{equation*}
    where 
    \begin{equation*}
        (n,h)\mapsto nh
    \end{equation*}
    is a bijective homomorphism homomorphism. Hence 
    \begin{equation*}
        G\cong N\rtimes_\varphi H
    \end{equation*}
\end{example}

% \textcolor{red}{what is happening}

Next we present some divisibility results.
\begin{prop}[Lagrange, Orbit-Stabilizer]
    We have the following divisibility results:
    \begin{itemize} 
        \item Let $H$ be a subgroup of $G$, let $[G:H]$ denote the number of cosets of $H$ in $G$, then 
        \begin{equation*}
            |G|=|H|[G:H]
        \end{equation*}
        \item Let $G$ be a finite group acting transitively on a finite set $A$, then for any $a\in A$, we have 
        \begin{equation*}
            |\text{Stab}_G(a)|\cdot|O_G(a)|=|G|
        \end{equation*}
    \end{itemize} 
\end{prop}
The class formula is when $G$ acts on itself by conjugation:
\begin{prop}[class formula]
    Let $G$ act on a finite set $S$, and let $Z$ denote fixed points of this action, then 
    \begin{equation*}
        |S|=|Z|+\sum_{a\in A}|O_G(a)|
    \end{equation*}
    where $A$ includes exactly one element from each nontrivial orbit.

    If $G$ acts on itself by conjugation, then 
    \begin{equation*}
        |G|=|Z(G)|+\sum_{g}|[g]|=|Z(G)|+\sum_{g}\frac{|G|}{|C_G(g)|}
    \end{equation*}
    where $[g]$ denote the conjugacy class of $g$, and the sum includes exactly one from each nontrivial conjugacy class in $G$.
\end{prop}


\begin{prob}[F2019-Q2]
    2. Let \( p, q \) be two prime numbers such that \( p \mid q - 1 \). Prove that  
    \begin{itemize}
    \item[(a)]there exists an integer \( r \neq 1 \mod q \) such that \( r^p \equiv 1 \mod q \);
    \item[(b)] there exists (up to an isomorphism) only one noncommutative group of order \( pq \).
    \end{itemize}
\end{prob}
\begin{proof}
    \begin{itemize}
        \item[(a)] We want to show that there exists an element $r\in(\Z/q\Z)^\times$ such that 
        \begin{equation*}
            r^p\equiv 1\mod q
        \end{equation*}
        We can do this because $(\Z/q\Z)^\times$ has order $(q-1)$ and $p\vert (q-1)$. Therefore by Cauchy's theorem, there exists an element of order $p$ in $(\Z/p\Z)^\times$.
        \item[(b)] Let $n_p,n_q$ denote the number of $p$, $q$-Sylow subgroups. We see that $n_q\vert p$ and $n_q\equiv 1\mod q$, since $p<q$, we must have $n_q=1$. Now $n_p=1$ or $q$ by the same reasoning. Suppose $n_q=1$, let $P,Q$ denote the normal subgroups of order $p,q$, then
        \begin{equation*}
            G\cong P\times Q
        \end{equation*}
        by a standard argument (included in the lemma below). Then $G$ is commutative. Since $G$ is noncommutative, we have $n_p=q$. Choose any $p$-Sylow subgroup $P$, we know that 
        \begin{equation*}
            G\cong Q\rtimes_\theta P
        \end{equation*} 
        where $Q$ is the normal subgroup of order $q$ and $\theta: P\to\text{Aut}(Q)=(\Z/q\Z)^\times$. We know either $\theta: 1\mapsto 1$, is the trivial map which produces  a commutative group; or $\theta: 1\mapsto r$, where $r\in(\Z/q\Z)^\times$ is some element of order $p$. 
        
        
        % By (a), such an element exists. Since $(q-1)$ is even, we know there only exists $1$ element in $(\Z/q\Z)^\times=\Z/(q-1)\Z$ of order $2$, namely, $r$. Therefore 
        % \begin{equation*}
        %     G\cong Q\rtimes_\theta P
        % \end{equation*}
        % where $\theta: P\to\text{Aut}(Q)=(\Z/q\Z)^\times$ is such that $\theta: 1\mapsto r$ defines the noncommutative group of order $pq$ up to isomorphisms.
    \end{itemize}
\end{proof}



\begin{lem}
    Let $p,q$ be two primes such that $q\nmid (p-1)$, and $N$, $H$ has order $p,q$ respectively, suppose that $N$ is normal in $G$, and $N\cap H=\{e\}$, then 
    \begin{equation*}
        G\cong N\times H
    \end{equation*}
\end{lem}
\begin{proof}
    We consider the map 
    \begin{equation*}
        \psi: N\times H\to G
    \end{equation*}
    such that 
    \begin{equation*}
        (n,h)\mapsto nh
    \end{equation*}
    We want to show that $\psi$ is a homomorphism and $\psi$ is injective (hence bijective by size argument). It is clearly injective: 
    \begin{equation*}
        nh=e\Rightarrow n, h\in N\cap H=\{e\}
    \end{equation*}
    It suffices to show that $\psi$ is a homomorphism. We see that this implies 
    \begin{equation*}
        n_1n_2h_1h_2=n_1h_1n_2h_2
    \end{equation*}
    Therefore it suffices to for any $n\in N, h\in H$, one has
    \begin{equation*}
        nh=hn
    \end{equation*}
    Consider the conjugation action 
    \begin{equation*}
        \varphi: H\to \text{Aut}(N)
    \end{equation*}
    where 
    \begin{equation*}
        h\mapsto \left( n\mapsto hnh^{-1}\right)
    \end{equation*}
    Then we claim that $\varphi$ is trivial. This is because $\ker(\varphi)$ has size either $1$ or $q$. If it has size $q$, then the map is trivial; if it has size $1$, then $H$ embeds in $\text{Aut}(N)$, however, $|H|=q, \text{Aut}(N)=p-1$, and $q\nmid(p-1)$, hence impossible. This shows that the map is trivial, i.e., for $n\in N, h\in H$, 
    \begin{equation*}
        hn=nh
    \end{equation*}
    as desired. 
\end{proof}


\begin{prob}[F2015-Q1]
    Prove every group of order $15$ is cyclic.
\end{prob}
\begin{proof}
    We will show that any group $G$ of order $15$ is isomorphic to 
    \begin{equation*}
        G\cong\frac{\Z}{3\Z}\times\frac{\Z}{5\Z}
    \end{equation*}
    For this, using the above lemma, it suffices to show that there is one normal subgroup of order $3$ and one normal subgroup of order $5$. We repeat the argument above, $n_5\mid 3$ and $n_5\equiv 1\mod 5$, hence $n_5=1$. Moreover, $n_3\mid 5$ and $n_3\equiv 1\mod 3$, hence $n_3=1$ as well. By the lemma above, we know that 
    \begin{equation*}
        G\cong\frac{\Z}{3\Z}\times\frac{\Z}{5\Z}
    \end{equation*}
    hence cyclic as desired.
\end{proof}

\begin{prob}[S2013-Q2]
Let \( p \) and \( q \) be primes with \( p < q \). Let \( G \) be a group of order \( pq \). Prove the following statements:

\begin{itemize}
    \item[(a)] If \( p \) does not divide \( q - 1 \) (i.e., \( p \nmid q - 1 \)), then \( G \) is cyclic.

    \item[(b)] If \( p \) divides \( q - 1 \) (i.e., \( p \mid q - 1 \)), then \( G \) is either cyclic or isomorphic to a non-abelian group on two generators. Give the presentation of this non-abelian group.
\end{itemize}
\end{prob}
\begin{proof}
    This question is exactly the same as  F19-Q2, we will only outline here.
    \begin{itemize}
        \item[(a)] We have $n_q=1$, and $n_p\mid q$, hence $n_p=1$ or $q$, moreover $n_p\equiv 1\mod p$. If $n_p=q$, this implies that $p\mid(q-1)$, hence $n_p=1$. Therefore by the above argument
        \begin{equation*}
            G\cong\frac{\Z}{p\Z}\times\frac{\Z}{q\Z}
        \end{equation*}
        \item[(b)] If $p\mid(q-1)$, then $n_p=1$ or $q$. Hence $G$ is either of the form above or isomorphic to the non-abelian group 
        \begin{equation*}
            G=Q\rtimes_\theta P
        \end{equation*}
        We know from F2019-Q2, the trivial $\theta$ defines the abelian, hence cyclic group $G=\frac{\Z}{p\Z}\times\frac{\Z}{q\Z}$. And $\theta: 1\mapsto r$, for some $r\in(\Z/q\Z)^\times$ of order $p$ defines a non-abelian group. So we have 
        \begin{equation*}
            G=\la g,h: g^q=h^p=e, hgh^{-1}=g^r\ra
        \end{equation*}
    \end{itemize}
\end{proof}

\begin{prob}[F2007-Q1]
    Prove that no group of order $148$ is simple.
\end{prob}
\begin{proof}
    We note the prime factorization of $148$ is 
    \begin{equation*}
        148=2^2\cdot 37
    \end{equation*}
    We see that $n_{37}\mid 4$ and $n_{37}\equiv 1\mod 37$, therefore $n_{37}=1$. This shows that there exists a normal subgroup of order $37$, i.e., the group is not simple.
\end{proof}


\begin{prob}[F2017-Q1]
    Show that there is no simple group of order $30$.
\end{prob}
\begin{proof}
    This is slightly more complicated, and we will use a counting argument. 
    Same reasoning as the above. The prime factorization of $30$ is as below:
    \begin{equation*}
        30=2\cdot 3\cdot 5
    \end{equation*}
    We see $n_5\mid 6$, and $n_5\equiv 1\mod 5$. Unfortunately, $n_5$ could either be $1$ or $6$. Now $n_3\mid 10$, and $n_3\equiv 1\mod 3$, unfortunately again $n_3$ could be $10$. However, we argue that $n_3=10$ and $n_5=6$ cannot happen at the same time. Suppose this is the case, then there are $20$ elements of order $2$ and $24$ elements of order $5$, but this is too many! Hence either $n_3=1$ or $n_5=1$, as desired.
\end{proof}




\begin{prob}[F2011-Q1]
    \phantom{}
        \begin{itemize}
            \item[(a)] Let \( G \) be a group of order 5046. Show that \( G \) cannot be a simple group. You may not appeal to the classification of finite simple groups.
            
            \item[(b)] Let \( p \) and \( q \) be prime numbers. Show that any group of order \( p^2q \) is solvable.
        \end{itemize}   
\end{prob}
\begin{proof} 
    The proof is very similar like above.
    \begin{itemize}
        \item[(a)] The prime factorization of $5049$ is as follows:
        \begin{equation*}
            5049=2\cdot 3\cdot 29^2
        \end{equation*}
        Hence we see $n_{29}=1$, i.e., there is a normal subgroup of order $29$, therefore not simple.
        \item[(b)] We will do discussion by cases.
        \begin{itemize}
            \item[(1)] $p>q$. Then $n_p=1$ or $q$ and $n_p\equiv 1\mod p$, therefore $n_p=1$. Let $P$ be the normal subgroup of $G$ of order $p^2$, we thus have 
            \begin{equation*}
                \{e\}\subset P\subset G
            \end{equation*}
            It is clear that $|G/P|=q$, thus abelian, and $|P|=p^2$ also abelian as well (by the lemma below). This shows that $G$ is solvable.
            \item[(2)] $p<q$. Then $n_p=1$ or $q$, and $n_q=1$ or $p^2$. Suppose that $n_q=1$, let $Q$ denote the normal subgroup of order $q$, then 
            \begin{equation*}
                \{e\}\subset Q\subset G
            \end{equation*}
            It is clear that $Q$ and $G/Q$ are both abelian. Suppose that $n_q=p^2$ instead, then there are only $p^2q-p^2(q-1)=p^2$ elements of order $\neq q$. Since any $p$-Sylow subgroup has $p^2$ elements with order $\neq q$, we must have $n_p=1$. 
            Hence we are in case (1) again. This shows that $G$ is solvable in either case $n_q=1, p^2$.
        \end{itemize}
    \end{itemize}
\end{proof}


\begin{lem}[$p^2$ abelian]
    Fix prime $p$, any group of order $p^2$ is abelian. 
\end{lem}
\begin{proof}
    For any nontrivial $p$ group, by the class formula, the center $Z(G)$ is nontrivial, thus the center has order either $p$ or $p^2$. If it has order $p^2$, then the group is abelian. If it has order $p$, then 
    \begin{equation*}
        \left|G/Z(G)\right|=p
    \end{equation*}
    is also cyclic, therefore $G$ is abelian (strictly speaking is a contradiction that $|Z(G)|=p$). In either case, we see that $G$ is abelian.
\end{proof}

\begin{prob}
    Any $p$-group is solvable, for any prime $p$. 
\end{prob}
\begin{proof}
    Suppose $|G|=p^r$ for some $r\geq 0$, we will use induction on $r$. If $r=0$, then the trivial group is trivally solvable. 
    \begin{itemize}
        \item Base case: if $r=1$, $|G|=p$, then $G$ is cyclic, hence solvable. 
        \item Induction step: suppose that $G$ is solvable for all $|G|=p^k$, where $0\leq k\leq r-1$. Now we want to show that $G$ of order $p^r$ is solvable. We know $G$ has a nontrivial center, suppose that $|Z(G)|=p^k$, where $1\leq k\leq r$, then 
        \begin{equation*}
            |G/Z(G)|=p^{r-k}, 0\leq r-k\leq r-1 
        \end{equation*}
        We know any group $G$ is solvable if and only if there exists a sequence of subgroups $G_0,\dots,G_k$
        \begin{equation*}
            \{e\}=G_0\subset\dots\subset G_k=G
        \end{equation*}
        such that $G_{i-1}$ is normal in $G_i$ and $G_i/G_{i-1}$ is solvable. Therefore we see when $|G|=p^{r}$, 
        \begin{equation*}
            \{e\}\subset Z(G)\subset G
        \end{equation*}
        has $Z(G)$ solvable, and $G/Z(G)$ also solvable by the induction hypothesis, so we close the induction. 
    \end{itemize}
    

    % We know that any $p$-group $G$ has nontrivial center $Z(G)$. Suppose that $|G|=p^r$ for some $r$. If $|Z(G)|=p^r$, then $G$ is solvable. If $|Z(G)|=p^{r-1}$, then 
    %  Then we also have $G$ is solvable. Suppose that $|Z(G)|=p^{r-2}$, then $|Z(G)|$ is 
\end{proof}

\begin{prob}[S2016-Q1]
    Classify all groups of order $66$, up to isomorphism.
\end{prob}
\begin{proof}
    By $66=2\cdot 3\cdot 11$, we know $n_{11}=1$. We claim that there is a normal subgroup isomorphic to $\Z/33\Z$. 
    \begin{enumerate}
        \item First we show that there is a subgroup of order $33$. Let $P_{11}$ denote the normal subgroup of order $11$ and let $P_{3}$ denote a $3$-Sylow subgroup of $G$. Then we claim that the following
        \begin{equation*}
            H=\{gh: g\in P_{11}, h\in P_{3}\}
        \end{equation*}
        forms a subgroup and is isomorphic to $\Z/33\Z$. By the Lemma 1.1, we see that 
        \begin{equation*}
            H\cong\frac{\Z}{11\Z}\times\frac{\Z}{3\Z}=\frac{\Z}{33\Z}
        \end{equation*}
        \item Now we show that it is normal. This follows from the following general lemma:
        \begin{lem}
            Let $p$ be the smallest prime factor of $|G|$, and let $H$ be a subgroup with index $p$, then $H$ is normal.
        \end{lem}
        \begin{proof}
            We will only prove in the case that $H$ is a subgroup of index $2$, i.e., $G=H\sqcup (G\setminus H)$. We see for all $g\in G$, 
            \begin{equation*}
                gH=Hg
            \end{equation*}
            since if $g\in H$, then the equality holds; if $g\not\in H$, then $gH=G\setminus H$, so is $Hg$.  
        \end{proof}
        Now since there is a subgroup of order $2$, we can write $G$ as a semidirect product 
        \begin{equation*}
            G=\frac{\Z}{33\Z}\rtimes_\theta \frac{\Z}{2\Z}
        \end{equation*}
        The number of nonisomorphic groups will depend on the choice of $\theta$. There are four different choices for $\theta: H\to\text{Aut}\left(\frac{\Z}{11\Z}\times\frac{\Z}{3\Z}\right)=\frac{\Z}{10\Z}\times\frac{\Z}{2\Z}$
        \begin{equation*}
            \begin{cases}
                \theta_1: 1\mapsto (0,0)\\
                \theta_2: 1\mapsto (0,1)\\
                \theta_3: 1\mapsto (5, 0)\\
                \theta_4: 1\mapsto (5,1)
            \end{cases}
        \end{equation*}
    \end{enumerate} 

    \textcolor{red}{what is happening}
\end{proof}



\begin{prob}[S2007-Q2]
    Prove that no group of order $224$ is simple.
\end{prob}
\begin{proof}
    The prime factorization is 
    \begin{equation*}
        224=2^5\cdot 7
    \end{equation*}
    If $n_2=1$ or $n_7=1$, then we are done; assume that $n_2=7$ instead, then we recall $G$ has a nontrivial transitive action on the set of $2$-Sylow subgroups, i.e., there is a homomorphism $\varphi:G\to S_7$. We know $\ker(\varphi)$ is a normal subgroup of $G$. Since the action is nontrivial transitive, we know $\ker(\varphi)\neq G$. If $\ker(\varphi)=\{e\}$, then $\varphi$ produces an embedding of $G$ into $S_7$. However, $|G|=224\nmid |S_7|$. This shows that $\ker(\varphi)$ is a nontrivial proper normal subgroup of $G$, concluding that $G$ is not simple.
\end{proof}

\begin{prob}[F2008-Q1]
    Show that no group of order $36$ is simple.
\end{prob}
\begin{proof}
    \begin{equation*}
        36=2^2\cdot 3^3
    \end{equation*}
    We know $n_2\mid 9, n_2\equiv 1\mod 2$, and $n_3\mid 4, n_3\equiv 1\mod 3$. We know $n_3=1$ or $4$, suppose that $n_3=4$, then there is a nontrivial action of $G$ on the set of $3$-Sylow subgroups, i.e., 
    \begin{equation*}
        \varphi: G\to S_4
    \end{equation*}
    Suppose that $G$ is simple, we know $\ker(\varphi)\neq G$ since the action is nontrivial, by assumption $\ker(\varphi)=\{e\}$, which implies that $\varphi$ is an embedding, but $|G|=32\nmid |S_4|$, which is a contradiction. This implies that $G$ is not simple.
\end{proof}

\begin{prob}[S2014-Q2]
    All groups of order less than $60$ are solvable, i.e., there exists a sequence of subgroups of $G$, $G_0, \dots, G_k$ such that $G_i$ is normal in $G_{i+1}$ and $G_{i+1}/G_i$ is abelian, and 
    \begin{equation*}
        1=G_0\subset\dots\subset G_k=G
    \end{equation*}
\end{prob}
\begin{proof}
    Groups of order $p, pq, p^2, p^2q$ are solvable.
    \begin{align*}
        \big\{ 1, 2, 3, 4, 5, 6, 7, 9, 10, 11, 12, 13, 14, 15, 17, 19, 20, 21, 22, 23, 25, 26, 28, 29, 30, \\ 31, 33, 34, 35, 37, 38, 39, 41, 43, 44, 45, 46, 47, 49, 50, 51, 52, 53, 55, 57, 58, 59 \big\}
    \end{align*}
    And any $p$-group is also solvable. 
    \begin{equation*}
        \{8, 16, 27, 32\}
    \end{equation*}
    The remaining groups are 
    \begin{equation*}
        \left\{24, 36, 40, 42, 48, 54, 56 \right\}
    \end{equation*}
    \begin{itemize}
        \item[24:] If $n_2=1$ or $n_3=1$, then we are done. We see $n_2=1$ or $3$, consider the action $\varphi:G\to S_3$. We see $\ker(\varphi)$ is a proper normal subgroup of $G$, this implies that 
        \begin{equation*}
            \{e\}\subset \ker(\varphi)\subset G
        \end{equation*}
        where $|\ker(\varphi)|$ is a known solvable group, hence we are done.
        \item[36:] Exactly same as above, we assume $n_3\neq 1$, therefore $n_3=4$, the action $\varphi: G\to S_4$ is not injective, hence $\ker(\varphi)$ is again a proper normal subgroup of $G$ that is solvable.
        \item[40:] We see $n_5=1$, therefore 
        \begin{equation*}
            \{e\}\subset \Z/5\Z\subset G
        \end{equation*}
        \item[42:] We see $n_7=1$.
        \item[48:] We see $n_2=1$ or $3$, the the action $\varphi: G\to S_3$ is not injective, hence $\ker(\varphi)$ is a proper normal subgroup of $G$ that is solvable.
        \item[54:] We see $n_3=1$. 
        \item[56:] We know $n_7=1$ or $8$ and $n_2=1$ or $7$. The group action argument does not work. We assume $n_7=8$, then there can be at most $56-8(7-1)=8$ elements of order $\neq 7$. This shows that $n_2=1$. Hence 
        \begin{equation*}
            \{e\}\subset P_2\subset G
        \end{equation*}
    \end{itemize}
\end{proof}

\begin{prob}[S2012-Q1]
    Let \( G \) be a group of order \( p^3 q^2 \), where \( p \) and \( q \) are prime integers. Show that for \( p \) sufficiently large and \( q \) fixed, \( G \) contains a normal subgroup other than \(\{1\}\) and \( G \).
\end{prob}
\begin{proof}
    We want to show that there exists a normal group of size $p^3$, i.e., $n_p=1$. We know $n_p\mid q^2, n_p\equiv 1\mod p$. Let $p$ be large enough such that $p>(q^2-1)$, then ths forces $n_p=1$, as desired.
\end{proof}


% Semidirect product: $N$ is normal, then there is an action of $H$ on $\text{Aut}(N)$, if $|H|=p$, then this map has kernel either be trivial or the entire group, in the case that it is trivial, this map is an embedding.

% Same argument for the simple group question.


\begin{prob}[F2014-Q4]
    \phantom{text}
    \begin{itemize}
        \item[(a)] Let \( G \) be a group of order \( p^2q^2 \), where \( p \) and \( q \) are distinct odd primes, with \( p > q \). Show that \( G \) has a normal subgroup of order \( p^2 \).
        
        \item[(b)] Can a solvable group contain a non-solvable subgroup? Explain.
    \end{itemize}
\end{prob}
\begin{proof}
    \begin{itemize}
        \item[(a)] We know $n_p=1$ or $q$ or $q^2$, and $n_p\equiv 1\mod p$. Since $p>q$, we know $n_p\neq q$. It suffices to show that $n_p\neq q^2$: suppose that $n_p=q^2$, then 
        \begin{equation*}
            p\mid (q^2-1)=(q+1)(q-1)
        \end{equation*}
        Since $p$ is prime, $p\mid (q+1)$ or $p\mid(q-1)$. The latter impossible since $q<p$. $p\mid (q+1)$ is also impossible because this implies that $q=p+1$, which implies that $q$ is even, a contradiction.
        \item[(b)] It is not possible. Suppose $G$ is a solvable group, let $H$ be a subgroup of $G$, then we know there exists sequence 
        \begin{equation*}
            \{e\}=G_0\subset\dots\subset G_k=G
        \end{equation*}
        such that $G_i$ is normal in $G_{i+1}$ and $\frac{G_{i+1}}{G_i}$ is abelian. We define $H_i=G_i\cap H$, then we see $H$ is solvable with sequence $H_0\subset\dots H_k$.
    \end{itemize}

\end{proof}


\begin{prob}[F2018-Q2]
    Let \( G \) be a group of order \( 24 \). Assume that no Sylow subgroup of \( G \) is normal in \( G \). Show that \( G \) is isomorphic to the symmetric group \( S_4 \).
\end{prob}
\begin{proof}
    By Sylow, we have $n_3=4, n_2=3$. Denote Syw$_3(G)=\{P_1,P_2,P_3,P_4\}$ and consider the transitive action by of $G$ by conjugation on this set, which embeds in $S_4$, i.e., $\varphi: G\to S_4$. By a size argument, it suffices to show that $\varphi$ is injective. We see that 
    \begin{equation*}
        \ker(\varphi)=\{g\in G: gP_ig^{-1}=P_i \text{ for each }i\}=\bigcap_{i=1}^4N_G(P_i)
    \end{equation*}
    By the orbit-stabilizer theorem, $|N_G(P_i)|=6$ for all $i$. However, for any $i\neq j$, $3$ does not divide $|N_G(P_i)\cap N_G(P_j)|$: if not, the intersection would include a $3$-Sylow subgroup but $P_i$ is the only $3$-Sylow subgroup in $N_G(P_i)$, thus this is impossible. It remains to see that $|\ker(\varphi)|\neq 2$. Suppose that it is, then $\im(\varphi)$ is an index $2$ subgroup of $S_4$, hence 
    \begin{equation*}
        \frac{G}{\ker\varphi}\cong A_4
    \end{equation*}
    and $K=\Z/2\Z\times\Z/2\Z$ is normal in $A_4$, hence so is $\varphi^{-1}(K)$ (it has size $8$) in $G$. This is a contradiction because this implies there is a normal $2$-Sylow subgroup.
\end{proof}


\begin{prob}[F2001-Q1]
    Let \( G \) be a finite group and let \( N \) be a normal subgroup of \( G \) such that \( N \) and \( G/N \) have relatively prime orders.
        
        \begin{enumerate}
            \item Assume that there exists a subgroup \( H \) of \( G \) having the same order as \( G/N \). Show that \( G = HN \). (Here \( HN \) denotes the set \(\{xy \mid x \in H, y \in N\}\).)
            
            \item Show that \( \phi(N) = N \), for all automorphisms \( \phi \) of \( G \).
        \end{enumerate}
\end{prob}
\begin{proof}
    \begin{enumerate}
        \item Since $N,H$ have relatively prime orders, $N\cap H=\{e\}$, thus we can write 
        \begin{equation*}
            G=N\rtimes_\theta H
        \end{equation*}
        where $\theta(h)n=hnh^{-1}$. One can show that the map $\varphi: N\rtimes_\theta H\to G$ as
        \begin{equation*}
            \varphi:(n,h)\mapsto nh
        \end{equation*}
        It is clear that $\varphi$ is a homomorphism and injective, thus by a size argument we have $\varphi$ is an isomorphism. This shows $G=NH$ and similarly $G=HN$.
        \item Any automorphism $\phi$ of $G$ permutes the $p$-Sylow subgroups. Suppose that $|G|=p_1^{i_1}\dots p_k^{i_k}$, then after rearranging,
        \begin{equation*}
            |N|=p_1^{i_1}\dots p_j^{i_j}
        \end{equation*}
        because $N$ and $G/N$ have relatively prime orders. Hence $N$ contains all the Sylow $p_i$-subgroups, hence $\phi(N)=N$ for all automorphisms $\phi$ of $G$.
    \end{enumerate}
\end{proof}


\begin{prob}[S2001-Q1]
    Let \( G \) be a finite group and \( p \) the smallest prime number dividing the order \( |G| \) of \( G \). Let \( H \) be a subgroup of \( G \) of index \( p \) in \( G \). Show that \( H \) is necessarily a normal subgroup of \( G \).
\end{prob}
\begin{proof}
    $G$ has an action on $G/H$ by left multiplication: $\varphi: G\to\text{Aut}(G/H)$ such that 
    \begin{equation*}
        \varphi(g)(\bar{g}H)=g\bar{g}H
    \end{equation*}
    We will show that $H=\ker(\varphi)$. First we see that $\ker(\varphi)\subset H$: 
    \begin{equation*}
        \ker(\varphi)=\{g\in G: g\bar{g}H=\bar{g}H: \text{ for all }\bar{g}\in G\}
    \end{equation*}
    letting $\bar{g}\in H$ we see $g\in\ker(\varphi)$ implies $g\in H$, i.e., $\ker(\varphi)\subset H$. 

    Now we use a size argument to show $|H|\leq|ker\varphi|$. We note that $\im(\varphi)$ is a subgroup of $\text{Aut}(G/H)=S_p$, thus 
    \begin{equation*}
        \frac{|G|}{|\ker(\varphi)|}\text{ divides } p!
    \end{equation*}
    because $\frac{|G|}{|\ker(\varphi)|}$ also divides $|G|$ and $p$ is the smallest prime that divides $p$, we must have 
    \begin{equation*}
        \frac{|G|}{|\ker(\varphi)|} \text{ divides } p
    \end{equation*}
    Note that $\frac{|G|}{|H|}=p$, this gives 
    \begin{equation*}
        |H|\leq|\ker(\varphi)|
    \end{equation*}
    which shows $H\subset\ker(\varphi)$, hence $H=\ker(\varphi)$.
\end{proof}




\begin{center}\textcolor{pink}{(End of Page 5)}
\end{center}

\section{Class Formula, Classification of $p$-groups}
\begin{defn}[nilpotent group]
    Let $G$ be a group. Define inductively an increasing sequence 
$\{e\}=Z_0 \subseteq Z_1 \subseteq Z_2 \subseteq \cdots$ of subgroups of $G$ 
as follows: for $i \geq 1$, $Z_i$ is the subgroup of $G$ corresponding 
to the center of $G/Z_{i-1}$. One can show that $Z_i$ is normal in $G$. A group is \emph{nilpotent} if $Z_m = G$ for some $m$.
\end{defn}
\begin{example}
    \phantom{text}
    \begin{itemize}
        \item $p$-groups are nilpotent.
        \item Nilpotent groups are solvable.
    \end{itemize}
\end{example}
\begin{prop}
    We have the following classification of groups of order $p, p^2, p^3$, for prime $p$.
    \begin{itemize}
        \item $|G|=p$ implies $G\cong\Z/p\Z$.
        \item $|G|=p^2$ implies 
        \begin{equation*}
            G\cong\frac{\Z}{p^2\Z} \quad\text{ or }\quad G\cong\frac{\Z}{p\Z}\oplus\frac{\Z}{p\Z}
        \end{equation*}
        \item $|G|=p^3$ implies that 
        \begin{equation*}
            G\cong\frac{\Z}{p^3\Z} \quad \text{ or } \quad G/Z(G)\cong\frac{\Z}{p\Z}\oplus\frac{\Z}{p\Z} \quad\text{ or }\quad [G,G]=Z(G)
        \end{equation*}
    \end{itemize}
\end{prop}
% \begin{proof}
%     \textcolor{red}{finish proof}
% \end{proof}

\begin{prob}[S2010-Q1]
    Let \( G \) be a non-abelian group of order \( p^3 \), where \( p \) is prime. Determine the number of distinct conjugacy classes in \( G \).
\end{prob}
\begin{proof}
    We know $G$ has a nontrivial center, and if $|Z(G)|=p^2$ or $p^3$, then $G$ is abelian, this shows that $|Z(G)|=p$, now let $g\in G\setminus Z(G)$, then 
    \begin{equation*}
        Z(G)\subsetneq Z_g(G)\subsetneq G
    \end{equation*}
    where $Z(G)\subsetneq Z_g(G)$ because $g\in Z_g(G)$, and $Z_g(G)\subsetneq G$ since $g\not\in Z(G)$. This shows that $Z_g(G)$ is a subgroup of order $p^2$, in other words, the size of the conjugacy class of any $g\in G\setminus Z(G)$ is 
    \begin{equation*}
        |[g]|=\left|\frac{G}{Z_g(G)}\right|=p
    \end{equation*}
    By the class formula, 
    \begin{equation*}
        |G|=|Z(G)|+\sum_{a\in A}|[a]|
    \end{equation*}
    where $A$ contains one $a$ from each nontrivial conjugacy class $[a]$. Thus we have 
    \begin{equation*}
        p^3=p+Np\Rightarrow N=p^2-1
    \end{equation*}
    Every element in $Z(G)$ is its own conjugacy class, thus the total number of conjugacy classes is 
    \begin{equation*}
        p^2+p-1
    \end{equation*}
\end{proof}

\begin{prob}[F2013-Q1]
    Let $p>2$ be a prime. Classify groups of order $p^3$ up to isomorphism. The two nonabelian groups of order \( p^3 \) (for \( p \neq 2 \)), up to isomorphism, are:

    \[
    \text{Heis}(\mathbb{Z}/(p)) = 
    \left\{ 
    \left( \begin{array}{ccc}
    1 & a & b \\
    0 & 1 & c \\
    0 & 0 & 1
    \end{array} \right) \middle| a, b, c \in \mathbb{Z}/(p)
    \right\}
    \]
    
    and
    
    \[
    G_p = 
    \left\{ 
    \left( \begin{array}{cc}
    a & b \\
    0 & 1
    \end{array} \right) \middle| a, b \in \mathbb{Z}/(p^2), a \equiv 1 \bmod p
    \right\}
    \]
    
    \[
    = 
    \left\{ 
    \left( \begin{array}{cc}
    1 + pm & b \\
    0 & 1
    \end{array} \right) \middle| m, b \in \mathbb{Z}/(p^2)
    \right\}
    \]
\end{prob}

\begin{prob}[F2014-Q5]
    \phantom{text}
    \begin{enumerate}
        \item[(a)] Prove that every group of order \( p^2 \) (with \( p \) prime) is abelian. Then classify such groups up to isomorphism.
    
        \item[(b)] Give an example of a non-abelian group of order \( p^3 \) for \( p = 3 \). \\
        \emph{Suggestion: Represent the group as a group of matrices.}
    \end{enumerate}    
\end{prob}
\begin{proof}
    \begin{itemize}
        \item[(a)] See Lemma 1.2. There are two abeliean groups: $\frac{\Z}{p^2\Z}, \frac{\Z}{p\Z}\times\frac{\Z}{p\Z}$.
        \item[(b)] See Problem 1.18.
    \end{itemize}
\end{proof}

\begin{prob}[F2019-Q4, S2015-Q3]
    Find all irreducible representations of a finite $p$-group over a field of characteristic $p$.
\end{prob}
\begin{proof}
    Let $G$ any finite $p$-group. Let $V$ be an irreducible representation over $\F_p$, consider the $[\F_pG]$-module $W$ generated by any $v\in V\setminus\{0\}$. We see $W$ is a finite-dimensional vector space over $\F_p$, i.e., 
    \begin{equation*}
        |W|=p^d
    \end{equation*}
    for some $d\geq 1$. We consider the action of $G$ on $W$, all the orbits of this action either has size $1$ or is a power of $p$, since $G$ is a $p$-group, by the class formula, let $N$ be the number of nontrivial orbits of size $1$, 
    \begin{equation*}
        |W|\equiv 1+N\mod p\Rightarrow 1+N\equiv 0\mod p
    \end{equation*}
    Hence there exists at least one nontrivial orbit $\{v\}$ of size $1$. We consider the vector space $\bar{W}$ generated by $v$ over $\F_p$: it is one-dimensional vector space contained in $V$, invariant under $G$, since $V$ is irreducible, we must have $V=\bar{W}$. The action of $G$ on $\bar{W}$ is the trivial action, thus all irreducible representations of a finite $p$-group over $\F_p$ are trivial.
\end{proof}


\section{Random Problems}

\begin{prob}[F2010-Q1]
    Let \( G \) be a group. Let \( H \) be a subset of \( G \) that is closed under group multiplication. Assume that \( g^2 \in H \) for all \( g \in G \). Show that:
    \begin{itemize}
        \item \( H \) is a normal subgroup of \( G \)
        \item \( G/H \) is abelian
    \end{itemize}
\end{prob}
\begin{proof}
    \begin{itemize}
        \item We first show that $H$ is subgroup. It remains to show that if $h\in H$, then $h^{-1}\in H$, we know $(h^{-1})^2\in H$, thus 
        \begin{equation*}
            h(h^{-1})^2=h^{-1}\in H
        \end{equation*}
        as desired.  Now we show that $H$ is normal: for any $h\in H$, $g\in G$, we want to show $ghg^{-1}\in H$.
        \begin{align*}
            ghg^{-1}&=(gh)^2(gh)^{-1}hg^{-1}\\
            &=(gh)^2h^{-1}g^{-1}hg^{-1}\\
            &=(gh)^2h^{-1}(g^{-1}h)^2(g^{-1}h)^{-1}g^{-1}\\
            &=(gh)^2h^{-1}(g^{-1}h)^2h^{-1}\in H
        \end{align*}
        as desired.
        \item It suffices to show that for any $g_1,g_2\in G$, we have 
        \begin{equation*}
            g_1g_2H\subset g_2g_1H
        \end{equation*}
        Take any $h\in H$, we want to show $(g_2g_1)^{-1}g_1g_2h\in H$,
        \begin{align*}
            (g_2g_1)^{-1}g_1g_2h&=(g_2g_1)^{-2}g_2g_1^2g_2h\\
            &=(g_2g_1)^{-2}(g_2g_1^2)^2(g_2g_1^2)^{-1}g_2h\\
            &=(g_2g_1)^{-2}(g_2g_1^2)^2g_1^{-2}h\in H
        \end{align*}
        as desired.
    \end{itemize}
\end{proof}

\begin{prob}[S2014-Q1]
    Find the number of colorings of the faces of a cube using 3 colors, where two colorings are considered equal if they can be transformed into each other by a rotation of the cube.
    
    [\textit{Hint}: Use Burnside's formula:
    
    \[
    |X/G| = \frac{1}{|G|} \sum_{g \in G} |X^g|,
    \]
    
    where a group $G$ acts on a set $X$, $X/G$ is the set of orbits, and for every $g \in G$, $X^g$ is the fixed subset of $g$ in $X$.]
\end{prob}
\begin{proof}
    Let $X$ be the set of all possible colorings of the cube (equal cubes allowed), we have $|X|=3^6$. We notice two things:
    \begin{enumerate}
        \item The group of rotations of a cube is $S_4$.
        \item For $\sigma_1,\sigma_2\in S_4$ that are conjugates of each other, $|X^{\sigma_1}|=|X^{\sigma_2}$. Therefore for the Burnside's formula becomes 
        \begin{equation*}
            |X/S_4|=\frac{1}{|S_4|}\sum_{[\sigma]
            \text{ conj classes}}|[\sigma]|\cdot|X^\sigma|
        \end{equation*}
    \end{enumerate}
    Now we analyze for each conjugacy class $[\sigma]$, what is $|X^\sigma|$.
    \begin{itemize}
        \item $(1+1+1+1)$, $|[e]|=1$ and $|X^e|=3^6$.
        \item $(1+1+2)$, $|[\sigma_1]|=6$ and $|X^{\sigma_1}|=3^3$.
        \item $(1+3)$, $|[\sigma_2]|=8$, and $|X^{\sigma_2}|=3^2$.
        \item $(2+2)$, $|[\sigma_3]|=6$, and $|X^{\sigma_3}|=3^4$.
        \item $(4)$, $|[\sigma_4]|=6$, and and $|X^{\sigma_4}|=3^3$.
    \end{itemize}
    Thus combining we get 
    \begin{equation*}
        |X/S_4|=\frac{1}{24}\left(3^6+6\cdot 3^3+8\cdot 3^2+6\cdot 3^4+6\cdot 3^3\right)=57
    \end{equation*}
\end{proof}

\begin{prob}[S2019-Q4]
    Let \( f \) be a polynomial with \( n \) variables and define
\[
\operatorname{Sym}(f) = \{ \sigma \in S_n \mid f(x_{\sigma(1)}, x_{\sigma(2)}, \ldots, x_{\sigma(n)}) = f(x_1, x_2, \ldots, x_n) \}.
\]

\begin{enumerate}
    \item Prove that \(\operatorname{Sym}(f)\) is a subgroup of \( S_n \).
    
    \item Prove that the dihedral group \( D_4 \) (the group of symmetries of the square) is isomorphic to \(\operatorname{Sym}(x_1 x_2 + x_3 x_4)\).
\end{enumerate}
\end{prob}
\begin{proof}
    \begin{enumerate}
        \item The group $S_n$ acts on the polynomial ring $k[x_1,\dots,x_n]$, by permuting the $x_i$ to $x_{\sigma(i)}$, and we see that Sym$(f)$ is the centralizer of a fixed element $f\in k[x_1,\dots,x_n]$, hence is a subgroup.
        \item We have a total of $8$ elements in Sym$(x_1x_2+x_3x_4)$: 
        \begin{equation*}
            \{e, (12), (34), (12)(34), (13)(24), (14)(23), (1324), (1423)\}
        \end{equation*}
        and we can by drawing a square tha this corresponds to the group $D_4$.
    \end{enumerate}
\end{proof}

\begin{prob}[S2011-Q1, F2004-Q1]
    \phantom{text}
    \begin{itemize}
    \item[(a)] Let \( H \) be a proper nontrivial subgroup of a finite group \( G \) (i.e., \( H \neq \{1\} \) and \( H \neq G \)).  
    Prove that \( G \) is not the union of all conjugates of \( H \) in \( G \).
    
    \item[(b)] Give an example of an infinite group \( G \) for which the assertion in part (a) fails.
    \end{itemize}
\end{prob}
\begin{proof}
    \begin{itemize}
        \item[(a)] If $H$ is normal, then all conjugations of $H$ is equal to $H$, but $H\subsetneq G$, this $G$ is not not the union of all conjugates of \( H \) in \( G \). Now suppose $H$ is not normal, assume the contrary that $G$ is the union of all conjugates of $H$, then the number of distinct conjugates of $H$ is $[G:N_G(H)]$, hence 
        \begin{equation*}
            |G|=[G:N_G(H)]\cdot|H|\iff [G:H]=[G:N_G(H)]\iff [N_G(H):H]=1
        \end{equation*}
        this is a contradiction since $H$ is not normal. Thus $G$ not the union of all conjugates of \( H \) in \( G \).
        \item[(b)] Consider 
        \begin{equation*}
            B=\left\{\begin{pmatrix}
                *&*\\
                0&*
            \end{pmatrix}\right\}\subset \text{GL}_2(\mathbb{C})
        \end{equation*}
        It is clear that conjugation of matrices in $B$ do not give matrices with nonzero left bottom entry.
    \end{itemize}
\end{proof}

\begin{prob}[S2009-Q1]
    Let \( H \) and \( K \) be two solvable subgroups of a group \( G \) such that \( G = HK \).
    
    \begin{enumerate}
        \item Show that if either \( H \) or \( K \) is normal in \( G \), then \( G \) is solvable.
        
        \item Give an example where \( G \) may not be solvable without the assumption in (a).
    \end{enumerate}
\end{prob}
\begin{proof}
    \begin{enumerate}
        \item WLOG suppose that $H$ is normal, then the composite map $\varphi=\pi\circ\iota$: 
        \[\begin{tikzcd}
            K & G & {G/H}
            \arrow["\iota", from=1-1, to=1-2]
            \arrow["\pi", from=1-2, to=1-3]
        \end{tikzcd}\]
        is surjective, therefore 
        \begin{equation*}
            \{e\}\subset H\subset G
        \end{equation*}
        $G/H\cong K/\ker(\varphi)$ is solvable, hence $G$ is solvable.
        \item The smallest nonsolvable group is $A_5$, we have 
        \begin{equation*}
            A_5=HK
        \end{equation*}
        where $H=\la(12345)\ra, K=A_4=\{\sigma\in A_5: \sigma(5)=5\}$. Now $H, K$ are both solvable, but $G$ is not.
    \end{enumerate}
\end{proof}

\begin{prob}[F2003-Q1]
    In a group \( G \), let \( 1 \) denote the identity element and let \([x, y] = xyx^{-1}y^{-1}\) denote the commutator of elements \(x, y \in G\).
    
    \begin{enumerate}
        \item Express \([z, xy]x\) in terms of \(x\), \([z, x]\), and \([z, y]\).
        
        \item Prove that if the identity \([[x, y], z] = 1\) holds in \(G\), then the following identities hold in \(G\):
        \[
        [x, yz] = [x, y][x, z] \quad \text{and} \quad [xy, z] = [x, z][y, z].
        \]
    \end{enumerate}
\end{prob}
\begin{proof}
    \begin{enumerate}
        \item We have 
        \begin{align*}
            [z,xy]x&=zxyz^{-1}y^{-1}x^{-1}x\\
            &=zxz^{-1}x^{-1}xzyz^{-1}y^{-1}\\
            &=[z,x]x[z,y]
        \end{align*}
        \item The identity $[[x,y],z]=1$ implies 
        \begin{equation*}
            [x,y]z=z[x,y]
        \end{equation*}
        Therefore using the identity in 1, we have 
        \begin{align*}
            [x,yz]&=[x,y]y[x,z]y^{-1}\\
            &=[x,y]yy^{-1}[x,z]\\
            &=[x,y][x,z]
        \end{align*}
        Similarly
        \begin{align*}
            [xy,z]&=xyzy^{-1}x^{-1}z^{-1}\\
            &=xyzy^{-1}z^{-1}zx^{-1}z^{-1}\\
            &=x[y,z]x^{-1}[x,z]\\
            &=[y,z][x,z]\\
            &=[x,z][y,z]
        \end{align*}
    \end{enumerate}
\end{proof}

\begin{prob}[S2005-Q1]
    Let $k$ be a field. Let $G = \mathrm{GL}_n(k)$ be the general linear group, where $n > 0$. Let $D$ be the subgroup of diagonal matrices, and let $N = N_G(D)$ be the normalizer of $D$ in $G$. Determine the quotient group $N/D$.
\end{prob}

\begin{prob}[F2009-Q1]
    Let \( G \) be a finite group, and let \( \mathrm{Aut}(G) \) be its automorphism group. Consider the group action \( \phi \colon \mathrm{Aut}(G) \times G \to G \) defined by \( \phi(\sigma, g) = \sigma(g) \). Assume \( G \) has exactly two orbits under this action.
    
    \begin{enumerate}
        \item Determine all such groups \( G \) up to isomorphism.
        
        \item For each case from (a), determine when \( \mathrm{Aut}(G) \) is solvable.
    \end{enumerate}
\end{prob}

\begin{prob}[F2016-Q1]
    Determine $\text{Aut}(S_3)$.
\end{prob}
\begin{proof}
    Every element $\sigma\in\text{Aut}(S_3)$ must send order $2$ elements $\{(12), (23), (13)\}$ to one another, and order $3$ elements $\{(123), (132)\}$ to each other. However, $\sigma$ is determined by how it permutes 
    \begin{equation*}
        \{(12), (23), (13)\}
    \end{equation*}
    Thus every $\sigma$ is an inner automorphism of the form $\sigma_g(h)=ghg^{-1}$ for $g,h\in S_3$ and $g$ is some transposition. Hence 
    \begin{equation*}
        \text{Aut}(S_3)\cong S_3
    \end{equation*}
\end{proof}


\chapter{Representation Theory}


% \begin{thm}[Maschke's theorem]

% \end{thm}

% \begin{lem}[Schur's Lemma]
    
% \end{lem}

\begin{prop}[properties of characters]
    
\end{prop}
\begin{prop}
    The character tables for $S_3, S_4, A_5, S_5$ are as follows:
\end{prop}

\begin{thm}[Compilation of theorems]
    Schur's lemma: 
    \begin{enumerate}
        \item If $\varphi: V\to W$ is a $G$-invariant map, i.e., 
        \begin{equation*}
            \varphi(\rho(g)(v))=\rho(g)\varphi(v)
        \end{equation*} 
        where $V,W$ are irreducible representations, then $\varphi=0$ or an isomorphism. This is true for any field $k$ that $V,W$ are over.
        \item If $\varphi: V\to V$ and everything as above, then 
        \begin{equation*}
            \varphi(v)=\lambda v
        \end{equation*}
        for some $\lambda\in k^\times$. This is only true when $k$ is algebraically clsoed.
        \item $\text{Hom}_G(V,W)\begin{cases}
            k \text{ if } V\cong W\\
            0 \text{ if  not}
        \end{cases}$, where $V,W$ are irreducible. This is true for $k$ algebraically closed.
        \item Mascheke's theorem: any finite dimensional representation $V$ of a finite group $G$ can be decomposed into a direct sum of irreducible representations.
        \begin{equation*}
            V=V_1^{r_1}\oplus\dots\oplus V_k^{r_k}
        \end{equation*}
        where $V_i$'s are irreducible. This is true when the characteristic $k$ does not divide $|G|$, notably this always holds for characteristic $0$ fields.
        \item Do not mix them up.
    \end{enumerate}
\end{thm}


\begin{prop}
    $G$ is abelian if and only if every irreducible representation $\rho$ is one-dimensional.
\end{prop}
\begin{proof}
    If $G$ is abelian, take any irreducible representation $\rho$, 
    \begin{equation*}
        \left\{\rho(g): g\in G\right\}
    \end{equation*}
    can be simultaneously diagonalized (minimal polynomial has no repeated factor), i.e., there exists an eigenbasis $\{e_1,\dots,e_n\}$ such that $\rho(g)$ is a diagonal matrix for all $g$. This implies that the vector space generated by $\{e_i\}$ for each $i$ is a $\rho$-invariant subspace, since $\rho$ is irreducible, $\rho$ must be one-dimensional. 

    Conversely, let $|G|=n$, if every irreducible $\rho$ is one-dimensional, then there are $n$ irreducible representations, i.e., $n$ conjugacy classes, i.e., $G$ is abelian.
\end{proof}

\section{Characters}

\begin{prob}[S2008-Q4]
    Let \( V \cong \mathbb{C}^n \) be an \( n \)-dimensional complex vector space with standard basis \( e_1, \ldots, e_n \). Consider the permutation action \( S_n \times V \to V \) defined by:
    \[
    \sigma \cdot e_i = e_{\sigma(i)} \quad \text{for } \sigma \in S_n
    \]
    Decompose \( V \) into irreducible \( \mathbb{C}[S_n] \)-modules.
\end{prob}

\begin{prob}[S2014-Q5]
    Find the table of characters for $S_4$.
\end{prob}

\begin{prob}[F2016-Q6]
    Find a table of characters for the alternating group $A_5$.
\end{prob}


\begin{prob}[F2015-Q3]
    Let \( G = S_4 \) (the symmetric group on four letters).

\begin{enumerate}
    \item[(a)] Prove that \( G \) has two non-equivalent irreducible complex representations of dimension 3; call them \( \rho_1 \) and \( \rho_2 \).
    
    \item[(b)] Decompose the tensor product representation \( \rho_1 \otimes \rho_2 \) into a direct sum of irreducible representations of \( G \).
\end{enumerate}
\end{prob}


\begin{prob}[F2011-Q4]
    Let \(\rho \colon S_3 \to \mathrm{GL}(2, \mathbb{C})\) be a two-dimensional irreducible representation of the symmetric group \(S_3\). 

\begin{enumerate}
    \item Decompose the tensor square \(\rho^{\otimes 2}\) into irreducible representations of \(S_3\).
    \item Decompose the tensor cube \(\rho^{\otimes 3}\) into irreducible representations of \(S_3\).
\end{enumerate}
\end{prob}

\begin{prob}[F2014-Q3]
    Let \( G = S_3 \) be the symmetric group on three elements.

\begin{enumerate}
    \item[(a)] Prove that \( G \) has an irreducible complex representation of dimension 2 (call it \( \rho \)), but none of higher dimension.
    
    \item[(b)] Decompose the triple tensor product \( \rho \otimes \rho \otimes \rho \) into a direct sum of irreducible representations of \( G \).
\end{enumerate}
\end{prob}

\begin{prob}[S2006-Q6]
    Let \( S_4 \) be the symmetric group on four elements.
    
    \begin{enumerate}
        \item Give an example of a non-trivial 8-dimensional complex representation of \( S_4 \).
        
        \item Show that every 8-dimensional complex representation of \( S_4 \) contains a 2-dimensional invariant subspace.
    \end{enumerate}
\end{prob}

\begin{prob}[F2007-Q5]
    Prove that every 5-dimensional complex representation of the alternating group \( A_4 \) (the alternating group of degree 4) contains a 1-dimensional invariant subspace.
\end{prob}

\begin{prob}[S2004-Q6]
    Consider complex representations of a finite group \( G \). Let \( \sigma_1, \ldots, \sigma_s \) be representatives of the conjugacy classes of \( G \), and let \( \chi_1, \ldots, \chi_s \) be the irreducible characters of \( G \).
    
    \begin{enumerate}
        \item Define an inner product on the \(\mathbb{C}\)-vector space of class functions on \( G \) such that \( \{\chi_1, \ldots, \chi_s\} \) forms an orthonormal basis.
        
        \item Let \( A = (a_{ij}) \) be the character table matrix of \( G \), where \( a_{ij} = \chi_i(\sigma_j) \) for \( 1 \leq i, j \leq s \). Prove that \( A \) is invertible.
    \end{enumerate}
\end{prob}

\begin{prob}[S2018-Q4, S2007-Q5]
    Is $S_4$ isomorphic to a subgroup of $\text{GL}_2(\mathbb{C})$?
\end{prob}


\begin{prob}[S2010-Q6]
    Let \( G \) be a group of order 24. Using representation theory, prove that \( G \neq [G, G] \), where \([G, G]\) denotes the commutator subgroup of \( G \).
\end{prob}
\begin{proof}
    Suppose $G=[G,G]$, then we claim the only $1$-dimensional representation $\rho:G\to\mathbb{C}^\times$ is the trivial one. This is because if $\rho$ is one-dim, then
    \begin{equation*}
        [G,G]\subset\ker(\rho)
    \end{equation*}
    i.e., $\rho$ is trivial. However, there is no way to write 
    \begin{equation*}
        |G|=24=1+d_1^2+\dots+d_k^2
    \end{equation*}
    where $d_i\geq 2$. Thus $G\neq [G,G]$.
\end{proof}

\begin{prob}[F2017-Q6]
    Let \( G \) be a finite group with center \( Z(G) \). Show that if \( G \) admits a faithful irreducible representation \( \rho \colon G \to \mathrm{GL}_n(k) \) for some positive integer \( n \in \mathbb{Z}^+ \) and some field \( k \), then the center \( Z(G) \) is cyclic.
\end{prob}
\begin{proof}
    (We will only do the case where $k$ is algebraically closed). For any $z\in Z(G)$, $\rho(z): V\to V$ is a $G$-map, i.e., 
    \begin{equation*}
        \rho(z)(\rho(g)v)=\rho(g)(\rho(z)v)
    \end{equation*}
    We know by Schur's lemma that $\rho(z)$ is a scalar multiplication: 
    \begin{equation*}
        \rho(z)\in k^\times
    \end{equation*}
    Because $\rho$ is faithful, $Z$ embeds into $k^\times$ via $\rho$. 
    \begin{lem}[Fact]
        Any finite subgroup of $k^\times$ for field $k$ is cyclic.
    \end{lem}
    Hence $Z$ is cyclic.
\end{proof}


\begin{prob}[S2005-Q6]
    Let \( V \) be a finite-dimensional vector space over a field \( k \), and let \( G \) be a finite group with an irreducible representation \( \varphi \colon G \to \mathrm{GL}(V) \). Suppose \( H \) is a finite abelian subgroup of \( \mathrm{GL}(V) \) contained in the centralizer of \( \varphi(G) \). Prove that \( H \) must be cyclic.
\end{prob}
\begin{proof}
    Just like above, we embed $H$ into $k^\times$. Let any $h\in H$, we note that $h$ is a $G$-map, i.e., for any $g\in G$, 
    \begin{equation*}
        h(\varphi(g)v)=\varphi(g)hv
    \end{equation*}
    this is because $h$ is contained in the centralizer of $\varphi(G)$, i.e., commutes with all $\varphi(g)$. By Schur's Lemma, we have 
    \begin{equation*}
        h=\lambda I, \text{ where }\lambda\in k^\times
    \end{equation*}
    One can define a homomorphism $\psi: H\to k^\times$ such that 
    \begin{equation*}
        \psi(\lambda I)=\lambda
    \end{equation*}
    This map embeds $H$ into $k^\times$, and we are done by again observing any finite subgroup of $k^\times$ is cyclic, 
\end{proof}






\begin{prob}[F2010-Q6]
    Let \( G \) be a non-abelian group of order \( p^3 \), where \( p \) is prime.
    
    \begin{enumerate}
        \item Determine the number of isomorphism classes of irreducible complex representations of \( G \), and find their dimensions.
        
        \item Which of these irreducible complex representations are faithful? Justify your answer.
    \end{enumerate}
\end{prob}
\begin{proof}
    \begin{enumerate}
        \item In S2010-Q1, we showed there are $p^2-1+p$ conjugacy classes in a non-abelian group $G$ of order $p^3$. There are $p^2$ one-dimensional irreducible representations because one dimensional representations of $G$ are equivalent to one-dimensional representations of $G/[G,G]$ which has size $p^2$, thus abelian and all irreducible representations are one-dimensional. 
        \begin{lem}[Fact]
            Let $V$ be an irreducible representation, then $\dim V$ divides $|G|$. (This is true when $k$ is algebraically closed and characteristic $0$).
        \end{lem}
        Thus it is clear that there are $p-1$ representations of dimension $p$. (Sanity check: $|G|=p^3=p^2+(p-1)p^2$).
        \item We claim that all the one-dimensional representations are not faithful and all the $p$-dimensional representations are. Recall $\rho$ is irreducible if and only if $\ker(\rho)=\{g: \rho(g)v=v \text{ for all }v\}=\{e\}$.
        \begin{lem}[Fact]
            Let $\rho:G\to\CC^\times$ be a one-dimensional irreducible representation, then
            \begin{equation*}
                [G,G]\subset\ker(\rho)
            \end{equation*}
        \end{lem}
        Thus if $\rho$ is one-dimensional, then $\rho$ is not faithful. Now for the higher dimensional case:
        \begin{lem}[Fact]
            If $\rho: G\to\text{GL}_p(\CC)$ is an irreducible representation, then $\bar{\rho}: \frac{G}{\ker\rho}\to\text{GL}_p(\CC)$ is also irreducible.
        \end{lem}
        If $\ker\rho$ is nontrivial, then it must divide the size of $|G|$, hence $\frac{G}{\ker\rho}$ is abelian, i.e., all irreducible representations are one-dimensional. This is a contradiction since $\rho$ is $p$-dimensional, thus $\ker(\rho)=\{e\}$, as desired.
    \end{enumerate}
\end{proof}


\begin{prob}[S2011-Q5]
    Let \( K \) be a field, and let \( \Phi \colon G \to \mathrm{GL}_n(K) \) be an \( n \)-dimensional matrix representation of a group \( G \). Define a \( G \)-action on the matrix ring \( M_n(K) \) by:
\[
(g, A) \mapsto \Phi(g) \cdot A \quad \text{(matrix multiplication)}
\]
for \( g \in G \) and \( A \in M_n(K) \). This action induces a group homomorphism \( \Psi \colon G \to \mathrm{GL}(M_n(K)) \). Express the character \( \chi_\Psi \) of \( \Psi \) in terms of \( \chi_\Phi \) (the character of \( \Phi \)).
\end{prob}

\begin{prob}[S2015-Q5]
    Prove that a tensor product of irreducible representations over an algebraically closed field is irreducible.
\end{prob}

\begin{prob}[S2001-Q3]
    Calculate the complete character table for $\Z/3\Z\times S_3$, where $S_3$ is the symmetric group in 3 letters.
\end{prob}

\section{Induced representations}


\begin{prob}[S2009-Q6]
    Let \( G = S_4 \) and consider the subgroup \( H = \langle (1\,2), (3\,4) \rangle \).
    
    \begin{enumerate}
        \item[(a)] Determine the number of irreducible complex characters of \( H \).
        
        \item[(b)] Choose a non-trivial irreducible character \( \psi \) of \( H \) over \( \mathbb{C} \) satisfying \( \psi((1\,2)(3\,4)) = -1 \). Compute the values of the induced character \( \operatorname{ind}_H^G(\psi) \) on all conjugacy classes of \( G \), and express it as a sum of irreducible characters of \( G \).
    \end{enumerate}
\end{prob}

\section{Frobenius Reciprocity}

\begin{prob}[S2017-Q6]
    Let $G$ be a finite group and H an abelian subgroup. Show that every
    irreducible representation of $G$ over $\mathbb{C}$ has dimension 
    $\leq[G : H]$.
\end{prob}
\begin{proof}
    We know that if $A$ is commutative, then all the irreducible representations $\rho$ of $A$ are one-dimensional. Now we induce $\rho$ to a representation on $G$:
    \begin{equation*}
        \bar{\rho}:G\to\text{GL}(\mathbb{C})
    \end{equation*}
    We have 
    \begin{equation*}
        \text{Ind}_A^G=\bigoplus_{i=1}^n g_iV
    \end{equation*}
    where $g_i$ is the representative for each coset $G/A$, and $n=[G:A]$. Therefore all representations of $G$ has dimension $[G:A]$. Since not all induced representataions are irreducible, any irreducible representation of $G$ has dimension $\leq[G:A]$, as desired.
\end{proof}

\begin{prob}[S2008-Q6]
    Give an example of non-isomorphic finite groups with same character
table. Construct the character table in detail.
\end{prob}

\begin{prob}[S2012-Q4]
    Let \( Q \) be the quaternion group with presentation:
\[
Q = \langle t, s_i, s_j, s_k \mid t^2 = 1,\ s_i^2 = s_j^2 = s_k^2 = s_i s_j s_k = t \rangle.
\]

\begin{enumerate}
    \item[(a)] Find four non-isomorphic 1-dimensional real representations of \( Q \).
    
    \item[(b)] Prove that the natural embedding \( \rho \colon Q \to \mathbb{H} \) given by:
    \[
    \rho(t) = -1, \quad \rho(s_i) = i, \quad \rho(s_j) = j, \quad \rho(s_k) = k
    \]
    defines an irreducible 4-dimensional real representation of \( Q \), where \( \mathbb{H} \) is the algebra of real quaternions.
    
    \item[(c)] Classify all irreducible complex representations of \( Q \) up to isomorphism.
\end{enumerate}
\end{prob}

\begin{prob}[F2004-Q6]
    Let \( D_8 \) be the dihedral group of order 8, with presentation:
    \[
    D_8 = \langle r, s \mid r^4 = 1 = s^2,\ rs = sr^{-1} \rangle.
    \]
    
    \begin{enumerate}
        \item Determine all conjugacy classes of \( D_8 \).
        
        \item Find the commutator subgroup \( D_8' \) of \( D_8 \) and determine the number of distinct degree-1 (linear) characters of \( D_8 \).
        
        \item Construct the complete complex character table of \( D_8 \).
    \end{enumerate}
\end{prob}

\begin{prob}[F2000-Q7]
    Let \( D_{10} \) be the dihedral group of order 10, with presentation:
    \[
    D_{10} = \langle r, s \mid r^5 = 1 = s^2,\ rs = sr^{-1} \rangle.
    \]
    
    \begin{enumerate}
        \item Determine all conjugacy classes of \( D_{10} \).
        
        \item Compute the commutator subgroup \( D_{10}' \) of \( D_{10} \).
        
        \item Prove that \( D_{10}/D_{10}' \cong \mathbb{Z}/2\mathbb{Z} \) and deduce that \( D_{10} \) has exactly two distinct degree-1 characters.
        
        \item Construct the complete complex character table of \( D_{10} \).
    \end{enumerate}
\end{prob}



















